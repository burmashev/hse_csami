\documentclass[a4paper,12pt]{article}

%%% Работа с русским языком
\usepackage{cmap}					% поиск в PDF
\usepackage{mathtext} 				% русские буквы в формулах
\usepackage[T2A]{fontenc}			% кодировка
\usepackage[utf8]{inputenc}			% кодировка исходного текста
\usepackage[english,russian]{babel}	% локализация и переносы
\usepackage{xcolor}
\usepackage{hyperref}
 % Цвета для гиперссылок
\definecolor{linkcolor}{HTML}{799B03} % цвет ссылок
\definecolor{urlcolor}{HTML}{799B03} % цвет гиперссылок

\hypersetup{pdfstartview=FitH,  linkcolor=linkcolor,urlcolor=urlcolor, colorlinks=true}

%%% Дополнительная работа с математикой
\usepackage{amsfonts,amssymb,amsthm,mathtools} % AMS
\usepackage{amsmath}
\usepackage{icomma} % "Умная" запятая: $0,2$ --- число, $0, 2$ --- перечисление

%% Номера формул
%\mathtoolsset{showonlyrefs=true} % Показывать номера только у тех формул, на которые есть \eqref{} в тексте.

%% Шрифты
\usepackage{euscript}	 % Шрифт Евклид
\usepackage{mathrsfs} % Красивый матшрифт

%% Свои команды
\DeclareMathOperator{\sgn}{\mathop{sgn}}

%% Перенос знаков в формулах (по Львовскому)
\newcommand*{\hm}[1]{#1\nobreak\discretionary{}
{\hbox{$\mathsurround=0pt #1$}}{}}
% графика
\usepackage{graphicx}
\graphicspath{{pictures/}}
\DeclareGraphicsExtensions{.pdf,.png,.jpg}
\author{Бурмашев Григорий, БПМИ-208}
\title{Линал. ИДЗ - 2 Вариант 2.}
\date{\today}
\begin{document}
\begin{center}
Бурмашев Григорий. 208. Дискра -- 8
\end{center}
\section*{Номер 1}
Степень многочлена равна n = 12, а степени в множителе равны:
\[
\alpha = (2, 2, 2, 2, 2, 2)
\]
Тогда коэффициент будет равен:
\[
\begin{pmatrix}
n \\
\alpha_1, \alpha_2, \ldots \alpha_6 \\ 
\end{pmatrix} = \frac{12!}{2!2!2!2!2!2!} = \frac{479001600}{64}  = 7484400
\]
\begin{center}
\textbf{Ответ:} 7484400
\end{center}

\subsection*{Номер 2}
Всего у нас 12 человек. Для первого человека есть 11 вариантов составить пару. Для второго человека остается (11 - 2) = 9  вариантов составить пару. Для третьего человека остается (9  - 2) = 7 вариантов составить пару и так далее (последнему человеку пару можно выбрать всего одним способом). Т.е:
\[
11 \cdot 9 \cdot 7 \cdot 5 \cdot 3 \cdot 1 = 10395
\]
\begin{center}
\textbf{Ответ:} 10395
\end{center}

\subsection*{Номер 3}
У нас есть $C_{10}^4$ способов расставить в слове длины 10 четыре буквы A. Остается 6 позиций, на каждую из которых мы можем поставить либо Б, либо В (всего $2^6 = 64$ способов). Значит:
\[
C_{10}^4 \cdot 64 = \frac{10!}{4!(10-4)!} \cdot = 210 \cdot 64 = 13440
\]
\begin{center}
\textbf{Ответ:} 13440
\end{center}

\newpage
\subsection*{Номер 4}
Пускай (x, y) -- число способов попасть в точку с координатами x, y. Тогда:
\begin{equation*}
\begin{gathered}
(4, 5) = (3, 5) + (4, 4) + (2, 3) = 189 \\
(4, 4) = (3, 4) + (4, 3) + (2, 2) = 101 \\ 
(3, 5) = (2, 5) + (3, 4) + (1, 3) = 76 \
(3, 4) = (2, 4) + (3, 3) + (1, 2) = 47 \\
(4, 3) = (3, 3) + (4, 2) + (2, 1) = 47 \\
(3, 3) = (2, 3) + (3, 2) + (1, 1) = 26\\
(2, 5) = (1, 5) + (2, 4) + (0, 5) = 25 \\
(4, 2) = (3, 2) + (4, 1) + (2, 0) = 18 \\
(2, 4) = (4, 2) = 18 \\
(3, 2) = (2, 2) + (3, 1) + (1, 0) = 12 \\
(2, 3) = (1, 3) + (2, 2) + (0, 1) = 12 \\
(2, 2) = (1, 2) + (2, 1) + (0, 0) = 7 \\ 
(1, 5) = (0,5) + (1, 4) = 6 \\
(1, 4) = (0, 4) + (1, 3) = 5\\
(4, 1) = (3, 1) + (4, 0) = 5 \\
(3, 1) = (2, 1) + (3, 0) = 4 \\
(1, 3) = (0, 3) + (1, 2) = 4 \\
(2, 1) = (1, 1) + (2, 0) = 3 \\
(1, 2) = (0, 2) + (1, 1) = 3 \\
(1, 1) = (0, 1) + (1, 0) = 2 \\
(0, 0) = (1, 0) = (2, 0) = (3, 0) = (4, 0) = (5, 0) = 1 \\
(0, 0) = (0, 1) = (0, 2) = (0, 3) = (0, 4) = (0, 5) = 1 \\
\end{gathered}
\end{equation*}
\begin{center}
\textbf{Ответ:} 189
\end{center}
\subsection*{Номер 5}
По формуле числа сочетаний с повторениями из 8 по 8 (т.к человек, как я понял, может голосовать в том числе сам за себя):
\[
C_{8 + 7}^{7} = C_{15}^7 = \frac{15!}{7!8!} = 6435
\]
\begin{center}
\textbf{Ответ:} 6435
\end{center}
\subsection*{Номер 6}
None

\subsection*{Номер 7}
ОБОРОНОСПОСОБНОСТЬ \\\\
Посчитаем количество букв:

О -- 7 шт

Б -- 2 шт

Р -- 1 шт

И -- 2 шт

С -- 3 шт

П -- 1 шт

Т -- 1 шт

Ь -- 1 шт 

Т.к никакие две буквы О не должны стоять рядом, расставим сначала их. 

Пусть мы расставили все 7 букв О рядом друг с другом:
\begin{center}
\underline{О} \underline{О} \underline{О} \underline{О} \underline{О} \underline{О} \underline{О} 
\end{center}
Нам понадобится выделить 6 позиций, чтобы разделить эти 7 букв друг от друга, т.е (где x -- что-нибудь, кроме О): 
\begin{center}
\underline{О} \underline{x} \underline{О} \underline{x} \underline{О} \underline{x} \underline{О} \underline{x} \underline{О} \underline{x} \underline{О} \underline{x} \underline{О}
\end{center}
Ну а позиции слева и справа от крайних О отделять не нужно, значит О мы можем расставить на 18 - 6 = 12 позиций. А всего букв О у нас 7, значит:
\[
C^7_{12} = \frac{12!}{7!5!}
\]
У нас остается 18 - 7 = 11 позиций, на которые нужно расставить оставшиеся буквы. По формуле мультиномиальных коэффициентов:
\[
\begin{pmatrix}
11 \\
2, 1, 2, 3, 1, 1, 1 \\ 
\end{pmatrix}= 
\frac{11!}{2!1!2!3!1!1!1!} = \frac{11!}{2!2!3!}
\]
Итого:
\[
\frac{12!}{7!5!}  \cdot \frac{11!}{2!2!3!} = 792 \cdot 1663200
\]
\begin{center}
\textbf{Ответ: } $ 792 \cdot 1663200 $
\end{center}
\newpage

\subsection*{Номер 8}
Воспользуемся методом перегородок:

Чтобы разложить 20 книг на 5 полок, нам понадобится 5 - 1 = 4 перегородки. Тогда нам нужно посчитать количество способов расставить 4 перегородки на 20 + 4 = 24 позиции:
\[
C_{24}^4
\]
На каждую из этих расстановок перегородок у нас есть 20! способов расставить книги. Значит всего способов:
\[
C_{24}^4 \cdot 20! = \frac{24!}{4! 20!} \cdot 20! = \frac{24!}{4!} 
\]
\begin{center}
\textbf{Ответ:} $\frac{24!}{4!} $
\end{center}
\subsection*{Номер 9}
None 

\subsection*{Номер 10}
\[
\begin{pmatrix}
1010 \\ 400 \\
\end{pmatrix} = \frac{1010!}{400!610!}
\]
\[
\begin{pmatrix}
1011 \\ 401 \\
\end{pmatrix} = \frac{1011!}{401!610!}
\]
\[
\begin{pmatrix}
1010 \\ 401 \\
\end{pmatrix} = \frac{1010!}{401!609!}
\]
\[
\begin{pmatrix}
1011 \\ 400 \\
\end{pmatrix} = \frac{1011!}{400!611!}
\]
Тогда сравним:
\[
\frac{1010! 1011!}{400!610!401!610}  \cup \frac{1010!1011!}{401!609!400!611!}
\]
\[
\frac{1}{400!610!401!610!}  \cup \frac{1}{401!609!400!611!}
\]
\[
\frac{1}{610!610!}  \cup \frac{1}{609!611!}
\]
\[
\frac{1}{610!609! \cdot 610}  \cup \frac{1}{609!611!}
\]
\[
\frac{1}{610!\cdot 610}  \cup \frac{1}{611!}
\]
\[
\frac{1}{610!\cdot 610}  \cup \frac{1}{610! \cdot 611}
\]
\[
\frac{1}{610}  \cup \frac{1}{611}
\]
\[
610 < 611 \rightarrow \frac{1}{610}  >  \frac{1}{611} \rightarrow \frac{1010! 1011!}{400!610!401!610}  > \frac{1010!1011!}{401!609!400!611!}
\]
Итого:
\[
\begin{pmatrix}
1010 \\ 400 \\
\end{pmatrix}
\begin{pmatrix}
1011 \\ 401 \\
\end{pmatrix} > 
\begin{pmatrix}
1010 \\ 401 \\
\end{pmatrix}
\begin{pmatrix}
1011 \\ 400 \\
\end{pmatrix} 
\]
\end{document}