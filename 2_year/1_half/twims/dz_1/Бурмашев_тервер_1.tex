\documentclass[a4paper,12pt]{article}

%%% Работа с русским языком
\usepackage{cmap}					% поиск в PDF
\usepackage{mathtext} 				% русские буквы в формулах
\usepackage[T2A]{fontenc}			% кодировка
\usepackage[utf8]{inputenc}			% кодировка исходного текста
\usepackage[english,russian]{babel}	% локализация и переносы
\usepackage{ulem}
\usepackage{xcolor}
\usepackage{hyperref}
 % Цвета для гиперссылок
\definecolor{linkcolor}{HTML}{799B03} % цвет ссылок
\definecolor{urlcolor}{HTML}{799B03} % цвет гиперссылок

\hypersetup{pdfstartview=FitH,  linkcolor=linkcolor,urlcolor=urlcolor, colorlinks=true}

%%% Дополнительная работа с математикой
\usepackage{amsfonts,amssymb,amsthm,mathtools} % AMS
\usepackage{amsmath}
\usepackage{icomma} % "Умная" запятая: $0,2$ --- число, $0, 2$ --- перечисление

%% Номера формул
%\mathtoolsset{showonlyrefs=true} % Показывать номера только у тех формул, на которые есть \eqref{} в тексте.

%% Шрифты
\usepackage{euscript}	 % Шрифт Евклид
\usepackage{mathrsfs} % Красивый матшрифт

%% Свои команды
\DeclareMathOperator{\sgn}{\mathop{sgn}}

%% Перенос знаков в формулах (по Львовскому)
\newcommand*{\hm}[1]{#1\nobreak\discretionary{}
{\hbox{$\mathsurround=0pt #1$}}{}}
% графика
\usepackage{graphicx}
\graphicspath{{pictures/}}
\DeclareGraphicsExtensions{.pdf,.png,.jpg}
\author{Бурмашев Григорий, БПМИ-208}
\title{\sout{Матан} ТВиМС, дз -- 1}
\date{\today}
\begin{document}
\maketitle
\section*{2. [С помощью условной вероятности]} 
\subsection*{а) Еще хотя бы один туз}

Сразу посчитаем важные для решения переменные:
\begin{itemize}
\item Ровно 1 туз в руке (6 карт):
\[
C^5_{32} \cdot C^1_4 
\]
\begin{center}
(C из 32 по 5 выбрать не тузовые карты и С из 4 по 1 выбрать 1 туз)
\end{center}

\item Ровно 2 туза в руке:
\[
C^4_{32} \cdot 	C^2_4
\]

\item Ровно 3 туза в руке:
\[
C^3_{32} \cdot C^3_4
\]

\item Ровно 4 туза в руке:
\[
C^2_{32} \cdot C^4_4
\]

\item 0 тузов в руке:
\[
C^6_{32}
\]

\item Общее число вариантов выбрать руку:
\[
C^6_{36}
\]
\end{itemize}
\clearpage
Т.к  нужно решить через условную вероятность, то задать искомую вероятность можно так : 

$P$(есть \textbf{еще} хотя бы 1 туз, при условии, что есть хотя бы 1 туз)
\\

Тогда введем события:
\begin{center}
$A$ -- есть \textbf{еще} хотя бы один туз
\\
$B$ -- есть хотя бы один туз
\end{center}

Ответом будет:
\[
P(A |B) = \frac{P(A \cap B)}{P(B)}
\]

Теперь считаем:
\begin{center}
$A \cap B$ -- есть хотя бы 2 туза 
\end{center}
\[
P(A \cap B) = \frac{C^4_{32} \cdot 	C^2_4 +C^3_{32} \cdot C^3_4 +  C^2_{32} \cdot C^4_4}{C^6_{36}}  = \frac{215760 + 19840 + 496}{1947792} = \frac{236096}{1947792}
\]

\[
P(B) = 1 - P(\text{нет тузов}) =  1 - \frac{C^6_{32}}{C^6_{36}} = 1 - \frac{906192}{1947792} = \frac{100}{187}
\]

\[
P(A | B) = \frac{\frac{236096}{1947792}}{\frac{100}{187}} = \frac{17}{75}
\]
\begin{center}
\textbf{Ответ: }  $\frac{17}{75}$
\end{center}
\clearpage
\subsection*{b) Еще хотя бы один туз, при условии туз пик}
\begin{center}Делаем все аналогично пункту $a)$\end{center}

Ищем вероятности для решения:
\begin{itemize}
\item Ровно 1 туз пик:
\[
\frac{C^5_{32} \cdot C^1_4}{4}
\]
\item 2 туза и есть туз пик:
\[
\frac{2 \cdot C^4_{32} \cdot C^2_4}{4}
\]
\item 3 туза и есть туз пик:
\[
\frac{3 \cdot C^3_{32} \cdot C^3_4}{4}
\]
\item 4 туза (аналогично пункту \textbf{a})
\end{itemize}

Ищем :
\begin{center}
$P$(есть еще хотя бы один туз, при условии, что есть хотя бы 1 туз пик)
\end{center}
События:
\begin{center}
$A$ -- есть еще хотя бы один туз \\
$B$ -- есть хотя бы 1 туз пик \\
$A \cap B$ -- хотя бы 2 туза, причем хотя бы 1 туз пик
\end{center}
\[
P(A\cap B) = \frac{\frac{2 \cdot C^4_{32} \cdot C^2_4 }{4} +\frac{3 \cdot C^3_{32} \cdot C^3_4}{4} + \frac{C^2_{32} \cdot C^4_4}{1} }{C^6_{36}}
\]
\[
P(B) = \frac{\frac{2 \cdot C^4_{32} \cdot C^2_4 }{4} +\frac{3 \cdot C^3_{32} \cdot C^3_4}{4} + \frac{C^2_{32} \cdot C^4_4}{1} + \frac{C^5_{32} \cdot C^1_4}{4}}{C^6_{36}} 
\]
\[
P(A|B) = \frac{P(A \cap B)}{P(B)} = \frac{\frac{2 \cdot C^4_{32} \cdot C^2_4 }{4} +\frac{3 \cdot C^3_{32} \cdot C^3_4}{4} + \frac{4 \cdot C^2_{32} \cdot C^4_4}{4}}{\frac{2 \cdot C^4_{32} \cdot C^2_4 }{4} +\frac{3 \cdot C^3_{32} \cdot C^3_4}{4} + \frac{4 \cdot C^2_{32} \cdot C^4_4}{4} + \frac{C^5_{32} \cdot C^1_4}{4}} = 
\]
\[
=
\frac{2\cdot C^4_{32} \cdot C^2_4 + 3 \cdot C^3_{32} \cdot C^3_4 + 4 \cdot C^2_{32} \cdot C^4_4}{2\cdot C^4_{32} \cdot C^2_4 + 3 \cdot C^3_{32} \cdot C^3_4 + 4 \cdot C^2_{32} \cdot C^4_4 + C^5_{32} \cdot C^1_4} = \frac{431520 + 59520 + 1984}{493024 +805504 } = \frac{493024}{1298528} =
\]
\[
= \frac{71}{187}
\]
\begin{center}
\textbf{Ответ: } $\frac{71}{187}$
\end{center}
\clearpage
\section*{5.}
\subsection*{а) Неразличимые шары}
Общее число исходов (по методу шаров и перегородок):
\[
C^{n -1}_{n + k - 1}
\]
Нам нужно, чтобы во всех коробках было хотя бы по одному шару, разложим в каждую из $n$ коробок по 1 шару, у нас останется $k - n$ шаров, их раскладываем уже произвольно (тем же методом, что и выше):
\\\\
Тогда наше событие:
\[
P(\text{нет пустых ящиков}) = \frac{C^{n - 1}_{k - n + n - 1}}{C^{n -1 }_{n + k - 1}} = \frac{C^{n - 1}_{k - 1}}{C^{n -1 }_{n + k - 1}}
\]
\begin{center}
\textbf{Ответ: } $\frac{C^{n - 1}_{k - 1}}{C^{n -1 }_{n + k - 1}}$
\end{center}
\subsection*{b) Различимые шары}
Общее число исходов:
\[
n^k
\]
Наше событие:
\[
P(\text{нет пустых ящиков}) = 1 - P(\text{есть пустой ящик})
\]
Найдем вероятность наличия хотя бы одного пустого ящика, пусть i-й ящик -- пустой. Тогда нам нужно будет (как угодно) разложить $k$ шаров по $n - 1$ ящику. Тогда вероятность наличия хотя бы одного пустого ящика будет определяться объединением события выше по всем $i$ от $1$ до $n$. Назовем такое событие A, т.е:
\[
A_i = \{\text{i-й ящик пустой}\}
\]
\[
P(\text{есть пустой ящик}) = P \left(A_1 \cup A_2 \cup \ldots \cup A_n \right) = P \left(\sum_{j = 1}^n A_j\right) = 
\]
\[
=
\sum_{m = 1}^{n } (-1)^{m - 1}  C_n^m \cdot \frac{(n - m)^k}{n^k}
\]
\begin{center}
\textbf{Ответ: } $ \sum\limits_{m = 1}^{n } (-1)^{m - 1}  C_n^m \cdot \frac{(n - m)^k}{n^k}$
\end{center}
\clearpage
\section*{9.}
\begin{center}
[Похожа на задачу 5]
\end{center}
\subsection*{a) Неразличимые шары}
Общее число исходов:
\[
C^{n - 1}_{n + k - 1}
\]
Теперь кладем в  $j$-й ящик $k_j$ шаров. У нас остается $k - k_j$ шаров и $n - 1$ ящиков, т.е:
\[
C^{(n - 1) - 1}_{(n - 1)+ (k - k_j) - 1} = C^{n - 2}_{n + k - k_j - 2}
\]
Тогда наше событие:
\[
P = \frac{C^{n - 2}_{n + k - k_j - 2}}{C^{n - 1}_{n + k - 1}}
\]
\begin{center}
\textbf{Ответ: } $P = \frac{C^{n - 2}_{n + k - k_j - 2}}{C^{n - 1}_{n + k - 1}}$
\end{center}
\subsection*{b) Различимые шары}
Общее число исходов:
\[
n^k
\]
Кладем в $j$-й ящик $k_j$ шаров, т.к шары различимы, это можно сделать $C_k^{k_j}$ (выбор $k_j$ шаров из $k$ шаров) способами. У нас остается $n - 1$ ящиков и $k - k_j$ шаров. Остальные шары раскладываем как хотим, на это уйдет $(n - 1)^{k - k_j}$ способов.

По итогу:
\[
P = \frac{C_k^{k_j} (n - 1)^{k - k_j}}{n^k}
\]
\begin{center}
\textbf{Ответ: } $P =  \frac{C_k^{k_j} (n - 1)^{k - k_j}}{n^k}$
\end{center}
\end{document}
