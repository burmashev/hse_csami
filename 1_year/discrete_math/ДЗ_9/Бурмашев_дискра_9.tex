\documentclass[a4paper,12pt]{article}

%%% Работа с русским языком
\usepackage{cmap}					% поиск в PDF
\usepackage{mathtext} 				% русские буквы в формулах
\usepackage[T2A]{fontenc}			% кодировка
\usepackage[utf8]{inputenc}			% кодировка исходного текста
\usepackage[english,russian]{babel}	% локализация и переносы
\usepackage{xcolor}
\usepackage{hyperref}
 % Цвета для гиперссылок
\definecolor{linkcolor}{HTML}{799B03} % цвет ссылок
\definecolor{urlcolor}{HTML}{799B03} % цвет гиперссылок

\hypersetup{pdfstartview=FitH,  linkcolor=linkcolor,urlcolor=urlcolor, colorlinks=true}

%%% Дополнительная работа с математикой
\usepackage{amsfonts,amssymb,amsthm,mathtools} % AMS
\usepackage{amsmath}
\usepackage{icomma} % "Умная" запятая: $0,2$ --- число, $0, 2$ --- перечисление

%% Номера формул
%\mathtoolsset{showonlyrefs=true} % Показывать номера только у тех формул, на которые есть \eqref{} в тексте.

%% Шрифты
\usepackage{euscript}	 % Шрифт Евклид
\usepackage{mathrsfs} % Красивый матшрифт

%% Свои команды
\DeclareMathOperator{\sgn}{\mathop{sgn}}

%% Перенос знаков в формулах (по Львовскому)
\newcommand*{\hm}[1]{#1\nobreak\discretionary{}
{\hbox{$\mathsurround=0pt #1$}}{}}
% графика
\usepackage{graphicx}
\graphicspath{{pictures/}}
\DeclareGraphicsExtensions{.pdf,.png,.jpg}
\author{Бурмашев Григорий, БПМИ-208}
\title{Линал. ИДЗ - 2 Вариант 2.}
\date{\today}
\begin{document}
\begin{center}
Бурмашев Григорий.  208. Дискра. Д/З -- 9
\end{center}
\newpage
\subsection*{Задание 1}
По определению композиции нужно для всех пар (x, y) найти z,  что:
\[
\frac{z}{y} > 0
\]
\[
\frac{x}{z} > 0
\]
Очевидно, что x, y, z не могут быть равны нулю. Тогда xy > 0 и $\frac{x}{y} > 0$

Поскольку $\frac{x}{y} > 0$,  то возможны случаи:
\\\\
а)
\[
\frac{x}{1} > 0
\]
\[
\frac{1}{y} > 0
\]
б)
\[
\frac{x}{-1} > 0
\]
\[
\frac{-1}{y} > 0
\]
Значит $ (x, y) \in R \circ R $, $R \circ R = R$
\begin{center}
\textbf{Ответ:} $R \circ R = R$
\end{center}

\subsection*{Задание 2}
Найдем $R^T$:
\[
\{   (a, 1), (2, b), (4, c), (8,  d), (8, e), (8, f), (8, g), (8, h) \}
\]
Найдем $R^T \circ R:$
\[
\{  
(a, a), (b, b), (c, c), (d, d), (e, e), (f, f), (g, g), (h, h) (d, e), (d, f),
\]
\[
 (d, g), (d, g), (d, h), (e, e), (e, d), (e, f), (e, g), (e, h), \]
\[(f, d), (f, e), (f, g), (f, h), (g, d), (g, e), (g, f), (g, h), (h, d), (h, e), (h, f), (h, g)
\}
\]
\begin{center}
Итого 28
\end{center}
Найдем $R \circ R^T$:
\[
\{ (1, 1), (2, 2), (4, 4), (8, 8)\}
\]
\begin{center}
Итого 4
\end{center}
\begin{center}
\textbf{Ответ:} В $R^T \circ R$ 28 элементов. В $R \circ R^T$ 4 элемента.
\end{center}

\subsection*{Задание 3}
Предположим, что  R1 не является функцией. Тогда в множестве пар в R1  может быть несколько значений. Но и в $R1 \cup R2$ будет тоже самое, $\rightarrow$ это не будет функцией. Аналогичная ситуация будет, если R2 -- не функция. А значит, в множестве пар как в R1, так и в R2 должны быть различные левые элементы. Т.е R1 и R2 --  функции. \textbf{Ч.Т.Д}

\subsection*{Задание 4}
По условию все элементы множества X переходят в B.

Положим:

X = \{1, 2\}.  Пусть все элементы переходят в 1. Тогда  B = \{1\}. При этом пусть Y = \{1, 10, 11\}. Все условия задачи сохраняются и выполняются, но при этом B $\neq$ Y.
\begin{center}
\textbf{Ответ:} нет
\end{center}
\subsection*{Задание 5}
Рассмотрим первый случай:
\[
f(A \triangle B) \subseteq f(A) \triangle f(B) 
\]
Положим:

\[
A = \{ 1, 2\}
\]
\[
B = \{ 3, 4\}
\]
Тогда:
\[
A \triangle B = \{1, 2, 3, 4\}
\]
Пускай при этом множество Y задано следующим образом:
\[
f(1) = f(3) = 5
\]
\[
f(2) = f(4) = 6
\]
А значит:
\[
f(A \triangle B) = \{5, 6\}
\]
\[
f(A) = \{5, 6\} = f(B)
\]
Но при этом:
\[
f(A) \triangle f(B) = \emptyset 
\]
\[
\{5, 6\} \subseteq \emptyset
\]
\begin{center}
\textbf{противоречие}
\end{center}

Рассмотрим второй случай:
\[
f(A \triangle B) \supseteq f(A) \triangle f(B) 
\] Т.к в условии задачи сказано, что при одном из знаков получается верное утверждение, а в первом случае мы уже получили противоречие, то заведомо предположим, что этот случай истиный. Тогда попробуем доказать от обратного. Пусть найдется такой $y$, что:
\[
y \notin f(A \triangle B) \cap y \in f(A) \triangle f(B).
\]
В таком случае $y = f(z)$, что  $f(z) \in f(A)$. А также $f(z) \in f(B)$. Отсюда $z \in f(A) \cap f(B)$.  Ну и $f(z) \notin f(A) \triangle f(B)$. Мы получили противоречие $\rightarrow$ такого быть не может.
\subsection*{Задание 8}
Функция от функции изменяет четность числа. Можно построить следующую биекцию, которая в квадрате будет делать именно то, что нужно:
\[
\begin{pmatrix}
1 & 2 & 3 & 4 \\
3 & 1 & 4  &2 \\
\end{pmatrix}
\]
Таким образом при взятии функции 2 раза:

1 переведется в 3 и затем в 4

2 переведется в 1 и затем в 3

3 переведется в 4 и затем в 2

4 переведется в 2 и затем в 1
\\\\
Т.е если мы разобьем наши числа на 4 группы, то сможем менять их четность  вызывая эту функцию 2 раза,
что и требовалось в условии задачи.
\end{document}