\documentclass[a4paper,12pt]{article}

%%% Работа с русским языком
\usepackage{cmap}					% поиск в PDF
\usepackage{mathtext} 				% русские буквы в формулах
\usepackage[T2A]{fontenc}			% кодировка
\usepackage[utf8]{inputenc}			% кодировка исходного текста
\usepackage[english,russian]{babel}	% локализация и переносы
\usepackage{xcolor}
\usepackage{hyperref}
 % Цвета для гиперссылок
\definecolor{linkcolor}{HTML}{799B03} % цвет ссылок
\definecolor{urlcolor}{HTML}{799B03} % цвет гиперссылок

\hypersetup{pdfstartview=FitH,  linkcolor=linkcolor,urlcolor=urlcolor, colorlinks=true}

%%% Дополнительная работа с математикой
\usepackage{amsfonts,amssymb,amsthm,mathtools} % AMS
\usepackage{amsmath}
\usepackage{icomma} % "Умная" запятая: $0,2$ --- число, $0, 2$ --- перечисление

%% Номера формул
%\mathtoolsset{showonlyrefs=true} % Показывать номера только у тех формул, на которые есть \eqref{} в тексте.

%% Шрифты
\usepackage{euscript}	 % Шрифт Евклид
\usepackage{mathrsfs} % Красивый матшрифт

%% Свои команды
\DeclareMathOperator{\sgn}{\mathop{sgn}}

%% Перенос знаков в формулах (по Львовскому)
\newcommand*{\hm}[1]{#1\nobreak\discretionary{}
{\hbox{$\mathsurround=0pt #1$}}{}}
% графика
\usepackage{graphicx}
\graphicspath{{pictures/}}
\DeclareGraphicsExtensions{.pdf,.png,.jpg}
\author{Бурмашев Григорий, БПМИ-208}
\title{Линал. 

Задание 4

Вариант 2.}
\date{\today}
\begin{document}
\maketitle
\newpage
\section*{Номер 1}
\begin{enumerate}
\item Найдем матрицу перехода от  $e$ к  $e'$:
\[
\begin{pmatrix}
-3 & 2 & 1 & \vrule&-8 & -4 & 3 & \\
-1 & 2 & -3 & \vrule &8 & 2 & -4 & \\
-1 & 1 & -3 &  \vrule& -4 & 8 & 9 & \\
\end{pmatrix}
=
\begin{pmatrix}
0 & -1 & 10 &\vrule& 4 & -28 & -24 & \\
-1 & 2 & -3 &\vrule& 8 & 2 & -4 & \\
-1 & 1 & -3 &\vrule& -4 & 8 & 9 & \\
\end{pmatrix}
=
\]
\[
=
\begin{pmatrix}
0 & -1 & 10 & \vrule&4 & -28 & -24 & \\
0 & 1 & 0 & \vrule&12 & -6 & -13 & \\
-1 & 1 & -3 &\vrule& -4 & 8 & 9 & \\
\end{pmatrix}
=
\begin{pmatrix}
-1 & 1 & -3 & \vrule&-4 & 8 & 9 & \\
0 & 1 & 0 & \vrule&12 & -6 & -13 & \\
0 & -1 & 10 &\vrule& 4 & -28 & -24 & \\
\end{pmatrix}
=
\]
\[
=
\begin{pmatrix}
-1 & 1 & -3 &\vrule& -4 & 8 & 9 & \\
0 & 1 & 0 &\vrule& 12 & -6 & -13 & \\
0 & 0 & 10 &\vrule& 16 & -34 & -37 & \\
\end{pmatrix}
=
\begin{pmatrix}
1 & 0 & 3 &\vrule& 16 & -14 & -22 & \\
0 & 1 & 0 & \vrule&12 & -6 & -13 & \\
0 & 0 & 10 &\vrule& 16 & -34 & -37 & \\
\end{pmatrix}
=
\]
\[
=
\begin{pmatrix}
1 & 0 & 3 &\vrule& 16 & -14 & -22 & \\
0 & 1 & 0 & \vrule&12 & -6 & -13 & \\
0 & 0 & 1 &\vrule& \frac{16}{10} & -\frac{34}{10}& -\frac{37}{10} & \\
\end{pmatrix}
=
\begin{pmatrix}
1 & 0 & 0 &\vrule& \frac{112}{10} & -\frac{38}{10} & -\frac{109}{10}& \\
0 & 1 & 0 & \vrule&12 & -6 & -13 & \\
0 & 0 & 1 &\vrule& \frac{16}{10} & -\frac{34}{10}& -\frac{37}{10} & \\
\end{pmatrix}
\]
А значит матрица перехода:
\[
C = 
\begin{pmatrix}
 \frac{112}{10} & -\frac{38}{10} & -\frac{109}{10}& \\
12 & -6 & -13 & \\
\frac{16}{10} & -\frac{34}{10}& -\frac{37}{10} & \\
\end{pmatrix}
\]
\item Найдем координаты вектора $v$ в базисе $e'$:

Для этого найдем $C^{-1}$:
\[
\begin{pmatrix}
112 & -38 & -109 & \vrule &10 & 0 & 0 & \\
12 & -6 & -13 & \vrule &0 & 1 & 0 & \\
16 & -34 & -37 & \vrule &0 & 0 & 10 & \\
\end{pmatrix}
=
\begin{pmatrix}
0 & 200 & 150 & \vrule &10 & 0 & -70 & \\
12 & -6 & -13 & \vrule &0 & 1 & 0 & \\
16 & -34 & -37 & \vrule &0 & 0 & 10 & \\
\end{pmatrix}
=
\]
\[
=
\begin{pmatrix}
0 & 20 & 15 & \vrule &1 & 0 & -7 & \\
12 & -6 & -13 & \vrule &0 & 1 & 0 & \\
4 & -28 & -24 & \vrule &0 & -1 & 10 & \\
\end{pmatrix}
=
\begin{pmatrix}
0 & 20 & 15 & \vrule &1 & 0 & -7 & \\
0 & 78 & 59 & \vrule &0 & 4 & -30 & \\
4 & -28 & -24 & \vrule &0 & -1 & 10 & \\
\end{pmatrix}
=
\]
\[
=
\begin{pmatrix}
4 & -28 & -24 & \vrule &0 & -1 & 10 & \\
0 & 78 & 59 & \vrule &0 & 4 & -30 & \\
0 & 20 & 15 & \vrule &1 & 0 & -7 & \\
\end{pmatrix}
=
\begin{pmatrix}
4 & -28 & -24 & \vrule &0 & -1 & 10 & \\
0 & 18 & 14 & \vrule &-3 & 4 & -9 & \\
0 & 20 & 15 & \vrule &1 & 0 & -7 & \\
\end{pmatrix}
=
\]
\[
=
\begin{pmatrix}
4 & -28 & -24 & \vrule &0 & -1 & 10 & \\
0 & 18 & 14 & \vrule &-3 & 4 & -9 & \\
0 & 2 & 1 & \vrule &4 & -4 & 2 & \\
\end{pmatrix}
=
\begin{pmatrix}
4 & -28 & -24 & \vrule &0 & -1 & 10 & \\
0 & 0 & 5 & \vrule &-39 & 40 & -27 & \\
0 & 2 & 1 & \vrule &4 & -4 & 2 & \\
\end{pmatrix}
=
\]
\[
=
\begin{pmatrix}
4 & -28 & -24 & \vrule &0 & -1 & 10 & \\
0 & 2 & 1 & \vrule &4 & -4 & 2 & \\
0 & 0 & 5 & \vrule &-39 & 40 & -27 & \\
\end{pmatrix}
=
\begin{pmatrix}
4 & 0 & 0 & \vrule &-22 & 23 & -16 & \\
0 & 2 & 1 & \vrule &4 & -4 & 2 & \\
0 & 0 & 5 & \vrule &-39 & 40 & -27 & \\
\end{pmatrix}
=
\]
\[
=
\begin{pmatrix}
1 & 0 & 0 & \vrule &-\frac{11}{2} & \frac{23}{4} & -4 & \\
0 & 1 & 0 & \vrule &\frac{59}{10}& -6 & \frac{37}{10} & \\
0 & 0 & 1& \vrule &-\frac{39}{5} & 8 & -\frac{27}{5}& \\
\end{pmatrix}
\]
\[
C^{-1} = 
\begin{pmatrix}
-\frac{11}{2} & \frac{23}{4} & -4 & \\
\frac{59}{10}& -6 & \frac{37}{10} & \\
-\frac{39}{5} & 8 & -\frac{27}{5}& \\
\end{pmatrix}
\]
Теперь найдем координаты v:
\[
\begin{pmatrix}
-\frac{11}{2} & \frac{23}{4} & -4 & \\
\frac{59}{10}& -6 & \frac{37}{10} & \\
-\frac{39}{5} & 8 & -\frac{27}{5}& \\
\end{pmatrix}
\cdot
\begin{pmatrix}
-1 \\
-4 \\
-3 \\
\end{pmatrix}
=
\left(\begin{matrix}
\frac{-11}{2} \\
7 \\
-8
\end{matrix}\right)
\]
\end{enumerate}
{\LARGE \begin{center}
\textbf{Ответ: } 
\[
\left(\begin{matrix}
\frac{-11}{2} \\
7 \\
-8
\end{matrix}\right)
\]
\end{center}}
\clearpage
\section*{Номер 2}
Докажем, что:
\[
R^4 = U \oplus W
\]
По лекциям:
\[
R^4 = U \oplus W \equiv \begin{cases}
R^4 =U  + W \\
U \cap W = 0
\end{cases}
\]
\begin{enumerate}
\item
Проверим, что $R^4 = U + W$:
\[
\begin{pmatrix}
-10 & -16 & 4 & 15 & \\
-2 & -6 & -5 & -12 & \\
9 & -7 & 4 & -5 & \\
0 & -32 & -13 & -25 & \\
\end{pmatrix}
=
\begin{pmatrix}
-1 & -23 & 8 & 10 & \\
-2 & -6 & -5 & -12 & \\
9 & -7 & 4 & -5 & \\
0 & -32 & -13 & -25 & \\
\end{pmatrix}
=
\]
\[
=
\begin{pmatrix}
-1 & -23 & 8 & 10 & \\
0 & 40 & -21 & -32 & \\
9 & -7 & 4 & -5 & \\
0 & -32 & -13 & -25 & \\
\end{pmatrix}
=
\begin{pmatrix}
-1 & -23 & 8 & 10 & \\
0 & 40 & -21 & -32 & \\
0 & -214 & 76 & 85 & \\
0 & -32 & -13 & -25 & \\
\end{pmatrix}
=
\]
\[
=
\begin{pmatrix}
-1 & -23 & 8 & 10 & \\
0 & 40 & -21 & -32 & \\
0 & -14 & -29 & -75 & \\
0 & -32 & -13 & -25 & \\
\end{pmatrix}
=
\begin{pmatrix}
-1 & -23 & 8 & 10 & \\
0 & -2 & -108 & -257 & \\
0 & -14 & -29 & -75 & \\
0 & -32 & -13 & -25 & \\
\end{pmatrix}
=
\]
\[
=
\begin{pmatrix}
-1 & -23 & 8 & 10 & \\
0 & -2 & -108 & -257 & \\
0 & 0 & 727 & 1724 & \\
0 & -32 & -13 & -25 & \\
\end{pmatrix}
=
\begin{pmatrix}
-1 & -23 & 8 & 10 & \\
0 & -2 & -108 & -257 & \\
0 & 0 & 727 & 1724 & \\
0 & 0 & 1715 & 4087 & \\
\end{pmatrix}
=
\]
\[
=
\begin{pmatrix}
-1 & 0 & 1250 & 2954 & \\
0 & 1 & 54 & 128 & \\
0 & 0 & 727 & 1724 & \\
0 & 0 & 1715 & 4087 & \\
\end{pmatrix}
=
\begin{pmatrix}
-1 & 0 & 1250 & 2954 & \\
0 & 1 & 54 & 128 & \\
0 & 0 & 727 & 1724 & \\
0 & 0 & 261 & 639 & \\
\end{pmatrix}
=
\]
\[
=
\begin{pmatrix}
-1 & 0 & 1250 & 2954 & \\
0 & 1 & 54 & 128 & \\
0 & 0 & -56 & -193 & \\
0 & 0 & 261 & 639 & \\
\end{pmatrix}
=
\begin{pmatrix}
-1 & 0 & 1250 & 2954 & \\
0 & 1 & 54 & 128 & \\
0 & 0 & -56 & -193 & \\
0 & 0 & 37 & -133 & \\
\end{pmatrix}
=
\]
\[
=
\begin{pmatrix}
-1 & 0 & 1250 & 2954 & \\
0 & 1 & 54 & 128 & \\
0 & 0 & -19 & -326 & \\
0 & 0 & 37 & -133 & \\
\end{pmatrix}
=
\begin{pmatrix}
-1 & 0 & 1250 & 2954 & \\
0 & 1 & 54 & 128 & \\
0 & 0 & -19 & -326 & \\
0 & 0 & -1 & -785 & \\
\end{pmatrix}
=
\]
\[
=
\begin{pmatrix}
1 & 0 & -1250 & -2954 & \\
0 & 1 & 54 & 128 & \\
0 & 0 & 0 & 1  & \\
0 & 0 & -1 & -785 & \\
\end{pmatrix}
=
\begin{pmatrix}
1 & 0 & 0 & 0 & \\
0 & 1 & 0 & 0 & \\
0 & 0 & 1 & 0 & \\
0 & 0 & 0& 1 & \\
\end{pmatrix}
\]
Мы видим, что векторы являются линейно независимыми, а значит они образуют всё $R^4$. К тому же $\text{dim } (U + W) = 4$
\\\\
Мы знаем, что $\text{dim } U = \text{dim } W = 2$ Из формулы $\text{dim }  ( U \cap W) + \text{dim } (U + W) = \text{dim }U + \text{dim } W$: $\text{dim }  ( U \cap W) = 0$. А значит $U \cap W = 0$ 
\item Найдем проекцию вектора $x = (-20, -10, 3, 48)$ на подпространство $W$ вдоль подпространства $U$:
\[
\begin{pmatrix}
-10 & -2 & 9 & 6 & \vrule &-20 & \\
-16 & -6& -7& -14 & \vrule &-10& \\
4 & -5 & 4 & 2 & \vrule &3 & \\
15 & -12 & -5 & 11 & \vrule &48 & \\
\end{pmatrix}
=
\begin{pmatrix}
-10 & -2 & 9 & 6 & \vrule &-20 & \\
0 & -26 & 9 & -6 & \vrule &2 & \\
4 & -5 & 4 & 2 & \vrule &3 & \\
-1 & 8 & -21 & 3 & \vrule &36 & \\
\end{pmatrix}
=
\]
\[
=
\begin{pmatrix}
0 & -82 & 219 & -24 & \vrule &-380 & \\
0 & -26 & 9 & -6 & \vrule &2 & \\
4 & -5 & 4 & 2 & \vrule &3 & \\
-1 & 8 & -21 & 3 & \vrule &36 & \\
\end{pmatrix}
=
\begin{pmatrix}
0 & -82 & 219 & -24 & \vrule &-380 & \\
0 & -26 & 9 & -6 & \vrule &2 & \\
0 & 27 & -80 & 14 & \vrule &147 & \\
-1 & 8 & -21 & 3 & \vrule &36 & \\
\end{pmatrix}
=
\]
\[
=
\begin{pmatrix}
0 & -82 & 219 & -24 & \vrule &-380 & \\
0 & -26 & 9 & -6 & \vrule &2 & \\
0 & 1 & -71 & 8 & \vrule &149 & \\
-1 & 8 & -21 & 3 & \vrule &36 & \\
\end{pmatrix}
=
\begin{pmatrix}
0 & 0 & -5603 & 632 & \vrule &11838 & \\
0 & -26 & 9 & -6 & \vrule &2 & \\
0 & 1 & -71 & 8 & \vrule &149 & \\
-1 & 8 & -21 & 3 & \vrule &36 & \\
\end{pmatrix}
=
\]
\[
=
\begin{pmatrix}
0 & 0 & -5603 & 632 & \vrule &11838 & \\
0 & 0 & -1837 & 202 & \vrule &3876 & \\
0 & 1 & -71 & 8 & \vrule &149 & \\
-1 & 8 & -21 & 3 & \vrule &36 & \\
\end{pmatrix}
=
\begin{pmatrix}
0 & 0 & -5603 & 632 & \vrule &11838 & \\
0 & 0 & -1837 & 202 & \vrule &3876 & \\
0 & 1 & -71 & 8 & \vrule &149 & \\
1 & 0 & -547 & 61 & \vrule &1156 & \\
\end{pmatrix}
=
\]
\[
=
\begin{pmatrix}
1 & 0 & -547 & 61 & \vrule &1156 & \\
0 & 1 & -71 & 8 & \vrule &149 & \\
0 & 0 & -1837 & 202 & \vrule &3876 & \\
0 & 0 & -92 & 26 & \vrule &210 & \\
\end{pmatrix}
=
\begin{pmatrix}
1 & 0 & -547 & 61 & \vrule &1156 & \\
0 & 1 & -71 & 8 & \vrule &149 & \\
0 & 0 & 3 & -318 & \vrule &-324 & \\
0 & 0 & -92 & 26 & \vrule &210 & \\
\end{pmatrix}
=
\]
\[
=
\begin{pmatrix}
1 & 0 & -547 & 61 & \vrule &1156 & \\
0 & 1 & -71 & 8 & \vrule &149 & \\
0 & 0 & 3 & -318 & \vrule &-324 & \\
0 & 0 & -2 & -9514 & \vrule &-9510 & \\
\end{pmatrix}
=
\begin{pmatrix}
1 & 0 & -547 & 61 & \vrule &1156 & \\
0 & 1 & -71 & 8 & \vrule &149 & \\
0 & 0 & 1 & -9832 & \vrule &-9834 & \\
0 & 0 & -2 & -9514 & \vrule &-9510 & \\
\end{pmatrix}
=
\]
\[
=
\begin{pmatrix}
1 & 0 & -547 & 61 & \vrule &1156 & \\
0 & 1 & -71 & 8 & \vrule &149 & \\
0 & 0 & 1 & -9832 & \vrule &-9834 & \\
0 & 0 & 0 & -29178 & \vrule &-29178 & \\
\end{pmatrix}
=
\begin{pmatrix}
1 & 0 & -547 & 0 & \vrule &1095 & \\
0 & 1 & -71 & 8 & \vrule &149 & \\
0 & 0 & 1 & -9832 & \vrule &-9834 & \\
0 & 0 & 0 & 1 & \vrule &1 & \\
\end{pmatrix}
=
\]
\[
=
\begin{pmatrix}
1 & 0 & -547 & 0 & \vrule &1095 & \\
0 & 1 & -71 & 0 & \vrule &141 & \\
0 & 0 & 1 & -9832 & \vrule &-9834 & \\
0 & 0 & 0 & 1 & \vrule &1 & \\
\end{pmatrix}
=
\begin{pmatrix}
1 & 0 & -547 & 0 & \vrule &1095 & \\
0 & 1 & -71 & 0 & \vrule &141 & \\
0 & 0 & 1 & 0 & \vrule &-2 & \\
0 & 0 & 0 & 1 & \vrule &1 & \\
\end{pmatrix}
=
\]
\[
=
\begin{pmatrix}
1 & 0 & -547 & 0 & \vrule &1095 & \\
0 & 1 & 0 & 0 & \vrule &-1 & \\
0 & 0 & 1 & 0 & \vrule &-2 & \\
0 & 0 & 0 & 1 & \vrule &1 & \\
\end{pmatrix}
=
\begin{pmatrix}
1 & 0 & 0 & 0 & \vrule &1 & \\
0 & 1 & 0 & 0 & \vrule &-1 & \\
0 & 0 & 1 & 0 & \vrule &-2 & \\
0 & 0 & 0 & 1 & \vrule &1 & \\
\end{pmatrix}
=
\]
\end{enumerate}
Тогда искомая проекция:
\[
v_3 \cdot (-2) + v_4 \cdot 1 =
\begin{pmatrix}
9 \\
-7\\
4 \\
-5\\
\end{pmatrix}
\cdot (-2) + 
\begin{pmatrix}
6\\-14\\2\\1
\end{pmatrix}
=
\left(\begin{matrix}
-12 \\
0 \\
-6 \\
11
\end{matrix}\right)
\]
{\LARGE \begin{center}
\textbf{Ответ: } 
\[
\left(\begin{matrix}
-12 \\
0 \\
-6 \\
11
\end{matrix}\right)
\]
\end{center}}
\clearpage
\section*{Номер 3}
Дополним наше подпространство U до $\mathbb{R}^5$:
\[
\begin{pmatrix}
14 & 6 & -13 & 8 & -15 & \\
7 & -11 & -12 & -11 & -11 & \\
10 & -13 & -7 & -10 & -6 & \\
-11 & -36 & 7 & -37 & 13 & \\
\end{pmatrix}
=
\begin{pmatrix}
0 & 28 & 11 & 30 & 7 & \\
7 & -11 & -12 & -11 & -11 & \\
10 & -13 & -7 & -10 & -6 & \\
-11 & -36 & 7 & -37 & 13 & \\
\end{pmatrix}
=
\]
\[
=
\begin{pmatrix}
0 & 28 & 11 & 30 & 7 & \\
7 & -11 & -12 & -11 & -11 & \\
-1 & -49 & 0 & -47 & 7 & \\
-11 & -36 & 7 & -37 & 13 & \\
\end{pmatrix}
=
\begin{pmatrix}
0 & 28 & 11 & 30 & 7 & \\
0 & -354 & -12 & -340 & 38 & \\
-1 & -49 & 0 & -47 & 7 & \\
-11 & -36 & 7 & -37 & 13 & \\
\end{pmatrix}
=
\]
\[
=
\begin{pmatrix}
0 & 28 & 11 & 30 & 7 & \\
0 & -354 & -12 & -340 & 38 & \\
-1 & -49 & 0 & -47 & 7 & \\
0 & 503 & 7 & 480 & -64 & \\
\end{pmatrix}
=
\begin{pmatrix}
1 & 49 & 0 & 47 & -7 & \\
0 & -354 & -12 & -340 & 38 & \\
0 & 28 & 11 & 30 & 7 & \\
0 & 149 & -5 & 140 & -26 & \\
\end{pmatrix}
=
\]
\[
=
\begin{pmatrix}
1 & 49 & 0 & 47 & -7 & \\
0 & -56 & -22 & -60 & -14 & \\
0 & 28 & 11 & 30 & 7 & \\
0 & 149 & -5 & 140 & -26 & \\
\end{pmatrix}
=
\begin{pmatrix}
1 & 49 & 0 & 47 & -7 & \\
0 & 0 & 0 & 0 & 0 & \\
0 & 28 & 11 & 30 & 7 & \\
0 & 149 & -5 & 140 & -26 & \\
\end{pmatrix}
=
\]
\[
=
\begin{pmatrix}
1 & 49 & 0 & 47 & -7 & \\
0 & 28 & 11 & 30 & 7 & \\
0 & 9 & -60 & -10 & -61 & \\
0 & 0 & 0 & 0 & 0 & \\
\end{pmatrix}
=
\begin{pmatrix}
1 & 49 & 0 & 47 & -7 & \\
0 & 1 & 191 & 60 & 190 & \\
0 & 9 & -60 & -10 & -61 & \\
0 & 0 & 0 & 0 & 0 & \\
\end{pmatrix}
=
\]
\[
=
\begin{pmatrix}
1 & 49 & 0 & 47 & -7 & \\
0 & 1 & 191 & 60 & 190 & \\
0 & 0 & -1779 & -550 & -1771 & \\
0 & 0 & 0 & 0 & 0 & \\
\end{pmatrix}
\]
Значит нам необходимы векторы из стандартного базиса $e_4$ и $e_5$.
\\
Тогда:
\begin{equation} \label{aye}
\mathbb{R}^5 = \;<v_1, v_2, v_3, v_4, e_4, e_5>
\end{equation}
Прибавим теперь к $e_4$ и $e_5$ какие-нибудь векторы из U. Пусть например:
\[
e_4 + v_1 = w_1
\]
\[
e_5 + v_2 = w_2
\]
Тогда пускай $W = \; <w_1, w_2>$
\\\\
Из \ref{aye} мы знаем, что $U + W = \mathbb{R}^5$ 

Чтобы доказать, что:
\[
R^5 = U \oplus W \equiv  \begin{cases}
R^5 =U  + W \\
U \cap W = 0
\end{cases}
\]

Нам нужно проверить пересечение, для этого воспользуемся формулой:
\[
\text{dim }  (U \cap W) + \text{dim } (U + W) = \text{dim }U + \text{dim } W
\]
\\
Мы знаем, что:
\[
\text{dim }U = 3
\]
\[
\text{dim } W = 2
\]
\[
\text{dim } (U + W) = 5
\]
А значит:
\[
\text{dim }  (U \cap W) = 0 \rightarrow U \cap W = 0
\]
\\\\
Тогда базис полученного подпространства $W$:
\[
\begin{pmatrix}
14 \\
6 \\
-13 \\
9\\
-15\\
\end{pmatrix}
,
\begin{pmatrix}
7\\
-11\\
-12\\
-11\\
-10\\
\end{pmatrix}
\]
{\LARGE \begin{center}
\textbf{Ответ: }
\[
W : \;
\begin{pmatrix}
14 \\
6 \\
-13 \\
9\\
-15\\
\end{pmatrix}
,
\begin{pmatrix}
7\\
-11\\
-12\\
-11\\
-10\\
\end{pmatrix}
\] 
\end{center}}
\clearpage
\section*{Номер 4}
$V = \mathbb{R}[x]_{x\leq 2}$. $\varphi : V \rightarrow \mathbb{R}^2$ в базисе $(-2 -2x +x^2, 1 + x - x^2, -2 -4x + 3x^2)$ пространства $V$ в базисе $((1,1), (3,2))$ пространства $\mathbb{R}^2$ имеет матрицу:
\[
A' = 
\begin{pmatrix}
-4 & -1 & -5 \\
-5 & -3 & -7\\
\end{pmatrix}
\]
Найдите $\varphi (4+8x - 5x^2)$  
\\\\
Пусть :
\begin{center}
$e$ -- базис $(-2 -2x +x^2, 1 + x - x^2, -2 -4x + 3x^2)$ в $V$
\end{center} 
\begin{center}$f$ -- базис $((1,1), (3,2))$ в $R^2$.  \end{center}
Назовем тогда стандартные базисы  $e'$ и $f'$, т.е:
\[
e':(1, x, x^2)
\]
\[
f':
\left((1,0), (0, 1)\right)
\]
Тогда по определению матрицы перехода:
\[
e= e' \cdot \begin{pmatrix}
-2 & 1  & -2 \\
-2 & 1 & -4\\
1 & -1 & 3 \\
\end{pmatrix}
\]
Пусть эта матрица -- C
\[
f = f' \cdot \begin{pmatrix}
1 & 3 \\1 & 2\\
\end{pmatrix}
\]
Пусть эта матрица -- D
\\\\
Тогда введём матрицу линейного отображения A:
\[
\varphi(4+8x -5x^2) = A \cdot \begin{pmatrix}
4 \\
8 \\
-5\\
\end{pmatrix}
\]
Мы знаем, что:
\[
A' = D^{-1} AC
\]
Отсюда:
\[
A = DA'C^{-1}
\]
Найдем $C^{-1}$:
\[
\begin{pmatrix}
-2 & 1 & -2 & \vrule &1 & 0 & 0 & \\
-2 & 1 & -4 & \vrule &0 & 1 & 0 & \\
1 & -1 & 3 & \vrule &0 & 0 & 1 & \\
\end{pmatrix}
=
\begin{pmatrix}
0 & -1 & 4 & \vrule &1 & 0 & 2 & \\
-2 & 1 & -4 & \vrule &0 & 1 & 0 & \\
1 & -1 & 3 & \vrule &0 & 0 & 1 & \\
\end{pmatrix}
=
\]
\[
=
\begin{pmatrix}
0 & -1 & 4 & \vrule &1 & 0 & 2 & \\
0 & -1 & 2 & \vrule &0 & 1 & 2 & \\
1 & -1 & 3 & \vrule &0 & 0 & 1 & \\
\end{pmatrix}
=
\begin{pmatrix}
1 & -1 & 3 & \vrule &0 & 0 & 1 & \\
0 & 0 & -2 & \vrule &-1 & 1 & 0 & \\
0 & -1 & 4 & \vrule &1 & 0 & 2 & \\
\end{pmatrix}
=
\]
\[
=
\begin{pmatrix}
1 & 0 & -1 & \vrule &-1 & 0 & -1 & \\
0 & 0 & -2 & \vrule &-1 & 1 & 0 & \\
0 & -1 & 4 & \vrule &1 & 0 & 2 & \\
\end{pmatrix}
=
\begin{pmatrix}
1 & 0 & -1 & \vrule &-1 & 0 & -1 & \\
0 & 1 & 0 & \vrule &1 & -2 & -2 & \\
0 & 0 & 2 & \vrule &1 & -1 & 0 & \\
\end{pmatrix}
=
\]
\[
\begin{pmatrix}
1 & 0 & 0& \vrule &-\frac{1}{2} & -\frac{1}{2} & -1 & \\
0 & 1 & 0 & \vrule &1 & -2 & -2 & \\
0 & 0 & 1 & \vrule &\frac{1}{2} & -\frac{1}{2}& 0 & \\
\end{pmatrix}
\]
\[
C^{-1} = 
\begin{pmatrix}
-\frac{1}{2} & -\frac{1}{2} & -1 & \\
1 & -2 & -2 & \\
\frac{1}{2} & -\frac{1}{2}& 0 & \\
\end{pmatrix}
\]
По итогу:
\[
\varphi(4+8x-5x^2) = \begin{pmatrix}
1 & 3 \\
1 & 2\\
\end{pmatrix}
\begin{pmatrix}
-4 & -1 & -5\\
-5 & -3 & -7\\
\end{pmatrix}
\begin{pmatrix}
-\frac{1}{2} & -\frac{1}{2} & -1 & \\
1 & -2 & -2 & \\
\frac{1}{2} & -\frac{1}{2}& 0 & \\
\end{pmatrix}
\begin{pmatrix}
4 \\
8 \\
-5\\
\end{pmatrix}
=
\]
\[
=
\left(\begin{matrix}
-19 & -10 & -26 \\
-14 & -7 & -19
\end{matrix}\right)
\begin{pmatrix}
-\frac{1}{2} & -\frac{1}{2} & -1 & \\
1 & -2 & -2 & \\
\frac{1}{2} & -\frac{1}{2}& 0 & \\
\end{pmatrix}
\begin{pmatrix}
4 \\
8 \\
-5\\
\end{pmatrix} = 
\]
\[
= 
\left(\begin{matrix}
\frac{-27}{2} & \frac{85}{2} & 39 \\
\frac{-19}{2} & \frac{61}{2} & 28
\end{matrix}\right)
\begin{pmatrix}
4 \\
8 \\
-5\\
\end{pmatrix}
= \left(\begin{matrix}
91 \\
66
\end{matrix}\right)
\]
{\LARGE \begin{center}
\textbf{Ответ: } 
\[
\varphi(4+8x-5x^2)  = \left(\begin{matrix}
91 \\
66
\end{matrix}\right)
\]
\end{center}}
\end{document}