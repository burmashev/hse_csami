\documentclass[a4paper,12pt]{article}

%%% Работа с русским языком
\usepackage{cmap}					% поиск в PDF
\usepackage{mathtext} 				% русские буквы в формулах
\usepackage[T2A]{fontenc}			% кодировка
\usepackage[utf8]{inputenc}			% кодировка исходного текста
\usepackage[english,russian]{babel}	% локализация и переносы
\usepackage{xcolor}
\usepackage{hyperref}
 % Цвета для гиперссылок
\definecolor{linkcolor}{HTML}{799B03} % цвет ссылок
\definecolor{urlcolor}{HTML}{799B03} % цвет гиперссылок

\hypersetup{pdfstartview=FitH,  linkcolor=linkcolor,urlcolor=urlcolor, colorlinks=true}

%%% Дополнительная работа с математикой
\usepackage{amsfonts,amssymb,amsthm,mathtools} % AMS
\usepackage{amsmath}
\usepackage{icomma} % "Умная" запятая: $0,2$ --- число, $0, 2$ --- перечисление

%% Номера формул
%\mathtoolsset{showonlyrefs=true} % Показывать номера только у тех формул, на которые есть \eqref{} в тексте.

%% Шрифты
\usepackage{euscript}	 % Шрифт Евклид
\usepackage{mathrsfs} % Красивый матшрифт

%% Свои команды
\DeclareMathOperator{\sgn}{\mathop{sgn}}

%% Перенос знаков в формулах (по Львовскому)
\newcommand*{\hm}[1]{#1\nobreak\discretionary{}
{\hbox{$\mathsurround=0pt #1$}}{}}
% графика
\usepackage{graphicx}
\graphicspath{{pictures/}}
\DeclareGraphicsExtensions{.pdf,.png,.jpg}
\author{Бурмашев Григорий}
\title{Дискра - 1}
\date{\today}
\begin{document}
\begin{center}
Бурмашев Григорий. 208. Дискра - 4
\end{center}
\section*{1.}
Т.к сумма степеней равна 20, то в графе у нас 10 ребер. Если мы возьмем 4 вершины, то максимально будет $\frac{4\cdot3}{2} = 6 $ ребер. Этого недостаточно, значит нужно взять 5 вершин, при этом мы получим$\frac{5\cdot 4}{2} = 10$ ребер
\begin{center}
\textbf{Ответ:} 5 вершин
\end{center}
\section*{2.}
У нас есть вершина степени 1, у нее всего 1 ребро. Тогда у нас остается $8-1 = 7$ вершин и $23-1 = 22$ ребра. Но при 7 вершинах возможно максимум $\frac{7\cdot6}{2} = 21$ ребро. Мы получили противоречие $\rightarrow$ это невозможно
\begin{center}
\textbf{Ответ:} нет
\end{center}
\section*{3.}
Независимое множество -- такое множество, в котором вершины не являются попарно смежными. В нашем случае это означает множество, в котором цифры у элементов не совпадают. Например: множество вида 
\[A = \{00, 11, 22, 33, 44, 55, 66, 77, 88, 99\}
\]
Его размер - 10. Никакой другой элемент в A мы добавить не можем, иначе мы используем цифру, которая уже есть в каком-то из элементов A и получим уже не независимое множество. Таким образом, максимальный размер независимого множества -- 10
\begin{center}
\textbf{Ответ:}  10
\end{center}
\section*{4.}
Всего двочных слов длины n = 4 у нас: $2^4 = 16$. Т.е в $Q_4$ всего 16 вершин
\\\\
При первом переходе (т.е при пути длины 1) мы можем инвертировать любую из 4х цифр двоичного слова. А при втором переходе (т.е уже при пути длины 2) мы можем инвертировать только 1 из трех оставшихся цифр (т.к по определению булевого куба можно инвертировать бит только в одной позиции). Всего у нас $\frac{4\cdot3}{2} = 6 $ комбинаций переходов.
\\\\
Итого для всего куба у нас должно быть $2^4 \cdot 6 = 96$ путей.  Но т.к $Q_n$ есть граф неориентированный, то мы должны выкинуть лишние случаи, (т.к путь из А в Б == путь из Б в А). Т.е поделить на два: $\frac{96}{2} = 48 $
\begin{center}
\textbf{Ответ:} 48 
\end{center}
\section*{6.}
Число делится на три, если сумма его цифр делится на 3 (т.е перестановка цифр местами в числе на делимость не влияет) 
\\\\
В вершину 9 можно попасть только из вершин 6 и 3 (т.к 69 и 39 делятся на три, и аналогично при перестановке цифр местами) Все остальные варианты (19, 29, 49, 59, 79, 89) на три не делятся. 
\\\\
В вершину 6 можно попасть из 9 и 3. (16, 26, 46, 56, 76, 86 на три не делятся). 
\\\\
В вершину 3 можно попасть из 9 и 6 (13, 23, 43, 53, 73, 83 на три не делятся).
\\\\
Таким образом, вершина 1 \textbf{не} соединена ни с 9, ни с 6, ни с 3. А значит из нее попасть в 9 невозможно.
\begin{center}
\textbf{Ответ:} нет
\end{center}
\section*{7.}
От противного:

Пусть есть вершина V, из которой можно добраться не во все оставшиеся вершины. Она соединена минимум с 7-ю вершинами. Пусть ровно с 7-ю. Тогда у нас во 2й компоненте связности остается $15-1-7 = 7$ вершин. Но тогда любая из этих 7-ми вершин соединена максимально с 6-ю вершинами, что противоречит условию задачи. Значит, 2й компонентны связности не существует и каждая из 7-ми вершин соединена как минимум с одной вершиной, которая соединена с V $\rightarrow$ из вершины V можно добраться в любую другую вершину.
\\\\
Пусть V соединена более чем с  7-ю вершинами. Тогда во 2й компоненте связности у нас остается еще меньше, чем 7 вершин. Аналогично вышенаписанному, любую вершину внутри компоненты связности из менее, чем 7-ми вершин, невозможно соединить с 7-ю вершинами, т.е условие задачи не выполняется и возникает противоречие. Тогда все же из любой вершины можно попасть в каждую
\begin{center}
\textbf{Ч.Т.Д}
\end{center}
\section*{8.}
\begin{itemize}
\item Рассмотрим элемент, в котором четное число нулей. Тогда количество единиц в нем тоже будет четно (т.к общая длина слова у нас 1000. Четное - четное = четное) Тогда он связан с вершиной, в которой ровно 400 различий. Тогда в этой вершине также четное количество как нулей, так и единиц. (т.к при изменении четного числа элементов общая четность не меняется)
\item Теперь рассмотрим элемент, в котором нечетное число нулей. Тогда количество единиц в нем также будет нечетно (Четное - нечетное = нечетное). Аналогично, он связан с вершиной, в которой ровно 400 различий. Тогда в этой вершине также нечетное количество как нулей, так и единиц.
\item Таким образом, если два элемента в этом графе связаны, то у них сохраняется одинаковая четность нулей и единиц. Если $Q_{1000,400}$ связен, тогда из каждой вершины можно попасть в любую другую. Но можно привести пример, когда это не выполняется:
 
Возьмем два элемента $Q_{1000,400}$:
\[
000\ldots0_{999}1
\]
\[
000\ldots0_{998}11
\]
В первом нечетное число нулей (999) и нечетное число единиц (1), а во втором четное число нулей (998) и четное число единиц (2). 
\\\\
Мы уже выяснили, что при переходе из одного элемента в другой четность сохраняется.  Но у этих двух элементов четность разная $\rightarrow$ они не могут быть связаны, а тогда и граф $Q_{1000,400}$ - несвязный
\end{itemize}
\begin{center}
\textbf{Ответ:} нет
\end{center}
\end{document}  