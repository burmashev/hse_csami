\documentclass[a4paper,12pt]{article}

%%% Работа с русским языком
\usepackage{cmap}					% поиск в PDF
\usepackage{mathtext} 				% русские буквы в формулах
\usepackage[T2A]{fontenc}			% кодировка
\usepackage[utf8]{inputenc}			% кодировка исходного текста
\usepackage[english,russian]{babel}	% локализация и переносы
\usepackage{xcolor}
\usepackage{hyperref}
 % Цвета для гиперссылок
\definecolor{linkcolor}{HTML}{799B03} % цвет ссылок
\definecolor{urlcolor}{HTML}{799B03} % цвет гиперссылок

\hypersetup{pdfstartview=FitH,  linkcolor=linkcolor,urlcolor=urlcolor, colorlinks=true}

%%% Дополнительная работа с математикой
\usepackage{amsfonts,amssymb,amsthm,mathtools} % AMS
\usepackage{amsmath}
\usepackage{icomma} % "Умная" запятая: $0,2$ --- число, $0, 2$ --- перечисление

%% Номера формул
%\mathtoolsset{showonlyrefs=true} % Показывать номера только у тех формул, на которые есть \eqref{} в тексте.

%% Шрифты
\usepackage{euscript}	 % Шрифт Евклид
\usepackage{mathrsfs} % Красивый матшрифт

%% Свои команды
\DeclareMathOperator{\sgn}{\mathop{sgn}}

%% Перенос знаков в формулах (по Львовскому)
\newcommand*{\hm}[1]{#1\nobreak\discretionary{}
{\hbox{$\mathsurround=0pt #1$}}{}}
% графика
\usepackage{graphicx}
\graphicspath{{pictures/}}
\DeclareGraphicsExtensions{.pdf,.png,.jpg}
\author{Бурмашев Григорий, БПМИ-208}
\title{}
\date{\today}
\begin{document}
\begin{center}
Бурмашев Григорий. 208. Дискра -- 18
\end{center}
\section*{Номер 1}
Воспользуемся методом деления пополам. Возьмем элемент, который находится в середине последовательности, а также 2 элемента: один слева и другой справа. Рассмотрим все возможные случаи расположения знаков (пусть i -- середина):
\begin{enumerate}
\item
\[
a\left[ i - 1 \right] > a\left[i \right] > a\left[i + 1\right]
\]
В этом случае мы условно попали правее искомого элемента. А значит нам нужно искать элемент в другой части (в левой). Размер множества уменьшится в 2 раза.
\item
\[
a\left[i - 1\right] < a\left[i\right] < a\left[i+ 1\right]
\]
В этом случае мы условно попали левее искомого элемента. А значит нам нужно искать элемент в другой части (в правой). Размер множества уменьшится в 2 раза.
\item
\[
a\left[i - 1\right] < a\left[i\right] > \left[i + 1\right]
\]
В этом случае мы аккурат попали в искомый нами элемент. И можем заканчивать поиск.
\end{enumerate}
Этот алгоритм повторяем до тех пор, пока не попадаем в искомый элемент. Размер множества каждый раз будет уменьшаться в 2 раза. А значит нам потребуется $3 \cdot \log N$ вопросов (очевидно, что N меньше n), т.е не более $O\left(\log n\right)$ ходов.
\begin{center}
\textbf{Ч.Т.Д} 
\end{center}
\section*{Номер 2}
Если монет четное количество, то случай очевидный. Берем и разбиваем все наши монеты попарно. Потом смотрим каждую из пар и взвешиваем. В какой-то одной из пар одна из монет окажется легче другой, она и будет фальшивой. Всего потребуется $[n / 2]$ взвешиваний, т.к из $n$ монет у нас получится $[n / 2]$ пар.

Рассмотрим случай, когда монет нечетное количество. Тогда также разбиваем все монеты попарно, но у нас останется одна лишняя. Взвешиваем все эти пары монет. Всего у нас будет $[n / 2]$ взвешиваний, но уже возможно 2 разных случая:
\begin{enumerate}
\item
В одной из пар одна из монет оказалась легче другой, это значит, что она и есть фальшивая. Поиск закончен.
\item Во всех парах все монеты оказались одного веса, ну тогда у нас остается всего 1 монета, а значит она и будет фальшивой (ибо фальшивая точно есть)
\end{enumerate}
По итогу у нас получится те же $[n / 2]$ взвешиваний, чего от нас и просят в задаче.
\begin{center}
\textbf{Ч.Т.Д} 
\end{center}
\section*{Номер 3}
Докажем от обратного, пусть мы можем решить за k взвешиваний. Причем $k < [n / 2]$. Ну а тогда $k \leq [n / 2] - 1$ (т.к  k  и n -- натуральные). Рассмотрим, аналогично предыдущей задаче, оба случая:
\begin{enumerate}
\item Если монет четное количество, то максимально мы взвесим $2k \leq (n - 2)$ монет, но у нас тогда останутся невзвешенные монеты  (как минимум две, а значит как пункт \textbf{2.} из второй задачи решить не получится) и мы можем не найти фальшивую. 
\item Если монет нечетное количество, то максимально мы взесим $2k \leq 2 \cdot [n / 2] - 2$ монет, но у нас все равно останутся невзвешенные монеты и мы можем не найти фальшивую.
\end{enumerate}
Из этих двух пунктов следует, что нам придется произвести $[n / 2]$ взвешиваний.
\begin{center}
\textbf{Ч.Т.Д} 
\end{center}
\section*{Номер 4}
Чтобы был логарифм по основанию три, надо монеты разбивать на три кучи. Так и будем делать, нам нужно, чтобы две кучи были одного размера. Тогда взвесим эти 2 кучи. У нас может быть 2 состояния:
\begin{enumerate}
\item
Кучи монет равны. Это означает, что фальшивой монеты там точно нет, ведь она весит меньше, чем настоящая. А значит фальшивая монета находится в третьей куче (которую мы не взвешивали). Откидываем эти две кучу и рассматриваем далее третьую, т.е мы уменьшили размер множества в три раза. 
\item Одна из куч оказалась меньше. Это означает, что фальшивая монета точно находится в этой куче, потому что фальшивая монета весит меньше настоящей, а кучи у нас одного размера. Откидываем две другие кучу и берем эту. Мы опять же уменьшили размер множества в три раза.
\end{enumerate}
Этот алгоритм мы продолжаем до самого конца, пока не найдем фальшивую монету. Потребуется нам на это как раз $\left[\log_3 n\right]$ взвешиваний.
\begin{center}
\textbf{Ч.Т.Д}
\end{center}
\section*{Номер 5}
Если мы будем решать алгоритмом из предыдущей задачи и делить кучу на три и уменьшать за каждый ход размер множества соотвественно в три раза, то, если мы попытаемся сделать меньше чем $\log_3 n$ ходов, то мы окажемся в рамках какого-то множества монет, причем размер множества не будет равен единице (ибо опять же за каждый ход размер уменьшается в три раза, а глубина дерева у нас будет $\log_3 n$). А значит у нас будет какое-то количество монет, среди которых есть одна фальшивая. Но поскольку ходы у нас закончились, то ничего с этим множеством мы сделать не сможем и найти монету у нас не получится. При попытке делить кучу пополам мы соотвественно за ход будем уменьшать размер множества в 2 раза, а значит такой алгоритм будет работать еще медленее. А если делить монеты на 4 и более куч, то нам потребуется более чем одно взвешивание для уменьшения размера в 3 раза, т.к мы условно взвесим две кучи, и если они будут равны, то нам придется взвешивать и две другие тоже, что явно медленее, чем при делении на 3 кучи. И тогда в худшем случае это будет $2 \cdot \log_4 n$, что медленнее $\log_3 n$. А значит нам в любом случае потребуется $[ \log_3 n]$ взвешиваний.
\begin{center}
\textbf{Ч.Т.Д}
\end{center}
\end{document}
