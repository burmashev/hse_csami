\documentclass[a4paper,12pt]{article}

%%% Работа с русским языком
\usepackage{cmap}					% поиск в PDF
\usepackage{mathtext} 				% русские буквы в формулах
\usepackage[T2A]{fontenc}			% кодировка
\usepackage[utf8]{inputenc}			% кодировка исходного текста
\usepackage[english,russian]{babel}	% локализация и переносы
\usepackage{xcolor}
\usepackage{hyperref}
 % Цвета для гиперссылок
\definecolor{linkcolor}{HTML}{799B03} % цвет ссылок
\definecolor{urlcolor}{HTML}{799B03} % цвет гиперссылок

\hypersetup{pdfstartview=FitH,  linkcolor=linkcolor,urlcolor=urlcolor, colorlinks=true}

%%% Дополнительная работа с математикой
\usepackage{amsfonts,amssymb,amsthm,mathtools} % AMS
\usepackage{amsmath}
\usepackage{icomma} % "Умная" запятая: $0,2$ --- число, $0, 2$ --- перечисление

%% Номера формул
%\mathtoolsset{showonlyrefs=true} % Показывать номера только у тех формул, на которые есть \eqref{} в тексте.

%% Шрифты
\usepackage{euscript}	 % Шрифт Евклид
\usepackage{mathrsfs} % Красивый матшрифт

%% Свои команды
\DeclareMathOperator{\sgn}{\mathop{sgn}}

%% Перенос знаков в формулах (по Львовскому)
\newcommand*{\hm}[1]{#1\nobreak\discretionary{}
{\hbox{$\mathsurround=0pt #1$}}{}}
% графика
\usepackage{graphicx}
\graphicspath{{pictures/}}
\DeclareGraphicsExtensions{.pdf,.png,.jpg}
\author{Бурмашев Григорий, БПМИ-208}
\title{}
\date{\today}
\begin{document}
\section*{Задача 5}
Т.к в задаче нет особо подробностей, то предполагаю, что ответ -- для любых натуральных $n$. Решаем методом мат.индукции и постараемся найти ответ. 

\begin{itemize}
\item База:

n = 1 : очевидно, что можно

n = 2: у нас квадрат из 16 клеток, поделим его на 4 квадрата $2 \times 2$, в одном из этих квадратов будет вырезанная клетка, его закрашиваем как в примере для $n = 1$, а три оставшиеся квадрата закрашиваем, как на лекции

\item Шаг:

Возьмем квадрат $2^{n + 1} \times 2^{n + 1}$. Поделим его на 4 квадрата $2^n \times 2^n$. В одном из них будет вырезанная клетка. Угол, который состоит из трех больших квадратов, можно закрасить аналогично лекции. У нас остается квадрат $2^n \times 2^n$, сведем за $n - 1$ шаг к квадрату $4 \times 4$, что мы уже закрашивали выше.
\end{itemize}
\begin{center}
\textbf{Ч.Т.Д} 
\end{center}
\begin{center}
\textbf{Ответ: } для любых натуральных $n$
\end{center}
\end{document}
