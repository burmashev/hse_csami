 \documentclass[a4paper,12pt]{article}

%%% Работа с русским языком
\usepackage{cmap}					% поиск в PDF
\usepackage{mathtext} 				% русские буквы в формулах
\usepackage[T2A]{fontenc}			% кодировка
\usepackage[utf8]{inputenc}			% кодировка исходного текста
\usepackage[english,russian]{babel}	% локализация и переносы
\usepackage{xcolor}
\usepackage{hyperref}
 % Цвета для гиперссылок
\definecolor{linkcolor}{HTML}{799B03} % цвет ссылок
\definecolor{urlcolor}{HTML}{799B03} % цвет гиперссылок

\hypersetup{pdfstartview=FitH,  linkcolor=linkcolor,urlcolor=urlcolor, colorlinks=true}

%%% Дополнительная работа с математикой
\usepackage{amsfonts,amssymb,amsthm,mathtools} % AMS
\usepackage{amsmath}
\usepackage{icomma} % "Умная" запятая: $0,2$ --- число, $0, 2$ --- перечисление

%% Номера формул
%\mathtoolsset{showonlyrefs=true} % Показывать номера только у тех формул, на которые есть \eqref{} в тексте.

%% Шрифты
\usepackage{euscript}	 % Шрифт Евклид
\usepackage{mathrsfs} % Красивый матшрифт

%% Свои команды
\DeclareMathOperator{\sgn}{\mathop{sgn}}

%% Перенос знаков в формулах (по Львовскому)
\newcommand*{\hm}[1]{#1\nobreak\discretionary{}
{\hbox{$\mathsurround=0pt #1$}}{}}
% графика
\usepackage{graphicx}
\graphicspath{{pictures/}}
\DeclareGraphicsExtensions{.pdf,.png,.jpg}
\author{Бурмашев Григорий, БПМИ-208}
\title{Диффуры, дз -- 3}
\date{\today}
\begin{document}
\maketitle
\clearpage
\section*{Номер 1}
\subsection*{a)}
\[
y'' - 3y' + 2y = \frac{1}{1 + e^x}
\]
Для начала решаем однородное (много раз делали такое, не буду подробно расписывать):
\[
\lambda^2 - 3 \lambda + 2 = 0 
\]
\[
(\lambda - 2)(\lambda - 1) = 0
\]
\[
\lambda_1 = 2, \lambda_2 = 1
\]
Оба корня кратности 1, отсюда получаем общее решение:
\[
y = C_1 \cdot e^{2x} + C_2 \cdot e^{x}
\]
Теперь варьируем постоянные:
\[
y = C_1(x) \cdot e^{2x} + C_2(x) \cdot e^{x}
\]
Получается система:
\[
\begin{cases}
C_1'(x) \cdot e^{2x}+ C_2'(x) \cdot e^{x}= 0 \\ 
C_1'(x) \cdot (2 e^{2x})+ C_2'(x)  \cdot e^{x}=  \frac{1}{1 + e^x}
\end{cases}
\]
Из этой системы надо найти константы, вычтем из второго уравнения первое:
\[
C_1'(x) \cdot (2 e^{2x})+ C_2'(x)  \cdot e^{x} - C_1'(x) \cdot e^{2x} - C_2'(x) \cdot e^{x}=  \frac{1}{1 + e^x} - 0 
\]
\[
C_1'(x) \cdot e^{2x} = \frac{1}{1 + e^x}
\]
\[
C_1'(x) = \frac{e^{-2x}}{1 + e^x}
\]
\[
C_1(x) = \int \frac{e^{-2x}}{1 + e^x}dx =  \begin{bmatrix}
u = e^x \\
du = e^x dx \\
dx = \frac{du}{e^{x}}
\end{bmatrix} = 
\int \frac{1}{u^3(1 + u)} du = \int \left( \frac{A}{u^3} + \frac{B}{u^2} +  \frac{C}{u} +  \frac{D}{1 + u} \right)du = (\times)
\]
Найдем коэффы:
\[
1 = A(1 + u) + B(u(1 + u)) + C(u^2(1 + u)) + D(u^3)
\]
\[
1 = A + Au  + Bu + Bu^2  + Cu^2 +  Cu^3  + Du^3
\]
\[
1 = A + u(A + B) + u^2(B + C) + u^3(C + D)
\]
\[
\begin{cases}
A = 1 \\
A + B = 0 \\
B + C = 0 \\
C + D = 0 \\
\end{cases}
\]
\[
\begin{cases}
A = 1 \\
B = -1 \\
C = 1 \\
D = -1
\end{cases}
\]
Возвращаемся к интегралу:
\[
(\times)  =
 \int \left( 
\frac{1}{u^3} - \frac{1}{u^2} +  \frac{1}{u} - \frac{1}{u + 1} 
\right)du 
=  \frac{1}{u} - \frac{1}{2u^2}  +  \ln |u| -  \ln |u + 1| + D_1= 
\]
\[
=
\frac{1}{e^{x}} - \frac{1}{2e^{2x}} + \ln |e^x| - \ln|e^x + 1| + D_1 = \frac{1}{e^{x}} - \frac{1}{2e^{2x}} + x - \ln(e^x + 1)+ D_1
\]
Теперь ищем вторую константу, для этого подставим в первое уравнение $C_1'(x) \cdot e^{2x}$:
\[
 \frac{1}{1 + e^x} + C_2'(x) \cdot e^{x}= 0 
\]
\[
C_2'(x) \cdot e^{x} = -  \frac{1}{1 + e^x}
\]
\[
C_2'(x) = - \frac{e^{-x}}{1 + e^x}
\]
\[
C_2(x) = - \int  \frac{e^{-x}}{1 + e^x}dx = \begin{bmatrix}
u = e^x \\
du = e^x dx \\
dx = \frac{du}{e^{x}}
\end{bmatrix} = - \int \frac{1}{u^2(1 + u)}du = - \int \left( \frac{1}{u^2} - \frac{1}{u} + \frac{1}{u + 1}\right) du   =
\]
\[
=
 \frac{1}{u}  + \ln |u|- \ln|u + 1| = \frac{1}{e^x} + x - \ln(e^x + 1)+ D_2
\]
Итоговый ответ:
\[
y = \left(\frac{1}{e^{x}} - \frac{1}{2e^{2x}} + x- \ln(e^x + 1)+ D_1\right) \cdot e^{2x} + \left( \frac{1}{e^x} + x - \ln(e^x + 1)+ D_2\right) \cdot e^{x} = 
\]
\[
=
\frac{1}{2} + e^x + (x - \ln(e^x + 1)) \cdot e^{2x}  + D_1 \cdot e^{2x}+ (x - \ln(e^x + 1)) \cdot e^x + D_2 \cdot e^x
\]
\begin{center}
\textbf{Ответ: } 
\[
\frac{1}{2} + e^x + (x - \ln(e^x + 1)) \cdot e^{2x}  + D_1 \cdot e^{2x}+ (x - \ln(e^x + 1)) \cdot e^x + D_2 \cdot e^x
\]
\end{center}
\clearpage
\subsection*{b)}
\[
y'' + 3y' = \frac{3x - 1}{x^2}
\]
Решаем однородное:
\[
\lambda^2+ 3 \lambda = 0
\]
\[
\lambda(\lambda + 3) = 0
\]
\[
\lambda_1 = 0, \lambda_2 = -3
\]
Получаем решение:
\[
y = D_1+ D_2 e^{-3x}
\]
Теперь варьируем постоянные:
\[
y = D_1(x) + D_2(x) e^{-3x}
\]
Получается система:
\[
\begin{cases}
D_1'(x) + D_2'(x) e^{-3x} = 0 \\
-3 D_2'(x) e^{-3x} = \frac{3x - 1}{x^2}
\end{cases}
\]
Отсюда получаем:
\[
D_2'(x) = -\frac{3x - 1}{x^2} \cdot \frac{e^{3x}}{3}
\]
\[
D_2(x) = \int \frac{-3x + 1}{x^2} \cdot \frac{e^{3x}}{3} dx= -\int \frac{e^{3x}}{x}  dx+ \int \frac{e^{3x}}{3x^2} dx =
\]
\[
=
- \int -\frac{e^{3x}}{x} dx + \int - \frac{e^{3x}}{x} dx -\frac{e^{3x}}{3x}  = -\frac{e^{3x}}{3x}  + C_2
\]
Теперь найдем $D_1(x)$:
\[
D_1'(x) - \frac{3x-1}{3x^2} = 0
\]
\[
D_1'(x)  =  \frac{3x-1}{3x^2} 
\]
\[
D_1(x) = \int   \frac{3x-1}{3x^2} dx = \int \frac{1}{x} dx - \int \frac{1}{3x^2}dx = \ln |x| + \frac{1}{3x} + C_1
\]
Итого:
\[
y = \ln |x| + \frac{1}{3x} + C_1 + \left(-\frac{e^{3x}}{3x} + C_2\right) e^{-3x} = 
\]
\[
=
\ln |x| + \frac{1}{3x} + C_1  - \frac{1}{3x} + C_2 \cdot e^{-3x} = 
\] 
\[
\ln |x| + C_1 + C_2 \cdot e^{-3x}
\]
\begin{center}
\textbf{Ответ: } 
\[
\ln |x| + C_1 + C_2 \cdot e^{-3x}
\]
\end{center}
\clearpage
\section*{Номер 2}
\subsection*{a)}
\[
4x^2y'' - 4xy' -5y = -4 \sqrt{x}
\]
Это уравнение Эйлера, поэтому делаем замену $x = e^t$ и тогда характеристическое уравнение будет иметь вид:
\[
4 \lambda (\lambda - 1) - 4 \lambda -5 = - 4 e^{\frac{t}{2}}
\]
Тогда ищем однородное:
\[
4 \lambda (\lambda - 1) - 4 \lambda -5 = 0
\]
\[
4\lambda^2 - 8 \lambda -5 = 0
\]
\[
(2 \lambda + 1)(2\lambda -5) = 0
\]
\[
\lambda_1 = - \frac{1}{2}, \lambda_2 = \frac{5}{2}
\]
А значит общее решение:
\[
y(t) = C_1 \cdot e^{-\frac{1}{2}t}  + C_2 \cdot e^{\frac{5}{2}t}
\]
Сделаем обратную замену:
\[
y(x) = C_1 \cdot x^{-\frac{1}{2}}  + C_2 \cdot x^{\frac{5}{2}}  =  \frac{C_1}{\sqrt{x}} + C_2 \cdot x^{\frac{5}{2}} 
\]
Ищем частное решение в виде:
\[
y_1 = d \cdot e^{\frac{t}{2}}, \;
y_1' = \frac{1}{2} d \cdot  e^{\frac{t}{2}}, \; 
y_1'' = \frac{1}{4} d \cdot  e^{\frac{t}{2}}
\]
Подставляем:
\[
(4 \cdot \frac{1}{4}d - 8 \cdot \frac{1}{2}d - 5)d e^{\frac{t}{2}} =  -4 e^{\frac{t}{2}}
\]
\[
-8d = -4 
\]
\[
d = \frac{4}{8} = \frac{1}{2}
\]
Отсюда:
\[
y_1 = \frac{1}{2}e^{\frac{t}{2}} = \frac{1}{2}\sqrt{x}
\]
Тогда получаем ответ:
\[
y =  \frac{1}{2}\sqrt{x}+ \frac{C_1}{\sqrt{x}} + C_2 \cdot x^{\frac{5}{2}} 
\]
\begin{center}
\textbf{Ответ: } 
\[
\frac{1}{2}\sqrt{x}+ \frac{C_1}{\sqrt{x}} + C_2 \cdot x^{\frac{5}{2}} 
\]
\end{center}
\clearpage
\subsection*{b)}
\[
x^2y'' - 6y = -16x^2 \ln x
\]
Делаем замену $x = e^t$, тогда:
\[
\lambda(\lambda - 1) - 6 = -16 e^{2t} \cdot t
\]
Ищем однородное:
\[
\lambda(\lambda - 1) - 6  = 0
\]
\[
\lambda^2 - \lambda -6 = 0
\]
\[
(\lambda - 3)(\lambda + 2) = 0
\]
\[
\lambda_1 = 3, \lambda_2 = -2 
\]
А значит:
\[
y(t) = C_1 \cdot e^{3t} + C_2 \cdot e^{-2t} 
\]
Обратная замена:
\[
y(x) =  C_1 \cdot x^{3} + C_2 \cdot x^{-2} 
\]
Теперь ищем частное, 2 -- не корень, поэтому ищем в виде:
\[
y_1 = e^{2t} (at + b)
\]
\[
y_1' = e^{2t} (2at + a + 2b)
\]
\[
y_1'' = 4 e^{2t} (a + b + a t)
\]
Тогда подставляем:
\[
4 e^{2t} (a + b + a t) - (e^{2t} (2at + a + 2b)) - 6  e^{2t} (at + b) = -16 e^{2t}\cdot  t
\]
\[
e^{2t} \cdot \left(
4a + 4b + 4at -2at - a -2b -6at - 6b
\right)
=-16 e^{2t}\cdot  t
\]
\[
3a - 4at -4b = -16 \cdot  t + 0
\]
Получаем систему:
\[
\begin{cases}
3a -4b = 0 \\
-4at = -16t
\end{cases}
\]
\[
\begin{cases}
a = 4 \\
b = 3
\end{cases}
\]
Тогда:
\[
y_1 = e^{2t}(4t + 3) = 4te^{2t} + 3e^{2t} = 4 \ln x \cdot x^2 + 3x^2
\]
Отсюда ответ:
\[
y = 4 \ln x \cdot x^2 + 3x^2 + C_1 \cdot x^{3} + C_2 \cdot x^{-2} 
\]
\begin{center}
\textbf{Ответ: } 
\[
4 \ln x \cdot x^2 + 3x^2 + C_1 \cdot x^{3} + \frac{C_2}{x^2}
\]
\end{center}
\clearpage
\section*{Номер 4}
\subsection*{a)}
\[
\begin{cases}
x' = -2x - y + 37 \sin t \\
y' = -4x - 5y
\end{cases}
\]
Решаем однородную систему:
\[
\begin{cases}
x' = -2x - y  \\
y' = -4x - 5y
\end{cases}
\]
\[
\begin{vmatrix}
-2 - \lambda & -1 \\
-4 & -5 - \lambda &
\end{vmatrix} 
=
\lambda^2 + 7 \lambda+ 6 = (\lambda + 6)(\lambda + 1)
\]
Имеем корни $-6$ и $-1$ кратности 1. Смотрим для $\lambda = -6$:
\[
\begin{pmatrix}
4& -1 \\
-4 & 1 
\end{pmatrix}
\cdot
\begin{pmatrix}
x \\ y
\end{pmatrix} = 
\left(\begin{matrix}
4x-y \\
-4x+y
\end{matrix}\right)
= 0
\]
Получаем собственный вектор 
$\begin{pmatrix}
1 \\ 4
\end{pmatrix}$.
А для $\lambda = -1$:
\[
\begin{pmatrix}
-1& -1 \\
-4 & -4 
\end{pmatrix}
\cdot
\begin{pmatrix}
x \\ y
\end{pmatrix} = 
\left(\begin{matrix}
-x-y \\
-4x-4y
\end{matrix}\right)
= 0
\]
Получаем
$\begin{pmatrix}
1 \\ -1
\end{pmatrix}$.
Тогда решение:
\[
\begin{pmatrix}
x \\ y
\end{pmatrix} 
=
C_1 \cdot e^{-6t} \cdot  \begin{pmatrix}
1 \\ 4
\end{pmatrix} 
+ C_2 \cdot e^{-t} \cdot  \begin{pmatrix}
1 \\ -1
\end{pmatrix} 
=
\begin{pmatrix}
C_1 \cdot e^{-6t} + C_2 \cdot e^{-t} \\
4 C_1 \cdot e^{-6t} - C_2 \cdot e^{-t}
\end{pmatrix}
\]
Выразим $x'$:
\[
y' = -4x -5 y
\]
\[
x = \frac{-5y-y' }{4}
\]
\[
-2x =  -\frac{-5y-y'}{2}
\]
\[
x' = \frac{-y'' - 5y'}{4}
\]
Тогда:
\[
-\frac{-5y- y'}{2} -y + 37 \sin t =  \frac{-y'' - 5y'}{4}
\]
\[
y'' + 7y' + 6y = - 148 \sin t 
\]
Ищем частное решение в виде:
\[
y_1 =a \cos t + b \sin t 
\]
\[
y_1' = -a \sin t + b \cos t 
\]
\[
y_1'' = -a \cos t - b \sin t 
\]
Подставляем:
\[
 -a \cos t - b \sin t + 7( -a \sin t + b \cos t ) + 6(a \cos t + b \sin t ) = - 148 \sin t 
\]
\[
(5 b - 7 a) \sin t + (5 a + 7 b) \cos t = -148 \sin t
\]
Получаем систему:
\[
\begin{cases}
5b - 7a = -148 \\
5a + 7b = 0 
\end{cases}
\]
\[
\begin{cases}
a = 14\\
b = -10
\end{cases}
\]
А значит:
\[
y_1 = 14 \cos t - 10 \sin t
\]
Вспоминаем что есть $x$:
\[
x = \frac{-5y-y' }{4}
\]
\[
y_1' = -14 \sin t - 10 \cos t
\]
\[
x = \frac{-5(14 \cos t - 10 \sin t)-(-14 \sin t - 10 \cos t)}{4}
\]
\[
x =16 \sin t - 15 \cos t
\]
\begin{center}
\textbf{Ответ: } 
\[
x = C_1 \cdot e^{-6t} + C_2 \cdot e^{-t} + 16 \sin t - 15 \cos t
\]
\[
y = 4 C_1 \cdot e^{-6t} - C_2 \cdot e^{-t} + 14 \cos t - 10 \sin t
\]
\end{center}
\clearpage
\subsection*{b)}
\[
\begin{cases}
x' = 3x -5y - 2e^t \\
y' = -4x - 5y
\end{cases}
\]
Решаем однородную систему:
\[
\begin{cases}
x' = 3x -5y\\
y' = -4x - 5y
\end{cases}
\]
\[
\begin{vmatrix}
3 - \lambda & -5 \\
-4 & -5 - \lambda
\end{vmatrix}
=\lambda^2 + 2\lambda -35 = (\lambda+ 7)(\lambda-5)
\]
Имеем корни $-7$ и $5$ кратности 1. Смотрим для $\lambda = -7$:
\[
\begin{pmatrix}
10 & -5 \\
-4 & 2 
\end{pmatrix}
\cdot
\begin{pmatrix}
x \\ y
\end{pmatrix}
=
\begin{pmatrix}
10x-5y \\
-4x+2y
\end{pmatrix}
\]
Получаем собственный вектор $\begin{pmatrix} 
1 \\ 2
\end{pmatrix}$ А для $\lambda = 5$:
\[
\begin{pmatrix}
-2 & -5 \\
-4 & -10
\end{pmatrix}
\begin{pmatrix}
x \\ y
\end{pmatrix}
=
\left(\begin{matrix}
-2x-5y \\
-4x-10y
\end{matrix}\right)
\]
Получаем собственный вектор $\begin{pmatrix} 
-5 \\ 2
\end{pmatrix}$. Тогда решение:
\[
\begin{pmatrix}
x \\ y
\end{pmatrix}
=
\begin{pmatrix}
C_1 \cdot e^{-7t} - 5C_2 \cdot e^{5t} \\
2C_1 \cdot e^{-7t} + 2C_2 \cdot e^{5t}
\end{pmatrix}
\]
Выразим $x'$:
\[
y' = -4x -5 y
\]
\[
x = \frac{-5y-y' }{4}
\]
\[
3x = \frac{-15y-3y' }{4}
\]
\[
x' = \frac{-y'' - 5y'}{4}
\]
Тогда:
\[
\frac{-y'' - 5y'}{4} = \frac{-15y-3y' }{4} -5y - 2e^t
\]
\[
y'' + 2y' - 35y = 8e^t
\]
Корни никакие не совпадают, ищем решение в виде:
\[
y_1 = a e^t
\]
Тогда:
\[
a e^t + 2ae^t - 35ae^t = 8e^t
\]
\[
-32 ae^t = 8e^t
\]
\[
a = - \frac{1}{4}
\]
А значит:
\[
y_1 = - \frac{e^t}{4}
\]
Вспоминаем что есть $x$:
\[
x = \frac{-5y-y' }{4}
\]
\[
x  = \frac{6\frac{e^t}{4}}{4}
\]
\[
x = \frac{3e^t}{8}
\]
\begin{center}
\textbf{Ответ: } 
\[
x = C_1 \cdot e^{-7t} - 5C_2 \cdot e^{5t} + \frac{3e^t}{8}
\]
\[
y = 2C_1 \cdot e^{-7t} + 2C_2 \cdot e^{5t} - \frac{e^t}{4}
\]
\end{center}
\clearpage
\subsection*{c)}
\[
\begin{cases}
x' = -5x - y\\
y' = x -3y -9e^{2t}
\end{cases}
\]
Решаем однородное:
\[
\begin{cases}
x' = -5x - y\\
y' = x -3y
\end{cases}
\]
\[
\begin{vmatrix}
-5 - \lambda& -1 \\
1 & -3 - \lambda 
\end{vmatrix}
= \lambda^2 + 8\lambda + 16 = (\lambda  +4)^2
\]
Имеем один корень $\lambda = -4$ кратности 2. Найдем собственный вектор:
\[
\begin{pmatrix}
-1 & -1 \\
1 & 1
\end{pmatrix}
\cdot
\begin{pmatrix}
x \\ y
\end{pmatrix}
=
\begin{pmatrix}
-x-y \\
x + y
\end{pmatrix}
=
0
\]
Всего один собственный вектор $h_1 = \begin{pmatrix}
-1 \\ 1
\end{pmatrix}$. Значит есть жорданова клетка ранга 2, придется искать присоединенный вектор:
\[
(A - \lambda E) h_2 = h_1
\]
\[
\begin{pmatrix}
-1 & -1 \\
1 & 1
\end{pmatrix}
\cdot
\begin{pmatrix}
x \\ y
\end{pmatrix}
=
\begin{pmatrix}
-x-y \\
x + y
\end{pmatrix}
=
\begin{pmatrix}
-1 \\ 1
\end{pmatrix}
\]
Например, $h_2 = \begin{pmatrix}
0 \\ 1
\end{pmatrix}$. Тогда выпишем решения по формуле:
\[
\begin{pmatrix}
x_1 \\ y_1 
\end{pmatrix}
=
\begin{pmatrix}
-1 \cdot e^{-4t} \\
1 \cdot e^{-4t}
\end{pmatrix}
\]
\[
\begin{pmatrix}
x_2 \\ y_2 
\end{pmatrix}
=
\begin{pmatrix}
e^{-4t} \cdot  (-t + 0)\\
e^{-4t} \cdot (t + 1) \\
\end{pmatrix}
\]
Тогда:
\[
x = -C_1 \cdot  e^{-4t} + C_2 \cdot e^{-4t} \cdot  (-t + 0)
\]
\[
y = C_1 \cdot e^{-4t} + C_2 \cdot e^{-4t} \cdot (t + 1) 
\]
Выразим $y'$:
\[
x' = -5x -y
\]
\[
y = -5x - x'
\]
\[
y' = -5x' -x''
\]
Подставляем:
\[
-5x' -x'' = x - 3( -5x - x') - 9e^{2t}
\]
\[
 x'' + 8x' + 16x = 9e^{2t} 
\]
Ищем решение в виде:
\[
x_1 = a e^{2t}
\]
\[
x_1' =2ae^{2t}
\]
\[
x_1'' = 4ae^{2t}
\]
Тогда:
\[
 4ae^{2t} + 8(2ae^{2t})+ 16 a e^{2t} = 9e^{2t} 
\]
\[
36ae^{2t} = 9e^{2t}
\]
\[
a = \frac{1}{4}
\]
Тогда:
\[
x_1 = \frac{e^{2t}}{4}
\]
\[
x_1' = \frac{e^{2t}}{2}
\]
Вспоминаем что есть $y$:
\[
y = -5x - x'
\]
\[
y = - 5\frac{e^{2t}}{4} - \frac{e^{2t}}{2}
\]
\[
y =- \frac{7e^{2t}}{4}
\]
\begin{center}
\textbf{Ответ: } 
\[
x = -C_1 \cdot  e^{-4t} + C_2 \cdot e^{-4t} \cdot  (-t + 0) +   \frac{e^{2t}}{4} 
\]
\[
y = C_1 \cdot e^{-4t} + C_2 \cdot e^{-4t} \cdot (t + 1)  - \frac{7e^{2t}}{4}
\]
\end{center}
\clearpage
\subsection*{d)}
\[
\begin{cases}
x' = 3x + 2y - e^{-t} \\
y' = -2x -2y - e^{-t}
\end{cases}
\]
Решаем однородное:
\[
\begin{cases}
x' = 3x + 2y \\
y' = -2x -2y 
\end{cases}
\]
\[
\begin{vmatrix}
3 - \lambda & 2 \\
-2 & -2 - \lambda 
\end{vmatrix}
=
\lambda^2 -\lambda -2 =  (\lambda + 1)(\lambda -2)
\]
Получили два корня $-1$ и $2$ кратности 1. Для $\lambda = -1$:
\[
\begin{pmatrix}
4 & 2 \\
-2 & -1
\end{pmatrix}
\cdot
\begin{pmatrix}
x \\ y
\end{pmatrix}
=
\begin{pmatrix}
4x + 2y \\
-2x - y
\end{pmatrix}
=
0
\]
Получаем собственный вектор $\begin{pmatrix}
-1 \\ 2
\end{pmatrix}$. Для $\lambda = 2$:
\[
\begin{pmatrix}
1 & 2  \\
-2 & -4
\end{pmatrix}
\cdot
\begin{pmatrix}
x \\ y
\end{pmatrix}
=
\left(\begin{matrix}
x+2y \\
-2x-4y
\end{matrix}\right)
=
0
\]
Получаем собственный вектор $\begin{pmatrix}
-2 \\ 1
\end{pmatrix}$. Тогда решение:
\[
\begin{pmatrix}
x \\ y
\end{pmatrix}
=
\begin{pmatrix}
-C_1 \cdot e^{-t} - 2C_2 \cdot e^{2t} \\
2C_1 \cdot e^{-t} + C_1 \cdot e^{2t}
\end{pmatrix}
\]
Выразим $y'$:
\[
x' = 3x + 2y - e^{-t}
\]
\[
-2y = 3x - e^{-t} - x'
\]
\[
y = -\frac{3x -e^{-t}  -x'}{2}
\]
\[
y' = \frac{-3x' - e^{-t} + x''}{2}
\]
\[
\frac{-3x' - e^{-t} + x''}{2} = -2x + 3x - e^{-t} - x' - e^{-t}
\]
\[
x' + 2x = x'' + 3e^{-t}
\]
\[
x'' - x' - 2x = -3e^{-t}
\]
Совпадает корень $\lambda = -1$ со степенью, а значит будем искать решение в виде:
\[
x_1 = e^{-t}(at + b)
\]
\[
x_1' = a e^{-t} - b e^{-t} - at e^{-t} 
\]
\[
x_1'' = -2 a e^{-t}+ b e^{-t} + at e^{-t}
\]
Тогда:
\[
 -2 a e^{-t}+ b e^{-t} + at e^{-t} - (a e^{-t} - b e^{-t} - at e^{-t} ) - 2( e^{-t}(at + b)) = -3e^{-t}
\]
\[
\begin{cases}
-3ae^{-t} = -3e^{-t} \\
b = C_3
\end{cases}
\]
\[
\begin{cases}
a = 1 \\
b = C_3
\end{cases}
\]
Отсюда:
\[
x_1 = e^{-t}(t + C_3)
\]
\[
x_1' = -e^{-t} ((t + C_3) - 1)
\]
Вспоминаем что есть $y$:
\[
y = -\frac{3x -e^{-t}  -x'}{2}
\]
\[
y = -\frac{3(e^{-t}(t + C_3))-e^{-t}  + e^{-t} ((t + C_3) - 1))}{2}
\]
\[
y = -e^{-t}(2t + 2C_3 - 1)
\]
\begin{center}
\textbf{Ответ: } 
\[
x  =-C_1 \cdot e^{-t} - 2C_2 \cdot e^{2t}  + e^{-t}(t + C_3)
\]
\[
y =
2C_1 \cdot e^{-t} + C_1 \cdot e^{2t}  -e^{-t}(2t + 2C_3 - 1)
\]
\end{center}
\end{document}
