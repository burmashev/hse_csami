\documentclass[a4paper,12pt]{article}

%%% Работа с русским языком
\usepackage{cmap}					% поиск в PDF
\usepackage{mathtext} 				% русские буквы в формулах
\usepackage[T2A]{fontenc}			% кодировка
\usepackage[utf8]{inputenc}			% кодировка исходного текста
\usepackage[english,russian]{babel}	% локализация и переносы
\usepackage{xcolor}
\usepackage{hyperref}
 % Цвета для гиперссылок
\definecolor{linkcolor}{HTML}{799B03} % цвет ссылок
\definecolor{urlcolor}{HTML}{799B03} % цвет гиперссылок

\hypersetup{pdfstartview=FitH,  linkcolor=linkcolor,urlcolor=urlcolor, colorlinks=true}

%%% Дополнительная работа с математикой
\usepackage{amsfonts,amssymb,amsthm,mathtools} % AMS
\usepackage{amsmath}
\usepackage{icomma} % "Умная" запятая: $0,2$ --- число, $0, 2$ --- перечисление

%% Номера формул
%\mathtoolsset{showonlyrefs=true} % Показывать номера только у тех формул, на которые есть \eqref{} в тексте.

%% Шрифты
\usepackage{euscript}	 % Шрифт Евклид
\usepackage{mathrsfs} % Красивый матшрифт

%% Свои команды
\DeclareMathOperator{\sgn}{\mathop{sgn}}

%% Перенос знаков в формулах (по Львовскому)
\newcommand*{\hm}[1]{#1\nobreak\discretionary{}
{\hbox{$\mathsurround=0pt #1$}}{}}
% графика
\usepackage{graphicx}
\graphicspath{{pictures/}}
\DeclareGraphicsExtensions{.pdf,.png,.jpg}
\author{Бурмашев Григорий, БПМИ-208}
\title{}
\date{\today}
\begin{document}
\begin{center}
Бурмашев Григорий. Дискра -- 17
\end{center}
\section*{Номер 1}
Пусть f -- выигрыш, он точно неотрицательный. Математическое ожидание выигрыша равно $40 \cdot 25\% = 10$. Положим a = 1000, тогда по неравенству Маркова:
\[
Pr[f \geq a] \leq \frac{E[f]}{a}
\]
\[
Pr[f \geq 1000] \leq \frac{10}{1000}  = 0.01
\]
Вероятность получить не менее 1000 рублей не превосходит 1\%
\section*{Номер 2}
Посчитаем вероятности:
\begin{enumerate}
\item Шестерка выпала ровно 0 раз:
\[
\frac{5^3}{6^3} = \frac{125}{216}
\]
\item Шестерка выпала ровно 1 раз:
\[
\frac{1}{6} \cdot \frac{5^2}{6^2} \cdot 3 = \frac{1}{2} \cdot \frac{5^2}{6^2} = \frac{25}{72}
\]
\item Шестерка выпала ровно 2 раза:
\[
\frac{1}{6^2} \cdot \frac{5}{6} \cdot 3 = \frac{1}{6^2} \cdot \frac{5}{2}  = \frac{5}{72}
\]
\item Шестерка выпала ровно 3 раза:
\[
\frac{1}{6^3} = \frac{1}{216}
\]
Тогда мат.ожидание:
\[
E(\text{Sum}) = \frac{1}{216} \cdot 3 + \frac{5}{72} \cdot 2 + \frac{25}{72} \cdot 1 - \frac{125}{216} = \frac{3 + 30 + 75 - 125 }{216} = -\frac{17}{216}
\]
\end{enumerate}
\begin{center}
\textbf{Ответ: } $-\frac{17}{216}$
\end{center}
\clearpage
\section*{Номер 3}
Очевидно, что:
\[
2^x \geq 64, \text{ если }  x \geq 6
\]
А значит:
\[
Pr[x \geq 6] = Pr[2^x \geq 64]
\]
Тогда по неравенству Маркова:
\[
Pr[2^x \geq 64] \leq \frac{E[2^x]}{64}
\]
Ну а по условию $E[2^x] = 5$, итого:
\[
Pr[x \geq 6] \leq \frac{5}{64} = \frac{50}{640} < \frac{1}{10} = \frac{64}{640}
\]
\section*{Номер 4}
Два события являются независимыми, а значит:
\[
E[LR] = E[L] \cdot E[R]
\]
\[
E[L] = E[R]
\]
Найдем $E[L]$:
\\\\
Вероятность получить i единиц равна:
\[
\frac{C^i_{50}}{2^{50}}
\]
Где:

$C^i_{50}$ -- количество спобосов расставить i единиц на 50 позиций

$2^{50}$ -- вероятность получения именно этой расстановки
\\\\
А значит:
\[
E[L] = 1 \cdot \frac{C^1_{50}}{2^{50}} + \ldots + 50 \cdot \frac{C^{50}_{50}}{2^{50}} =\sum_{i = 0}^{50} i \cdot \frac{C^i_{50}}{2^{50}}
\]
Тогда:
\[
E[LR]  = \left(\sum_{i = 0}^{50} i \cdot \frac{C^i_{50}}{2^{50}} \right)^2
\]
(Не очень понимаю, как можно красиво посчитать эту штуку, поэтому оставил как есть)
\begin{center}
\textbf{Ответ: } $E[LR]  = \left(\sum_{i = 0}^{50} i \cdot \frac{C^i_{50}}{2^{50}} \right)^2$
\end{center}
\section*{Номер 5}
Введем следующую индикаторную функцию $I$, которая будет обозначать количество элементов, которые перешли сами в себя, т.е:
\[
I = I_1 + I_2 + \ldots I_n
\]
Где:
\[
I_i = 
\begin{cases}
1, \text {если i переходит в перестановке сам в себя} \\
0, \text {иначе}
\end{cases}
\]
Тогда:
\[
E[I] = E[I_1 + \ldots I_n] = E[I_1] + \ldots + E[I_n]
\]
Посчитаем $E[I_i]$:
\[
E[I_i] = 1 \cdot Pr[\text{i перешла в перестановке сама в себя}] + 
\]
\[
+
0 \cdot Pr[\text{i перешла не сама в себя}] = Pr[\text{i перешла в перестановке сама в себя}] 
=
\]
\[
= \frac{(n-1)!}{n!} = \frac{1}{n}
\]
Тогда:
\[
I = \frac{1}{n} \cdot n = 1
\]
\begin{center}
\textbf{Ответ: } $1$ элемент
\end{center}
\section*{Номер 6}
Введем следующую индикаторную функцию I, которая будет обозначать количество элементов, попавших  в пересечение:
\[
I = I_1 + I_2 + \ldots I_n
\]
Где:
\[
I_i = 
\begin{cases}
1, \text {если i-й элемент попал в пересечение} \\
0, \text {иначе}
\end{cases}
\]
Тогда:
\[
E[I] = E[I_1 + \ldots I_n] = E[I_1] + \ldots + E[I_n]
\]
Посчитаем $E[I_i]$:
\[
E[I_i] = 1 \cdot Pr[\text{i-й в пересечении}] + 
\]
\[
+
0 \cdot Pr[\text{i-й не в пересечении}] = Pr[\text{i-й  в пересечении}] 
\]
Количество способов задать множество X или Y:
\[
C^k_n
\]
В свою очередь количество способов задать множество X или Y с i-тым элементом:
\[
C^{k-1}_{n-1}
\]
А значит:
\[
E[I_i] = \frac{\left(C^{k-1}_{n-1}\right)^2}{\left(C^k_n\right)^2} = \left(\frac{(n-1)!}{(k-1)!(n-k)!} \cdot \frac{k!(n-k)!}{n!}\right)^2 =  \left(\frac{k}{n}\right)^2
\]
По итогу:
\[
E[I] =\left( \frac{k}{n}\right)^2 \cdot n  = \frac{k^2}{n}
\]
\begin{center}
\textbf{Ответ: } $\frac{k^2}{n}$
\end{center}
\end{document}
