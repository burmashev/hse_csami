\documentclass[a4paper,12pt]{article}

%%% Работа с русским языком
\usepackage{cmap}					% поиск в PDF
\usepackage{mathtext} 				% русские буквы в формулах
\usepackage[T2A]{fontenc}			% кодировка
\usepackage[utf8]{inputenc}			% кодировка исходного текста
\usepackage[english,russian]{babel}	% локализация и переносы
\usepackage{xcolor}
\usepackage{hyperref}
 % Цвета для гиперссылок
\definecolor{linkcolor}{HTML}{799B03} % цвет ссылок
\definecolor{urlcolor}{HTML}{799B03} % цвет гиперссылок

\hypersetup{pdfstartview=FitH,  linkcolor=linkcolor,urlcolor=urlcolor, colorlinks=true}

%%% Дополнительная работа с математикой
\usepackage{amsfonts,amssymb,amsthm,mathtools} % AMS
\usepackage{amsmath}
\usepackage{icomma} % "Умная" запятая: $0,2$ --- число, $0, 2$ --- перечисление

%% Номера формул
%\mathtoolsset{showonlyrefs=true} % Показывать номера только у тех формул, на которые есть \eqref{} в тексте.

%% Шрифты
\usepackage{euscript}	 % Шрифт Евклид
\usepackage{mathrsfs} % Красивый матшрифт

%% Свои команды
\DeclareMathOperator{\sgn}{\mathop{sgn}}

%% Перенос знаков в формулах (по Львовскому)
\newcommand*{\hm}[1]{#1\nobreak\discretionary{}
{\hbox{$\mathsurround=0pt #1$}}{}}
% графика
\usepackage{graphicx}
\graphicspath{{pictures/}}
\DeclareGraphicsExtensions{.pdf,.png,.jpg}
\author{Бурмашев Григорий, БПМИ-208}
\title{}
\date{\today}
\begin{document}
\section*{Задание 1}
Пусть $S = (n + 0) + (n + 1) + \ldots + (n + 2019)$.

Всего у нас 2020 скобок, в каждой есть $n$, значит все эти $n-$ки можем заменить на $2020 \cdot n$. Тогда вынесем это за скобки и получим вид:
\[
S = 2020 \cdot n \cdot (0 + 1 + \ldots + 2019)
\]
0 ничего не значит и его мы можем отбросить, сумму чисел 1 от до 2019 можем посчитать с помощью формулы : $S_n = \frac{n(n+1)}{2}$. Докажем её корректность методом мат.индукции:
\begin{itemize}
\item База:
\[
S_1 = \frac{1(1 + 1)}{2} = 1
\]
\item Индукция:

пусть $S_n = \frac{n(n+1)}{2}$, тогда для $n + 1$:
\[
S_{n + 1} = S_n + (n + 1) = \frac{n(n+1)}{2} + n + 1 = \frac{n^2 + n + 2n + 2}{2} =
\]
\[
=
\frac{n^2 + 3n + 2}{2} = \frac{(n+1)(n+2)}{2} = S_{n + 1}
\]
\end{itemize}
\begin{center}
\textbf{Корректность доказана} 
\end{center}
Теперь считаем:
\[
S_{2019} = \frac{2019 \cdot 2020}{2} = 2039190
\]
По итогу получаем:
\[
S = (n + 0) + (n + 1) + \ldots + (n + 2019) = 2020 \cdot n + 2039190
\]
\begin{center}
\textbf{Ответ: } $2020 \cdot n + 2039190$
\end{center}
\end{document}
