\documentclass[a4paper,12pt]{article}

%%% Работа с русским языком
\usepackage{cmap}					% поиск в PDF
\usepackage{mathtext} 				% русские буквы в формулах
\usepackage[T2A]{fontenc}			% кодировка
\usepackage[utf8]{inputenc}			% кодировка исходного текста
\usepackage[english,russian]{babel}	% локализация и переносы
\usepackage{xcolor}
\usepackage{hyperref}
 % Цвета для гиперссылок
\definecolor{linkcolor}{HTML}{799B03} % цвет ссылок
\definecolor{urlcolor}{HTML}{799B03} % цвет гиперссылок

\hypersetup{pdfstartview=FitH,  linkcolor=linkcolor,urlcolor=urlcolor, colorlinks=true}

%%% Дополнительная работа с математикой
\usepackage{amsfonts,amssymb,amsthm,mathtools} % AMS
\usepackage{amsmath}
\usepackage{icomma} % "Умная" запятая: $0,2$ --- число, $0, 2$ --- перечисление

%% Номера формул
%\mathtoolsset{showonlyrefs=true} % Показывать номера только у тех формул, на которые есть \eqref{} в тексте.

%% Шрифты
\usepackage{euscript}	 % Шрифт Евклид
\usepackage{mathrsfs} % Красивый матшрифт

%% Свои команды
\DeclareMathOperator{\sgn}{\mathop{sgn}}

%% Перенос знаков в формулах (по Львовскому)
\newcommand*{\hm}[1]{#1\nobreak\discretionary{}
{\hbox{$\mathsurround=0pt #1$}}{}}
% графика
\usepackage{graphicx}
\graphicspath{{pictures/}}
\DeclareGraphicsExtensions{.pdf,.png,.jpg}
\author{Бурмашев Григорий, БПМИ-208}
\title{}
\date{\today}
\begin{document}
\begin{center}
Бурмашев Григорий. Алгебра -- 9
\end{center}
\begin{center}
\textbf{[ПОСЛЕДНЯЯ]}
\end{center}
\section*{Номер 1}
Расписываем $\mathbb{F}_9$:
\[
\mathbb{F}_9 = \mathbb{Z}_3[x]/x^2 +2x + 2 = \{0, 1, 2, x, x + 1, x + 2, 2x, 2x + 1, 2x + 2\}
\]
Заметим,  что $x^2 + 2x + 2$ неприводим, т.к у него нет корней
\\\\
Формула понижения степени:
\[
x^2 = x + 1 
\]
Составим таблицу умножения для $\mathbb{F}_9$:

\begin{center}
\begin{tabular}{|c|c|c|c|c|c|c|c|c|c|}
\hline
$\cdot$& 0 & 1 & 2 & x & x + 1 & x + 2 & 2x & 2x + 1 & 2x + 2 \\
\hline
0 & 0 & 0 & 0 & 0 & 0 & 0 & 0 & 0 & 0 \\
\hline
1 & 0 & 1 & 2 & x & x+1 & x+2 & 2x & 2x+1 & 2x+2 \\
\hline
2 & 0 & 2 & 1 & 2x & 2x+2 & 2x+1 & x & x+2 & x+1 \\
\hline
x & 0 & x & 2x & x+1 & 2x+1 & 1 & 2x+2 & 2 & x+2 \\
\hline
x +1  & 0 & x+1 & 2x+2 & 2x+1 & 2 & x & x+2 & 2x & 1 \\
\hline
x + 2 & 0 & x+2 & 2x+1 & 1 & x & 2x+2 & 2 & x+1 & 2x \\
\hline
2x & 0 & 2x & x & 2x+2 & x+2 & 2 & x+1 & 1 & 2x+1 \\
\hline
2x + 1 & 0 & 2x+1 & x+2 & 2 & 2x & x+1 & 1 & 2x+2 & x \\
\hline
2x + 2 & 0 & 2x+2 & x+1 & x+2 & 1 & 2x & 2x+1 & x & 2 \\
\hline
\end{tabular}
\end{center}

Из лекции знаем, что $\forall a \in F_q$ $a$ является корнем многочлена $x^q - x$, а значит $a^{q-1} = 1$, в нашем случае $q = 9$, а значит $a^8 = 1$ и нас интересуют элементы порядка 8, нужно их найти
\clearpage

А теперь для каждого элемента возводим его в степень, пока не придем в единицу. Для возведения будем просто идти по таблице умножения до тех пор, пока не упремся в 1.  Таким образом и посчитаем порядок каждого элемента.

Через $\rightarrow$ обозначаю возведение исходного числа в очередную степень (т.е $x^2$ будет $x \rightarrow x + 1$ соотвественно)

\begin{itemize}
\item 0:

Нас не интересует, т.к не является порождающим

\item 1:

\[
1^1 = 1
\]
Т.е порядок 1
\item 2:

\[
2 \rightarrow 1
\]
Т.е порядок 2 
\item $x$:
\[
x \rightarrow x + 1 \rightarrow 2x+1 \rightarrow 2 \rightarrow 2x \rightarrow 2x+ 2 \rightarrow x+2 \rightarrow 1
\]
\textbf{Т.е порядок 8}

\item $x+1$:
\[
x +1  \rightarrow 2 \rightarrow 2x+2 \rightarrow 1
\]
Т.е порядок 4

\item $x + 2$:

\[
x + 2 \rightarrow 2x +2 \rightarrow 2x \rightarrow 2 \rightarrow 2x+1 \rightarrow x+1 \rightarrow x \rightarrow 1
\]
\textbf{Т.е порядок 8}

\item $2x$:

\[
2x \rightarrow \rightarrow x+1 \rightarrow x + 2 \rightarrow 2 \rightarrow x \rightarrow 2x + 2 \rightarrow 2x + 1 \rightarrow 1
\]
\textbf{Т.е порядок 8}

\item $2x + 1$:
\[
2x + 1 \rightarrow 2x +2 \rightarrow x \rightarrow 2 \rightarrow x + 2 \rightarrow  x + 1 \rightarrow 2x \rightarrow 1
\]
\textbf{Т.е порядок 8}

\item $2x + 2$:
\[
2x +2 \rightarrow 2 \rightarrow x + 1 \rightarrow 1 
\]
Т.е порядок 4
\end{itemize}


Посчитали все порядки, по итогу:

\begin{center}
\begin{tabular}{|c|c|}
\hline
Элемент: & Порядок: \\
\hline
1 & 1 \\
\hline
2 & 2 \\
\hline
x & 8 \\
\hline
x+1 & 4 \\
\hline
x+2 & 8 \\
\hline
2x & 8 \\
\hline
2x+1 & 8 \\
\hline
2x+2 & 4 \\
\hline
\end{tabular}
\end{center}
Берем только те, у которых порядок 8
\begin{center}
\textbf{Ответ: } 
\[
x, x + 2 , 2x, 2x + 1
\]
\end{center}

\clearpage

\section*{Номер 2}
Задача с семинара, поэтому делаю как на семинаре:

\[
p = 5,  n = 2
\]
\[
h_1 = x^2 + 3
\]
\[
h_2 = y^2 + y + 2
\]
Для начала сделаем легкую часть, проверим на приводимость:

В $\mathbb{Z}_5$ возможные корни:
\[
0, 1, 2, 3, 4
\]
Проверяем:
\begin{itemize}
\item $h_1$:
 \[
h_1(0) = 3
\]
\[
h_1(1) = 4
\]
\[
h_1(2) = 4 + 3 = 2
\]
\[
h_1(3) = 4 + 3 = 2
\]
\[
h_1(4) = 1 + 3 = 4
\]
\item $h_2$:
\[
h_2(0) = 2
\]
\[
h_2(1) = 1 + 1 + 2 = 4
\]
\[
h_2(2) = 4 + 2 + 2 = 3
\]
\[
h_2(3) = 4 + 3 + 2 = 4
\]
\[
h_2(4) = 1 + 4 + 2 = 2
\]
\end{itemize}
Нигде не получили нулей, значит они действительно неприводимы над $\mathbb{Z}_5$
\\\\
Нам нужно построить явно изоморфизм вида:
\[
F_1 = \mathbb{Z}_5 /(h_1) \simeq F_2 = \mathbb{Z}_5/(h_2) 
\]

\[
\exists \alpha \in F_2 : h_1(\alpha) = 0
\]
\clearpage
Гомоморфизм:
\[
\varphi : \mathbb{Z}_5[x] \rightarrow F_2
\]
\[
f \Rightarrow f(\alpha)
\]
На лекции доказывалось, что:
\[
h_1 \in \text{Ker} \varphi \leadsto \text{Ker} \varphi = (h_1)
\]
Теорема о гомоморфизме $\Rightarrow$ изоморфизм вида:
\[
F_1 = \mathbb{Z}_p[x] / (h_1) \simeq \text{Im} \varphi \subseteq F_2
\] 
В $F_1$ лежит $q$ элементов, в $F_2$ тоже $q$, тогда получается, что образ тоже содержит $q$ элементов, а тогда образ совпадает с $F_2$ и мы действительно получаем изоморфизм.
\\\\
Остается найти такое $\alpha \in F_2  : h_1(\alpha) = 0$

Знаем, что в $F_2$:
\[
y^2 = -y - 2 = 4y + 3
\]
Теперь находим $\alpha$:
\[
\alpha = a\overline{y} + b
\]
\[
h_1(\alpha) = (a\overline{y}+b)^2 + 3 = a^2\overline{y}^2 + 2 a\overline{y}b + b^2 + 3= 0
\]
Пользуемся фактом про понижение степени:
\[
a^2(4\overline{y}+ 3) +  2a \overline{y}b + b^2 + 3 = 0
\]
\[
4a^2\overline{y} + 3a^2 + 2a \overline{y}b + b^2 + 3 = 0
\]
\[
\overline{y} (4a^2 + 2ab) + (3a^2 + b^2 + 3) = 0
\]
Получаем СЛУ:
\[
\begin{cases}
4a^2 + 2ab = 0 \\3a^2 + b^2 + 3 = 0
\end{cases}
\]
\[
\begin{cases}
2a(2a  + b) = 0 \\3a^2 + b^2 + 3 = 0
\end{cases}
\]
Пытаемся угадать вариант, их не так уж и много в рамках $\mathbb{Z}_5$
\begin{itemize}
\item $a = 1 $:
\[
\begin{cases}
2(2 + b) = 0  \\ 
b^2 + 1 = 0 
\end{cases}
\]
\[
\begin{cases}
b = 3  \\ 
b^2 + 1 = 3^2 + 1 = 4 + 1 = 0 
\end{cases}
\]
\end{itemize}
Все выполняется, а значит угадывание можно останавливать и нам подходит вариант вида:
\[
\alpha = \overline{y} + 3
\]
\begin{center}
\textbf{Ответ: } 

изоморфизм $F_1 \simeq F_2$ определяется так:
\[
a\overline{x} + b \rightarrow a(\overline{y} + 3) + b
\]

\begin{center}
\textbf{Вот и всё :(}
\end{center}
\end{center}
 \end{document}



