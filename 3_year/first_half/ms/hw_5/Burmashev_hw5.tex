\documentclass[a4paper,12pt]{article}

%%% Работа с русским языком
\usepackage{cmap}					% поиск в PDF
\usepackage{mathtext} 				% русские буквы в формулах
\usepackage[T2A]{fontenc}			% кодировка
\usepackage[utf8]{inputenc}			% кодировка исходного текста
\usepackage[english,russian]{babel}	% локализация и переносы
\usepackage{xcolor}
\usepackage{hyperref}
 % Цвета для гиперссылок
\definecolor{linkcolor}{HTML}{799B03} % цвет ссылок
\definecolor{urlcolor}{HTML}{799B03} % цвет гиперссылок

\hypersetup{pdfstartview=FitH,  linkcolor=linkcolor,urlcolor=urlcolor, colorlinks=true}

%%% Дополнительная работа с математикой
\usepackage{amsfonts,amssymb,amsthm,mathtools} % AMS
\usepackage{amsmath}
\usepackage{icomma} % "Умная" запятая: $0,2$ --- число, $0, 2$ --- перечисление

%% Номера формул
%\mathtoolsset{showonlyrefs=true} % Показывать номера только у тех формул, на которые есть \eqref{} в тексте.

%% Шрифты
\usepackage{euscript}	 % Шрифт Евклид
\usepackage{mathrsfs} % Красивый матшрифт

%% Свои команды
\DeclareMathOperator{\sgn}{\mathop{sgn}}

%% Перенос знаков в формулах (по Львовскому)
\newcommand*{\hm}[1]{#1\nobreak\discretionary{}
{\hbox{$\mathsurround=0pt #1$}}{}}
% графика
\usepackage{graphicx}
\graphicspath{{pictures/}}
\DeclareGraphicsExtensions{.pdf,.png,.jpg}
\author{Бурмашев Григорий, БПМИ-208}
\title{}
\date{\today}
\begin{document}
\maketitle
\clearpage
\section*{Номер 1}
Являются ли общезначимыми следующие формулы:
\subsection*{a)}
\[
(\exists x P(x) \rightarrow \exists y Q(y))
\rightarrow \exists y \forall (P(x) \rightarrow Q(y))
\]
Видим, что:
\[
(\exists x P(x) \rightarrow \exists y Q(y))
\equiv 
\overline{(\exists x P(x))} \vee \exists y Q(y) 
\equiv
\forall x \exists y \overline{(P(x)} \vee Q(y))
\]
Второе:
\[
\exists y \forall (P(x) \rightarrow Q(y))
\equiv
\exists y \forall x (\overline{P(x)} \vee Q(y))
\]
С помощью этих преобразований исходная формула принимает вид:
\[
\forall x \exists y \overline{(P(x)} \vee Q(y))
\rightarrow
\exists y \forall x (\overline{P(x)} \vee Q(y))
\]
Пусть существует такая модель, что формула ложна, т.е принимает вид $1 \rightarrow 0$. Тогда заметим, что левая и правая часть не зависят друг от друга из-за перемены $x$ и $y$ местами, следовательно левая часть может быть истина только при выполнении хотя бы одного из:
\begin{itemize}
\item $\forall x \overline{P(x)}$
\item $\exists y Q(y)$
\end{itemize}
Но тогда мы получаем истинность заключения в нашей формуле, из этого следует, что формула \textbf{общезначима}
\clearpage
\subsection*{b)}
\[
(\forall x P(x, x) \wedge \forall x, y, z ((P(x, y) \wedge P(y, z)) \rightarrow P(x, z))) \rightarrow \forall x, y (P(x, y) \rightarrow P(y, x))
\]
Рассмотрим такую модель, где $P(x, y)  \leftrightarrow x \ge y $, тогда можем заметить:
\[
\forall x (x \ge x) \equiv 1
\]
\[
\forall x, y, z, ((x \ge y, y \ge z ) \rightarrow x \ge z) \equiv 1
\]
Таким образом левая часть нашей исходной формулы принимает вид:
\[
\forall x P(x, x) \wedge \forall x, y, z  ((P(x, y) \wedge P(y, z) \rightarrow P(x, z)) \leftrightarrow 1 \wedge 1 \equiv 1
\]
Теперь, аналогично, для правой части формулы (заключения) получаем:
\[
\forall x, y (x \ge y \rightarrow y \ge x)
\equiv 0 
\]
Тобишь:
\[
\forall x, y (P(x, y) \rightarrow P(y, x))
\equiv 0 
\]
Из всего полученного выше получаем, что:
\[
(\forall x P(x, x) \wedge \forall x, y, z ((P(x, y) \wedge P(y, z)) \rightarrow P(x, z))) \rightarrow \forall x, y (P(x, y) \rightarrow P(y, x)) \equiv 1 \rightarrow 0 \equiv 0
\]
Следовательно формула \textbf{не является} общезначимой
\begin{center}
\textbf{Ответ: } 
\begin{itemize}
\item a) да, является
\item b) нет, не является
\end{itemize}
\end{center}
\end{document}
