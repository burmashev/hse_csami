\documentclass[a4paper,12pt]{article}

%%% Работа с русским языком
\usepackage{cmap}					% поиск в PDF
\usepackage{mathtext} 				% русские буквы в формулах
\usepackage[T2A]{fontenc}			% кодировка
\usepackage[utf8]{inputenc}			% кодировка исходного текста
\usepackage[english,russian]{babel}	% локализация и переносы
\usepackage{xcolor}
\usepackage{hyperref}
 % Цвета для гиперссылок
\definecolor{linkcolor}{HTML}{799B03} % цвет ссылок
\definecolor{urlcolor}{HTML}{799B03} % цвет гиперссылок

\hypersetup{pdfstartview=FitH,  linkcolor=linkcolor,urlcolor=urlcolor, colorlinks=true}

%%% Дополнительная работа с математикой
\usepackage{amsfonts,amssymb,amsthm,mathtools} % AMS
\usepackage{amsmath}
\usepackage{icomma} % "Умная" запятая: $0,2$ --- число, $0, 2$ --- перечисление

%% Номера формул
%\mathtoolsset{showonlyrefs=true} % Показывать номера только у тех формул, на которые есть \eqref{} в тексте.

%% Шрифты
\usepackage{euscript}	 % Шрифт Евклид
\usepackage{mathrsfs} % Красивый матшрифт

%% Свои команды
\DeclareMathOperator{\sgn}{\mathop{sgn}}

%% Перенос знаков в формулах (по Львовскому)
\newcommand*{\hm}[1]{#1\nobreak\discretionary{}
{\hbox{$\mathsurround=0pt #1$}}{}}
% графика
\usepackage{graphicx}
\graphicspath{{pictures/}}
\DeclareGraphicsExtensions{.pdf,.png,.jpg}
\author{Бурмашев Григорий, БПМИ-208}
\title{Диффуры, дз -- 5}
\date{\today} 
\begin{document}
\maketitle
\section*{Номер 2}
\subsection*{a)}
\[
(2x - x^2)y'' + 2y' - \frac{2}{x}y = (2-x)^2 xe^{-x}
\]
Избавимся от скобки, для этого поделим на $(2x-x^2)$, заведомо заметив, что $x = 0$ и $x = 2$ -- не являются корнями уравнения, получаем:
\[
y'' + \frac{2}{(2-x)x} y'  - \frac{2}{(2 - x)x^2} y = (2 - x)e^{-x}
\]
Разберемся сначала с однородным уравнением, замечаем, что $y_1 = x$ -- частное решение, тогда применяем формулу Луивилля--Остроградского и получаем:
\[
xy_2' - y_2 = C \cdot e^{\int \frac{2}{x^2 - 2x}dx}
\]
Выражаем:
\[
xy_2' - y_2 = C \cdot e^{\ln(2 - x) - \ln (x)}
\]
\[
xy_2' - y_2 = C \cdot \frac{2 - x}{x}
\]
\[
\frac{xy_2' - y_2}{x^2} =C \cdot \frac{2 - x}{x^3}
\]
Тогда делим на $y_1^2 = x^2 $ и получаем:
\[
\left(
\frac{y_2}{x}
\right)'_x = C \cdot \frac{2 - x}{x^3}
\]
\[
\frac{y_2}{x} = C \cdot \int \frac{2-x}{x^3} = C \cdot \frac{x-1}{x^2} + D = \frac{C}{x} - \frac{C}{x^2} + D
\]
\[
y_2 = C - \frac{C}{x} + D \cdot x
\]
Назовем $C = C_1$ и $D = C_2$, тогда:
\[
y_2 = C_1(x) - \frac{C_1(x)}{x } +  C_2(x) \cdot x
\]
\[
y_2 = C_1(x) \left(
1 - \frac{1}{x} 
\right) + C_2(x) \cdot x
\]
Теперь решим неоднородное методом вариации постоянных:
\[
\begin{cases}
C_1' \cdot \left(
1 - \frac{1}{x}
\right)+ C_2' x = 0  \\
C_1' \cdot \frac{1}{x^2} + C_2' = (2 - x)e^{-x}
\end{cases}
\]
Выражаем производные в системе:
\[
\begin{cases} 
C_1' = x^2 e^{-x} \\
C_2' = (1 - x)e^{-x}
\end{cases}
\]
А теперь находим сами С-шки:
\[
C_1 = \int x^2 e^{-x} dx =  - x^2 e^{-x} + 2 \int e^{-x} xdx = -x^2e^{-x} - 2e^{-x} x + 2 \int e^{-x} dx = 
\]
\[
=
-e^{-x} \left(
x^2 + 2x + 2
\right) + D_1
\]
\[
C_2 = \int (1 - x) e^{-x} dx = \int (e^{-x} - x e^{-x}) dx =  \int e^{-x} dx - \int x e^{-x} dx =
\] 
\[
=
-e^{-x}  + e^{-x} x + e^{-x} + D_2 = 
e^{-x} x + D_2
\]
Отсюда получаем решение:
\[
y = \left(
-e^{-x} (x^2 + 2x +2) + D_1 
\right) \cdot \left(
 1- \frac{1}{x}
\right)
+ (e^{-x}x + D_2) x = 
\]
\[
=
e^{-x} x^2 + D_2 \cdot x +\left(
-e^{-x} (x^2 + 2x +2) + D_1 
\right) - \frac{\left(
-e^{-x} (x^2 + 2x +2) + D_1 
\right)}{x} = 
\]
\[
=
e^{-x} x^2 + D_2 \cdot x - e^{-x}(x^2 + 2x + 2) + D_1 + \frac{e^{-x}(x^2 + 2x + 2)}{x} - \frac{D_1}{x}
\]
\begin{center}
\textbf{Ответ: } 
\[
y = e^{-x} x^2 + D_2 \cdot x - e^{-x}(x^2 + 2x + 2) + \frac{e^{-x}(x^2 + 2x + 2)}{x} + D_1 \cdot \left(
1 - \frac{1}{x}
\right)
\]
\end{center}
\clearpage
\subsection*{b)}
\[
xy'' - (4x + 2)y' + (4x + 4)y = x^2 e^{2x}
\]
Поделим на $x$, заведомо посмотрев, что $x = 0$ не является решением, получаем:
\[
y'' - \frac{4x + 2}{x} y' + \frac{4x + 4}{x} y = x e^{2x}
\]
Сначала решаем однородное, для этого замечаем, что $y_1 = e^{2x}$ является частным решением и после используем формулу  аналогично предыдущему пункту:
\[
e^{2x} y'  - 2e^{2x} y = C \cdot e^{\int \frac{4x  + 2}{x} dx}
\]
\[
e^{2x} y_2'  - 2e^{2x} y_2  = C \cdot e^{4x + 2 \ln x}
\]
\[
e^{2x} y_2'  - 2e^{2x} y_2   = C \cdot e^{4x} \cdot x^2
\]
Аналогично делим и получаем:
\[
\frac{e^{2x} y_2'  - 2e^{2x} y_2 }{e^{4x}} = C \cdot x^2
\]
\[
\left(
\frac{y_2}{y_1}
\right) ' = C x^2
\]
\[
\frac{y_2}{y_1} = C \cdot \int x^2 dx  = C \cdot \frac{x^3}{3} + D 
\]
\[
y_2 = C \frac{x_3}{3} \cdot y_1  = C \cdot \frac{x^3}{3} \cdot e^{2x} + D \cdot e^{2x} 
\]
Итого получаем (закинув тройку в константу):
\[
y_2 =  C_1(x) \cdot  x^3 \cdot e^{2x} + C_2 \cdot e^{2x}
\]
Теперь применяем метод вариации переменных:
\[
\begin{cases}
C_1' \cdot x^3 \cdot e^{2x} + C_2' \cdot e^{2x} = 0 \\
C_1' \cdot (3x^2 e^{2x} + x^32 e^{2x})  + C_2' \cdot 2e^{2x} = xe^{2x}
\end{cases}
\]
Отсюда:
\[
\begin{cases}
C_1' = \frac{1}{3x} \\
C_2' = -\frac{x^2}{3}
\end{cases}
\]
Получаем:
\[
C_1 = \frac{\ln x}{3} + D_1 
\]
\[
C_2 = - \frac{x^3}{9} + D_2
\]
Отсюда ответ:
\[
y = 
\left(
 \frac{\ln x}{3} + D_1 
\right) x^3 \cdot e^{2x} + 
\left(
- \frac{x^3}{9} + D_2
\right) \cdot e^{2x} = \frac{\ln x \cdot x^3 \cdot e^{2x}}{3} + x^3 \cdot e^{2x} \cdot D_1  - \frac{x^3 \cdot e^{2x}}{9} + e^{2x} \cdot D_2
\]
\begin{center}
\textbf{Ответ: } 
\[
y =  \frac{\ln x \cdot x^3 \cdot e^{2x}}{3} + x^3 \cdot e^{2x} \cdot D_1  - \frac{x^3 \cdot e^{2x}}{9} + e^{2x} \cdot D_2
\]
\end{center}
\clearpage
\section*{Номер 3}
\[
a(x) y'' + b(x)y' + c(x) y = f(x), \; f(x) = 1
\]
Из условия фундаментальные решения:
\[
y_1 = x, \; y_2 = x^2 - 1
\]
Сразу выразим производные для определителя:
\[
y_1' = 1, \; y_1'' = 0 
\]
\[
y_2' = 2x, \; y_2' = 2
\]
Тогда:
\[
\begin{vmatrix}
x & x^2 - 1 & y \\
1 & 2x & y' \\
0 & 2 & y''
\end{vmatrix} = 0
\]
Cчитаем:
\[
x^2 y'' - 2xy' + y'' + 2y = 0
\]
\[
y''(x^2 + 1) - 2xy' + 2y = 0
\]
Ну все, вид уравнения нашли, теперь вспоминаем про $f(x) = 1$ и ищем общее решение:
\[
y''(x^2 + 1) - 2xy' + 2y = 1 
\]
Ищем частное решение в виде:
\[
y_3 = Ax^2 + Bx + C 
\]
\[
y_3' = A \cdot 2x + B 
\]
\[
y_3'' = 2A
\]
Тогда подставляем:
\[
2A(x^2 + 1) - 2x(2x A + B) + 2(Ax^2 + Bx + C) = 1
\]
\[
2Ax^2 + 2A - 4x^2A - 2xB + 2Ax^2 + 2Bx + 2C = 1
\]
Отсюда:
\[
x^2(2A - 4A + 2A) + x(-2B + 2B) + 2A + 2C = 1
\]
\[
\begin{cases}
2A - 4A + 2A = 0 \\
-2B + 2B = 0 \\
2A + 2C = 1
\end{cases}
\]
Тогда положим самый простой из вариантов:
\[
\begin{cases}
A = 0 \\
B = 1 \\
C = \frac{1}{2}
\end{cases}
\]
И получаем:
\[
y_3 =  x + \frac{1}{2}
\]
Тогда общее решение:
\[
y = C_1 \cdot x + C_2 \cdot (x^2 - 1) + x + \frac{1}{2}
\]
\begin{center}
\textbf{Ответ: } 
\\
вид уравнения:
\[
y''(x^2 + 1) - 2xy' + 2y = 1 
\]
общее решение:
\[
y = C_1 \cdot x + C_2 \cdot (x^2 - 1) + x + \frac{1}{2}
\]
\end{center}
\clearpage
\section*{Номер 4}
Знаем:
\[
y'' + p(x) y = 0, \; x \geq 0, \; p(x) \text{ -- непрерывна }
\]
К тому же есть два решения со свойствами:
\[
y_1(x) \overset{x \rightarrow + \infty }{\longrightarrow} 0 
\]
\[
y_2(x) \overset{x \rightarrow + \infty }{\longrightarrow} 0 
\]
\[
y_1'(x), y_2'(x) \text{  ограничены на } x \geq 0 
\]
Тогда смотрим, по Луивиллю-Остроградскому:
\[
\begin{vmatrix}
y_1 & y_2 \\
y_1' & y_2'
\end{vmatrix} = y_1 y_2' - y_1' y_2 = C \cdot e^{0} = C, \; C \text{ -- const }
\]
Из вида уравнения получили, что справа просто стоит константа, нужно ее оценить, мы можем это сделать из условия об ограниченности производных:
\[
|y_1 y_2' - y_1' y_2| \leq \left(
|y_1(x)| + |y_2(x)| 
\right) M 
\]
Но (опять же из условия) обе функции стремятся к нулю, поэтому если взять достаточно большие $x$, $|y_1(x)| + |y_2(x)|$ станет очень и очень малой величиной. Но при этом $C$ -- константа и мы ее ограничили чем-то, что очень близко к нулю, значит $C = 0$, тобишь определитель равен нулю, из чего по определению $y_1(x)$ и $y_2(x)$ являются зависимыми
\begin{center}
\textbf{Ч.Т.Д} 
\end{center}
\clearpage
\section*{Номер 5}
\subsection*{a)}
\[
y'' + \frac{1}{1 + x^2} y = 0, \; x \geq 0 
\]
Доказываем, нужно по теореме Штурма искать такое:
\[
y'' + q(x) \cdot  y = 0
\]
Что:
\[
q(x) \leq \frac{1}{1 + x^2}
\] 
Совсем нетрудно заметить, что мы можем оценить как (взяв $x \geq 1$):
\[
\frac{1}{2x^2} \leq \frac{1}{1 + x^2}
\]
Тогда получаем понятное нам уравнение:
\[
y'' + \frac{1}{2x^2} y = 0
\]
Домножаем:
\[
2x^2y'' + y = 0
\]
А такое уравнение мы решать умеем, это уравнение Эйлера, решаем:
\[
2 \lambda ( \lambda -  1) + 1 = 0 
\]
\[
2\lambda^2 - 2 \lambda + 1  = 0
\]
Отсюда корни:
\[
\lambda_{1, 2} = \frac{1 \pm i }{2 }
\]
Значит:
\[
y = e^{\frac{t}{2}} \left(
C_1 \cos \frac{t}{2} + C_2 \sin \frac{t}{2}
\right)
\]
Возвращаемся от стандартной замены:
\[
y = \sqrt{x} \left(
D_1 \cos \frac{\ln |x| }{2} + D_2 \sin \frac{\ln |x| }{2}
\right)
\]
Видим, что здесь бесконечно много корней (периодичные функции), ну тогда по теореме Штурма и для нашего исходного уравнения тоже бесконечно много корней
\begin{center}
\textbf{Ч.Т.Д} 
\end{center}
\clearpage
\subsection*{b)}
\[
y'' - xy' + y = 0, \; (-\infty, + \infty) 
\]
Пусть:
\[
y(x) = a(x) \cdot z(x) 
\]
Найдем производные:
\[
y' = a' z + az' 
\]
\[
y'' = a'' z + az'' + 2a'z'
\]
\[
a'' z + az'' + 2a'z' -  xa'z  - xaz + az = 0
\]
Отсюда:
\[ 
2a' - xa = 0
\]
\[
2(a)'_x = xa
\]
\[
2 \int \frac{a'}{a} = \int x dx 
\]
\[
2 \ln |a| = \frac{x^2}{2}  + C
\]
\[
\ln |a| = \frac{x^2}{4} + C
\]
\[
a = C \cdot e^{\frac{x^2}{4}}
\]
Пусть $C = 1$, тогда:
\[
y(x) = e^{\frac{x^2}{4}} \cdot z(x)
\]
Теперь будем подставлять, посчитаем:
\[
a = e^{\frac{x^2}{4}}
\]
\[
a' = \frac{x}{2} \cdot e^{\frac{x^2}{4}}
\]
\[
a'' = \left(\frac{1}{2} + \frac{x^2}{4} \right) e^{\frac{x^2}{4}}
\]
\[
e^{\frac{x^2}{4}}  \cdot z'' + 
\left(
\frac12 + 
\frac{x^2}{4}
- \frac{x^2}{2}
+1
\right)
\cdot 
z \cdot 
e^{\frac{x^2}{4}}  = 0
\]
Тогда уравнение:
\[
z'' +  \left(
\frac{6 - x^2}{4 }
\right) z= 0
\]
Хотим как-то применять Штурма, Заметим, что на $[-\sqrt{6}, \sqrt{6}]$ дробь $\frac{6 - x^2}{4}$ положительна. Тогда оценим тут как:
\[
\frac{6 - x^2}{4} \leq \frac{3}{2},  x \in [-\sqrt{6}, \sqrt{6}]
\]
Тогда рассматриваем более простое уравнение:
\[
z'' + \frac32 z = 0
\]
И решаем его:
\[
\lambda^2 + \frac{3}{2} = 0 
\]
\[
\lambda_{1, 2} = \pm \sqrt{\frac{3}{2}}i
\]
Отсюда:
\[
z = C_1 \cos\left( \sqrt{\frac{3}{2}} x  \right)  + C_2 \sin \left( \sqrt{\frac{3}{2}} x  \right)
\]
Замечаем, что корней здесь не более двух в силу периодичности функций $\sin, \cos$ (в наш отрезок тупо не влезет больше двух нулей).  Еще сами точки $-\sqrt{6}, \sqrt{6}$ могут иметь нули, тогда набирается четыре корня, теперь осталось проверить что происходит вне этого промежутка (должен быть 1 корень для условия задачи), там дробь отрицательна, значит можем оценить:
\[
\frac{6 - x^2}{4} \leq 0
\]
Тогда по теореме Штурма можем смотреть на уравнение:
\[
z'' + 0 z = 0
\]
У него корни имеют вид:
\[
y = Ax + B
\]
Значит их тут не более одного. Итого суммируем и получаем, что на $(-\infty, +\infty)$ у нашего уравнения не более 5 корней, что от нас и требовалось получить.
\begin{center}
\textbf{Ч.Т.Д} 
\end{center}
\end{document}
