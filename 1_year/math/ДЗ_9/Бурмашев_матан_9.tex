\documentclass[a4paper,12pt]{article}

%%% Работа с русским языком
\usepackage{cmap}					% поиск в PDF
\usepackage{mathtext} 				% русские буквы в формулах
\usepackage[T2A]{fontenc}			% кодировка
\usepackage[utf8]{inputenc}			% кодировка исходного текста
\usepackage[english,russian]{babel}	% локализация и переносы
\usepackage{xcolor}
\usepackage{hyperref}
 % Цвета для гиперссылок
\definecolor{linkcolor}{HTML}{00FFFF} % цвет ссылок
\definecolor{urlcolor}{HTML}{4682B4} % цвет гиперссылок

\hypersetup{pdfstartview=FitH,  linkcolor=linkcolor,urlcolor=urlcolor, colorlinks=true}

%%% Дополнительная работа с математикой
\usepackage{amsfonts,amssymb,amsthm,mathtools} % AMS
\usepackage{amsmath}
\usepackage{icomma} % "Умная" запятая: $0,2$ --- число, $0, 2$ --- перечисление

%% Номера формул
%\mathtoolsset{showonlyrefs=true} % Показывать номера только у тех формул, на которые есть \eqref{} в тексте.

%% Шрифты
\usepackage{euscript}	 % Шрифт Евклид
\usepackage{mathrsfs} % Красивый матшрифт

%% Свои команды
\DeclareMathOperator{\sgn}{\mathop{sgn}}

\usepackage{enumerate}
%% Перенос знаков в формулах (по Львовскому)
\newcommand*{\hm}[1]{#1\nobreak\discretionary{}
{\hbox{$\mathsurround=0pt #1$}}{}}
% графика
\usepackage{graphicx}
\graphicspath{{picture/}}
\DeclareGraphicsExtensions{.pdf,.png,.jpg}
\author{Бурмашев Григорий, 208. \href{https://teleg.run/burmashev}{@burmashev}}
\title{Дискретная математика. Коллок -- 1. Определения и задачи по ним.}
\date{}
\begin{document}
\begin{center}
Бурмашев Григорий.  208. Матан. Д/з -- 9
\end{center}
\begin{center}
\includegraphics[scale=0.8]{Ye-Yz56_hB4.jpg}
\end{center}
\section*{Номер 8}
Какие из следующих утверждений справедливы при $x \rightarrow 0$?
\subsection*{a)}
\[
\overline{o}(x^2) + \overline{o}(x)= \overline{o}(x)
\]
Пусть:
\[
\overline{o} (x^2) = \frac{F(x)}{x^2}
\]
\[
\overline{o} (x) = \frac{G(x)}{x}
\]
\[
 \frac{F(x) + G(x)}{x} = \frac{F(x) \cdot x}{x^2} \rightarrow 0  \equiv \frac{F(x)}{x^2} + \frac{G(x)}{x}= \overline{o}(x)
\]
\begin{center}
\textbf{Ответ:} верно
\end{center}
\subsection*{b)}
\[
\overline{o}(x) + x^2 = \overline{o}(x)
\]
\[
\frac{x^2}{x} \rightarrow 0 \equiv x^2 = \overline{o}(x)
\]
А значит:
\[
\overline{o}(x) + x^2  = \overline{o}(x) + \overline{o}(x) = \overline{o}(x)
\]
\begin{center}
\textbf{Ответ:} верно
\end{center}
\subsection*{с)}
\[
(x + \overline{o}(x))(2x^2 + \overline{o}(x^2)) = 2x^3 + \overline{o}(x^3)
\]
\[
2x^3 + \overline{o}(x) \cdot \overline{o}(x^2) + 2x^2 \cdot  \overline{o}(x)  +  x \cdot \overline{o}(x^2)  = 2x^3 + \overline{o}(x^3)
\]
Упростим каждый из множителей:
\begin{itemize}
\item 
\[
\overline{o}(x) \cdot \overline{o}(x^2) = x \cdot \overline{o}(1) \cdot x^2 \cdot \overline{o}(1) = x^3 \cdot  \overline{o}(1) 
\]
\[
\frac{x^3 \cdot \overline{o}(1)}{x^3} \rightarrow 0 \equiv \overline{o}(x) \cdot \overline{o}(x^2) = \overline{o}(x^3)
\]
\item
\[
2x^2 \cdot \overline{o}(x) = 2x^2 \cdot x \cdot \overline{o}(1) = 2x^3 \cdot \overline{o}(1)
\]
\[
\frac{2x^3 \cdot \overline{o}(1)}{x^3} \rightarrow 0 \equiv 2x^2 \cdot \overline{o}(x) = \overline{o}(x^3)
\]
\item
\[
x \cdot \overline{o}(x^2) = x \cdot x \cdot \overline{o}(1) = x^2 \overline{o}(1)
\]
\[
\frac{x^3 \cdot \overline{o}(1)}{x^3} \rightarrow 0 \equiv x \cdot \overline{o}(x^2) = \overline{o}(x^3)
\]
А значит:
\[
2x^3  + \overline{o}(x^3) + \overline{o}(x^3) + \overline{o}(x^3) = 2x^3 + \overline{o}(x^3)
\]
\begin{center}
\textbf{Ответ:} верно
\end{center}
\end{itemize}
\subsection*{d)}
\[
\overline{o}(1) - \overline{o}(1) = 0
\]
Пусть левый множитель:
\[
\overline{o}(1) = F(x)
\]
А правый:
\[
\overline{o}(1) = G(x)
\]
Тогда:
\[
\frac{F(x) - G(x)}{1} \rightarrow 0 \equiv \overline{o}(1) - \overline{o}(1) = \overline{o}(1)
\]
А значит, исходное выражение \textbf{неверное}
\begin{center}
\textbf{Ответ:} неверно
\end{center}
\end{document}
