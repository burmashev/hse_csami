\documentclass[a4paper,12pt]{article}

%%% Работа с русским языком
\usepackage{cmap}					% поиск в PDF
\usepackage{mathtext} 				% русские буквы в формулах
\usepackage[T2A]{fontenc}			% кодировка
\usepackage[utf8]{inputenc}			% кодировка исходного текста
\usepackage[english,russian]{babel}	% локализация и переносы
\usepackage{xcolor}
\usepackage{hyperref}
 % Цвета для гиперссылок
\definecolor{linkcolor}{HTML}{799B03} % цвет ссылок
\definecolor{urlcolor}{HTML}{799B03} % цвет гиперссылок

\hypersetup{pdfstartview=FitH,  linkcolor=linkcolor,urlcolor=urlcolor, colorlinks=true}

%%% Дополнительная работа с математикой
\usepackage{amsfonts,amssymb,amsthm,mathtools} % AMS
\usepackage{amsmath}
\usepackage{icomma} % "Умная" запятая: $0,2$ --- число, $0, 2$ --- перечисление

%% Номера формул
%\mathtoolsset{showonlyrefs=true} % Показывать номера только у тех формул, на которые есть \eqref{} в тексте.

%% Шрифты
\usepackage{euscript}	 % Шрифт Евклид
\usepackage{mathrsfs} % Красивый матшрифт

%% Свои команды
\DeclareMathOperator{\sgn}{\mathop{sgn}}

%% Перенос знаков в формулах (по Львовскому)
\newcommand*{\hm}[1]{#1\nobreak\discretionary{}
{\hbox{$\mathsurround=0pt #1$}}{}}
% графика
\usepackage{graphicx}
\graphicspath{{pictures/}}
\DeclareGraphicsExtensions{.pdf,.png,.jpg}
\author{Бурмашев Григорий, БПМИ-208}
\title{}
\date{\today}
\begin{document}
\begin{center}
Бурмашев Григорий. 208. Дискра -- 19
\end{center}
\section*{Номер 1}
Найдем верхнюю и нижнюю оценку:
\begin{itemize}
\item Верхняя : 

$\leq n$. Задаем вопрос про каждую переменную. Тогда, зная значения всех переменных, мы можем определить и значение функции.

\item Нижняя:

$\geq n$. Каждая переменная существенная. Пусть у нас есть алгоритм, который выдает значение функции за менее чем n. Тогда пусть например мы спросили про значения n - 1 переменной и они все были нулями. Алгоритм выдаст нам ответ 0. Но последняя n -- я переменная может быть равна 1, тогда верный ответ на самом деле будет 1, но наш алгоритм будет выдавать неправильный ответ. Значит нам все таки нужно узнать значения всех переменных. 
\begin{center}
\textbf{Ответ: } Сложность вычисления дизъюнкции равна n
\end{center}
\end{itemize}
\section*{Номер 3}
У нас 7 элементов, чтобы узнать значение MAJ, нам нужно знать значение хотя бы 4 элементов (чтобы было больше половины). Значит все элементы в многочлене, где менее 4 элементов мы откидываем. Посмотрим разные случаи при разном количестве единичек в MAJ
\begin{itemize}
\item Если 4 единички:

Один из $C_7^4$ конъюнктов из 4х элементов будет равен единице (тот, в котором нашлась нужная комбинация из вот этих элементов, равных единичке). Ну 1 -- нечетное, значит эти конъюнкты мы должны точно оставить, иначе у нас не получится узнать ответ.

\item Если 5 единичек:

Один из $C_7^5$ конъюнктов из 5ти элементов будет равен единице (аналогично предыдущему случаю), но помимо этого еще $C_5^4$ конъюнктов из 4х элементов тоже будут равны единичке (те, в которых набрались любые 4 из 5 элементов, равных единичке). Но тогда $C_5^4$ + 1 -- четное число, а многочлен Жигалкина по сути есть сложение по модулю два, значит наш алгоритм будет работать неверно, если мы сохраним эти множители. Следовательно мы должны их выкинуть

\item Если 6 единичек:

Аналогично предыдущему пункту, один из $C_7^6$ конъюнктов из 6ти элементов будет равен единице, и еще $C_6^4$ из четырех элементов будут равны единице, но $C_6^4$ + 1 -- четное, $\rightarrow$ опять выкидываем

\item Если 7 единичек:

Всего 1 конъюнкт из 7 элементов равен единичке ($C_7^7 = 1$), но еще $C_7^4$ конъюнктов из 4х элементов принимают истину $\rightarrow$ опять выкидываем.

По итогу мы видим, что нам нужно оставить только конъюнкты из 4х элементов, всего их $C_7^4 = 35$ штук.

\begin{center}
\textbf{Ответ: } $35$
\end{center}
\end{itemize}
\section*{Номер 4}
Логично, что есть смысл рассматривать четные и нечетные значения n. Пусть n -- четное. Если у нас поровну нулей и единичек, то MAJ у нас будет равна нулю и очевидно,  что $\overline{0} \neq 0$ (если мы переворачиваем значения нашиъ переменных), и самодвойственности мы не получаем. Если же у нас нечетное n, тогда у нас будет какое-то наиболее часто встречающееся число. Пусть это 1. Тогда, если мы реверсим наши переменные, то наиболее частым будет 0. Но $\overline{0} = 1$. Т.е мы получили самодвойственность (аналогично для случая, если наиболее часто встречающееся число -- 0)

\begin{center}
\textbf{Ответ: } при нечетных n
\end{center} 

\section*{Номер 5}
От противного: пусть есть такая $f(x, y)$, что $f(x, y) = \overline{f} (\overline{x}, \overline{y})$ и она еще и зависит как от x, так и от y. Положим тогда две переменные $a$ и $b$, принимающие соотвественно 0 или 1. Пускай, например, $f(1, 1) = a$, ну тогда по предположению $f(0, 0) = \overline{a}$. И пусть $f(1, 0) = b$. Тогда, аналогично, $f(0, 1) = \overline{b}$. x существенна, тогда $a \neq \overline{b}$. Но тогда $a = b$. А также y существенна, но тогда $\overline{a} \neq \overline{b}$, а значит $a = b$. Мы получили противоречие, значит такой функции не существует
\begin{center}
\textbf{Ч.Т.Д} 
\end{center}
\section*{Номер 6}
Оценка сверху:

Наша функция равна $y_x$. Положим тогда в ДНФ $2^k$ конъюнктов. У нас $y_0, y_1, \ldots y_{2^k - 1}$ элементов. Их всего $2^k$. Соотвественно каждому числу y соотвествует свой конъюнкт. 
\\\\
Оценка снизу:

Пусть нам хватит $2^k - 1$ конъюнктов. Приведем контрпример, когда все сломается. Пусть k = 3. Тогда всего у нас 7 конъюнктов. "y" внутри функции у нас тогда 8 штук ($y_0, y_1, y_2 \ldots y_7$). Тогда функция будет иметь вид:
\[
\text{Ind} (x_1, x_2, x_3, y_0, y_1 \ldots y_7)
\]
Заметим, что т.к у нас всего 7 конъюнктов, мы можем составить всего 7 индексов, а игриков-то у нас 8. Значит для одного игрика нам не хватит конъюнкта и их должно быть 8. Значит все же нужно $2^k$.
\begin{center}
\textbf{Ч.Т.Д } 
\end{center}
\end{document}
