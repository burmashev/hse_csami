\documentclass[a4paper,12pt]{article}

%%% Работа с русским языком
\usepackage{cmap}					% поиск в PDF
\usepackage{mathtext} 				% русские буквы в формулах
\usepackage[T2A]{fontenc}			% кодировка
\usepackage[utf8]{inputenc}			% кодировка исходного текста
\usepackage[english,russian]{babel}	% локализация и переносы
\usepackage{xcolor}
\usepackage{hyperref}
 % Цвета для гиперссылок
\definecolor{linkcolor}{HTML}{799B03} % цвет ссылок
\definecolor{urlcolor}{HTML}{799B03} % цвет гиперссылок

\hypersetup{pdfstartview=FitH,  linkcolor=linkcolor,urlcolor=urlcolor, colorlinks=true}

%%% Дополнительная работа с математикой
\usepackage{amsfonts,amssymb,amsthm,mathtools} % AMS
\usepackage{amsmath}
\usepackage{icomma} % "Умная" запятая: $0,2$ --- число, $0, 2$ --- перечисление

%% Номера формул
%\mathtoolsset{showonlyrefs=true} % Показывать номера только у тех формул, на которые есть \eqref{} в тексте.

%% Шрифты
\usepackage{euscript}	 % Шрифт Евклид
\usepackage{mathrsfs} % Красивый матшрифт

%% Свои команды
\DeclareMathOperator{\sgn}{\mathop{sgn}}

%% Перенос знаков в формулах (по Львовскому)
\newcommand*{\hm}[1]{#1\nobreak\discretionary{}
{\hbox{$\mathsurround=0pt #1$}}{}}
% графика
\usepackage{graphicx}
\graphicspath{{pictures/}}
\DeclareGraphicsExtensions{.pdf,.png,.jpg}
\author{Бурмашев Григорий}
\title{Матанализ  -3}
\date{\today}
\begin{document}
\begin{center}
Бурмашев Григорий.  208. Матан. Д/з - 3 
\begin{center}
\includegraphics[scale=0.3]{cat.jpg}
\end{center}
\end{center}
\newpage
\section*{№ 10}
Исследуйте следующие рекуррентные последовательности на сходимость
\subsection*{a)}
\[
a_{n+1} = \sqrt{2a_n}, \; a_1 = \sqrt{2}
\]
\begin{itemize}
\item Докажем,  что $a_n \leq 2$:
\\
Воспользуемся методом математической индукции:
\\
База:  $n = 1 $
\[
\sqrt{2} \leq 2
\]
\begin{center}
Верно
\end{center}
Переход: пусть верно для  $n$, докажем, что это верно для $n+1$:
\[
a_{n+1} \leq 2
\]
\[
\sqrt{2a_n} \leq 2
\]
\[
2a_n \leq 4
\]
\[
a_n \leq 2
\]
\begin{center}
Верно
\end{center}
\item Докажем, что $a_{n+1} \geq a_n $
\\
\[
a_{n+1} \geq a_n
\]
\[
\sqrt{2a_n} \geq a_n
\]
\[
2a_n \geq a_n^2
\]
\[
a^2_n - 2a_n \leq 0
\]
\[
a_n(a_n-2) \leq 0
\]
Т.к мы доказали, что $a_n \leq 2$, то это верно 
\\
\item Т.к последовательность не убывает и ограничена сверху, то по теореме Вейерштрасса:
\[
\exists \lim_{n \rightarrow \infty} a_n = a
\]
Тогда:
\[
a = \sqrt{2a}
\]
\[
a^2 = 2a
\]
\[
a(a-2) = 0
\]
Т.к $a_1 > 0$ и последовательнось не убывает, то:
\[
\lim_{n \rightarrow \infty} a_n = 2
\]
\textbf{Ответ:} $\lim_{n \rightarrow \infty} a_n = 2$
\end{itemize}
\subsection*{b)}
\[
a_{n+1} = \sqrt{6 + a_n}, \; a_1 = 0
\]
\begin{itemize}
\item Докажем, что $a_n \leq 3$:
\\
Воспользуемся методом математической индукции:
\\
База: n = 1 
\[
0 \leq 3
\]
\begin{center}
Верно
\end{center}
Переход: пусть верно для $n$, докажем, что верно для $n+1$:
\[
a_{n+1} \leq 3
\]
\[
\sqrt{6 + a_n} \leq 3
\]
\[
6 + a_n \leq 9 
\]
\[
a_n \leq 3
\]
\begin{center}
Верно
\end{center}
\item Докажем, что $a_{n+1} \geq a_n $:
\[
a_{n+1} \geq a_n
\]
\[
\sqrt{6+a_n} \geq a_n
\]
\[
6 + a_n \geq a_n^2
\]
\[
a_n^2 - a_n -6 \leq 0
\]
\[
(a - 3)(a +2) \leq 0
\]
Т.к мы доказали, что $ a_n \leq 3$,  то это верно
\\
\item Т.к последовательность не убывает и ограничена сверху, значит:
\[
\exists \lim_{n \rightarrow \infty } a_n = a
\]
Тогда:
\[
a = \sqrt{6 + a}
\]
\[
a^2 = 6 + a
\]
\[
a^2 - a - 6 = 0
\]
\[
(a-3)(a+2) = 0
\]
Значит $(a = 3) \vee (a = -2)$. Но $a_1 \geq 0$ и последовательность не убывает, тогда:
\[
\lim_{n \rightarrow \infty} a_n = 3
\]
\textbf{Ответ:} $\lim_{n \rightarrow \infty} a_n = 3$
\end{itemize}
\subsection*{c)}
\[
a_{n+1} = \frac{1}{3} \times (2a_n + \frac{3}{a_n^2}), \; a_1 = 3
\]
\begin{itemize}
\item Докажем, что $a_n \geq \sqrt[3]{3}$:\\
Воспользуемся методом математической индукции:
\\
База: n = 1
\\
\[
3 \geq \sqrt[3]{3}
\]
\[
3^3 \geq 3
\]
\[
27 \geq 3
\]
\begin{center}
Верно
\end{center}
Переход: пусть верно для  $n$,  докажем, что это верно и для $n+1$:
\[
a_{n+1} \geq \sqrt[3]{3}
\]
\[
\frac{1}{3} \times (2a_n + \frac{3}{a_n^2}) \geq \sqrt[3]{3}
\]
\[
\frac{a_n + a_n + \frac{3}{a_n^2} }{3} \geq \sqrt[3]{3}
\]
Видно, что слева от знака $ \geq $ находится среднее арифметическое, при этом мы знаем, что среднее арифметическое больше, чем среднее геометрическое, тогда:
\[
\frac{a_n + a_n + \frac{3}{a_n^2} }{3} \geq \sqrt[3]{a_n \times a_n \times \frac{3}{a_n^2}} \geq \sqrt[3]{3}
\]
\[
\sqrt[3]{\frac{3a_n^2}{3}} \geq \sqrt[3]{3}
\]
\[
\sqrt[3]{3} \geq \sqrt[3]{3}
\]
\begin{center}
Верно
\end{center}
\item Докажем, что $a_{n+1} \leq a_n$:
\[
a_{n+1} \leq a_n
\]
\[
\frac{1}{3}(2a_n + \frac{3}{a_n^2}) \leq a_n
\]
\[
2a_n + \frac{3}{a_n^2} \leq 3a_n
\]
\[
\frac{3}{a_n^2} \leq a_n
\]
\[
3 \leq a_n \times a_n^2
\]
\[
a_n^3 \geq 3
\]
\[
a_n \geq \sqrt[3]{3}
\]
Мы это уже доказали, значит это верно
\\
\item Т.к последовательность не возрастает и ограничена снизу, значит:
\[
\exists \lim_{n \rightarrow \infty} a_n = a
\]
\[
a = \frac{1}{3} (2a + \frac{3}{a^2})
\]
\[
3a = 2a + \frac{3}{a^2}
\]
\[
a = \frac{3}{a^2}
\]
\[
a^3 = 3
\]
\[
a = \sqrt[3]{3}
\]
Значит:
\[
\lim_{n \rightarrow \infty} a_n = \sqrt[3]{3}
\]
\textbf{Ответ:} $\lim_{n \rightarrow \infty} a_n = \sqrt[3]{3}$
\section*{№ 11}
Докажите,  что:
\[
\left(\frac{n}{e}\right)^n \leq n! \leq e\times \left(\frac{n}{2}\right)^n
\]
Для начала докажем нижнюю границу:
\[
\left(\frac{n}{e}\right)^n \leq n!
\]
Воспользуемся методом математической индукции:
\\
База: n = 1
\\
\[
\frac{1}{e} \leq 1
\]
Мы знаем, что $e \geq 1$, значит неравенство выполняется
\\
Переход:
\\
Пусть верно для $ n $, докажем,  что это верно и для $n+1$:
\[
\left(\frac{n+1}{e}\right)^{n+1} \leq (n+1)!
\]
\[
\left(\frac{n+1}{e}\right)^{n+1} \leq n! \times (n+1)
\]
\[
\left(\frac{n+1}{e}\right)^{n+1} \leq \left(\frac{n}{e}\right)^n \times (n+1)
\]
\[
\left(\frac{n+1}{e}\right)^{n} \times \frac{n+1}{e} \leq \frac{n^n}{e^n} \times (n+1)
\]
Умножим на $e^n$:
\[
(n+1)^n \times \frac{n+1}{e} \leq n^n \times (n+1)
\]
Поделим на $(n+1)$:
\[
\frac{(n+1)^n}{e} \leq n^n
\]
\[
(n+1) ^ n \leq n^n \times e
\]
\[
\left(1 + \frac{1}{n}\right)^n \leq e
\]
Мы знаем про второй замечательный предел, докажем, что $x_n =  \left(1 + \frac{1}{n}\right)^n $ не убывает, посмотрим на:
\[
\frac{x_{n+1}}{x_n} = \left(\frac{(1+\frac{1}{n+1})^{n+1}}{(1+\frac{1}{n})^n}\right) = \frac{(n+2)^{n+1}\times n^n}{(n+1)^{2n+1}} = \frac{n+2}{n+1} \times \frac{(n^2+2n)^n}{(n^2+2n+1)^n} = 
\]
\[
= \frac{n+2}{n+1} \times \left(1 - \frac{1}{n^2+2n+1}\right)^n \geq \frac{n+2}{n+1} \times \left(1 - \frac{n}{n^2+2n+1}\right) = \frac{(n+2)(n^2+n+1)}{(n+1)^3} = 
\]
\[
= \frac{n^3 + 3n^2 +3n + 2}{n^3 + 3n^2 +3n + 1} > 1 
\]
\begin{center}
Т.к $\frac{x_n{n+1}}{x_n} > 1$, то $x_n$ не убывает
\end{center}
$ \lim_{n \rightarrow \infty} \left(1 + \frac{1}{n}\right)^n = e $, а также $ \left(1 + \frac{1}{n}\right)^n $ не убывает, значит:
\[
\left(1 + \frac{1}{n}\right)^n \leq e
\]
Теперь докажем верхнюю границу:
\[
n! \leq e \times \left(\frac{n}{2}\right) ^ n
\]
Воспользуемся методом математической индукции:
\\
База: n = 1
\\
\[
1 \leq e \times 0.5
\]
\[
e \geq 2 \rightarrow 0.5 \times e \geq 1
\]
\begin{center}
Верно
\end{center}
Переход: пусть верно для n, докажем,  что это верно и для  $n+1$:
\[
(n+1)! \leq e \times \left(\frac{n+1}{2}\right)^{n+1}
\]
\[
n! \times (n+1) \leq e \times \left(\frac{n+1}{2}\right)^{n+1}
\]
\[
e \times \left( \frac{n}{2}\right) ^ n \times (n+1) \leq e \times \left(\frac{n+1}{2}\right)^{n+1}
\]
\[
\left( \frac{n}{2}\right) ^ n \leq  \left(\frac{n+1}{2}\right)^{n+1}
\]
\[
n^n \times (n+1) \leq \frac{(n+1)^n \times(n+1)}{2}
\]
\[
2n^n \leq (n+1)^n
\]
\[
\left(1+\frac{1}{n}\right) ^ n \geq 2
\]
При n = 1: $\left(1+\frac{1}{n}\right) ^ n = 2$, и при этом эта функция не убывает (мы это уже доказали), значит это верно $\forall n, \; n \geq 1$\\
\textbf{Ч.Т.Д}
\\\\
Итог: мы доказали как нижнюю, так и верхнюю границу, что и требовалось от нас в задаче
\section*{№ 12}
Рассмотрим последовательности:
\[
a_n = \frac{1}{\sqrt{1}} + \frac{1}{\sqrt{2}} + \ldots + \frac{1}{\sqrt{n}} - 2\sqrt{n}
\]
\[
b_n = \frac{1}{\sqrt{1}} + \frac{1}{\sqrt{2}} + \ldots + \frac{1}{\sqrt{n}} -2 \sqrt{n+1}
\]
\subsection*{a)}
Доказать, что $a_n \geq b_n $, $a_n$ - не возрастает, $b_n$ - не убывает
\item \[
a_n \geq b_n
\]
\[
\frac{1}{\sqrt{1}} + \frac{1}{\sqrt{2}} + \ldots + \frac{1}{\sqrt{n}} - 2\sqrt{n} \geq
\]
\[
\geq \frac{1}{\sqrt{1}} + \frac{1}{\sqrt{2}} + \ldots + \frac{1}{\sqrt{n}} -2 \sqrt{n+1}
\]
\[
-2\sqrt{n} + 2\sqrt{n+1} \geq 0
\]
\[
-\sqrt{n} + \sqrt{n+1} \geq 0
\]
\[
\sqrt{n+1} \geq \sqrt{n}
\]
\[
n+1 \geq n
\]
\[
1 \geq 0
\]
\begin{center}
Верно
\end{center}
\item
\[
a_{n+1} \leq a_n
\]
\[
\frac{1}{\sqrt{1}} + \frac{1}{\sqrt{2}} + \ldots + \frac{1}{\sqrt{n}} + \frac{1}{\sqrt{n+1}}- 2\sqrt{n+1} \leq 
\]
\[
\leq \frac{1}{\sqrt{1}} + \frac{1}{\sqrt{2}} + \ldots + \frac{1}{\sqrt{n}} -2 \sqrt{n}
\]
\[
\frac{1}{\sqrt{n+1}} - 2 \sqrt{n+1} \leq -2\sqrt{n}
\]
\[
\frac{2-2n}{\sqrt{n+1}} \leq -2 \sqrt{n}
\]
\[
\frac{1-n}{\sqrt{n+1}} \leq -\sqrt{n}
\]
\[
\frac{1-2n+n^2}{\sqrt{n+1}} \leq -\sqrt{n}
\]
\[
n^2-2n+1 \leq n^2+n
\]
\[
1 \leq 3n
\]
\[
n \geq \frac{1}{3}
\]
\begin{center}
Верно
\end{center}
\item
\[
b_{n+1} \geq b_n
\]
\[
\frac{1}{\sqrt{1}} + \frac{1}{\sqrt{2}} + \ldots + \frac{1}{\sqrt{n}} + \frac{1}{\sqrt{n+1}} - 2\sqrt{n+2} \geq
\]
\[
\geq \frac{1}{\sqrt{1}} + \frac{1}{\sqrt{2}} + \ldots + \frac{1}{n} - 2\sqrt{n+1}
\]
\[
\frac{1}{\sqrt{n+1}} - 2\sqrt{n+1} \geq -2 \sqrt{n+1}
\]
\[
2(n+1) - 2\sqrt{(n+1)(n+2)} +1 \geq 0
\]
\[
2n+3 \geq 2 \sqrt{n^2 +3n + 2}
\]
\[
4n^2 + 12n + 0 \geq 4n^2 + 12n + 8
\]
\[
9 \geq 8
\]
\begin{center}
Верно
\end{center}
\subsection*{b)}
Из пункта a) получите, что $a_k \geq b_m$ для произвольных индексов $k, m \in \mathbb{N}$ и обе последовательности ограничены
\item
Возьмем произвольный k такой, что $k < n$, тогда:
\[
a_k \geq a_n \geq b_n \rightarrow a_k \geq b_n
\]
Аналогично, возьмем произвольный m такой, что $ m > n$, тогда:
\[
b_n \leq b_m \leq a_n
\]
Соединяя, получим:
\[
a_k \geq a_n \geq b_m \geq b_n 
\]
Т.е:
\[
a_k \geq b_m \; \forall k, m \in \mathbb{N}
\]
\item
Т.к $a_n \geq b_n$ и $a_n$ не возрастает, то $a_n$ \textbf{ограничена} снизу $ b_n $
\\\\
Т.к  $a_n \geq b_n$ и $b_n$ не убывает, то  $b_n$  \textbf{ограничена} сверху $a_n$
\subsection*{c)}
Докажите, что:
\[
\frac{1}{\sqrt{1}} + \frac{1}{\sqrt{2}} + \ldots + \frac{1}{\sqrt{n}} = 2 \sqrt{n} - c + \alpha_n \text{ где:}
\]
\begin{center}
c -  положительное число
\[
\lim_{n \rightarrow \infty} \alpha_n = 0
\]
Пусть:
\[
\lim_{n \rightarrow \infty } a_n = A
\]
\[
\lim_{n \rightarrow \infty} b_n = B
\]
Мы знаем, что:
\[
a_n = b_n + 2\sqrt{n+1} - 2\sqrt{n}
\]
\[
A = \lim\limits_{n \rightarrow \infty} (b_n + 2\sqrt{n+1} - 2\sqrt{n}) = B + \lim_{n \rightarrow \infty} \left( \frac{2(n+1-n)}{\sqrt{n+1} + \sqrt{n}} \right) = B
\]
\[
 \lim_{n \rightarrow \infty}  \left(\frac{2}{\sqrt{n+1} + \sqrt{n}}\right) = 0 \text{  т.к при n стремящемся к бесконечности} 
\]
\[
\text{знаменатель увеличивается, а числитель не меняется}
\]
Пускай:
\[
a_n = -1,\; \;\; b_n \leq a_n \rightarrow b_n \leq -1
\]
Если $b_n \leq -1$, то и $a_n \leq -1$
\[
a_n = A + \alpha_n \text{ ,где } \alpha_n \rightarrow 0
\]
\[
\frac{1}{\sqrt{1}} + \frac{1}{\sqrt{2}} + \ldots + \frac{1}{\sqrt{n}} = 2\sqrt{n} + A + \alpha_n
\]
Пусть $c = -A$, c > 0 (т.к $ A < 0$, $ -A >0$)
\[
\frac{1}{\sqrt{1}} + \frac{1}{\sqrt{2}} + \ldots + \frac{1}{\sqrt{n}} = 2\sqrt{n} - c + \alpha_n
\]
\\
\textbf{Ч.Т.Д}
\end{center}
\end{itemize}
\end{document}
