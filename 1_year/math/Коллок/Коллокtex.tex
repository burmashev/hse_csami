\documentclass[a4paper,12pt]{article}

%%% Работа с русским языком
\usepackage{cmap}					% поиск в PDF
\usepackage{mathtext} 				% русские буквы в формулах
\usepackage[T2A]{fontenc}			% кодировка
\usepackage[utf8]{inputenc}			% кодировка исходного текста
\usepackage[english,russian]{babel}	% локализация и переносы
\usepackage{xcolor}
\usepackage{hyperref}
 % Цвета для гиперссылок
\definecolor{linkcolor}{HTML}{799B03} % цвет ссылок
\definecolor{urlcolor}{HTML}{799B03} % цвет гиперссылок

\hypersetup{pdfstartview=FitH,  linkcolor=linkcolor,urlcolor=urlcolor, colorlinks=true}

%%% Дополнительная работа с математикой
\usepackage{amsfonts,amssymb,amsthm,mathtools} % AMS
\usepackage{amsmath}
\usepackage{icomma} % "Умная" запятая: $0,2$ --- число, $0, 2$ --- перечисление

%% Номера формул
%\mathtoolsset{showonlyrefs=true} % Показывать номера только у тех формул, на которые есть \eqref{} в тексте.

%% Шрифты
\usepackage{euscript}	 % Шрифт Евклид
\usepackage{mathrsfs} % Красивый матшрифт

%% Свои команды
\DeclareMathOperator{\sgn}{\mathop{sgn}}

\newcommand*{\hm}[1]{#1\nobreak\discretionary{}
{\hbox{$\mathsurround=0pt #1$}}{}}
% графика
\usepackage{graphicx}
\graphicspath{{pictures/}}
\DeclareGraphicsExtensions{.pdf,.png,.jpg}
\author{Бурмашев Григорий}
\title{Матан, коллок - 1 }
\date{\today}
\begin{document}
\maketitle
\newpage
\section*{Билет № 1}
\subsection*{Рациональные числа}
Рациональные числа ($\mathbb{Q}$)-- числа вида $ \frac{p}{q} $, где q - натуральное,  а p - целое. 
\\
Два рациональных числа задают одно и тоже число, если $p_1q_2 = p_2q_1 $
\subsection*{Вещественные числа}
Множество вещественных чисел $\mathbb{R}$ отождествляется с множеством всех бесконечных десятичных дробей вида $\pm a_0a_1a_2 \ldots$, где $a_0 = 0$, $a_j \in \{0, \ldots 9\}$ и записи в которых с какого-то момента стоят только девятки -- запрещены. Число $\pm 0,000\ldots$ -- совпадает с числом 0 и называется нулем. На множесте вещественных чисел определены все операции множества рациональных чисел. Для вещественных чисел определен модуль числа $|a|$, такой, что:
\begin{equation*}
\begin{cases}
|a| = a & a \geq 0\\
|a| = -a & a \leq 0 \\
\end{cases}
\end{equation*}
Важно помнить о \textbf{неравенстве треугольника}:
\[
|a + b| \leq |a| + |b|
\]
\[
| |a| - |b| | \leq |a + b |
\]
\subsection*{Принцип полноты}
\textbf{Левее/правее + разделение}:

Множество А лежит \textbf{левее} множества B, если $a \leq b\; \; \forall \; a \in A,  b \in B$.  Число <<c>> \textbf{разделяет} множества A и B, если $ a \leq c \; \; \forall \; a \in A$ и $ c \leq b \; \; \forall \; b \in B$
\\\\
Если для произвольных непустых множеств A левее B найдется разделяющий их элемент, то выполняется так называемый \textbf{принцип полноты}

На множестве вещественных чисел выполняется принцип полноты.

\textbf{Доказательство}:

Пусть есть множества A и B. Пусть A лежит левее B. Если A состоит только из неположительных чисел, а B -- только из неотрицательных, тогда разделителем является ноль. Пускай теперь в A есть неотрицательный элемент, тогда в B есть только положительные числа (т.к A левее B). Построим число -- разделитель c = $c_0c_1c_2\ldots$

Рассмотрим множество всех натуральных чисел, с которых начинаются элементы множества B. Пусть $b_0$ -- наименьшее из таких чисел и $b_0 = c_0$. Теперь среди всех чиcел в B, начинающихся с $b_0$ найдем наименьшую следующую цифру, пусть теперь она равна $b_1$ и $b_1 = c_1$. Теперь посмотрим на все числа в B, начинающиеся с $b_0b_1$ и проделаем ту же самую операцию. 

Мы получили бесконечную десятичную дробь $c_0c_1c_2\ldots$ Стоит заметить, что подряд идущих девяток в нем не будет, т.к мы запретили такие записи в B. Покажем теперь, что это число -- разделитель множеств A и B. 

Во первых, $c \leq b$ $\forall b \in B$. Либо b = c (тогда все окей), либо $b \neq c$. Во втором случае пусть $b_0 = c_0, \ldots, b_{k-1} = c_{k-1}$ и $b_k \neq c_k$. Тогда, по построению числа c $c_k < b_k \rightarrow c < b$.

Покажем теперь, что  $a \leq c$. От обратного: пусть $a > c$, т.е $a \geq c$ и $ a \neq c$. Тогда найдется позиция k, для которой $a_0 = c_0, \ldots, a_{k-1} = c_{k-1}$ и $a_k > c_k$. Но по построению числа c есть такой b, что $b_0 = c_0, \ldots b_k = c_k$ и получается, что a > b, что противоречит условию. Значит c -- действительно разделитель для двух множеств A и B.
\subsection*{Иррациональность числа $\sqrt{2}$}
Пусть $\frac{p}{q} = x^2 = 2$, тогда $p^2 = 2q^2$ и $p$ -- четное, т.е его можно представить как  $p = 2p_1$, откуда $2p^2_1 = q^2$, а значит и q -- четное. Но тогда $\frac{p}{q}$ не является конечным решением и противоречит нашему предположению об отсутствии общих делителей. Мы знаем, что на множестве вещественных чисел выполняется принцип полноты (теорема 4 из лекции 1). А значит если взять два множества А и B, такие, что:
\[
A = \{a : a > 0, a^2 \leq 2\}
\]
\[
B = \{ b: b > 0, b^2 \geq 2\}
\]
Если их элементы принадлежат множеству вещественных чисел, то, согласно принципу полноты, найдется элемент c, который будет их разделять, причем такой c, что $c^2 = 2$

















\newpage
\section*{Билет № 2}
\subsection*{Предел последовательности}
Предел последовательности a:
\[
\lim_{n \rightarrow \infty} a_n = a
\]
Последовательность $a_n$ сходится к числу a, если:
\[
\forall \varepsilon > 0 \; \exists N(\varepsilon) \in \mathbb{N}: \forall n > N(\varepsilon) \; |a_n - a| < \varepsilon
\]
\subsection*{Единственность предела}
Пусть $
\lim\limits_{n \rightarrow \infty} a_n = a $
,\; $ \lim\limits_{n \rightarrow \infty} a_n = b$,\; $ a = b$
\\\\

\textbf{Доказательство}:

От обратного: пусть $a \neq b$, тогда $|a - b| = \varepsilon_0 > 0$. 
Но по определению предела найдется номер $N_1$, что $|a_n - a| < \frac{\varepsilon_0}{2} $ при $n > N_1$ и найдется номер 
$N_2$, что $|a_n - b| < \frac{\varepsilon_0}{2} $ при $ n > N_2$

Тогда при n > $ {max} \{ N_1, N_2\} $:
\[
\varepsilon_0 = | a - b| = |a - a_n + a_n - b| \leq |a - a_n| + |a_n - b| < \varepsilon_0
\]
\begin{center}
\textbf{Противоречие}
\end{center}
\subsection*{Арифметические свойства}
Пусть
$
\lim\limits_{n \rightarrow \infty} a_n = a
$, \;
$
\lim\limits_{n \rightarrow \infty} b_n = b
$.
Тогда:
\[
\lim_{n \rightarrow \infty} (a_n + b_n) = a + b
\]
\[
\lim_{n \rightarrow \infty} a_nb_n = ab 
\]
\[
\lim_{n \rightarrow \infty} \frac{a_n}{b_n} = \frac{a}{b}
\]

\subsection*{Ограниченность сходящейся последовательности}
Последовательность $a_n$ называется ограниченной, если существуют такие числа C, c $\in \mathbb{R}$, что $ c \leq a_n \leq C$ для каждого $n \in \mathbb{N}$
\\\\
\textbf{Сходящаяся последовательность ограничена}

\textbf{Доказательство по Шапошникову}:

Начиная с какого-то номера N все элементы последовательности попадают в интервал $(a-  \alpha, a + \beta )$, где a -- предел последовательности.  Возьмем элементы $a_1, a_2, \ldots a_N$. Возьмем самый минимальный (пусть A) из них и самый максимальный (пусть B). Нам нужно взять интервал вида $[min(\alpha, A), max(\beta, B)]$ и тогда мы точно сможем захватить все элементы последовательности в интервал. 

\textbf{Доказательство по Косову}:

Для $N \in \mathbb{N}: $
$|a_n - a | < 1 $ при n > N

Тогда:
\[
|a_n| = |a_n - a + a| \leq |a_n - a| + |a| < 1 + |a|
\]

Значит:
\[
|a_n| \leq M = max \{1 + |a|, |a_1|, \ldots, |a_N| \}
\]

Т.е:
\[
-M = c \leq a_n \leq C = M
\]
\subsection*{Отделимость}
Если $ \lim\limits_{n \rightarrow \infty} a_n = a$ и $ a > 0 $, то найдется номер N $\in \mathbb{N}$, для которого:
\[
a_n> \frac{a}{2} > 0
\]

\textbf{Доказательство по Шапошникову:}

Начиная с некоторого  N, все элементы начнут попадать в интервал вида $\left(\frac{a}{2},\frac{3a}{2}\right)$.  И в частности они оказываются больше, чем $\frac{a}{2}$.
\begin{center}\textbf{Ч.Т.Д}\end{center}




\textbf{Доказательство по Косову:}

Взяв $ \varepsilon = \frac{|a|}{2} $ мы получим номер N $\in \mathbb{N}$, для которого $
|a_n - a| < \frac{|a|}{2}
$
при n > N. Тогда, при n > N, выполнено $|a| - |a_n| \leq |a_n - a| < \frac{|a|}{2}$, что равносильно доказываемому утверждению
\newpage
\section*{Билет № 3}
\subsection*{Переход к пределу в неравенствах}
Если $a_n \leq b_n$ при n > N для некоторого N, то $a \leq b$
\\

\textbf{Доказательство по Шапошникову:}

Пусть b < a. Тогда по рисунку:



\textbf{Доказательство по Косову:}


Пусть $a - b = \varepsilon_0 > 0$. Тогда найдутся номера $N_1, N_2$ такие, что $|a_n - a| < \frac{\varepsilon_0}{2}$ при n > $N_1$ и $|b_n - b| < \frac{\varepsilon_0}{2}$ при n > $N_2$.
\\
Тогда:
\[
\varepsilon_0 = a - b = a - a_n  + a_n - b_n + b_n - b \leq  - a_n + b_n - b < \varepsilon_0
\]
\begin{center}
\textbf{Противоречие}
\end{center}
\subsection*{Лемма о зажатой последовательности}
Пусть:
\[
\lim_{n \rightarrow \infty} a_n = \lim_{n \rightarrow \infty} b_n = a
\]
\[
a_n \leq c_n \leq b_n 
\]
Тогда:
\[
\lim_{n \rightarrow \infty} c_n = a
\]
\\

\textbf{Доказательство по Шапошникову:}

По рисунку:

\textbf{Доказательство по Косову:}

Для каждого $\varepsilon > 0$ найдутся номера  $N_1 \in \mathbb{N},N_2 \in \mathbb{N}$, для которых $|a_n - a| < \varepsilon$ и $|b_n -  a| < \varepsilon$. Тогда при $n > max \{N, N_1, N_2\}$ выполнено:
\[
a - \varepsilon < a_n \leq c_n \leq b_n < b + \varepsilon
\]
\subsection*{Принцип вложенных отрезков}
Всякая последовательность вложенных отрезков имеет общую точку. Кроме того, если длины отрезков стремятся к нулю, то такая общая точка только одна.
\\

\textbf{Доказательство:}

Пусть A -- множество всех возможных начал отрезков, а B -- множество всех возможных концов отрезков. Тогда:
\[
\forall \; n, m \in \mathbb{N} :\; [a_{n+m};b_{n+m}] \subset [a_n; b_n] \rightarrow a_n \leq a_{n+m}
\]
\[
\forall \; n, m \in \mathbb{N} :\; [a_{n+m};b_{n+m}] \subset [a_n; b_n] \rightarrow b_{n+m} \leq b_m
\]
А значит $
\forall \; n, m \in \mathbb{N}: a_n \leq a_{n+m} \leq b_{n+m} \leq b_m
$.
Т.е $
a_n \leq b_m $
Тогда по принципу полноты найдется такое с, которое будет разделять эти два множества, т.е $a_n \leq c \leq b_m$,  в частности $a_n \leq c \leq b_n$, т.е $ c \in [a_n, b_n] $ для любых n $\in \mathbb{N}$

Пускай общих точек две: $c$ и $c'$ и при этом $c < c'$. Тогда $a_n \leq c < c' \leq b_n$ и $ c' - c \leq b_n - a_n$, что противоречит тому, что  $\lim_{n \rightarrow \infty}(b_n - a_n) = 0 $. Найдется номер $N \in \mathbb{N}$, для которого $b_n - a_n < c'- c$ при каждом n > N
\subsection*{Геометрическая интерпретация $\mathbb{R}$}
Сопоставим десятичной дроби $0a_1a_2a_3\ldots$ последовательность вложенных отрезков по следующему правилу. Разделим отрезок [0, 1] на 10 равных частей и выберем из получившихся 10-ти отрезков $a_1 + 1$й по счету.  Теперь проделываем ту же самую операцию и берем $a_2 + 1$й по счету и так далее... Получаем последовательность вложенных отрезков, причем длина отрезка на n-ом шаге равна $\frac{1}{10^n}$ По уже доказанной выше теореме существует единственная общая точка построенной последовательности вложенных отрезков, причем только одна, совпадающая с нашим исходным числом

\newpage
\section*{Билет № 4}
\subsection*{Точные верхние и нижние грани}
Число b называется \textbf{верхней гранью} множества А, если $a \leq b$ для каждого числа $a \in A$ Если есть хотя бы одна верхняя грань, то множество называется \textbf{ограниченным сверху}. Наименьшая из верхних граней множества A называется \textbf{точной верхней гранью} множества A и обозначается как sup A (супремум).

Аналогично для \textbf{нижней грани}, только $b \leq a$ и 
называется это inf A (инфимум).
\\
Ограниченное и сверху и снизу множество называется \textbf{ограниченным}.
\\\\
Пусть A - непустое ограниченное сверху (снизу) множество. Тогда существует точная верхняя (нижняя) грань sup A (inf A)

\textbf{Доказательство}:
\\
Пусть А -- непустое ограниченное сверху множество из условия. B -- непустое (по условию) множество его верхних граней.  Тогда A лежит левее B и существует разделяющий их элемент c. Он является верхней гранью для A и c $\leq b$ для каждой верхней грани множества А.  По определению c = sup A

Наличие inf A доказывается аналогично или переходом к множеству -A

\subsection*{Теорема Вейерштрасса}
Пускай последовательность $\{a_n\}_{n=1}^{\infty}$ не убывает ($a_n \leq a_{n+1}$) и ограничена сверху. Тогда эта последовательность сходится к своему супремуму.

Аналогично, пусть последовательность $\{a_n\}_{n=1}^{\infty}$ не возрастает ($a_n \geq a_{n+1}$) и ограничена снизу. Тогда эта последовательность сходится к своему инфимуму.

\textbf{Доказательство}:

Пусть M = sup $\{a_n : n \in \mathbb{N} \}$ = sup $a_n$. Тогда для каждого $\varepsilon > 0$ найдется номер $N \in \mathbb{N} $, для которого M - $\varepsilon < a_N$ (иначе $M - \varepsilon$ -- верхняя грань, чего не может быть). В силу того, что последовательность неубывающая, при каждом n > N выполнено:
\[
M - \varepsilon < a_N \leq a_n \leq M < M + \varepsilon
\]
Тем самым, по определению M = lim $a_n$
\subsection*{Пример рекуррентной формулы для вычисления $\sqrt{2}$}
Пусть:
\[
a_{n+1} = \frac{1}{2}\left(a_n + \frac{2}{a_n}\right), \; \; a_1 = 2
\]
Можно заметить, что:
\[
a_{n+1} = \frac{1}{2}\left(a_n + \frac{2}{a_n}\right) \geq \frac{1}{2} \cdot 2\sqrt{a_n \cdot \frac{2}{a_n}} = \sqrt{2}
\]
А значит $a_n \geq \sqrt{2}$. Кроме того:
\[
a_{n+1} =  \frac{1}{2}\left(a_n + \frac{2}{a_n}\right) \leq \frac{1}{2}\left(a_n + \frac{a_n^2}{a_n}\right) = a_n
\]
\[
a_{n+1} \leq a_n
\]
По Вейерштрассу у этой последовательности существует предел a. Т.к $a_n \geq 0$, то и $a \geq 0$. По арифметике пределов получаем:
\[
a = \frac{1}{2} \left(a + \frac{2}{a}\right) = \sqrt{2}
\]
\subsection*{Оценка скорости сходимости}
Хз

\newpage
\section*{Билет № 5}
\subsection*{Фундаментальная последовательность}
Последовательность $a_n$ фундаментальна (или удовл. условию Коши) если:
\[
\forall \; \varepsilon > 0 \; \exists  N : \forall n, m > N \; |a_n - a_m| < \varepsilon
\]
Если последовательность $a_n$ сходится, то  $a_n$  -- фундаментальна.

\textbf{Доказательство:}

Пусть предел равен а. Это значит, что:
\[
\forall \; \varepsilon > 0 \; \exists N\; \forall\; n > N \;|a_n - a| < \varepsilon
\]
Пусть $m > N$ и $n > N$. Тогда:
\[
|a_m - a_n| \leq |a_m - a| + |a_n - a| \leq 2\varepsilon
\]
\\\\
Если последовательность фундаментальна, то она ограничена

\textbf{Доказательство:}

Пусть $\varepsilon = 1$. Тогда:
\[
\exists \; N : \forall \; n, m > N \;  |a_n - a_m| < 1
\]

СМ листок
\\\\
Если $a_n$ фундаментальна + ограничена (из пунктяяяа выше),  то она сходится

СМ листок
\\\\

\subsection*{Критерий Коши}
Последовательность $a_n$ сходится к конечному пределу тогда и только тогда, когда она фундаментальна.

Если сходится $\rightarrow$ фундаментальна.

Если фундаментальна $\rightarrow$ сходится

\subsection*{Цепная дробь $\sqrt{2}$}
Пускай:
\[
a_{n+1} = 1 + \frac{1}{1+a_n}, \; a_1 = 1
\]
Заметим, что $a_n \geq 1$ и:
\[
|a_{n+1} - a_n| = \left| \frac{1}{1+a_n} +  \frac{1}{1+a_{n-1}}  \right| = \frac{|a_n - a_{n-1}|}{(1+a_n)(1+a_{n-1})} \leq \frac{1}{4}|a_n-a_{n-1}| \leq \left(\frac{1}{4}\right)^{n-1}|a_2-a_1| =\]
\[
= \left(\frac{1}{4}\right)^{n-1}\frac{1}{2}
\]
Отсюда при m > n:
\[
|a_m - a_n| \leq |a_m - a_{m-1} |+ \ldots + |a_{n+1} - a_n |  \leq \frac{1}{2}
\cdot \left(\left( \left(\frac{1}{4}\right)^{m-2} + \ldots + \left(\frac{1}{4}\right)^{n-1}    \right)\right) = 
\]
\[
= \frac{1}{2} \cdot \left(\frac{1}{4}\right)^{n-1} \cdot 
\frac{1 - (\frac{1}{4})^{m-n}}{1-\frac{1}{4}} \leq
\frac{8}{3} (\frac{1}{4})^n
\]
Т.к $(\frac{1}{4})^n \rightarrow 0$, то для любого $\varepsilon > 0 \; \; \exists N:$ $\frac{1}{4}^n < \varepsilon$

Таким образом, для последовательности выполнен критерий Коши, а значит существует $\lim\limits_{n\rightarrow \infty} a_n = a$ По арифметике предела число a  удовлетворяет уравнению:
\[
a(1+a) = 1 + a + 1 \leftrightarrow a^2 = a \leftrightarrow a = \sqrt{2}
\]

\section*{Билет № 6}
\subsection*{Числовые ряды}
\textbf{Числовым рядом} с членами $a_n$ называется выражение:
\[
a_1 + a_2 + a_3 + \ldots = \sum_{k=1}^{\infty}a_k
\]
Конечные суммы $S_n :=\sum\limits_{k=1}^{\infty}a_k$ называют \textbf{частичными суммами} ряда
$\sum\limits_{k=1}^{\infty}a_k$
\\\\
Число A называют суммой ряда, если предел $S_n = A$

Пишут:
\[
A = \sum_{n=1}^{\infty} a_n
\]
Если предел $S_n$ конечен, то говорят, что \textbf{ряд сходится}. Если предел бесконечен или не существует, то говорят, что \textbf{ряд расходится}.
\subsection*{Переформулировка критерия Коши}
Ряд $\sum\limits_{k=1}^{\infty}a_k$ сходится тогда и только тогда, когда для каждого $\varepsilon> 0 $ найдется такой номер N, что для всех n > m > N выполнено:
\[
\left| \sum_{k=m+1}^{n} a_k\right| = |S_n - S_m| < \varepsilon
\]
\subsection*{Необходимое условие сходимости ряда}
Если ряд сходится, то его слагаемые стремятся к нулю. \textbf{Обратное -- неверно}

\textbf{Доказательство:}

По условию сходимости существует $\lim\limits_{n\rightarrow\infty}S_n =  A$.
$S_n$ стремится к А. $S_{n-1}$ тоже стремится к A. Тогда:
\[
a_n = S_n - S_{n-1} \rightarrow A - A  = 0
\]
\subsection*{Расходимость ряда $\frac{1}{n}$}
Заметим, что:
\[
\frac{1}{n+1} + \frac{1}{n+2} + \ldots + \frac{1}{2n} \geq n \cdot \frac{1}{2n} = \frac{1}{2}
\]
Возьмем и сгруппируем $S_{2^m} $:
\[
(\frac{1}{1} + \frac{1}{2} + \frac{1}{3} + \frac{1}{4}) + (\frac{1}{5} + \frac{1}{6} + \frac{1}{7} + \frac{1}{8}) + \ldots \ldots (\frac{1}{2^{m-1}+1} + \ldots + \frac{1}{2^m})
\]
Каждая $S_{2^m} \geq \frac{1}{2}$. И всего таких группировок у нас m штук. Значит $S_{2^m} \geq \frac{m}{2}$. Помимо этого, каждая следующая $S_n$ больше предыдущей (ибо мы к предыдущей сумме прибавляем какой-то положительный член) $\rightarrow$ ряд расходится.
\subsection*{Условная и абсолютная сходимость}
Говорят, что ряд сходится \textbf{абсолютно}, если сходится ряд $\sum\limits_{k=1}^{\infty}|a_k|$.
\\
Говорят, что ряд сходится \textbf{условно}, если $\sum\limits_{k=1}^{\infty}|a_k|$ расходится, а ряд $\sum\limits_{k=1}^{\infty}a_k$ сходится
\\\\
Из сходимости ряда $\sum\limits_{k=1}^{\infty}|a_k|$ следует сходимость ряда $\sum\limits_{k=1}^{\infty}a_k$
\newpage
\section*{Билет № 7}
\subsection*{Сходимость рядов с неотрицательными слагаемыми}
Если $a_n \geq 0$, то ряд сходится тогда и только тогда, когда последовательность его частичных сумм ограничена. 

\textbf{Доказательство:}
Последовательность частичных сумм не убывает, т.к $a_n \geq 0$. Каждая следующая сумма будет больше предыдущей, и если последовательность частичных сумм не будет ограничена, то она просто уйдет в бесконечность и сходится не будет. 
\subsection*{Признак сравнения}
Пусть $ 0 \leq a_n \leq b_n$, тогда:

Если ряд из $b_n$ сходится, то и ряд из $a_n$ сходится. 

Если ряд из $a_n$ расходится, то и $b_n$ расходится.

\textbf{Доказательство:}
\[
S_n^a = a_1 + \ldots + a_n \leq b_1 + \ldots + b_n = S_n^b
\]
Если последовательность частичных сумм $S_n^b$ ограничена,  то и $S_n^a$ ограничена. А если $S_n^a$ неограниченны, то и $S_n^b$ неограниченны.
\subsection*{Признак Коши}
Пусть $a_n \geq 0$, $a_n$ сходится тогда, когда сходится $2^n a_{2^n}$

\textbf{Доказательство:}

Нужно как-то связать ограниченность частичных сумм.
Пусть $2^m \leq n < 2^{m+1}$
\[
a_2 + 2a_4 + 4a_8 + \ldots + 2^{m-1}a_{2^m}
\geq 
a_1 + a_2 + a_3 + \ldots a_n \leq a_1 + 2a_2 + 4a_4 + 8a_8 + \ldots + 2^{m}\cdot a_{2^m}
\]
\subsection*{Сходимость и расходимость $\frac{1}{n^p}$}
По доказанному выше он сходится, если сходится:
\[
2^n \cdot \frac{1}{(2^{n})^p} = \left( \frac{1}{2^{p-1}} \right)^n
\]
А это геометрическая прогрессия, она сходится тогда, когда то, что внутри меньше единицы, а именно при $p > 1$

Т.е наш ряд сходится тогда и только тогда, когда  $p > 1 $ Иначе он расходится.
\newpage
\section*{Билет № 8}
\subsection*{Подпоследовательность}

Пусть задана последовательность $a_n$ и последовательность возрастающих номеров $n_1 < n_2 < n_3 \ldots n_k < n_{k+1}$, то последовательность $a_{n_k} $ называется подпоследовательностью последовательности $a_n$.

Если последовательность сходится к a, то всякая подпоследовательность тоже сходится к a. 

1. $n_k \geq k$

Докажем по индукции:

База:

k = 1 $n_1 \geq 1$. Верно, т.к $n_1$ -- натурально.

Шаг:

Пусть верно для k, докажем, что верно для $k +1$:

$n_{k+1} > n_k \geq k \rightarrow n_{k+1} \geq n_k + 1 \geq k + 1$

2. По определению предела:
\[
\forall \varepsilon > 0 \exists N : \forall n > N |a_n - a| < \varepsilon
\]
Пусть k > N. Тогда $n_k \geq k > N$. А значит неравенство выполняется и для $n_k$, т.е:
\[
|a_{n_k} - a | < \varepsilon
\]
А значит lim $a_n$ = lim $a_{n_k}$
\subsection*{Теорема Больцано}
Если последовательность $a_n$ ограничена, то в ней есть сходящаяся подпоследовательность.

\textbf{Доказательство:}
Все элементы лежат внутри отрезка. Давайте поделим его пополам. Тогда очевидно, что хотя бы в одной половине бесконечно много элементов. Делим эту половину еще раз пополам и так далее. Мы получили последовательность вложенных отрезков, в каждом из них бесконечно много элементов. В системе вложенных отрезков есть общая точка c. Возьмем ее. А также в каждом из наших отрезков выберем элементы по порядку $a_{n_1}, a_{n_2}, a_{n_3}$ и так далее. Причем каждый из этих элементов отличается от c на длину отрезка. А если длина первого отрезка была l, то:
\[
|a_{n_k} - c | \leq \frac{l}{2^{k-1}}
\]
Следовательно все элементы приближаются к  с и c -- предел.
\subsection*{Частичные пределы}
Предел подпоследовательности называется \textbf{частичным пределом}. Задача: описать множество частичных пределов.
\\\\
1) $M_n \geq M_{n+1}$

\textbf{Доказательство}:

Т.к  $M_n$ -- верхняя грань для \{$a_{n+1}, a_{n+2}, \ldots$\}. То она и верхняя грань для \{$a_{n+2}, a_{n+3}, \ldots$\}. Следовательно $M_{n+1} \leq M_n$
\\\\
Т.к $a_n$ ограниченна, то и $M_n$ ограниченна.

По теореме Вейерштрасса $M_n$ невозрастает и ограничен и существует предел $M_n = M$
\\\\
2)
Для inf -- аналогично.
\\\\
3) Докажем, что M -- частичный предел. Т.е нужно предьявить такую $a_{n_k}$,  что $a_{n_k} \rightarrow M$

$n_1$: $M_1 - a_{n_1} < 1$
($M_1 - 1$ -- не верхняя грань для $a_2, a_3\ldots \rightarrow \exists a_{n_1}$ из них: $M_1 \geq a_{n_1} > M_1 - 1$

$n_2$: $0 \leq M_{n_1} - a_{n_2} < \frac{1}{2}$

Если уже построена $n_k$, то $n_{k+1} > n_k$ и:
\[
n_{k+1} : 0 \geq M_{n_k} - a_{n_{k+1}} < \frac{1}{k+1}
\] 
Мы получили подпоследовательность $a_{n_k}$:
\[
M_{n_{k-1}} - \frac{1}{k} \leq a_{n_{k}} \leq M_{n_k}
\]
И левая, и правая часть сходится к M, а значит $a_{n_k}$ сходится к М по теореме о зажатой последовательности и M -- частичный предел.
\\\\
Докажем, что любой частичный предел лежит между [m, M]. Т.е если произвольная подпоследовательность $a_{n_k} \rightarrow a$, то $a \in [m,M]$
\[
m_{n_{k-1}} \leq a_{n_k} \leq M_{n_{k-1}}
\]
\[
m \leq a \leq M
\]
\subsection*{Критерий сходимости в последовательности в терминах структуры множеств частичных пределов}
Пусть $a_n$ -- ограниченная последовательность. $a_n$ сходится тогда и только тогда, когда M = m = a, т.е верхний предел совпадает с нижним пределом. Верхний и нижний предел равны пределу последовательности.

\textbf{Доказательство:}

В прямую сторону:
Если последовательность сходится, то и ее подпоследовательность сходится к тому же самому.

В обратную сторону:
\[
\inf_{k>n-1} a_k\leq a_n \leq \sup_{k>n-1} a_k
\]
\[
m \leq a \leq M
\]

\end{document}
