\documentclass[a4paper,12pt]{article}

%%% Работа с русским языком
\usepackage{cmap}					% поиск в PDF
\usepackage{mathtext} 				% русские буквы в формулах
\usepackage[T2A]{fontenc}			% кодировка
\usepackage[utf8]{inputenc}			% кодировка исходного текста
\usepackage[english,russian]{babel}	% локализация и переносы
\usepackage{xcolor}
\usepackage{hyperref}
 % Цвета для гиперссылок
\definecolor{linkcolor}{HTML}{00FFFF} % цвет ссылок
\definecolor{urlcolor}{HTML}{4682B4} % цвет гиперссылок

\hypersetup{pdfstartview=FitH,  linkcolor=linkcolor,urlcolor=urlcolor, colorlinks=true}

%%% Дополнительная работа с математикой
\usepackage{amsfonts,amssymb,amsthm,mathtools} % AMS
\usepackage{amsmath}
\usepackage{icomma} % "Умная" запятая: $0,2$ --- число, $0, 2$ --- перечисление

%% Номера формул
%\mathtoolsset{showonlyrefs=true} % Показывать номера только у тех формул, на которые есть \eqref{} в тексте.

%% Шрифты
\usepackage{euscript}	 % Шрифт Евклид
\usepackage{mathrsfs} % Красивый матшрифт

%% Свои команды
\DeclareMathOperator{\sgn}{\mathop{sgn}}

\usepackage{enumerate}
%% Перенос знаков в формулах (по Львовскому)
\newcommand*{\hm}[1]{#1\nobreak\discretionary{}
{\hbox{$\mathsurround=0pt #1$}}{}}
% графика
\usepackage{graphicx}
\graphicspath{{picture/}}
\DeclareGraphicsExtensions{.pdf,.png,.jpg}
\author{Бурмаzшев Григорий, 208. \href{https://teleg.run/burmashev}{@burmashev}}
\title{Дискретная математика. Коллок -- 1. Определения и задачи по ним.}

\begin{document}
\begin{center}
Бурмашев Григорий. Дискра -- 12
\end{center}
\subsection*{Номер 1}
Две последние цифры -- это остаток при делении на 100. 99 сравнима с -1 по 100, т.к $99 - (-1) = 100$ делится на 100. Тогда:
\[
99^{1000} \stackrel{100}{\equiv}(-1)^{1000}  = 1
\]
Значит $99^{1000}$ оканчивается на 01.
\begin{center}
\textbf{Ответ:} 01
\end{center}
\subsection*{Номер 2}
Мы знаем, что $a^2 - b^2$ делится на $a - b$, т.к $a^2 - b^2 = (a-b)(a+b)$.  Отсюда следует, что числа $a^2$ и $b^2$ дают одинаковые остатки при делении на $ a - b $. (поскольку мы знаем, что $x$ сравимо с $y$ по модулю $N$ тогда и только тогда, когда $x - y$ делится на $N$)
\newpage  
\subsection*{Номер 3}
\textbf{(1)}

Предположим, что $a+b$ делится на $c$.


Число $b$ можно представить как $b = (a + b) - a$. При этом по условию $a$ делится на $c$, $b$ \textbf{не} делится на $c$. По предположению $ a + b$ делится на $c$.  Слева от знака равенства у нас число не делится на $c$, а справа -- разность чисел, делящихся на $c$. Мы получили противоречие $\rightarrow$ $a+ b$ \textbf{не} делится на $c$.
\begin{center}
\textbf{Ответ:} да
\end{center}
\textbf{(2)}

Можно привести контрпример:
\[
a = 4, b = 5, c = 3
\]
4 не делится на 3, 5 не делится на 3, при этом $4 + 5 = 9$ делится на 3.
\begin{center}
\textbf{Ответ:} нет
\end{center}
\textbf{(3)}

Можно привести контрпример:
\[
a = 2, b = 3, c = 6
\]
2 не делится на 6, 3 не делится на 6, при этом $2 \cdot 3 = 6$ делится на 6.
\begin{center}
\textbf{Ответ:} нет
\end{center}
\textbf{(4)}

Поскольку $a$ делится на $b$ и $b$ делится на $c$, то и $a$ делится на $c$. Ну а отсюда следует, что $a \cdot b$ делится на $c^2$.
\begin{center}
\textbf{Ответ:} да
\end{center}
\subsection*{Номер 4}
\begin{itemize}
\item Когда $x+10y$ делится на 13:

Домножим на 4:
\[
4x + 40y = (4x + y) + 39y 
\]
$39y$ делится на 13, а значит и $(4x+y)$ делится на 13.

\item Когда $y+4x$ делится на 13:

Домножим на 4:
\[
10y + 40x = 39x + (x + 10y)
\]
$39x$ делится на 13, а значит и $x + 10y$ делится на 13. 
\end{itemize}
\newpage
\subsection*{Номер 5}
\[
53x \equiv 1 (\text{mod} \; 42)
\]
\[
53x = 42 \cdot q  + 1 
\]
\[
53x - 42 \cdot q = 1
\]
Cделаем замену, чтобы привести к удобному виду:
\[
y = - q
\]
\[
53x + 42 y = 1
\]
По расширенному алгоритму Евклида посмотрим:
\[
\text{НОД}(53, 42) = \text{НОД} (11, 42) = \text{НОД}(11, 9) = \text{НОД} (2, 9) = \text{НОД} (2, 1) = 1 
\]
Найдем x:
\[
1 = 2 \cdot 0 + 1 \cdot 1 = 2 \cdot 0 + (9 - 2 \cdot 4) \cdot 1 = 9 \cdot 1 - 2 \cdot 4 = 9 \cdot 1 - (11 - 9) \cdot 4 = 
\]
\[
= 5 \cdot 9 - 11 \cdot 4 = 5 \cdot (42 - 11 \cdot 3) - 11 \cdot 4 = 5 \cdot 42 - 19 \cdot 11 = 5 \cdot 42 - 19 \cdot (53 - 42) = 24 \cdot 42 - 19 \cdot 53 
\]
Т.е:
\[
x = -19
\]
\[
y = 24 
\]
\begin{center}
\textbf{Ответ:} $x = -19$
\end{center}
\subsection*{Номер 6}
Чтобы доказать несократимость дроби, нужно показать, что НОД равен 1, тогда:
\[
\text{НОД} (n^2 - n + 1, n^2 + 1) = \text{НОД}(n^2 - n + 1, n) = \text{НОД}(n^2 + 1, n)
\]
Поскольку у нас n -- положительное целое, то мы можем из $n^2 + 1$ вычесть $n$ $k$ раз, где $k$ будет равно $n$, т.е:
\[
\text{НОД}(n^2 + 1,n) = \text{НОД} (1, n) = 1
\]

\subsection*{Номер 7}
Пусть наше число это x. Оно состоит из 100 нулей, 100 единиц и 100 двоек. Его сумма цифр равна 300 ($0 \cdot 100 + 1 \cdot 100 + 2 \cdot 100 = 300$). А значит x делится на 3 (по признаку делимости на 3) и не делится на 9 (по признаку делимости на 9). Предположим, что $x$ -- это точный квадрат какого-то числа $y$. Тогда $x = y^2$. При этом $x$ делится на 3, запишем его в виде $x = 3 \cdot q$, где $q$ -- какое-то целое число. Тогда $y^2 = 3 \cdot q$.  Отсюда следует, что $y$ делится на 3. Т.е $y = 3 \cdot r$, где $r$ -- какое-то целое число. Тогда $x = y^2 = (3 \cdot r)^2 = 9 \cdot r^2$, т.е $x$ делится на 9, что является противоречием $\rightarrow$ $x$ \textbf{не} является точным квадратом
\begin{center}
\textbf{Ответ:} нет, не может
\end{center}
\subsection*{Номер 8}
Число N такое, что сумма цифр чисел  N и N+1 делится на 7. 

Если наше число N не заканчивается на 9, то в N + 1 сумма цифр просто увеличится на 1 и мы не сможем получить искомый случай (будет 2 суммы цифр, которые различаются на 1). Значит N оканчивается на какое-то количество девяток (пусть их k). Когда мы рассматриваем число N+1, девятка превращается в ноль (сумма цифр уменьшается на 9) и предыдущая цифра увеличивается на 1 (cумма цифр увеличивается на 1). Если у нас k девяток, то сумма цифр уменьшится соотвественно на $k \cdot 9 - 1$. Пусть сумма цифр была x и делилась на 7 (чтобы подходило условию задачи). Тогда $x - (k \cdot 9 - 1)$ тоже делится на 7, т.е $k \cdot 9 - 1$ делится на 7. Найдем наименьшее k, когда это возможно:
\[
k = 1; 9 - 1 = 8 
\]
\[
k = 2;18 - 1 = 17 
\]
\[
k = 3; 27 - 1 = 26
\]
\[ 
k = 4; 36 - 1 = 35 
\]
35 делится на 7 $\rightarrow$ при k = 4 возможен такой случай.

Чтобы число было наименьшим, перед 4мя девятками должна стоять одна цифра (пусть a).
Число имеет вид $\overline{a9999}$. Найдем наименьшее a, при котором сумма цифр будет делится на 7. Т.е $36 + a$ должно делится на 7. Это возможно при a = 6. ($36 + 6 = 42$ делится на 7). Тогда искомое число -- $69999$
\begin{center}
\textbf{Ответ:} $69999$
\end{center}
\end{document}
