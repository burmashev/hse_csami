\documentclass[a4paper,12pt]{article}

%%% Работа с русским языком
\usepackage{cmap}					% поиск в PDF
\usepackage{mathtext} 				% русские буквы в формулах
\usepackage[T2A]{fontenc}			% кодировка
\usepackage[utf8]{inputenc}			% кодировка исходного текста
\usepackage[english,russian]{babel}	% локализация и переносы
\usepackage{xcolor}
\usepackage{hyperref}
 % Цвета для гиперссылок
\definecolor{linkcolor}{HTML}{799B03} % цвет ссылок
\definecolor{urlcolor}{HTML}{799B03} % цвет гиперссылок

\hypersetup{pdfstartview=FitH,  linkcolor=linkcolor,urlcolor=urlcolor, colorlinks=true}

%%% Дополнительная работа с математикой
\usepackage{amsfonts,amssymb,amsthm,mathtools} % AMS
\usepackage{amsmath}
\usepackage{icomma} % "Умная" запятая: $0,2$ --- число, $0, 2$ --- перечисление

%% Номера формул
%\mathtoolsset{showonlyrefs=true} % Показывать номера только у тех формул, на которые есть \eqref{} в тексте.

%% Шрифты
\usepackage{euscript}	 % Шрифт Евклид
\usepackage{mathrsfs} % Красивый матшрифт

%% Свои команды
\DeclareMathOperator{\sgn}{\mathop{sgn}}

%% Перенос знаков в формулах (по Львовскому)
\newcommand*{\hm}[1]{#1\nobreak\discretionary{}
{\hbox{$\mathsurround=0pt #1$}}{}}
% графика
\usepackage{graphicx}
\graphicspath{{pictures/}}
\DeclareGraphicsExtensions{.pdf,.png,.jpg}
\author{Бурмашев Григорий, БПМИ-208}
\title{}
\date{\today}
\begin{document}
\begin{center}
Бурмашев Григорий. 208. Матан -- 6 
\end{center}
\begin{center}
\includegraphics[scale=1]{2.png}
\end{center}
\begin{center}
\includegraphics[scale=0.4]{1.jpg}
\end{center}
\clearpage
\section*{Номер 6}
\section*{1)}
\[
\int \frac{dx}{x(x+1)(x+2)}
\]
$\text{deg } Q = 3$, а значит:
\[
 \int \left( \frac{a}{x} + \frac{b}{x+1} + \frac{c}{x+2} \right) \, dx
\]
Найдем коэффы:
\[
(x+1)(x+2)a + x(x+2)b + x(x+1)c = 1
\]
Воспользуемся методом постановки x:
\[ 
x = 0 : 2a +0b + 0c = 1; a = \frac{1}{2}
\]
\[
x = -1: 0a + (-b) + 0c = 1; b = -1
\]
\[
x = -2: 0a + 0b + 2c =1 ; c = \frac{1}{2}
\]
Тогда:
\[
\int \left( \frac{\frac{1}{2}}{x} + \frac{-1}{x+1} + \frac{\frac{1}{2}}{x+2} \right) \, dx = \int \frac{1}{2x} \,dx - \int \frac{1}{1+x} \, dx + \int \frac{1}{2(x+2)} \, dx = 
\]
\[
= \frac{\ln|x|}{2} - \ln |x+1| + \frac{\ln |x+2|}{2} + C
\]
{\Large \begin{center}
\textbf{Ответ: } 
\[
\frac{\ln|x|}{2} - \ln |x+1| + \frac{\ln |x+2|}{2} + C
\]
\end{center}}
\subsection*{2)}
\[
\int \frac{x^2 + 5x + 4}{x^4 + 5x^2 + 4} \, dx = \int \frac{x^2 + 5x + 4}{(x^2+4)(x^2+1)} \, dx
\]
$\text{deg Q} = 4 < \text{deq P}$, а значит:
\[
\int \left( \frac{ax + b}{x^2+4} + \frac{cx + d}{x^2 + 1}  \right) \, dx
\]
Найдем коэфы:
\[
(x^2+4)(cx+d)  + (x^2+1)(ax+b) = x^2 + 5x + 4
\]
\[
(a+c)x^3 + (b+d)x^2 + (a+4c)x + (b+4d) = x^2 + 5x + 4
\]
А значит:
\[
\begin{cases}
a + c = 0 \\ b + d = 1 \\ a + 4c = 5 \\ b + 4d = 4\\
\end{cases}
\]
\[
\begin{cases}
a = - \frac{5}{3} \\b =  0 \\c =  \frac{5}{3} \\ d = 1
\end{cases}
\]
Тогда интеграл имеет вид:
\[
\int \frac{-\frac{5}{3} x +0 }{x^2+4} + \frac{\frac{5}{3}x + 1}{x^2+1} \, dx = 
\]
\[
=
\int \frac{-\frac{5}{3} x +0 }{x^2+4} \, dx + \int \frac{\frac{5}{3}x + 1}{x^2+1} \, dx  
\]
Посчитаем отдельно:
\begin{itemize}
\item
\[
\int \frac{-\frac{5}{3} x +0 }{x^2+4} \, dx = -\frac{5}{3} \int \frac{x}{x^2+4} \, dx = -\frac{5}{3} \cdot \frac{1}{2} \ln (x^2+4)  + C\] 
\item
\[
 \int \frac{\frac{5}{3}x + 1}{x^2+1} = \frac{1}{3} \int \left( \frac{5x}{x^2+1} + \frac{3}{x^2+1} \right) \,dx  = \frac{1}{3} \cdot \frac{5}{2} \ln(x^2 + 1) + \frac{1}{3} \cdot 3 \;\text{arctg} (x) + C
\]
\end{itemize}
И по итогу получаем:
\[
=-\frac{5}{6} \ln(x^2+4) + \frac{5}{6} \ln(x^2 +1) + \text{arctg} (x) + C = 
\frac{5}{6}\left[\ln(x^2 + 1) - \ln(x^2 + 4) \right] + \text{arctg} (x) + C
\]
\begin{center}
\textbf{Ответ: } 
\[
\frac{5}{6}\left[\ln(x^2 + 1) - \ln(x^2 + 4) \right] + \text{arctg} (x) + C
\]
\end{center}
\end{document}
