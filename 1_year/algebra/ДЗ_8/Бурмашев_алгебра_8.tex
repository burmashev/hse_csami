\documentclass[a4paper,12pt]{article}
\usepackage[left=2cm,right=2cm,
    top=4cm,bottom=4cm,bindingoffset=0cm]{geometry}
%%% Работа с русским языком
\usepackage{cmap}					% поиск в PDF
\usepackage{mathtext} 				% русские буквы в формулах
\usepackage[T2A]{fontenc}			% кодировка
\usepackage[utf8]{inputenc}			% кодировка исходного текста
\usepackage[english,russian]{babel}	% локализация и переносы
\usepackage{xcolor}
\usepackage{hyperref}
 % Цвета для гиперссылок
\definecolor{linkcolor}{HTML}{799B03} % цвет ссылок
\definecolor{urlcolor}{HTML}{799B03} % цвет гиперссылок

\hypersetup{pdfstartview=FitH,  linkcolor=linkcolor,urlcolor=urlcolor, colorlinks=true}

%%% Дополнительная работа с математикой
\usepackage{amsfonts,amssymb,amsthm,mathtools} % AMS
\usepackage{amsmath}
\usepackage{icomma} % "Умная" запятая: $0,2$ --- число, $0, 2$ --- перечисление

%% Номера формул
%\mathtoolsset{showonlyrefs=true} % Показывать номера только у тех формул, на которые есть \eqref{} в тексте.

%% Шрифты
\usepackage{euscript}	 % Шрифт Евклид
\usepackage{mathrsfs} % Красивый матшрифт

%% Свои команды
\DeclareMathOperator{\sgn}{\mathop{sgn}}

%% Перенос знаков в формулах (по Львовскому)
\newcommand*{\hm}[1]{#1\nobreak\discretionary{}
{\hbox{$\mathsurround=0pt #1$}}{}}
% графика
\usepackage{graphicx}
\graphicspath{{pictures/}}
\DeclareGraphicsExtensions{.pdf,.png,.jpg}
\author{Бурмашев Григорий, БПМИ-208}
\title{}
\date{\today}
\begin{document}
\begin{center}
Бурмашев Григорий. 208. Алгебра -- 8
\end{center}
\section*{Номер 1}
\[
\frac{9 - 40\sqrt[3]{6} -6 \sqrt[6]{36}}{1 - \sqrt[3]{6} - 3\sqrt[3]{36}} \in \mathbb{Q}(\sqrt[6]{6})
\]

\begin{center}
Делаю по алгосу с семинара.
\end{center}

Пусть $x = \sqrt[3]{6}$, тогда $x^3 = 6$

x есть алгебр. над $\mathbb{Q}$ с минимальным многочленом $h = x^3 - 6$. А также:

\[
\mathbb{Q}(\sqrt[3]{6}) \simeq \mathbb{Q}[x]/(h)
\]

Тогда:

\[
\frac{9-40x-6x^2}{1-x-3x^2} = \beta_0 + \beta_1x + \beta_2x^2
\]
\[
9-40x - 6x^2 = ( \beta_0 + \beta_1x + \beta_2x^2) \cdot (1 - x -3x^2)
\]
\[
9-40x-6x^2 = -3\beta_0x^2 - \beta_0 x + \beta_0 - 3\beta_1x^3 - \beta_1x^2 + \beta_1 x - 3\beta_2x^4 - \beta_2x^3 + \beta_2x^2
\]
\[
9-40x-6x^2 = -3\beta_0x^2 - \beta_0 x + \beta_0 - 18\beta_1 - \beta_1x^2 + \beta_1 x - 18\beta_2x - 6\beta_2x + \beta_2x^2
\]
\[
9-40x-6x^2 = (\beta_0 - 18\beta_1 - 6\beta_2) - (\beta_0 - \beta_1 + 18\beta_2)x - (3\beta_0 +\beta_1 - \beta_2)x^2
\]

Получаем СЛУ, которую нам нужно решить, а именно:

\[
\begin{cases}
\beta_0 - 18\beta_1 - 6\beta_2 = 9 \\
-(\beta_0 - \beta_1 + 18\beta_2)  = -40\\
 -(3\beta_0 + \beta_1 - \beta_2) = -6
\end{cases}
\]

\[
\begin{cases}
\beta_0 - 18\beta_1 - 6\beta_2 = 9 \\
\beta_0 - \beta_1 + 18\beta_2 = 40\\
3\beta_0 + \beta_1 - \beta_2 = 6
\end{cases}
\]

Переводим в матричный вид и считаем:
\[
\begin{pmatrix}
1 & -18 & -6 & \vrule &9 \\
1 & -1 & 18 & \vrule &40  \\
3 & 1 & -1 & \vrule &6 \\
\end{pmatrix}
=
\begin{pmatrix}
1 & -18 & -6 & \vrule &9 \\
1 & -1 & 18 & \vrule &40 \\
0 & 4 & -55 & \vrule &-114  \\
\end{pmatrix}
=
\]
\[
=
\begin{pmatrix}
0 & -17 & -24 & \vrule &-31  \\
1 & -1 & 18 & \vrule &40 \\
0 & 4 & -55 & \vrule &-114  \\
\end{pmatrix}
=
\begin{pmatrix}
1 & -1 & 18 & \vrule &40  \\
0 & -1 & -244 & \vrule &-487 \\
0 & 4 & -55 & \vrule &-114  \\
\end{pmatrix}
=
\]
\[
=
\begin{pmatrix}
1 & -1 & 18 & \vrule &40  \\
0 & 1 & 244 & \vrule &487 \\
0 & 0 & -1031 & \vrule &-2062  \\
\end{pmatrix}
=
\begin{pmatrix}
1 & -1 & 18 & \vrule &40  \\
0 & 1 & 0 & \vrule &-1  \\
0 & 0 & 1 & \vrule &2 \\
\end{pmatrix}
=
\]
\[
=
\begin{pmatrix}
1 & -1 & 0 & \vrule &4  \\
0 & 1 & 0 & \vrule &-1  \\
0 & 0 & 1 & \vrule &2  \\
\end{pmatrix}
=
\begin{pmatrix}
1 & 0 & 0 & \vrule &3 \\
0 & 1 & 0 & \vrule &-1  \\
0 & 0 & 1 & \vrule &2 \\
\end{pmatrix}
\]
Получаем, что:
\[
\begin{cases}
\beta_0 = 3\\
\beta_1 = -1\\
\beta_2 = 2\\
\end{cases}
\]
А значит получаем вид:
\[
2x^2 - x + 3 = 2 \sqrt[3]{36} - \sqrt[3]{6} + 3
\]
\begin{center}
\textbf{Ответ: } 
\[
2 \sqrt[3]{36} - \sqrt[3]{6} + 3
\]
\end{center}

\clearpage
\section*{Номер 2}
\[
\sqrt{7} - \sqrt{3} + 1
\]

Ищем какой-нибудь многочлен:

\[
\alpha = \sqrt{7} - \sqrt{3} + 1
\]
\[
\alpha - 1 = \sqrt{7} - \sqrt{3}
\]
\[
(\alpha - 1)^2 = 10 - 2\sqrt{21}
\]
\[
(\alpha - 1)^2 - 10 = -2\sqrt{21}
\]
\[
((\alpha - 1)^2 - 10 )^2 = 84
\]
\[
(a^2 - 2a - 9)^2 = 84
\]
\[
a^4 - 4a^3 - 14a^2 + 36a + 81 = 84
\]
\[
a^4 - 4a^3 - 14a^2 + 36a -3 = 0
\]

Теперь нужно показать, что найденный многочлен является минимальным, т.е $[\mathbb{Q}(\alpha) : \mathbb{Q}] = 4$.

Важно заметить, что:

\[
1 \in \mathbb{Q} \rightarrow \mathbb{Q}(1) = \mathbb{Q}
\]

Далее:

\[
\mathbb{Q} \subseteq \mathbb{Q}(\sqrt{3}) \subseteq \mathbb{Q}(\sqrt{3})(\sqrt{7}) = K
\]
\[
[\mathbb{Q}(\sqrt{3}) : \mathbb{Q}] = 2
\]

Покажем теперь, что $\sqrt{7} \notin \mathbb{Q}(\sqrt{3})$:

Пусть лежит, тогда:
\[
\sqrt{7} = a + b\sqrt{3}
\]

\[
7 = a^2 + 2\sqrt{3}ab + 3b^2
\]

Значит:
\[
\begin{cases}
7 = a^2 + 3b^2 \\
2\sqrt{3}ab = 0
\end{cases}
\]

А значит либо $a = 0$, либо $b = 0$ $\rightarrow$ нет решений. 

Итак, $\sqrt{7} \notin \mathbb{Q}(\sqrt{ 3}) \rightarrow$ $x^2 - 7$ это мин.многочлен для $\sqrt{7}$ над $\mathbb{Q}(\sqrt{3})$, тогда:

\[
[K : \mathbb{Q}(\sqrt{3})] = 2 \rightarrow [K : \mathbb{Q}] = 4
\]

Причем базис $K$ над $\mathbb{Q}$ -- это $1, \sqrt{3}, \sqrt{7}, \sqrt{21}$

\[
\alpha = \sqrt{7} - \sqrt{3} + 1 \in \mathbb{Q}(\alpha)
\]

Тогда:

\[
(\alpha - 1)^2\in \mathbb{Q}(\alpha) 
\]
\[
(\alpha - 1)^2 = 10 - 2\sqrt{21} \rightarrow 10 \in \mathbb{Q}(\alpha) \rightarrow \sqrt{21} \in \mathbb{Q}(\alpha)
\]

Но тогда и:

\[
\alpha \sqrt{21} \in \mathbb{Q}(\alpha) \rightarrow 7\sqrt{3} - 3\sqrt{7} \in \mathbb{Q}(\alpha)
\]

А также:
\[
\alpha = \sqrt{7} - \sqrt{3} + 1 \in \mathbb{Q}(\alpha) 
\]
\[
\alpha - 1 = \sqrt{7} - \sqrt{3} \in \mathbb{Q}(\alpha) \rightarrow \sqrt{7} - \sqrt{3} \in \mathbb{Q}(\alpha)
\]

\[
\begin{cases}
7\sqrt{3} - 3\sqrt{7} \in \mathbb{Q}(\alpha)\\
 \sqrt{7} - \sqrt{3} \in \mathbb{Q}(\alpha)
\end{cases} \rightarrow \sqrt{3} \in \mathbb{Q}(\alpha), \sqrt{7} \in \mathbb{Q}(\alpha)
\]

Вывод:

\[
1, \sqrt{3}, \sqrt{7}, \sqrt{21} \in \mathbb{Q}(\alpha) 
\]

Важно:

\[
\mathbb{Q}(\alpha) \subseteq K = \mathbb{Q}(\sqrt{3})(\sqrt{7}) \rightarrow 
\]
\[
\rightarrow \mathbb{Q}(\alpha) = K  \rightarrow [\mathbb{Q}(\alpha) : \mathbb{Q}] = 4
\]
Значит это действительно минимальный многочлен
\begin{center}
\textbf{Ответ: } 
\[
a^4 - 4a^3 - 14a^2 + 36a -3
\]
\end{center}

\clearpage
\section*{Номер 3}
\begin{center}
Поле $\mathbb{F}_8$
\end{center}
Заметим, что $8 = 2^3$

Ищем неприводимый многочлен в $\mathbb{Z}_2[x]$ степени 3: $h = x^3 + ax^2 + bx + c$

Чтобы он был неприводимым, требуем:

\[
\begin{cases}
h(0) \neq 0 : c \neq 0 \\
h(1) \neq 0 : 1 + a + b + c \neq 0
\end{cases}
\]

Подойдет $x^3 + x + 1$, т.к в 0 он обращается в 1, в 1 обращается в 3 (условие выше выполнено)

Тогда:

\begin{center}
[не рисую $\overline{\text{overline}}$, т.к получается как-то очень некрасиво + чтобы сэкономить время]
\end{center}
\[
\mathbb{F}_8 = \mathbb{Z}[x] / (x^3 + x + 1) = \{0, 1, x, x+1, x^2, x^2 + 1, x^2 + x, x^2 + x, x^2 + x + 1\}
\]

\begin{center}
Теперь составляем таблицу сложения:
\end{center}

\begin{center}
{\footnotesize \begin{tabular}{|c|c|c|c|c|c|c|c|c|}
\hline
 $+$ & $0$ & $1$ & $x$ & $x + 1 $ &  $x^2$& $ x^2 + 1$ & $x^2 + x$  &$  x^2 + x + 1 $\\
\hline
 $0$&  $0$&$1$  &$x$  &$x+1$  &$x^2$  &$x^2+1$ &$x^2+x$   & $x^2 + x + 1$ \\
\hline
$1$ & $1$ &  $0$& $x + 1$  &$x$  &$x^2 + 1$  & $x^2$ &  $x^2 + x + 1$& $x^2 + x$ \\
\hline
 $x$& $x$ &  $x+1$&  $0$&$1$  & $x^2 + x$ &  $x^2 + x + 1$&  $x^2$& $x^2 + 1$ \\
\hline
$x + 1$ & $x + 1$ & $x$& $1$ &$0$  &$x^2 + x  +1$  &$x^2 + x$  & $x^2 + 1$ &$x^2$  \\
\hline
$x^2$ &$x^2$  &$x^2 + 1$  &$x^2 + x$  & $x^2 + x + 1$ & $0$ & $1$ &$x$  &$x+1$  \\
\hline
 $x^2 + 1$&$x^2 + 1$  & $x^2$ & $x^2 + x +1$ & $x^2 + x$ &  $1$&$0$  &  $x + 1$&  $x$\\
\hline
 $x^2 + x$&$x^2 + x$  &$x^2 + x + 1$  & $x^2$ & $x^2 + 1$ &  $x$& $x+1$ &$0$  & $1$ \\
\hline
 $x^2 + x + 1$& $x^2 + x +1$ & $x^2+x$ &$x^2+1$  &$x^2$  &$x + 1$  & $x$ & $1$ & $0$ \\
\hline
\end{tabular}
}
\end{center}

\begin{center}
Формула понижения степени:
\end{center}
\[
x^3 = x + 1
\]

\begin{center}
Теперь составляем таблицу умножения:
\end{center}

\begin{center}
{\footnotesize \begin{tabular}{|c|c|c|c|c|c|c|c|c|}
\hline
 $\times$ & $0$ & $1$ & $x$ & $x + 1 $ &  $x^2$& $ x^2 + 1$ & $x^2 + x$  &$  x^2 + x + 1 $\\
\hline
 $0$&  $0$&$0$  &$0$  &$0$  &$0$  &$0$ &$0$   & $0$ \\
\hline
$1$ & $0$ &  $1$& $x $  &$x + 1$  &$x^2$  & $x^2 + 1$ &  $x^2 + x $& $x^2 + x + 1$ \\
\hline
 $x$& $0$ &  $x$&  $x^2$&$x^2  + x$  & $x+ 1$ &  $1$&  $x^2 + x + 1$& $x^2+ 1$ \\
\hline
$x + 1$ & $0$ & $x  + 1$& $x^2 + x$ &$x^2 + 1$  &$x^2 + x + 1$  &$x^2$  & $1$ &$x$  \\
\hline
$x^2$ &$0$  &$x^2$  &$x + 1$  & $x^2 + x + 1$ & $x^2 + x$ & $x$ &$x^2 + 1$  &$1$  \\
\hline
 $x^2 + 1$&$0$  & $x^2 + 1$ & $1$ & $x^2$ &  $x$&$x^2 + x + 1$  &  $x + 1$&  $x^2 + x$\\
\hline
 $x^2 + x$&$0$  &$x^2 + x$  & $x^2 + x + 1$ & $1$ &  $x^2 +1$& $x+1$ &$x$  & $x^2$ \\
\hline
 $x^2 + x + 1$& $0$ & $x^2+x + 1$ &$x^2+1$  &$x$  &$1$  & $x^2 + x$ & $x^2$ & $x+1$ \\
\hline
\end{tabular}
}
\end{center}
\begin{center}
Умер пока считал + техал :(
\end{center}
\end{document}
