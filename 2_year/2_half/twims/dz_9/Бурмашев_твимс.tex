\documentclass[a4paper,12pt]{article}

%%% Работа с русским языком
\usepackage{cmap}					% поиск в PDF
\usepackage{mathtext} 				% русские буквы в формулах
\usepackage[T2A]{fontenc}			% кодировка
\usepackage[utf8]{inputenc}			% кодировка исходного текста
\usepackage[english,russian]{babel}	% локализация и переносы
\usepackage{xcolor}
\usepackage{hyperref}
 % Цвета для гиперссылок
\definecolor{linkcolor}{HTML}{799B03} % цвет ссылок
\definecolor{urlcolor}{HTML}{799B03} % цвет гиперссылок

\hypersetup{pdfstartview=FitH,  linkcolor=linkcolor,urlcolor=urlcolor, colorlinks=true}

%%% Дополнительная работа с математикой
\usepackage{amsfonts,amssymb,amsthm,mathtools} % AMS
\usepackage{amsmath}
\usepackage{icomma} % "Умная" запятая: $0,2$ --- число, $0, 2$ --- перечисление

%% Номера формул
%\mathtoolsset{showonlyrefs=true} % Показывать номера только у тех формул, на которые есть \eqref{} в тексте.

%% Шрифты
\usepackage{euscript}	 % Шрифт Евклид
\usepackage{mathrsfs} % Красивый матшрифт

%% Свои команды
\DeclareMathOperator{\sgn}{\mathop{sgn}}

%% Перенос знаков в формулах (по Львовскому)
\newcommand*{\hm}[1]{#1\nobreak\discretionary{}
{\hbox{$\mathsurround=0pt #1$}}{}}
% графика
\usepackage{graphicx}
\graphicspath{{pictures/}}
\DeclareGraphicsExtensions{.pdf,.png,.jpg}
\author{Бурмашев Григорий, БПМИ-208}
\title{}
\date{\today}
\begin{document}
\section*{Номер 1}
Собственно проверяем все свойства:
\subsubsection*{Состоятельность:}
Воспользуемся ЗБЧ (аналогично семинарской таске). Для этого нужно найти мат.ожидание, можем это сделать, зная плотность:
\[
\mathbb{E} X_1 = \int\limits_{-\infty}^{+\infty} x \cdot \frac{3x^2}{\theta^3} \cdot I_{x \in [0, \theta]} = \int\limits_0^{\theta} \frac{3x^3}{\theta^3} dx =
\frac{3}{\theta^3}
\cdot 
\int\limits_0^{\theta} x^3  dx
=
\frac{3}{4} \cdot \theta 
 \]
Теперь по ЗБЧ:
\[
\frac{4}{3} \cdot \overline{X} \overset{P}{\longrightarrow} \frac{4}{3} \cdot \frac{3}{4} \theta = \theta
\]
\begin{center}
\textbf{Ответ: } состоятельность сть
\end{center}
\subsubsection*{Сильная состоятельность:}
Тут будем использовать уЗБЧ, по нему:
\[
\frac{4}{3} \overline{X} \overset{\text{п.н}}{\longrightarrow} \frac{4}{3} \cdot \frac{3}{4} \theta  = \theta 
\]
\begin{center}
\textbf{Ответ: } сильная состоятельность сть
\end{center}
\subsubsection*{Несмещенность:}
\[
\mathbb{E} \left[ \frac{4}{3} \overline{X} \right]  = \frac{4}{3} \cdot \mathbb{E} \left[
\frac{X_1 + \ldots + X_n}{n}
\right] = \frac{4}{3n} \cdot n \cdot \frac{3}{4} \cdot \theta = \theta
\]
Получили:
\[
\mathbb{E} \left[ \frac{4}{3} \overline{X} \right]   = \theta
\]
\begin{center}
\textbf{Ответ: } несмещенность есть
\end{center}
\clearpage
\section*{Номер 10 [листок 6]}
Положим оценку:
\[
\stackrel{\wedge}{\theta} = X_{(n)} - X_{(1)}
\]
Где $X_{(n)} $ -- максимальное значение из выборки, $X_{(1)}$ -- минимальное значение из выборки. 
Ну а теперь проверим её на несмещенность и состоятельность. Почти аналогичная семинару задача, будем опираться на него.
\subsubsection*{Состоятельность:}
По определению хотим проверить:
\[
X_{(n)} - X_{(1)} \overset{P}{\longrightarrow} \theta \; ? 
\]
Делаем аналогично семинару:
\[
P \left(
|X_{(1)} - a | \geq \varepsilon
\right) = 
P \left(
X_{(1)} - a
 \leq -\varepsilon \right)
+ 
P \left(
X_{(1)} - a
\geq \varepsilon \right)  = 
P \left(
X_{(1)} - a
\geq \varepsilon \right)  = 
\]
\[
=
1 - P \left(
X_{(1)} - a
\leq \varepsilon \right)  = 
1 - P \left(
X_{(1)}
\leq \varepsilon  + a\right) = 
\begin{cases}
0,  b - \varepsilon  < a \\
\left(
\frac{-\varepsilon - a + b}{b - a} \right)^n , \; a \leq b  - \varepsilon 
\end{cases}
\]
Оба в системе стремятся к нулю, значит по определению:
\[
X_{(1)} \overset{P}{\longrightarrow} a
\]
Теперь проделываем аналогичную операцию, только с  $X_{(n)}$ и$b$:
\[
P \left(
|X_{(n)} - b | \geq \varepsilon
\right)  = P \left(
X_{(n)} \leq b - \varepsilon
\right) =
\begin{cases}
0, \; b - \varepsilon < a \\
\left(
\frac{b - a - \varepsilon}{b - a}
\right)^n, \; b - \varepsilon \geq a
\end{cases}
\]
Тут также все к нулю, получаем:
\[
X_{(n)} \overset{P}{\longrightarrow} b
\]
Cуммируем и получаем:
\[
\stackrel{\wedge}{\theta}  
=
X_{(n)} - X_{(1)} \overset{P}{\longrightarrow} b - a = \theta 
\]
А значит есть состоятельность!
\subsubsection*{Несмещенность:}
По определению хотим проверить:
\[
\mathbb{E} \left[
X_{(n)} - X_{(1)}
\right] = \theta ?
\]
Ну, нужно найти мат.ожидания, так что ищем их, для этого надо выразить плотности, выражаем (берем с сема, хех):
\[
\rho_{X_{(n)}} (t) = 
I_{[a, b]}
\cdot \frac{n \cdot (t - a)^{n - 1}}{(b - a)^n}
\]
\[
\rho_{X_{(1)}} (t) =I_{[a, b]} \cdot  \frac{n \cdot (b - t)^{n - 1}}{(b - a)^n}
\]
Тогда:
\[
\mathbb{E}\left[
X_{(n)}
\right] = \int\limits_a^b  t \cdot
 \frac{n \cdot (t - a)^{n - 1}}{(b - a)^n} dt = \frac{n}{(b - a)^n} \int\limits_a^b (t - a)^n dt + \frac{na}{(b - a)^n} \int\limits_a^b (t - a)^{n - 1} dt = \frac{a + bn }{n + 1}
\]
А для $X_{(1)}$:
\[
\mathbb{E} \left[
X_{(1)}
\right]
=
\int\limits_a^b  t  \cdot  \frac{n \cdot (b - t)^{n - 1}}{(b - a)^n}dt = n \cdot  \int\limits_a^b  \frac{(t -b )(b -t)^{n - 1}}{(b - a)^n } dt + nb \int\limits_a^b \frac{(b - t)^{n - 1}}{(b -a)^n} dt =  \frac{an + b}{n + 1}
\]
Тогда нужное нам матожидание:
\[
\mathbb{E} \left[
X_{(n)} - X_{(1)}
\right]  = \mathbb{E} \left[
X_{(n)}
\right] - \mathbb{E} \left[
X_{(1)}
\right]  =  \frac{a + bn }{n + 1} - \frac{an + b}{n + 1}  = \frac{(n-1)(b - a)}{n + 1}\neq b-a = \theta 
\]
Нужного нам равенства никак не получаем, значит нет несмещенности
\begin{center}
\textbf{Ответ: } есть состоятельность, нет несмещенности
\end{center}
\end{document}
