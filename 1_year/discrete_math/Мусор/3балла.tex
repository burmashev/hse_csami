\documentclass[a4paper,12pt]{article}

%%% Работа с русским языком
\usepackage{cmap}					% поиск в PDF
\usepackage{mathtext} 				% русские буквы в формулах
\usepackage[T2A]{fontenc}			% кодировка
\usepackage[utf8]{inputenc}			% кодировка исходного текста
\usepackage[english,russian]{babel}	% локализация и переносы
\usepackage{xcolor}
\usepackage{hyperref}
 % Цвета для гиперссылок
\definecolor{linkcolor}{HTML}{799B03} % цвет ссылок
\definecolor{urlcolor}{HTML}{799B03} % цвет гиперссылок

\hypersetup{pdfstartview=FitH,  linkcolor=linkcolor,urlcolor=urlcolor, colorlinks=true}

%%% Дополнительная работа с математикой
\usepackage{amsfonts,amssymb,amsthm,mathtools} % AMS
\usepackage{amsmath}
\usepackage{icomma} % "Умная" запятая: $0,2$ --- число, $0, 2$ --- перечисление

%% Номера формул
%\mathtoolsset{showonlyrefs=true} % Показывать номера только у тех формул, на которые есть \eqref{} в тексте.

%% Шрифты
\usepackage{euscript}	 % Шрифт Евклид
\usepackage{mathrsfs} % Красивый матшрифт

%% Свои команды
\DeclareMathOperator{\sgn}{\mathop{sgn}}

%% Перенос знаков в формулах (по Львовскому)
\newcommand*{\hm}[1]{#1\nobreak\discretionary{}
{\hbox{$\mathsurround=0pt #1$}}{}}
% графика
\usepackage{graphicx}
\graphicspath{{pictures/}}
\DeclareGraphicsExtensions{.pdf,.png,.jpg}
\author{Бурмашев Григорий, БПМИ-208}
\title{}
\date{\today}
\begin{document}
Доказательства 3-х балльные:
\begin{itemize}
\item
критерий обратимости (ОЧЕНЬ МНОГО)
\item
теорема Поста
\item
формула включений исключений для вероятностей
\item
формула включений исключений через индикаторную функцию
\item
рамсей
\item
дилоурс
\item
вероятность пересечения двух k -- элементных множеств в n -- элементном множестве
\item нижняя и верхняя оценка для поиска самого тяжелого и легкого объекта
\item cвязность графа (34 док-во)
\end{itemize}
Задачи 3-х балльные:
\begin{itemize}
\item самолет  100 пассажиров
\item набор значений переменных задает граф на множестве вершин (функция TREE)
\item индексная функция IND
\item арифметическая прогрессия a
\item числитель несократимой дроби делится на p
\item a и b независимы, b и c независимы
\item обозначим через N множество  натуральных чисел, докажите, что существует такая функция
\item голосование MAJ()
\item равенство $\sum \phi(d) = n$
\item в симметрическую разность входят все элементы (глина какая-то)
\item задача про яйца (двое играют в бой яиц)
\item про неотрицательную случайную величину известно + мат ожидание
\item король предлагает сыграть в игру
\item множество $\mathbb{N}^d$ упорядоченно покординатно. есть ли бесконечная антицепь
\item в графе на n вершинах + для любой пары вершин
\item вася и петя бросают монету, вася 10 раз, петя 11
\item докажите что сравнение по модулю $ax \equiv b (mod N)$ либо не имеет решений либо что-то там
\end{itemize}
\end{document}
