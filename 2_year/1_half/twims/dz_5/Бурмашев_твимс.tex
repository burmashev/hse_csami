\documentclass[a4paper,12pt]{article}

%%% Работа с русским языком
\usepackage{cmap}					% поиск в PDF
\usepackage{mathtext} 				% русские буквы в формулах
\usepackage[T2A]{fontenc}			% кодировка
\usepackage[utf8]{inputenc}			% кодировка исходного текста
\usepackage[english,russian]{babel}	% локализация и переносы
\usepackage{xcolor}
\usepackage{hyperref}
 % Цвета для гиперссылок
\definecolor{linkcolor}{HTML}{799B03} % цвет ссылок
\definecolor{urlcolor}{HTML}{799B03} % цвет гиперссылок

\hypersetup{pdfstartview=FitH,  linkcolor=linkcolor,urlcolor=urlcolor, colorlinks=true}

%%% Дополнительная работа с математикой
\usepackage{amsfonts,amssymb,amsthm,mathtools} % AMS
\usepackage{amsmath}
\usepackage{icomma} % "Умная" запятая: $0,2$ --- число, $0, 2$ --- перечисление

%% Номера формул
%\mathtoolsset{showonlyrefs=true} % Показывать номера только у тех формул, на которые есть \eqref{} в тексте.

%% Шрифты
\usepackage{euscript}	 % Шрифт Евклид
\usepackage{mathrsfs} % Красивый матшрифт

%% Свои команды
\DeclareMathOperator{\sgn}{\mathop{sgn}}

%% Перенос знаков в формулах (по Львовскому)
\newcommand*{\hm}[1]{#1\nobreak\discretionary{}
{\hbox{$\mathsurround=0pt #1$}}{}}
% графика
\usepackage{graphicx}
\graphicspath{{pictures/}}
\DeclareGraphicsExtensions{.pdf,.png,.jpg}
\author{Бурмашев Григорий, БПМИ-208}
\title{ТВиМС, дз -- 5}
\date{\today}
\begin{document}
\maketitle 
\section*{Номер 1 (пункт б)}
Cчитаем распределение:
\begin{center}
Вытянули 0 красных шаров: $\frac{5}{12} \cdot \frac{5}{12}$
\end{center}
\begin{center}
Вытянули 1 красный шар: $\frac{7}{12} \cdot \frac{5}{12} + \frac{5}{12}\cdot \frac{7}{12}$
\end{center}
\begin{center}
Вытянули 2 красных шара: $\frac{7}{12} \cdot \frac{7}{12}$
\end{center}
Cчитаем математическое ожидание:
\[
E[x] = 0 \cdot \left(\frac{5}{12} \cdot \frac{5}{12}\right) + 1 \cdot \left(\frac{7}{12} \cdot \frac{5}{12} + \frac{5}{12}\cdot \frac{7}{12}\right) + 2 \cdot \left(\frac{7}{12} \cdot \frac{7}{12}\right) = \frac{7}{6}
\]
Считаем дисперсию:
\[
D[x] = E[x^2] - (E[x])^2 = 0^2 \cdot \left(\frac{5}{12} \cdot \frac{5}{12}\right) + 1^2 \cdot \left(\frac{7}{12} \cdot \frac{5}{12} + \frac{5}{12}\cdot \frac{7}{12}\right) + 2^2 \cdot \left(\frac{7}{12} \cdot \frac{7}{12}\right) - \frac{7^2}{6^2} = \frac{35}{72}
\]
\begin{center}
\textbf{Ответ: } \[
E[x] = \frac{7}{6}
\]
\[
D[x] =  \frac{35}{72}
\]
\end{center}
\clearpage
\section*{Номер 6} 
Считаем распределение (оформлю сразу таблицей, там очев, например получить сумму 4 это 2 + 2, 1 +3 или 3 + 1, а всего вариантов $6 \cdot 6 = 36$, аналогично для остальных сумм): 
\begin{center}
\begin{tabular}{|c|c|c|c|c|c|c|c|c|c|c|}
\hline
 2& 3 &4  & 5 &  6&  7& 8 & 9 &  10&  11& 12 \\
\hline
$\frac{1}{36}$ & $\frac{2}{36}$  &$\frac{3}{36}$   &$\frac{4}{36}$   & $\frac{5}{36}$  &$\frac{6}{36}$   &$\frac{5}{36}$   & $\frac{4}{36}$  & $\frac{3}{36}$  &$\frac{2}{36}$   & $\frac{1}{36}$ \\
\hline
\end{tabular}
\end{center}
Теперь посчитаем математическое ожидание:
\[
E[x] = \frac{2 \cdot 1}{36} + \frac{3\cdot 2}{36} + \frac{4\cdot 3}{36} + \frac{5\cdot 4}{36} + \frac{6\cdot 5}{36} + \frac{7\cdot 6}{36} + \frac{8 \cdot 5}{36} + \frac{9\cdot 4}{36} +\frac{10\cdot 3}{36} +\frac{11\cdot 2}{36} + \frac{12\cdot 1}{36} = 7
\]
\[
D[x] = \frac{2^2\cdot 1}{36} + \frac{3^2\cdot 2}{36} + \frac{4^2\cdot 3}{36} + \frac{5^2\cdot 4}{36} + \frac{6^2\cdot 5}{36} + \frac{7^2\cdot 6}{36} + \frac{8^2 \cdot 5}{36} + \frac{9^2\cdot 4}{36} +\frac{10^2\cdot 3}{36} +\frac{11^2\cdot 2}{36} + \frac{12^2\cdot 1}{36} - 7^2 = 
\]
\[
=
\frac{35}{6}
\]
\begin{center}
\textbf{Ответ: } 
\[
E[x] = 7
\]
\[
D[x] = \frac{35}{6}
\]
\end{center}
\section*{Номер 7}
Пусть у нас $N$ рыб, нас интересует максимизация вероятности $P(X = 1)$. Всего у нас $C_N^2$ способов выбрать 2 рыбы из $N$, а выбрать 1 из 5 помеченных рыб среди всех $N$ у нас $C_5^1 \cdot C^1_{N - 5}$, тогда:
\[
P(X = 1) = \frac{C_5^1 \cdot C^1_{N - 5}}{C_N^2} = \frac{10N - 50}{(N - 1)N}
\]
Для максимизации этой вероятности считаем производную:
\[
\left(\frac{10N - 50}{(N - 1)N}\right)' = \frac{10(N^2 - N) - (10N - 50) \cdot (2N - 1)}{(N^2 - N)^2} = \frac{10(-N^2 +10N -5)}{(N^2 - N)^2}
\]
Теперь приравниваем производную к нулю, знаменатель у нас всегда больше нуля ($N$ как минимум 5 из условия), коэффициент 10 нас не интересует, значит нужно занулить $-N^2 + 10N - 5$):
\[-
N^2 -  10N + 5 = 0
\]
Чудесами вычислений получаем точки:
\[
N = 5 + 2\sqrt{5}
\]
\[
N = 5 - 2 \sqrt{5}
\]
Точка максимума в $N = 5 + 2 \sqrt{5}$


У нас рыбы, поэтому нужно взять натуральное, нам подходят ближайшие слева и справаа $N = 9, N = 10$, проверим их:
\[
P(X = 1, N = 9) = \frac{90 - 50}{8 \cdot 9 } = \frac59
\]
\[
P(X = 1, N = 10) = \frac{100 - 50}{9 \cdot 10 } = \frac59
\]
А значит подходят оба, теперь считаем распределение, математическое ожидание и дисперсию (это и есть \textbf{ответ}):
\begin{itemize}
\item $N = 9$
\begin{center}
0 помеченных рыб :
\[
P(X = 0) = \frac{C_{N-5}^2}{C_N^2} = \frac{C^2_4}{C^2_9} = \frac{6}{36} =  \frac{1}{6}
\]
1 помеченная рыба:
\[
P(X = 1) = \frac{5}{9}
\]
2 помеченные рыбы:
\[
P(X = 2) = \frac{C^2_5}{C^2_9} = \frac{10}{36} = \frac{5}{18}
\]
\end{center}
\[
E[x] = 0 \cdot \frac16 + 1 \cdot \frac59 + 2 \cdot \frac{5}{18} = \frac{10}{9}
\]
\[
D[x] = 1^2 \cdot \frac59 + 2^2 \cdot \frac{5}{18} - \frac{10^2}{9^2} = \frac{35}{81}
\]
\clearpage
\item N = 10:
\begin{center}
0 помеченных рыб :
\[
P(X = 0) = \frac{C_{N-5}^2}{C_N^2} = \frac{C^2_5}{C^2_{10}} = \frac{10}{45} = \frac{2}{9}
\]
1 помеченная рыба:
\[
P(X = 1) = \frac{5}{9}
\]
2 помеченные рыбы:
\[
P(X = 2) = \frac{C^2_5}{C^2_{10}} = \frac{10}{45} = \frac{2}{9}
\]
\end{center}
\[
E[x] = \frac{5}{9} + 2 \cdot \frac29 = 1
\]
\[
D[x] = \frac{5}{9} + 4 \cdot \frac{2}{9} - 1 = \frac{4}{9}
\]
\end{itemize}
\end{document}
