\documentclass[a4paper,12pt]{article}

%%% Работа с русским языком
\usepackage{cmap}					% поиск в PDF
\usepackage{mathtext} 				% русские буквы в формулах
\usepackage[T2A]{fontenc}			% кодировка
\usepackage[utf8]{inputenc}			% кодировка исходного текста
\usepackage[english,russian]{babel}	% локализация и переносы
\usepackage{xcolor}
\usepackage{hyperref}
 % Цвета для гиперссылок
\definecolor{linkcolor}{HTML}{799B03} % цвет ссылок
\definecolor{urlcolor}{HTML}{799B03} % цвет гиперссылок

\hypersetup{pdfstartview=FitH,  linkcolor=linkcolor,urlcolor=urlcolor, colorlinks=true}

%%% Дополнительная работа с математикой
\usepackage{amsfonts,amssymb,amsthm,mathtools} % AMS
\usepackage{amsmath}
\usepackage{icomma} % "Умная" запятая: $0,2$ --- число, $0, 2$ --- перечисление

%% Номера формул
%\mathtoolsset{showonlyrefs=true} % Показывать номера только у тех формул, на которые есть \eqref{} в тексте.

%% Шрифты
\usepackage{euscript}	 % Шрифт Евклид
\usepackage{mathrsfs} % Красивый матшрифт

%% Свои команды
\DeclareMathOperator{\sgn}{\mathop{sgn}}

%% Перенос знаков в формулах (по Львовскому)
\newcommand*{\hm}[1]{#1\nobreak\discretionary{}
{\hbox{$\mathsurround=0pt #1$}}{}}
% графика
\usepackage{graphicx}
\graphicspath{{pictures/}}
\DeclareGraphicsExtensions{.pdf,.png,.jpg}
\author{Бурмашев Григорий, БПМИ-208}
\title{Язык SQL, дз -- 1}
\date{\today}
\begin{document}
\maketitle
\clearpage
\section*{Предметная область}
Я учусь на факультете компьютерных наук в вышке. Уже как 2 или 3 года большинство наших пар (как лекции, так и семинары) проходят онлайн. Даже если пара ведется очно, у нее все равно как правило идет параллельная запись в формате видео. Записи этих пар выкладываются в ручном режиме. Кто-то выкладывает их на youtube, кто-то на яндекс диск. Все записи находятся на разных каналах и на разных плейлистах. Помимо самих записей пар есть много сопутствующего материала -- книги, конспекты, тетрадки jupyter notebook, таблицы с оценками и прочее. Некоторые конспекты делают преподаватели, некоторые -- студенты. Находить и отслеживать все это сейчас -- крайне неудобно. Иногда могут потребоваться материалы за прошлые учебные года, их найти еще сложнее. По моему мнению создание базы данных, которая будет хранить в себе все указанные выше материалы, сильно упростит жизнь как студентам, так и самим преподавателям.
\section*{Требования к данным}
Я предлагаю хранить в базе данных следующее:
\begin{itemize}
\item \textbf{Преподаватель}

О каждом преподавателе должны хранится такие данные, как ФИО, адрес почты для связи, список предметов, которые этот преподаватель ведет

\item \textbf{Предмет}

У каждого предмета должен хранится год его проведения, список преподавателей, а также все связи с записями пар по этому предмету

\item \textbf{Запись пары}

У каждой записи пары должны хранится такие данные, как дата записи, преподаватель, название предмета, тип пары (лекция/семинар/иное), ссылка на сам файл записи (youtube/яндекс диск, думаю что хранить исходники видео в базе данных -- не лучшая идея)

\item \textbf{Материал}

У каждого материала должны хранится такие данные, как автор материала (преподаватель/ФИО студента), предмет, к которому относится материал, запись пары, к которой относится этот материал, тип материала (конспект/книга/таблица с оценками), ссылка на сам файл материала.
\end{itemize}
\section*{Типичные запросы}
Примерный список типичных запросов:
\begin{enumerate}
\item Создать новый предмет

\item Добавить запись пары для предмета

\item Добавить материал к записи пары

\item Посмотреть все записи пар конкретного года по конкретному предмету

\item Посмотреть все возможные материалы по конкретному предмету

\item Посмотреть записи по всем предметам текущего года с даты A по дату B 

\end{enumerate}
\end{document}
