\documentclass[a4paper,12pt]{article}

%%% Работа с русским языком
\usepackage{cmap}					% поиск в PDF
\usepackage{mathtext} 				% русские буквы в формулах
\usepackage[T2A]{fontenc}			% кодировка
\usepackage[utf8]{inputenc}			% кодировка исходного текста
\usepackage[english,russian]{babel}	% локализация и переносы
\usepackage{xcolor}
\usepackage{hyperref}
 % Цвета для гиперссылок
\definecolor{linkcolor}{HTML}{799B03} % цвет ссылок
\definecolor{urlcolor}{HTML}{799B03} % цвет гиперссылок

\hypersetup{pdfstartview=FitH,  linkcolor=linkcolor,urlcolor=urlcolor, colorlinks=true}

%%% Дополнительная работа с математикой
\usepackage{amsfonts,amssymb,amsthm,mathtools} % AMS
\usepackage{amsmath}
\usepackage{icomma} % "Умная" запятая: $0,2$ --- число, $0, 2$ --- перечисление

%% Номера формул
%\mathtoolsset{showonlyrefs=true} % Показывать номера только у тех формул, на которые есть \eqref{} в тексте.

%% Шрифты
\usepackage{euscript}	 % Шрифт Евклид
\usepackage{mathrsfs} % Красивый матшрифт

%% Свои команды
\DeclareMathOperator{\sgn}{\mathop{sgn}}

%% Перенос знаков в формулах (по Львовскому)
\newcommand*{\hm}[1]{#1\nobreak\discretionary{}
{\hbox{$\mathsurround=0pt #1$}}{}}
% графика
\usepackage{graphicx}
\graphicspath{{pictures/}}
\DeclareGraphicsExtensions{.pdf,.png,.jpg}





\author{Бурмашев Григорий}
\title{Матанализ  -2}
\date{\today}
\begin{document}
\begin{center}
Бурмашев Григорий, 208. Матан. Д/з - 2\\
\url{https://www.instagram.com/burmashev_/}  \\
\textbf{Кто не подписался - тот лох!} \\
\center{\includegraphics[scale=0.5]{hearth}}
\end{center}
\section*{№ 10 (листок 1)}
\subsection*{a) }
Указав $ N(\varepsilon) $, вычислите предел:
\begin{equation*}
\begin{gathered}
\lim_{n\to\infty} \frac{n^2+6}{n^2-10n+26} =
\lim_{n\to\infty} \frac{1+\frac{6}{n^2}}{1-\frac{10}{n}+\frac{26}{n^2}}=
\lim_{n\to\infty} \frac{1}{1} = 1\\\\
\forall \varepsilon > 0  \exists N(\varepsilon) \forall n > N:\\\\
|\frac{n^2+6}{n^2-10n+26} -1 | < \varepsilon\\\\
|\frac{10n-20}{n^2-10n+26}| < \varepsilon\\\\
Если \; n \geq 2:\\\\
\frac{10n-20}{n^2-10n+26} < \varepsilon\\\\
\frac{\frac{10}{n}-\frac{20}{n^2}}{1 -\frac{10}{n} +\frac{26}{n^2}} <\varepsilon\\\\
\frac{10}{n-10} < \varepsilon\\\\
n > \frac{10}{\varepsilon} + 10\\\\
N = \left[\frac{10}{\varepsilon}+10\right]+ 1\\\\
N = \left[\frac{10}{\varepsilon}\right]+ 11\\
\end{gathered}
\end{equation*}
{\Large \textbf{Ответ:}\; $max(N = \left[\frac{10}{\varepsilon}\right]+ 11, 2)$}
\subsection*{б) }
Указав $ N(\varepsilon) $, вычислите предел:
\begin{equation*}
\begin{gathered}
\lim_{n\to\infty} \frac{\log_an}{n}, a > 1\\\\
\text{Т.к }a > 1 ,\log_an \text{ растет медленнее, чем }n \text{, значит:}\\
\lim_{n\to\infty} \frac{\log_an}{n} = 0\\\\
\forall \varepsilon > 0  \exists N(\varepsilon) \forall n > N:\\\\
|\frac{\log_an}{n}| < \varepsilon\\\\
|\frac{\lg(n)}{\lg(a) \times n}| < \varepsilon\\\\
\lg(n) < \varepsilon \times \lg(a) \times n\\\\
\lg(n) \text{ растет примерно, как}\;\sqrt{n} \text{, тогда можно заменить:}\\\\
\sqrt{n} < \varepsilon \times \lg(a) \times n\\\\
n < \varepsilon^2 \times \lg^2(a) \times n^2\\\\
\frac{1}{n} < \varepsilon^2 \times \lg^2(a)\\
\frac{1}{\varepsilon^2 \times \lg^2 (a)} < n\\\\
n > \frac{1}{\varepsilon^2 \times \lg^2 (a)}\\\\
N = \left[ \frac{1}{\varepsilon^2 \times \lg^2(a)} \right] + 1
\end{gathered}
\end{equation*}
{\Large \textbf{Ответ:}  $ N = \left[ \frac{1}{\varepsilon^2 \times \lg^2(a)} \right] + 1 $}
\subsection*{с) }
Указав $ N(\varepsilon) $, вычислите предел:
\begin{equation*}
\begin{gathered}
\lim_{n\to\infty} \sqrt[n]{n}\\
\text{При } n \to \infty \text{ степень } \frac{1}{n} \to 0 \text{, значит:}\\\\
\lim_{n\to\infty} \sqrt[n]{n} = 1\\\\
\forall \varepsilon > 0  \exists N(\varepsilon) \forall n > N:\\\\
|\sqrt[n]{n} - 1| < \varepsilon\\
\sqrt[n]{n} < \varepsilon + 1\\
n < (1 + \varepsilon)^n\\
\text{По биному Ньютона: } \\
n < 1 + n\varepsilon + \frac{n(n-1)}{2} \varepsilon^2 + C_n^3\varepsilon^3 + \ldots + C_n^{n-1}\varepsilon^{n-1} + \varepsilon^n\\
\text{Пусть:}\\
n <  + n\varepsilon + \frac{n(n-1)}{2}\varepsilon^2\\
1 < \varepsilon + \frac{(n-1)}{2} \varepsilon^2\\
2 < 2\varepsilon + (n-1)\varepsilon^2\\
\frac{2-2\varepsilon}{\varepsilon^2} < n-1\\
\frac{2-2\varepsilon}{\varepsilon^2} + 1 < n\\
N = \left[ \frac{2-2\varepsilon}{\varepsilon^2} +1 \right] + 1\\
N = \left[ \frac{2-2\varepsilon}{\varepsilon^2} \right] + 2\\
\end{gathered}
\end{equation*}
{\Large \textbf{Ответ:} $N = \left[ \frac{2-2\varepsilon}{\varepsilon^2} \right] + 2\ $}\\
\newpage
Я надеюсь, что мое решение \textbf{№10 c)} верное и достаточное, а если нет, то:\\\\
\begin{center}
\includegraphics[scale=0.25]{durka}\\
\text{Мои полномочия всё}
\end{center}
\section*{№9 (листок 2)}
\subsection*{a) }
\begin{equation*}
\begin{gathered}
\lim_{n\to\infty} (\sqrt{4+2n+n^2} - \sqrt{n^2-n+1}) =
\lim_{n\to\infty} \frac{4+2n+n^2-(n^2-n+1)}{\sqrt{4+2n+n^2} + \sqrt{n^2-n+1}} = \\\\=
\lim_{n\to\infty} \frac{3n+3}{\sqrt{4+2n+n^2} + \sqrt{n^2-n+1}} =
\lim_{n\to\infty} \frac{3+\frac{3}{n}}{\sqrt{\frac{4}{n^2}+\frac{2}{n}+1} + \sqrt{1-\frac{1}{n} + \frac{1}{n^2}}} = \\\\ = 
\lim_{n\to\infty} \frac{3}{\sqrt{1} + \sqrt{1}} = \lim_{n\to\infty} \frac{3}{2}\\\\
\end{gathered} 
\end{equation*}
{\Large\textbf{Ответ:} $\frac{3}{2} $}
\subsection*{б) }
\begin{equation*}
\begin{gathered}
\lim_{n\to\infty} \sqrt[n]{\frac{7+5^n+3^n}{3+2^n}}\\\\
\text{По арифметике пределов это эквивалентно пределам числителя}\\ \text{и знаменателя:}\\\\
\;\;1)\lim_{n\to\infty} \sqrt[n]{7+5^n+3^n} \\
\;\;2)\lim_{n\to\infty} \sqrt[n]{3+2^n} \\\\
\textbf{1)}\\
\text{При } n\to\infty \; 5^n \text{ будет несоизмеримо больше, чем } 3^n\\ \text{ а 7 и вовсе перестанет играть роли}\\
\text{Наша функция будет стремится к виду: }\\
\sqrt[n]{5^n} \\
\lim_{n\to\infty} \sqrt[n]{5^n} = 5\\\\
\text{Отсюда следует, что: }\\
\lim_{n\to\infty} \sqrt[n]{7+5^n+3^n} = 5\\
\textbf{2)}\\
\text{Аналогично пункту 1), при } n \to\infty\\ \text{3 перестанет влиять на функцию и она будет стремится к:}\\
\sqrt[n]{2^n} \\
\lim_{n\to\infty} \sqrt[n]{2^n} = 2\\
\text{Возвращаясь к исходному пределу:}\\\\
\lim_{n\to\infty} \sqrt[n]{\frac{7+5^n+3^n}{3+2^n}} = \frac{\lim_{n\to\infty} \sqrt[n]{7+5^n+3^n}}{\lim_{n\to\infty} \sqrt[n]{3+2^n}} = \frac{5}{2}
\end{gathered}
\end{equation*}
{\Large \textbf{Ответ: } $\frac{5}{2} $}
\end{document}
