\documentclass[a4paper,12pt]{article}

%%% Работа с русским языком
\usepackage{cmap}					% поиск в PDF
\usepackage{mathtext} 				% русские буквы в формулах
\usepackage[T2A]{fontenc}			% кодировка
\usepackage[utf8]{inputenc}			% кодировка исходного текста
\usepackage[english,russian]{babel}	% локализация и переносы
\usepackage{xcolor}
\usepackage{hyperref}
 % Цвета для гиперссылок
\definecolor{linkcolor}{HTML}{799B03} % цвет ссылок
\definecolor{urlcolor}{HTML}{799B03} % цвет гиперссылок

\hypersetup{pdfstartview=FitH,  linkcolor=linkcolor,urlcolor=urlcolor, colorlinks=true}

%%% Дополнительная работа с математикой
\usepackage{amsfonts,amssymb,amsthm,mathtools} % AMS
\usepackage{amsmath}
\usepackage{icomma} % "Умная" запятая: $0,2$ --- число, $0, 2$ --- перечисление

%% Номера формул
%\mathtoolsset{showonlyrefs=true} % Показывать номера только у тех формул, на которые есть \eqref{} в тексте.

%% Шрифты
\usepackage{euscript}	 % Шрифт Евклид
\usepackage{mathrsfs} % Красивый матшрифт

%% Свои команды
\DeclareMathOperator{\sgn}{\mathop{sgn}}

%% Перенос знаков в формулах (по Львовскому)
\newcommand*{\hm}[1]{#1\nobreak\discretionary{}
{\hbox{$\mathsurround=0pt #1$}}{}}
% графика
\usepackage{graphicx}
\graphicspath{{pictures/}}
\DeclareGraphicsExtensions{.pdf,.png,.jpg}
\author{Бурмашев Григорий, БПМИ-208}
\title{}
\date{\today}
\begin{document}
\maketitle
\clearpage
\section*{Номер 1}
\begin{itemize}
\item
Докажите, что существует неперечислимое множество, дополнение которого тоже неперечислимо
\end{itemize}
Для доказательства воспользуемся фактом существования неперечислимого множества. Возьмем некое неперечислимое подмножество натуральных чисел: $D \subset \mathbb{N}$. Тогда рассмотрим два подмножества $\mathbb{N}$, такие что:
\begin{enumerate}
\item 
\[
A = \{ 
2k : k \in D, \; 2k + 1 : k \not \in  D
\}
\]
\item 
\[
B = \{
2k : k \not \in D, \; 2k + 1: k \in D 
\}
\]
\end{enumerate}
Легко заметить, что $\mathbb{N} \setminus A = B$, поскольку $A \cup B = \mathbb{N}$ и $A \cup B = \emptyset$. Теперь нужно показать, что $A$ и $B$ -- неперечислимы.
\\\\
Предположим, что множество $А$ является перечислимым. Тогда будем перебирать все $2k : k \in D$, а оставшиеся элементы пропускать. Поделим эти элементы на два, мы получаем алгоритм перечисления множества $D$,  но оно \textbf{неперечислимо}, значит и $A$ также неперечислимо
\\\\
Предположим, что множество $B$ является перечислимым. Тогда будем перебирать все $2k + 1 : k \in D$, а оставшиеся элементы пропускать. Вычтем из каждого выбранного элемента единицу и после поделим на два, тогда мы снова получим алгорим перечисления множества $D$, но оно \textbf{неперечислимо}, значит и $B$ также неперечислимо
\\\\
В конце концов получаем, что мы смогли предоставить такое неперечислимое множество $A$, что его дополнение также является неперечислимым.
\begin{center}
\textbf{Ч.Т.Д} 
\end{center}
\clearpage
\section*{Номер 2}
\end{document}
