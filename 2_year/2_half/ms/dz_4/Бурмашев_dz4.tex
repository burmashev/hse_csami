\documentclass[a4paper,12pt]{article}

%%% Работа с русским языком
\usepackage{cmap}					% поиск в PDF
\usepackage{mathtext} 				% русские буквы в формулах
\usepackage[T2A]{fontenc}			% кодировка
\usepackage[utf8]{inputenc}			% кодировка исходного текста
\usepackage[english,russian]{babel}	% локализация и переносы
\usepackage{xcolor}
\usepackage{hyperref}
 % Цвета для гиперссылок
\definecolor{linkcolor}{HTML}{799B03} % цвет ссылок
\definecolor{urlcolor}{HTML}{799B03} % цвет гиперссылок

\hypersetup{pdfstartview=FitH,  linkcolor=linkcolor,urlcolor=urlcolor, colorlinks=true}

%%% Дополнительная работа с математикой
\usepackage{amsfonts,amssymb,amsthm,mathtools} % AMS
\usepackage{amsmath}
\usepackage{icomma} % "Умная" запятая: $0,2$ --- число, $0, 2$ --- перечисление

%% Номера формул
%\mathtoolsset{showonlyrefs=true} % Показывать номера только у тех формул, на которые есть \eqref{} в тексте.

%% Шрифты
\usepackage{euscript}	 % Шрифт Евклид
\usepackage{mathrsfs} % Красивый матшрифт

%% Свои команды
\DeclareMathOperator{\sgn}{\mathop{sgn}}

%% Перенос знаков в формулах (по Львовскому)
\newcommand*{\hm}[1]{#1\nobreak\discretionary{}
{\hbox{$\mathsurround=0pt #1$}}{}}
% графика
\usepackage{graphicx}
\graphicspath{{pictures/}}
\DeclareGraphicsExtensions{.pdf,.png,.jpg}
\author{Бурмашев Григорий, БПМИ-208}
\title{}
\date{\today}
\usepackage{listings}
\begin{document}
\maketitle
\clearpage
\section*{Номер 1}
Ссылка на код : http://turingmachinesimulator.com/shared/goiteaybih

\begin{verbatim}
//------------ Алгоритм -------------------------------------------------|
//
// Будем идти слева направо и заменять пары (a, b) на (x, x)
// 
// Если нашли a, то заменяем на x, идем направо и ищем b, 
// после нахождения b меняем его на x,
// возвращаемся в начальное состояние и снова пытаемся найти пару.
//
// Аналогично для (b, a).
// 
// В конечном итоге мы как либо придем в конец строки,
// если пришли в состоянии поиска
// любого символа, значит у нас равное количество символов, 
// если же мы пришли в  состоянии поиска конкретного символа,
// значит у нас неравное количество символов
//
// После прихода в конец выставляем значение функции и
// идем менять оставшиеся символы на пробельные
// 
// В конце концов возвращаемся к значению функции, двигаем каретку налево
// и завершаемся
//------------ Состояния-------------------------------------------------|
//
// q0 - идем направо и ищем a или b      
// q1 - смогли найти a, теперь ищем b для удаления
// q2 - смогли найти b, теперь ищем a для удаления
// q3 - удалили пару (a, b), теперь хотим вернуться в начало 
// q4 - удалили все что могли, теперь подчищаем вспомогательные символы
// q5 - после отчищения возвращаемся в конец
// qAccept - функция посчитана, заканчиваем работу
//
//-----------------------------------------------------------------------|

name: Burmashev Grigory
init: q0
accept: qAccept

//-------------------Состояние q0----------------------------------------|
// нашли a, заменили его на x, переходим к поиску b
q0,a
q1,x,>

// нашли b, заменили его на x, переходим к поиску a
q0,b
q2,x,>

// пропускаем символ x
q0,x
q0,x,>

// встретили конец в состоянии q0, значит символов равное количество,
// ставим a как результат функции и идем подчищать x
q0,_
q4,a,<

//-------------------Состояние q1----------------------------------------|
// пропускаем a, т.к ищем b
q1,a
q1,a,>

// встретили b, заменяем его на x, переходим к возвращению в начало
q1,b
q3,x,<

// пропускаем символ x
q1,x
q1,x,>

// встретили конец в состоянии q1, значит символов НЕ равное количество,
// ставим b как результат функции и идем подчищать x
q1,_
q4,b,<

//-------------------Состояние q2----------------------------------------|
// пропускаем b, т.к ищем a
q2,b
q2,b,>

// встретили a, заменяем его на x, переходим к возвращению в начало
q2,a
q3,x,<

// пропускаем символ x
q2,x
q2,x,>

// встретили конец в состоянии q1, значит символов НЕ равное количество,
// ставим b как результат функции и идем подчищать x
q2,_
q4,b,<

//-------------------Состояние q3----------------------------------------|
// Здесь мы просто возвращаемся в самое начало
q3,a
q3,a,<

q3,b
q3,b,<

q3,x
q3,x,<

// Вернулись в самое начало, переходим к поиску новой пары (a, b)
q3,_
q0,_,>

//-------------------Состояние q4----------------------------------------|
// Просто подчищаем символы и возвращаемся в начало
q4,a
q4,_,<

q4,b
q4,_,<

q4,x
q4,_,<

// Вернулись в начало, теперь возвращаемся к результату функции
q4,_
q5,_,>

//-------------------Состояние q5----------------------------------------|
// Заканчиваем работу с f() = a
q5,a
qAccept,a,<
// Заканчиваем работу с f() = b
q5,b
qAccept,b,<

// Идем влево
q5,x
q5,_,>

// Идем влево
q5,_
q5,_,>
\end{verbatim}
\end{document}
