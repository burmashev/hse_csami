\documentclass[a4paper,12pt]{article}

%%% Работа с русским языком
\usepackage{cmap}					% поиск в PDF
\usepackage{mathtext} 				% русские буквы в формулах
\usepackage[T2A]{fontenc}			% кодировка
\usepackage[utf8]{inputenc}			% кодировка исходного текста
\usepackage[english,russian]{babel}	% локализация и переносы
\usepackage{xcolor}
\usepackage{hyperref}
 % Цвета для гиперссылок
\definecolor{linkcolor}{HTML}{799B03} % цвет ссылок
\definecolor{urlcolor}{HTML}{799B03} % цвет гиперссылок

\hypersetup{pdfstartview=FitH,  linkcolor=linkcolor,urlcolor=urlcolor, colorlinks=true}

%%% Дополнительная работа с математикой
\usepackage{amsfonts,amssymb,amsthm,mathtools} % AMS
\usepackage{amsmath}
\usepackage{icomma} % "Умная" запятая: $0,2$ --- число, $0, 2$ --- перечисление

%% Номера формул
%\mathtoolsset{showonlyrefs=true} % Показывать номера только у тех формул, на которые есть \eqref{} в тексте.

%% Шрифты
\usepackage{euscript}	 % Шрифт Евклид
\usepackage{mathrsfs} % Красивый матшрифт

%% Свои команды
\DeclareMathOperator{\sgn}{\mathop{sgn}}

%% Перенос знаков в формулах (по Львовскому)
\newcommand*{\hm}[1]{#1\nobreak\discretionary{}
{\hbox{$\mathsurround=0pt #1$}}{}}
% графика
\usepackage{graphicx}
\graphicspath{{pictures/}}
\DeclareGraphicsExtensions{.pdf,.png,.jpg}
\author{Бурмашев Григорий, БПМИ-208}
\title{Линал. ИДЗ - 2 Вариант 2.}
\date{\today}
\begin{document}
\maketitle
\newpage
\section*{1.}
Решите матричное уравнение AX = B, где:
\[
A = \begin{pmatrix}
-18 & 6 & -1 & 0\\
-23 & -2 & 3 & 2\\
-6 & -1 & 1& 2\\
\end{pmatrix},
B = \begin{pmatrix}
-20 & 27 & 28 \\
-23 & 14 & 17\\
-10 & 4 & -2\\
\end{pmatrix}
\]
Решим:
\[
\begin{pmatrix}
-18 & 6 & -1 & 0 & \vrule &-20 & 27 & 28 & \\
-23 & -2 & 3 & 2 & \vrule &-23 & 14 & 17 & \\
-6 & -1 & 1 & 2 & \vrule &-10 & 4 & -2 & \\
\end{pmatrix} = \begin{pmatrix}
-18 & 6 & -1 & 0 & \vrule &-20 & 27 & 28 & \\
-5 & -8 & 4 & 2 & \vrule &-3 & -13 & -11 & \\
-6 & -1 & 1 & 2 & \vrule &-10 & 4 & -2 & \\
\end{pmatrix} = 
\]
\[
= \begin{pmatrix}
-18 & 6 & -1 & 0 & \vrule &-20 & 27 & 28 & \\
-5 & -8 & 4 & 2 & \vrule &-3 & -13 & -11 & \\
-1 & 7 & -3 & 0 & \vrule &-7 & 17 & 9 & \\
\end{pmatrix} = \begin{pmatrix}
0 & -120 & 53 & 0 & \vrule &106 & -279 & -134 & \\
-5 & -8 & 4 & 2 & \vrule &-3 & -13 & -11 & \\
-1 & 7 & -3 & 0 & \vrule &-7 & 17 & 9 & \\
\end{pmatrix}= 
\]
\[
= 
\begin{pmatrix}
0 & -120 & 53 & 0 & \vrule &106 & -279 & -134 & \\
0 & -43 & 19 & 2 & \vrule &32 & -98 & -56 & \\
-1 & 7 & -3 & 0 & \vrule &-7 & 17 & 9 & \\
\end{pmatrix} = \begin{pmatrix}
0 & 9 & -4 & -6 & \vrule &10 & 15 & 34 & \\
0 & -43 & 19 & 2 & \vrule &32 & -98 & -56 & \\
-1 & 7 & -3 & 0 & \vrule &-7 & 17 & 9 & \\
\end{pmatrix} = 
\]
\[
= \begin{pmatrix}
0 & 9 & -4 & -6 & \vrule &10 & 15 & 34 & \\
0 & -7 & 3 & -22 & \vrule &72 & -38 & 80 & \\
-1 & 7 & -3 & 0 & \vrule &-7 & 17 & 9 & \\
\end{pmatrix} = \begin{pmatrix}
0 & 9 & -4 & -6 & \vrule &10 & 15 & 34 & \\
0 & -7 & 3 & -22 & \vrule &72 & -38 & 80 & \\
-1 & 0 & 0 & -22 & \vrule &65 & -21 & 89 & \\
\end{pmatrix} = 
\]
\[
= \begin{pmatrix}
0 & 2 & -1 & -28 & \vrule &82 & -23 & 114 & \\
0 & -7 & 3 & -22 & \vrule &72 & -38 & 80 & \\
-1 & 0 & 0 & -22 & \vrule &65 & -21 & 89 & \\
\end{pmatrix} = \begin{pmatrix}
0 & 2 & -1 & -28 & \vrule &82 & -23 & 114 & \\
0 & -1 & 0 & -106 & \vrule &318 & -107 & 422 & \\
-1 & 0 & 0 & -22 & \vrule &65 & -21 & 89 & \\
\end{pmatrix} =
\]
\[
= \begin{pmatrix}
0 & 0 & -1 & -240 & \vrule &718 & -237 & 958 & \\
0 & -1 & 0 & -106 & \vrule &318 & -107 & 422 & \\
-1 & 0 & 0 & -22 & \vrule &65 & -21 & 89 & \\
\end{pmatrix} = 
 \begin{pmatrix}
1 & 0 & 0 & 22 & \vrule &-65 & 21 & -89 & \\
0 & 1 & 0 & 106 & \vrule &-318 & 107 & -422 & \\
0 & 0 & -1 & 240 & \vrule &-718 & 237 & -958 & \\
\end{pmatrix}
\]
Таким образом:
\[
X = 
\begin{pmatrix}
-65 - 22a_1 & 21 - 22a_2 & -89 - 22a_3 \\
-318 - 106a_1 & 107 - 106a_2 & -422 - 106a_3 \\
-718 - 240a_1 & 237 - 240a_2 & -958 - 240a_3\\
a_1 & a_2 & a_3 \\
\end{pmatrix}, \; \forall a_1, a_2, a_3
\]
\begin{center}
\textbf{Ответ:}
\end{center}
\begin{center}
\[
\begin{pmatrix}
-65 - 22a_1 & 21 - 22a_2 & -89 - 22a_3 \\
-318 - 106a_1 & 107 - 106a_2 & -422 - 106a_3 \\
-718 - 240a_1 & 237 - 240a_2 & -958 - 240a_3\\
a_1 & a_2 & a_3 \\
\end{pmatrix}, \; \forall a_1, a_2, a_3
\]
\end{center}
\newpage
\section*{2.} Решите уравнение относительно неизвестной перестановки X:
\[
\left(
\begin{pmatrix}
1 & 2 & 3 & 4 & 5 & 6 & 7 & 8 \\
2 & 7 & 1 & 5 & 6 & 3 & 4 & 8 \\
\end{pmatrix}^{12}
\cdot
\begin{pmatrix}
1 & 2 & 3 & 4 & 5 & 6 & 7 & 8 \\
8 & 1 & 3& 6 & 2 & 7 & 4 & 5 \\
\end{pmatrix}^{-1} 
\right)^{115} 
X
= 
\begin{pmatrix}
1 & 2 & 3 & 4 & 5 & 6 & 7 & 8 \\
6 & 7& 4& 2 & 8 & 5 & 3 & 1 \\
\end{pmatrix}
\]
Приведем левую часть к циклам и упростим:
\[
\left(
(1274563)^{12}(8)^{12} \cdot \left( (1852)(3)(467) \right)^{-1}
\right)^{115}
\]
\[
\left( 
(1274563)^{12}(8)^{12} \cdot (1258)(3)(476)
\right)^{115}
\]
\[
\left( 
(1274563)^{5}(8)\cdot (1258)(3)(476)
\right)^{115}
\]
\[
\left(
(1642357)(8) \cdot (1258)(3)(476)
\right)^{115}
\]
\[
(1642357)^{3}(8) \cdot (1258)^{3}(3)(476)
\]
\[
(1274563)(8) \cdot (1852)(3)(476)
\]
Умножим эти две перестановки:
\[
\begin{pmatrix}
1 & 2 & 3 & 4 & 5 & 6 & 7 & 8 \\
2 & 7 & 1 & 5 & 6 & 3 & 4 & 8\\
\end{pmatrix} 
\cdot
\begin{pmatrix}
1 & 2 & 3 & 4 & 5 & 6 & 7 & 8 \\
8 & 1 & 3 & 7 & 2 & 4& 6 & 5\\
\end{pmatrix} 
=
\begin{pmatrix}
1 & 2 & 3 & 4 & 5 & 6 & 7 & 8 \\
8 & 2 & 1 & 4 & 7 & 5 & 3 & 6\\
\end{pmatrix} 
\]
Мы знаем, что:
$aX = b \rightarrow X = a^{-1} \cdot b$, тогда: 
\[
\begin{pmatrix}
1 & 2 & 3 & 4 & 5 & 6 & 7 & 8 \\
8 & 2 & 1 & 4 & 7 & 5 & 3 & 6\\
\end{pmatrix}^{-1} 
=
\begin{pmatrix}
1 & 2 & 3 & 4 & 5 & 6 & 7 & 8 \\
3 & 2 & 7 & 4 & 6 & 8 & 5 & 1\\
\end{pmatrix}
\]
\[
\begin{pmatrix}
1 & 2 & 3 & 4 & 5 & 6 & 7 & 8 \\
3 & 2 & 7 & 4 & 6 & 8 & 5 & 1\\
\end{pmatrix}
\cdot
\begin{pmatrix}
1 & 2 & 3 & 4 & 5 & 6 & 7 & 8 \\
6 & 7& 4& 2 & 8 & 5 & 3 & 1 \\
\end{pmatrix}
= 
\begin{pmatrix}
1 & 2 & 3 & 4 & 5 & 6 & 7 & 8 \\
8 & 5 & 4 & 2 & 1 & 6 & 7 & 3 \\
\end{pmatrix}
\]
\\
\begin{center}
\textbf{Ответ:}
\end{center}
\[
\begin{pmatrix}
1 & 2 & 3 & 4 & 5 & 6 & 7 & 8 \\
8 & 5 & 4 & 2 & 1 & 6 & 7 & 3 \\
\end{pmatrix}
\]
\newpage
\section*{3.}
Определите чётность перестановки:
\[
\begin{pmatrix}
1 & 2 & \ldots & 21 & 22 & \ldots & 97 & 98 & \ldots & 115 \\
95 & 96 & \ldots & 115 & 19 & \ldots & 94 & 1 & \ldots &18 \\
\end{pmatrix}
\]
\begin{itemize}
\item От 1го до 21го у нас 21 элемент. Они все идут в порядке возрастания и каждый из них больше, чем все оставшиеся элементы справа, начиная с 22го (их всего 76 + 18 = 94):
\[
21 \cdot 94 = 1974
\]
\item От 22го до 97го у нас 76 элемент. Они все идут в порядке возрастания и каждый из них больше, чем все оставшиеся элементы справа, начиная с 98го (их всего 18):
\[
76 \cdot 18 = 1368
\]
\item С 98го по 115й элементы идут в порядке возрастания и инверсий там нет.
\end{itemize}
Значит:
\[
1974 + 1368 = 3342
\]
\[
(-1)^{3342} = 1 \rightarrow \text{перестановка чётная}
\]
\begin{center}
\textbf{Ответ:} 

перестановка чётная
\end{center}

\newpage
\section*{4.}
Вычислите определитель:
\[
\left|\begin{matrix}
0 & 0 & 0 & 7 & x & 0 \\
x & 0 & 2 & x & 0 & x \\
6 & 9 & 0 & 4 & 1 & 7 \\
0 & 9 & 6 & 8 & 8 & 1 \\
0 & 8 & 2 & 8 & x & x \\
0 & 2 & 8 & 6 & 2 & 3
\end{matrix}\right| 
=
(-1)^5 \cdot 7 
\left|\begin{matrix}
x & 0 & 2 & 0 & x \\
6 & 9 & 0 & 1 & 7 \\
0 & 9 & 6 & 8 & 1 \\
0 & 8 & 2 & x & x \\
0 & 2 & 8 & 2 & 3
\end{matrix}\right|
+
(-1)^6 \cdot x 
\left|\begin{matrix}
x & 0 & 2 & x & x \\
6 & 9 & 0 & 4 & 7 \\
0 & 9 & 6 & 8 & 1 \\
0 & 8 & 2 & 8 & x \\
0 & 2 & 8 & 6 & 3
\end{matrix}\right|
\]
\begin{itemize}
\item Найдем значение определителя при коэффициенте -7:
\[
\left|\begin{matrix}
x & 0 & 2 & 0 & x \\
6 & 9 & 0 & 1 & 7 \\
0 & 9 & 6 & 8 & 1 \\
0 & 8 & 2 & x & x \\
0 & 2 & 8 & 2 & 3
\end{matrix}\right| 
= 
x
\left|\begin{matrix}
9 & 0 & 1 & 7 \\
9 & 6 & 8 & 1 \\
8 & 2 & x & x \\
2 & 8 & 2 & 3
\end{matrix}\right|
- 6 
\left|\begin{matrix}
0 & 2 & 0 & x \\
9 & 6 & 8 & 1 \\
8 & 2 & x & x \\
2 & 8 & 2 & 3
\end{matrix}\right|
\]
\[
\left|\begin{matrix}
9 & 0 & 1 & 7 \\
9 & 6 & 8 & 1 \\
8 & 2 & x & x \\
2 & 8 & 2 & 3
\end{matrix}\right| 
=
\left|\begin{matrix}
0 & -6 & -7 & 6 \\
1 & 4 & 8-x & 1-x \\
0 & -30 & x-8 & x-12 \\
0 & 0 & -14+2x & 1+2x
\end{matrix}\right|
=
-1 \cdot 
\begin{vmatrix}
-6 & -7 & 6\\
-30 &x-8 & x - 12\\
0 & -14+2x & 1+2x\\
\end{vmatrix} =
\]
\[
=
-1 \cdot (-918x + 3366) = 918x - 3366
\]
\[
\left|\begin{matrix}
0 & 2 & 0 & x \\
9 & 6 & 8 & 1 \\
8 & 2 & x & x \\
2 & 8 & 2 & 3
\end{matrix}\right| = 
(-1)^{3} \cdot 2 \left|\begin{matrix}
9 & 8 & 1 \\
8 & x & x \\
2 & 2 & 3
\end{matrix}\right| + (-1)^5 \cdot x \left|\begin{matrix}
9 & 6 & 8 \\
8 & 2 & x \\
2 & 8 & 2
\end{matrix}\right| = 
\]
\[
= -2 \cdot (23x-176) -x (-60x+420) = -46x + 352 + 60x^2 - 420x = 60x^2 -466x + 35
\]

Таким образом:
\[
x(918x-3366) -6(60x^2 - 466x + 35) = 918x^2 - 3366x - 360x^2 + 2796x - 2112 =
\]
\[
= 558x^2 - 570x - 2112
\]
\item Теперь найдем значение определителя при коэффициенте x:
\[
\left|\begin{matrix}
x & 0 & 2 & x & x \\
6 & 9 & 0 & 4 & 7 \\
0 & 9 & 6 & 8 & 1 \\
0 & 8 & 2 & 8 & x \\
0 & 2 & 8 & 6 & 3
\end{matrix}\right| = (-1)^2 \cdot x \left|\begin{matrix}
9 & 0 & 4 & 7 \\
9 & 6 & 8 & 1 \\
8 & 2 & 8 & x \\
2 & 8 & 6 & 3
\end{matrix}\right| + (-1)^3 \cdot 6  \left|\begin{matrix}
0 & 2 & x & x \\
9 & 6 & 8 & 1 \\
8 & 2 & 8 & x \\
2 & 8 & 6 & 3
\end{matrix}\right|
\]
\[
\left|\begin{matrix}
9 & 0 & 4 & 7 \\
9 & 6 & 8 & 1 \\
8 & 2 & 8 & x \\
2 & 8 & 6 & 3
\end{matrix}\right| = 
\left|\begin{matrix}
1 & -2 & -4 & 7-x \\
0 & 6 & 4 & -6 \\
0 & -30 & -16 & x-12 \\
0 & 12 & 14 & -11+2*x
\end{matrix}\right| = (-1)^2\left|\begin{matrix}
6 & 4 & -6 \\
-30 & -16 & x-12 \\
12 & 14 & -11+2*x
\end{matrix}\right| = 
\]
\[
= 12x+3536
\]
\[
\left|\begin{matrix}
0 & 2 & x & x \\
9 & 6 & 8 & 1 \\
8 & 2 & 8 & x \\
2 & 8 & 6 & 3
\end{matrix}\right|
=
\left|\begin{matrix}
0 & 2 & x & x \\
1 & 4 & 0 & 1-x \\
0 & -30 & -16 & x-12 \\
0 & 0 & 6 & 1+2*x
\end{matrix}\right| = (-1)^{3} \left|\begin{matrix}
2 & x & x \\
-30 & -16 & x-12 \\
0 & 6 & 1+2*x
\end{matrix}\right| =
\]
\[
 = -1 \cdot (60x^2 - 226x + 112) = -60x^2 + 226x - 112
\]
Таким образом:
\[
x(12x+3536) -6(-60x^2+226x-112) = 12x^2 + 3536x +360x^2 + 672 = 372x^2 + 180x + 672
\]
\item Итого:
\[
-7(558x^2 -570x - 2112) + x(372x^2+180x+672) =
\]
\[
= -3906x^2+ 3990x + 14784 + 372x^3  + 180x^2 + 672x = 
\]
\[
= 372x^3 - 3726x^2 + 4662x + 14784
\]
\end{itemize}
\begin{center}
\textbf{Ответ:} 

$372x^3 - 3726x^2 + 4662x + 14784$
\end{center}
\newpage
\section*{5.} 
Найдите коэффициент при $x^5$ в выражении определителя:
\[
\left|\begin{matrix}
2 & 1 & 8 & 5 & x & 7 & 0 \\
1 & 1 & 6 & x & 7 & 5 & 6 \\
8 & 6 & x & 7 & 9 & 1 & 5 \\
5 & x & 7 & 9 & 8 & 2 & 9 \\
x & 7 & 9 & 8 & 1 & 4 & x \\
7 & 5 & 1 & 2 & 4 & x & 0 \\
0 & 6 & 5 & 9 & x & 0 & 6
\end{matrix}\right|
\]
Упростим матрицу: вычтем из первой строки последнюю и из самого левого столбца самый правый (при таких преобразованиях det не изменится):
\[
\left|\begin{matrix}
8 & -5 & 3 & -4 & 0 & 7 & -6 \\
-5 & 1 & 6 & x & 7 & 5 & 6 \\
3 & 6 & x & 7 & 9 & 1 & 5 \\
-4 & x & 7 & 9 & 8 & 2 & 9 \\
0 & 7 & 9 & 8 & 1 & 4 & x \\
7 & 5 & 1 & 2 & 4 & x & 0 \\
-6 & 6 & 5 & 9 & x & 0 & 6
\end{matrix}\right|
\]
У нас остается всего 6 иксов, значит максимальная степень у икса в определителе будет 6, а чтобы получить 5ю степень, нужно рассматривать перестановки со всеми иксами, кроме какого-то одного.  Рассмотрим все возможные такие случаи ($a_{ij}$ -- элемент в нашей матрице на i-й строке и j-м столбце):
\begin{itemize}
\item Если не берем $a_{24}$:

Тогда мы точно берем $a_{42},  a_{33}, a_{57}, a_{66}, a_{75}$, чтобы получить пятую степень у икса. Остаются на выбор варианты $a_{11}, a_{14}, a_{21}$. Мы из них можем взять $a_{21}$ и $a_{14}$ (чтобы выполнялось условие на единственность элементов из каждой строки и каждого столцба). Коэффициент в перестановке получается $-5 \cdot (-4) = 20 $.

Итоговая перестановка:
\[
a_{42} a_{33}a_{57} a_{66} a_{75}a_{21}a_{14}
\]
\\
Перестановка $\begin{pmatrix}
1 & 2 & 3 & 4 & 5 & 6 & 7\\
4 & 1 & 3 & 2 & 7 & 6 & 5 \\
\end{pmatrix}$ нечетна (7 элементов и 4 цикла), значит знак в определителе будет отрицательный, т.е в итоге -20.
\item Если не берем $a_{33}$:

Тогда мы точно берем $a_{42}, a_{24}, a_{57}, a_{66}, a_{76}$. Остаются на выбор варианты $a_{11}, a_{13}, a_{31}$. Мы берем $a_{31}$ и $a_{13}$ и получаем коэффициент $3 \cdot 3 = 9$. 

Итоговая перестановка:
\[
a_{42} a_{24} a_{57} a_{66} a_{76} a_{31} a_{13}
\]
\\
Перестановка $\begin{pmatrix}
1 & 2 & 3 & 4 & 5 & 6 & 7\\
3 & 4 & 1& 2 & 7 & 6 & 5 \\
\end{pmatrix}$ нечетна (7 элементов и 4 цикла), значит знак в определителе будет отрицательный, т.е в итоге -9.
\item Если не берем $a_{42}$:

Тогда мы точно берем $a_{24}, a_{33}, a_{57}, a_{66}, a_{75}$. Остаются на выбор варианты $a_{11}, a_{12}, a_{41}$. Мы берем $a_{41}, a_{12}$ и получаем коэффициент $-4 \cdot (-5) = 20$. 

Итоговая перестановка:
\[
a_{24}a_{33}a_{57}a_{66} a_{75}a_{41} a_{12}
\]
\\
Перестановка $\begin{pmatrix}
1 & 2 & 3 & 4 & 5 & 6 & 7\\
2 & 4 & 3& 1& 7 & 6 & 5 \\
\end{pmatrix}$  нечетна (7 элементов и 4 цикла), значит знак в определителе будет отрицательный, т.е в итоге -20.
\item Если не берем $a_{57}$

Тогда мы точно берем $a_{24}, a_{33}, a_{42}, a_{66}, a_{75}$. Остаются на выбор варианты $a_{11}, a_{17}, a_{51}$. Мы должны взять $a_11$ и $a_{51}$.  Но $a_{51} = 0$, значит перестановка будет равна нулю.

\item Если не берем $a_{66}$:

Тогда мы точно берем $a_{24}, a_{33}, a_{42}, a_{57}, a_{75}$. Остаются на выбор варианты $a_{11}, a_{16}, a_{61}$. Мы должны взять $a_{16} $ и $a_{61}$. Коэффициент в перестановке получается $7 \cdot 7 = 49$. 

Итоговая перестановка:
\[
a_{24}a_{33}a_{42}a_{57}a_{75} a_{16} a_{61}
\]
\\
Перестановка $\begin{pmatrix}
1 & 2 & 3 & 4 & 5 & 6 & 7\\
6 & 4 & 3& 2& 7 & 1& 5 \\
\end{pmatrix}$ нечетна (7 элементов и 4 цикла), значит знак в определителе будет отрицательный, т.е в итоге -49.
\item Если не берем $a_{75}$:

Тогда мы точно берем $a_{24}, a_{33}, a_{42}, a_{57}$. Остаются на выбор варианты $a_{11}, a_{15}, a_{71}$.  Мы должны взять $a_{15}$  b $a_{71}$. Но $a_{15} = 0$, значит перестановка будет равна нулю.
\end{itemize}

\textbf{Итого:} 
\\\\
$-20 - 9 - 20 - 49 = -98$. Т.е коэффициент при $x^5$ будет равен -98

\begin{center}
\textbf{Ответ:}

-98
\end{center}



























\end{document}