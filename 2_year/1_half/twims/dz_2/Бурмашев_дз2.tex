\documentclass[a4paper,12pt]{article}

%%% Работа с русским языком
\usepackage{cmap}					% поиск в PDF
\usepackage{mathtext} 				% русские буквы в формулах
\usepackage[T2A]{fontenc}			% кодировка
\usepackage[utf8]{inputenc}			% кодировка исходного текста
\usepackage[english,russian]{babel}	% локализация и переносы
\usepackage{xcolor}
\usepackage{hyperref}
 % Цвета для гиперссылок
\definecolor{linkcolor}{HTML}{799B03} % цвет ссылок
\definecolor{urlcolor}{HTML}{799B03} % цвет гиперссылок

\hypersetup{pdfstartview=FitH,  linkcolor=linkcolor,urlcolor=urlcolor, colorlinks=true}

%%% Дополнительная работа с математикой
\usepackage{amsfonts,amssymb,amsthm,mathtools} % AMS
\usepackage{amsmath}
\usepackage{icomma} % "Умная" запятая: $0,2$ --- число, $0, 2$ --- перечисление

%% Номера формул
%\mathtoolsset{showonlyrefs=true} % Показывать номера только у тех формул, на которые есть \eqref{} в тексте.

%% Шрифты
\usepackage{euscript}	 % Шрифт Евклид
\usepackage{mathrsfs} % Красивый матшрифт

%% Свои команды
\DeclareMathOperator{\sgn}{\mathop{sgn}}

%% Перенос знаков в формулах (по Львовскому)
\newcommand*{\hm}[1]{#1\nobreak\discretionary{}
{\hbox{$\mathsurround=0pt #1$}}{}}
% графика
\usepackage{graphicx}
\graphicspath{{pictures/}}
\DeclareGraphicsExtensions{.pdf,.png,.jpg}
\author{Бурмашев Григорий, БПМИ-208}
\title{ТВиМС, дз -- 2}
\date{\today}
\begin{document}
\maketitle
\section*{Номер 8 [листок 1]}
\subsection*{a) В прикупе два туза}
Общее число исходов (2 любые туза из общего числа):
\[
C^2_{32}
\]
\\
Нам подходит выбор 2 тузов из 4:
\[
C^2_4
\]
Итого:
\[
P = \frac{C^2_4}{C^2_{32}} = \frac{6}{496} = \frac{3}{248}
\]
\begin{center}
\textbf{Ответ: } $\frac{3}{248}$
\end{center}
\subsection*{a) Вы один из игроков}
При таком раскладе мы точно знаем, что одному из игроков раздали 10 карт и тузов у него \textbf{не} оказалось. 
\\\\
Тогда общее число исходов будет выбрать 2 карты в прикуп из 22: 
\[
C^2_{22}
\]
Нам подходят все тот же выбор 2 тузов из 4:
\[
C^2_4
\]
Итого:
\[
P = \frac{C^2_4}{C^2_{22}} = \frac{6}{231} = \frac{2}{77}
\]
\begin{center}
\textbf{Ответ: } $\frac{2}{77}$
\end{center}
\clearpage
\section*{Задача 10 [листок 1]}
Т.к все люди разные, то всего у нас способов рассадить $k$ человек по $n$ вагонам:
\[
n^k
\]
Выберем теперь $r$ из $n$ вагонов, на это нужно:
\[
C^r_n
\]
Теперь работаем с $r$ вагонами, рассадим $k$ человек по $r$ вагонам так, чтобы ни один из $r$ вагонов не был пустым, для этого воспользуемся формулой для числа сюръекций из $k$ -- элементного множества в $r$ -- элементное, а именно:
\[
\sum_{i = 0}^{r} (-1)^i C^i_r (r - i)^k
\]
Тогда получаем ответ:
\[
P = \frac{C_n^r \cdot \left(\sum\limits_{i = 0}^{r} (-1)^i C^i_r (r - i)^k\right)}{n^k}
\]
\clearpage
\section*{Задача 8 [листок 2]}
Введем события:
\begin{center}
A -- орел выпал ровно 2 раза
\end{center}
\begin{center}
B -- выпало четное число орлов
\end{center}
Нас интересует:
\[
P(A | B) = \frac{P(A \cap B)}{P(B)}
\]
Тогда:
\[
A \cap B  \text{ -- выпало всего 2 орла}
\]
Знаем вероятность $B$:
\[
P(B) = \frac12
\]
Посчитаем вероятность $A$, расставим 2 орлов на $N$ позиций, это будет $C_n^2$, вероятность выбрать 2 орла будет $\left( \frac{1}{2} \right)^2$. На оставшиеся $N - 2$ позиции нужно поставить решек, вероятность этого будет $ \left( \frac{1}{2} \right)^{N -2}$, итого:
\[
P(A \cap B) = \frac{C_N^2 \cdot \left( \frac{1}{2} \right)^2 \cdot\left( \frac{1}{2} \right)^{N -2} }{1} = C_N^2 \left(\frac{1}{2}\right)^{N}
\]
 Тогда ответ:
\[
P(A | B) = \frac{C_N^2 \cdot \left( \frac{1}{2}\right)^N}{\frac12} = C_N^2 \cdot \left(\frac12\right)^{N- 1}
\]
\begin{center}
\textbf{Ответ: } $ C_N^2 \cdot \left(\frac12\right)^{N- 1}$
\end{center}
\end{document}
