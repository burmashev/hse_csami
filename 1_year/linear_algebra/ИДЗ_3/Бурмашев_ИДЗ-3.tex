\documentclass[a4paper,12pt]{article}

%%% Работа с русским языком
\usepackage{cmap}					% поиск в PDF
\usepackage{mathtext} 				% русские буквы в формулах
\usepackage[T2A]{fontenc}			% кодировка
\usepackage[utf8]{inputenc}			% кодировка исходного текста
\usepackage[english,russian]{babel}	% локализация и переносы
\usepackage{xcolor}
\usepackage{hyperref}
 % Цвета для гиперссылок
\definecolor{linkcolor}{HTML}{799B03} % цвет ссылок
\definecolor{urlcolor}{HTML}{799B03} % цвет гиперссылок

\hypersetup{pdfstartview=FitH,  linkcolor=linkcolor,urlcolor=urlcolor, colorlinks=true}

%%% Дополнительная работа с математикой
\usepackage{amsfonts,amssymb,amsthm,mathtools} % AMS
\usepackage{amsmath}
\usepackage{icomma} % "Умная" запятая: $0,2$ --- число, $0, 2$ --- перечисление

%% Номера формул
%\mathtoolsset{showonlyrefs=true} % Показывать номера только у тех формул, на которые есть \eqref{} в тексте.

%% Шрифты
\usepackage{euscript}	 % Шрифт Евклид
\usepackage{mathrsfs} % Красивый матшрифт

%% Свои команды
\DeclareMathOperator{\sgn}{\mathop{sgn}}

%% Перенос знаков в формулах (по Львовскому)
\newcommand*{\hm}[1]{#1\nobreak\discretionary{}
{\hbox{$\mathsurround=0pt #1$}}{}}
% графика
\usepackage{graphicx}
\graphicspath{{pictures/}}
\DeclareGraphicsExtensions{.pdf,.png,.jpg}
\author{Бурмашев Григорий, БПМИ-208}
\title{Линал. 

Задание 3

Вариант 2.}
\date{\today}
\begin{document}
\maketitle
\newpage
\section*{Номер 1}

Посчитаем для начала ранг матрицы A:
\[
\begin{pmatrix}
14 & 6 & -1 & -3 & 1 & \\
4 & 1 & -2 & -2 & 1 & \\
-23 & -12 & -3 & 16 & -2 & \\
-3 & -1 & 1 & 4 & -1 & \\
1 & 3 & 6 & 1 & -2 & \\
\end{pmatrix} 
=
\begin{pmatrix}
0 & -36 & -85 & -17 & 29 & \\
4 & 1 & -2 & -2 & 1 & \\
-23 & -12 & -3 & 16 & -2 & \\
-3 & -1 & 1 & 4 & -1 & \\
1 & 3 & 6 & 1 & -2 & \\
\end{pmatrix}
=
\]
\[
=
\begin{pmatrix}
0 & -36 & -85 & -17 & 29 & \\
0 & -11 & -26 & -6 & 9 & \\
-23 & -12 & -3 & 16 & -2 & \\
-3 & -1 & 1 & 4 & -1 & \\
1 & 3 & 6 & 1 & -2 & \\
\end{pmatrix}
=
\begin{pmatrix}
0 & -36 & -85 & -17 & 29 & \\
0 & -11 & -26 & -6 & 9 & \\
0 & 57 & 135 & 39 & -48 & \\
-3 & -1 & 1 & 4 & -1 & \\
1 & 3 & 6 & 1 & -2 & \\
\end{pmatrix}
=
\]
\[
=
\begin{pmatrix}
0 & -36 & -85 & -17 & 29 & \\
0 & -11 & -26 & -6 & 9 & \\
0 & 57 & 135 & 39 & -48 & \\
0 & 8 & 19 & 7 & -7 & \\
1 & 3 & 6 & 1 & -2 & \\
\end{pmatrix}
=
\begin{pmatrix}
1 & 3 & 6 & 1 & -2 & \\
0 & -11 & -26 & -6 & 9 & \\
0 & 1 & 2 & -10 & 1 & \\
0 & 8 & 19 & 7 & -7 & \\
0 & -36 & -85 & -17 & 29 & \\
\end{pmatrix}
=
\]
\[
=
\begin{pmatrix}
1 & 3 & 6 & 1 & -2 & \\
0 & 0 & -4 & -116 & 20 & \\
0 & 1 & 2 & -10 & 1 & \\
0 & 8 & 19 & 7 & -7 & \\
0 & -36 & -85 & -17 & 29 & \\
\end{pmatrix}
=
\begin{pmatrix}
1 & 3 & 6 & 1 & -2 & \\
0 & 0 & -4 & -116 & 20 & \\
0 & 1 & 2 & -10 & 1 & \\
0 & 0 & 3 & 87 & -15 & \\
0 & -36 & -85 & -17 & 29 & \\
\end{pmatrix}
=
\]
\[
=
\begin{pmatrix}
1 & 3 & 6 & 1 & -2 & \\
0 & 0 & -4 & -116 & 20 & \\
0 & 1 & 2 & -10 & 1 & \\
0 & 0 & 3 & 87 & -15 & \\
0 & 0 & -13 & -377 & 65 & \\
\end{pmatrix}
=
\begin{pmatrix}
1 & 3 & 6 & 1 & -2 & \\
0 & 1 & 2 & -10 & 1 & \\
0 & 0 & -1 & -29 & 5 & \\
0 & 0 & 3 & 87 & -15 & \\
0 & 0 & -13 & -377 & 65 & \\
\end{pmatrix}
=
\]
\[
=
\begin{pmatrix}
1 & 3 & 6 & 1 & -2 & \\
0 & 1 & 2 & -10 & 1 & \\
0 & 0 & -1 & -29 & 5 & \\
0 & 0 & 0 & 0 & 0 & \\
0 & 0 & -13 & -377 & 65 & \\
\end{pmatrix}
=
\begin{pmatrix}
1 & 3 & 6 & 1 & -2 & \\
0 & 1 & 2 & -10 & 1 & \\
0 & 0 & -1 & -29 & 5 & \\
0 & 0 & 0 & 0 & 0 & \\
0 & 0 & 0 & 0 & 0 & \\
\end{pmatrix} =
\]
\[
=
\begin{pmatrix}
1 & 3 & 6 & 1 & -2 & \\
0 & 1 & 2 & -10 & 1 & \\
0 & 0 & -1 & -29 & 5 & \\
\end{pmatrix}
\]
Мы видим, что $\text{rk} \; A = 3$, т.к у нас всего 3 ведущих элемента.
\\\\
Приводим к улучшенному  ступенчатому виду:
\[
\begin{pmatrix}
1 & 0 & 0 & 31 & -5 & \\
0 & 1 & 2 & -10 & 1 & \\
0 & 0 & -1 & -29 & 5 & \\
\end{pmatrix}
=
\begin{pmatrix}
1 & 0 & 0 & 31 & -5 & \\
0 & 1 & 0 & -68 & 11 & \\
0 & 0 & 1 & 29 & -5 & \\
\end{pmatrix}
\]
Представим в виде суммы трех матриц ранга 1:
\[
A_1 = 
\begin{pmatrix}
14 & 0& 0& 14 \cdot 31 & 1 4В\cdot (-5) & \\
4 & 0 & 0 & 4 \cdot 31& 4 \cdot (-5) & \\
-23 & 0 & 0 & -23  \cdot 31& -23  \cdot (-5)& \\
-3 & 0 & 0& -3 \cdot 31& -3  \cdot (-5)& \\
1 & 0& 0 & 1  \cdot 31& 1\cdot ( -5)& \\
\end{pmatrix} 
=
\begin{pmatrix}
14 & 0& 0& 434 & -70 & \\
4 & 0 & 0 & 124& -20& \\
-23 & 0 & 0 & -713& 115& \\
-3 & 0 & 0& -93& 15& \\
1 & 0& 0 & 31& -5& \\
\end{pmatrix} 
\]
\[
A_2 = 
\begin{pmatrix}
0 & 6 & 0 & 6 \cdot (-68) & 6\cdot (11)& \\
0 & 1 & 0&  1 \cdot (-68) & 1\cdot (11) & \\
0& -12 & 0 & -12  \cdot (-68)  & -12 \cdot (11)& \\
0 & -1 & 0 & -1  \cdot (-68) & -1 \cdot (11)& \\
0& 3 & 0 & 3  \cdot (-68)  & 3 \cdot (11) & \\
\end{pmatrix} 
=
\begin{pmatrix}
0 & 6 & 0 & -408 & 66& \\
0 & 1 & 0&   -68 & 11& \\
0& -12 & 0 & 816 & -132& \\
0 & -1 & 0 & 68 & -11& \\
0& 3 & 0 & -204 & 33 & \\
\end{pmatrix} 
\]
\[
A_3 = 
\begin{pmatrix}
0 & 0 & -1 & -1 \cdot 29& -1 \cdot (-5)& \\
0 & 0 & -2 & -2\cdot 29& -2 \cdot (-5)& \\
0 & 0 & -3 & -3 \cdot 29& -3 \cdot (-5) & \\
0 & 0& 1 & 1 \cdot 29& 1  \cdot (-5)& \\
0& 0 & 6 & 6\cdot 29 & 6 \cdot (-5) & \\
\end{pmatrix} 
=
\begin{pmatrix}
0 & 0 & -1 & -29& 5& \\
0 & 0 & -2 & -58& 10& \\
0 & 0 & -3 & -87& 15 & \\
0 & 0& 1 & 29& -5& \\
0& 0 & 6 & 174 & -30 & \\
\end{pmatrix} 
\]
Проверка:
\[
\begin{pmatrix}
14 & 0& 0& 434 & -70 & \\
4 & 0 & 0 & 124& -20& \\
-23 & 0 & 0 & -713& 115& \\
-3 & 0 & 0& -93& 15& \\
1 & 0& 0 & 31& -5& \\
\end{pmatrix}  
+
\begin{pmatrix}
0 & 6 & 0 & -408 & 66& \\
0 & 1 & 0&   -68 & 11& \\
0& -12 & 0 & 816 & -132& \\
0 & -1 & 0 & 68 & -11& \\
0& 3 & 0 & -204 & 33 & \\
\end{pmatrix} 
=
\left(\begin{matrix}
14 & 6 & 0 & 26 & -4 \\
4 & 1 & 0 & 56 & -9 \\
-23 & -12 & 0 & 103 & -17 \\
-3 & -1 & 0 & -25 & 4 \\
1 & 3 & 0 & -173 & 28
\end{matrix}\right)
\]
\[
\left(\begin{matrix}
14 & 6 & 0 & 26 & -4 \\
4 & 1 & 0 & 56 & -9 \\
-23 & -12 & 0 & 103 & -17 \\
-3 & -1 & 0 & -25 & 4 \\
1 & 3 & 0 & -173 & 28
\end{matrix}\right)
+
\begin{pmatrix}
0 & 0 & -1 & -29& 5& \\
0 & 0 & -2 & -58& 10& \\
0 & 0 & -3 & -87& 15 & \\
0 & 0& 1 & 29& -5& \\
0& 0 & 6 & 174 & -30 & \\
\end{pmatrix} 
=
\left(\begin{matrix}
14 & 6 & -1 & -3 & 1 \\
4 & 1 & -2 & -2 & 1 \\
-23 & -12 & -3 & 16 & -2 \\
-3 & -1 & 1 & 4 & -1 \\
1 & 3 & 6 & 1 & -2
\end{matrix}\right)
\]
Получили исходную матрицу А
\begin{center}
\textbf{Ответ:}
\[
\begin{pmatrix}
14 & 0& 0& 434 & -70 & \\
4 & 0 & 0 & 124& -20& \\
-23 & 0 & 0 & -713& 115& \\
-3 & 0 & 0& -93& 15& \\
1 & 0& 0 & 31& -5& \\
\end{pmatrix} 
+
\begin{pmatrix}
0 & 6 & 0 & -408 & 66& \\
0 & 1 & 0&   -68 & 11& \\
0& -12 & 0 & 816 & -132& \\
0 & -1 & 0 & 68 & -11& \\
0& 3 & 0 & -204 & 33 & \\
\end{pmatrix} 
+
\begin{pmatrix}
0 & 0 & -1 & -29& 5& \\
0 & 0 & -2 & -58& 10& \\
0 & 0 & -3 & -87& 15 & \\
0 & 0& 1 & 29& -5& \\
0& 0 & 6 & 174 & -30 & \\
\end{pmatrix} 
\]
\end{center}
\clearpage

\section*{Номер 2}
\begin{itemize}
\item
Проверим принадлежность $v_1$ к $U$:
\[
\begin{pmatrix}
1 & 3 & 5 & 2 & -11 & \\
-2 & -11 & 0 & -2 & 1 & \\
-1 & -8 & 5 & -7 & -1 & \\
-1 & -3 & -5 & 5 & 8 & \\
\end{pmatrix}
=
\begin{pmatrix}
1 & 3 & 5 & 2 & -11 & \\
-2 & -11 & 0 & -2 & 1 & \\
0 & -5 & 10 & -12 & -9 & \\
-1 & -3 & -5 & 5 & 8 & \\
\end{pmatrix}
=
\]
\[
=
\begin{pmatrix}
0 & 0 & 0 & 7 & -3 & \\
-2 & -11 & 0 & -2 & 1 & \\
0 & -5 & 10 & -12 & -9 & \\
-1 & -3 & -5 & 5 & 8 & \\
\end{pmatrix}
=
\begin{pmatrix}
0 & 0 & 0 & 7 & -3 & \\
0 & -5 & 10 & -12 & -15 & \\
0 & -5 & 10 & -12 & -9 & \\
-1 & -3 & -5 & 5 & 8 & \\
\end{pmatrix}
=
\]
\[
=
\begin{pmatrix}
0 & 0 & 0 & 7 & -3 & \\
0 & 0 & 0 & 0 & -6 & \\
0 & -5 & 10 & -12 & -9 & \\
-1 & -3 & -5 & 5 & 8 & \\
\end{pmatrix}
=
\begin{pmatrix}
-1 & -3 & -5 & 5 & 8 & \\
0 & -5 & 10 & -12 & -9 & \\
0 & 0 & 0 & 7 & -3 & \\
0 & 0 & 0 & 0 & -6 & \\
\end{pmatrix}
\]
Мы видим, что  $v_1$ \textbf{не} принадлежит $U$.
\item
Проверим принадлежность $v_2$ к $U$:
\[
\begin{pmatrix}
1 & 3 & 5 & 2 & -2 & \\
-2 & -11 & 0 & -2 & 9 & \\
-1 & -8 & 5 & -7 & 7 & \\
-1 & -3 & -5 & 5 & 2 & \\
\end{pmatrix}
=
\begin{pmatrix}
1 & 3 & 5 & 2 & -2 & \\
-2 & -11 & 0 & -2 & 9 & \\
0 & -5 & 10 & -12 & 5 & \\
-1 & -3 & -5 & 5 & 2 & \\
\end{pmatrix}
=
\]
\[
=
\begin{pmatrix}
0 & 0 & 0 & 7 & 0 & \\
-2 & -11 & 0 & -2 & 9 & \\
0 & -5 & 10 & -12 & 5 & \\
-1 & -3 & -5 & 5 & 2 & \\
\end{pmatrix}
=
\begin{pmatrix}
0 & 0 & 0 & 7 & 0 & \\
0 & -5 & 10 & -12 & 5 & \\
0 & -5 & 10 & -12 & 5 & \\
-1 & -3 & -5 & 5 & 2 & \\
\end{pmatrix}
=
\]
\[
=
\begin{pmatrix}
0 & 0 & 0 & 7 & 0 & \\
0 & 0 & 0 & 0 & 0 & \\
0 & -5 & 10 & -12 & 5 & \\
-1 & -3 & -5 & 5 & 2 & \\
\end{pmatrix}
=
\begin{pmatrix}
-1 & -3 & -5 & 5 & 2 & \\
0 & -5 & 10 & -12 & 5 & \\
0 & 0 & 0 & 7 & 0 & \\
0 & 0 & 0 & 0 & 0 & \\
\end{pmatrix}
\]
Из 3й строчки видно, что ОСЛУ будет совместна и $v_2 \in U$.

P.S Я не стал писать через вертикальную черту, а просто поменял знаки у векторов $v_1$ и $v_2$ в матрице (перенес их налево, если рассматривать ОСЛУ)
\item
Дополним вектор $v_2$ до базиса всего подпространства $U$:
\[
\begin{pmatrix}
0 & 0 & 0 & 0 & 7 & \\
-9 & -2 & -11 & 0 & -2 & \\
-7 & -1 & -8 & 5 & -7 & \\
-2 & -1 & -3 & -5 & 5 & \\
\end{pmatrix}
=
\begin{pmatrix}
0 & 0 & 0 & 0 & 7 & \\
-2 & -1 & -3 & -5 & 5 & \\
-7 & -1 & -8 & 5 & -7 & \\
-2 & -1 & -3 & -5 & 5 & \\
\end{pmatrix}
=
\]
\[
=
\begin{pmatrix}
0 & 0 & 0 & 0 & 7 & \\
-2 & -1 & -3 & -5 & 5 & \\
-7 & -1 & -8 & 5 & -7 & \\
\end{pmatrix}
=
\begin{pmatrix}
0 & 0 & 0 & 0 & 7 & \\
-2 & -1 & -3 & -5 & 5 & \\
-1 & 2 & 1 & 20 & -22 & \\
\end{pmatrix}
=
\]
\[
=
\begin{pmatrix}
0 & 0 & 0 & 0 & 7 & \\
0 & -5 & -5 & -45 & 49 & \\
-1 & 2 & 1 & 20 & -22 & \\
\end{pmatrix}
=
\begin{pmatrix}
-1 & 2 & 1 & 20 & -22 & \\
0 & -5 & -5 & -45 & 49 & \\
0 & 0 & 0 & 0 & 7 & \\
\end{pmatrix}
\]
А значит базис составляют векторы $v_2$, $u_2$ и $u_4$
\end{itemize}
{\large \begin{center}
\textbf{Ответ:}

$v_2$ лежит в $U$

Базис составляют векторы $v_2$, $u_2$, $u_4$
\end{center}}
\section*{Номер 3}
Запишем векторы в строки матрицы A и найдем ФСР $AX = 0$: 
\[
A =
\begin{pmatrix}
3 & -5 & 4 & 10 & \\
-3 & 2 & 2 & -13 & \\
1 & 2 & -6 & 7 & \\
-1 & 2 & -2 & -3 & \\
\end{pmatrix}
=
\begin{pmatrix}
0 & -3 & 6 & -3 & \\
-3 & 2 & 2 & -13 & \\
1 & 2 & -6 & 7 & \\
-1 & 2 & -2 & -3 & \\
\end{pmatrix}
=
\]
\[
=
\begin{pmatrix}
0 & -3 & 6 & -3 & \\
-3 & 2 & 2 & -13 & \\
0 & 4 & -8 & 4 & \\
-1 & 2 & -2 & -3 & \\
\end{pmatrix}
=
\begin{pmatrix}
0 & -3 & 6 & -3 & \\
0 & -4 & 8 & -4 & \\
0 & 4 & -8 & 4 & \\
-1 & 2 & -2 & -3 & \\
\end{pmatrix}
=
\]
\[
=
\begin{pmatrix}
0 & -3 & 6 & -3 & \\
0 & 4 & -8 & 4 & \\
-1 & 2 & -2 & -3 & \\
\end{pmatrix}
=
\begin{pmatrix}
0 & -3 & 6 & -3 & \\
0 & 1 & -2 & 1 & \\
-1 & 2 & -2 & -3 & \\
\end{pmatrix}
=
\]
\[
=
\begin{pmatrix}
0 & 1 & -2 & 1 & \\
-1 & 2 & -2 & -3 & \\
\end{pmatrix}
=
\begin{pmatrix}
1 & 0 & -2 & 5 & \\
0 & 1 & -2 & 1 & \\
\end{pmatrix}
\]
Отсюда:
\[
\begin{cases}
x_1 -2x_3 + 5x_4 = 0 \\
x_2 - 2x_3 + x_4 = 0
\end{cases}
\]
Тогда ФСР:
\[
\begin{pmatrix}
2 & 2 & 1 & 0 \\
-5 & -1 & 0 & 1\\
\end{pmatrix}
\]
Тогда ответом является однородная система уравнений:
\[
\begin{cases}
2x_1+ 2x_2 + x_3 = 0 \\
-5x_1 - x_2 + x_4  = 0\\ 
\end{cases}
\]
{\large \begin{center}
\textbf{Ответ:} 
\[
\begin{cases}
2x_1+ 2x_2 + x_3 = 0 \\
-5x_1 - x_2 + x_4  = 0\\ 
\end{cases}
\]
\end{center}}
\clearpage
\section*{Номер 4}
\begin{itemize}
\item
Запишем векторы в столбцы и найдем базис и размерность $L_1$:
\[
\begin{pmatrix}
2 & -5 & 16 & 19 & \\
-1 & 0 & -3 & -2 & \\
-5 & -1 & -13 & -7 & \\
-3 & -4 & -1 & 6 & \\
\end{pmatrix}
=
\begin{pmatrix}
0 & -5 & 10 & 15 & \\
-1 & 0 & -3 & -2 & \\
-5 & -1 & -13 & -7 & \\
-3 & -4 & -1 & 6 & \\
\end{pmatrix}
=
\]
\[
=
\begin{pmatrix}
0 & -5 & 10 & 15 & \\
-1 & 0 & -3 & -2 & \\
-2 & 3 & -12 & -13 & \\
-3 & -4 & -1 & 6 & \\
\end{pmatrix}
=
\begin{pmatrix}
0 & -5 & 10 & 15 & \\
-1 & 0 & -3 & -2 & \\
0 & 3 & -6 & -9 & \\
-3 & -4 & -1 & 6 & \\
\end{pmatrix}
=
\]
\[
=
\begin{pmatrix}
0 & -5 & 10 & 15 & \\
-1 & 0 & -3 & -2 & \\
0 & 3 & -6 & -9 & \\
0 & -4 & 8 & 12 & \\
\end{pmatrix}
=
\begin{pmatrix}
-1 & 0 & -3 & -2 & \\
0 & -5 & 10 & 15 & \\
0 & 3 & -6 & -9 & \\
0 & -4 & 8 & 12 & \\
\end{pmatrix}
=
\]
\[
=
\begin{pmatrix}
-1 & 0 & -3 & -2 & \\
0 & -5 & 10 & 15 & \\
0 & 3 & -6 & -9 & \\
0 & -1 & 2 & 3 & \\
\end{pmatrix}
=
\begin{pmatrix}
-1 & 0 & -3 & -2 & \\
0 & -5 & 10 & 15 & \\
0 & 0 & 0 & 0 & \\
0 & -1 & 2 & 3 & \\
\end{pmatrix}
=
\]
\[
=
\begin{pmatrix}
-1 & 0 & -3 & -2 & \\
0 & 0 & 0 & 0 & \\
0 & 0 & 0 & 0 & \\
0 & -1 & 2 & 3 & \\
\end{pmatrix}
=
\]
\begin{equation*}
=
\begin{pmatrix}
1 & 0 & 3 & 2 & \\
0 & 1 & -2 & -3 & \\
\end{pmatrix}
\end{equation*}
А значит базисными являются векторы $a_1$ и $a_2$, $\text{dim } L_1 = 2$
\item
Запишем векторы в столбцы и найдем базис и размерность $L_2$:
\[
\begin{pmatrix}
9 & -17 & -22 & -4 & \\
-2 & 4 & 5 & 1 & \\
-9 & 11 & 19 & 1 & \\
-2 & 6 & 6 & 2 & \\
\end{pmatrix}
=
\begin{pmatrix}
9 & -17 & -22 & -4 & \\
-2 & 4 & 5 & 1 & \\
-9 & 11 & 19 & 1 & \\
0 & 2 & 1 & 1 & \\
\end{pmatrix}
=
\]
\[
=
\begin{pmatrix}
0 & -6 & -3 & -3 & \\
-2 & 4 & 5 & 1 & \\
-9 & 11 & 19 & 1 & \\
0 & 2 & 1 & 1 & \\
\end{pmatrix}
=
\begin{pmatrix}
0 & -6 & -3 & -3 & \\
-2 & 4 & 5 & 1 & \\
-1 & -5 & -1 & -3 & \\
0 & 2 & 1 & 1 & \\
\end{pmatrix}
=
\]
\[
=
\begin{pmatrix}
0 & -6 & -3 & -3 & \\
0 & 14 & 7 & 7 & \\
-1 & -5 & -1 & -3 & \\
0 & 2 & 1 & 1 & \\
\end{pmatrix}
=
\begin{pmatrix}
-1 & -5 & -1 & -3 & \\
0 & 14 & 7 & 7 & \\
0 & -6 & -3 & -3 & \\
0 & 2 & 1 & 1 & \\
\end{pmatrix}
=
\]
\[
=
\begin{pmatrix}
-1 & -5 & -1 & -3 & \\
0 & 14 & 7 & 7 & \\
0 & 0 & 0 & 0 & \\
0 & 2 & 1 & 1 & \\
\end{pmatrix}
=
\begin{pmatrix}
-1 & -5 & -1 & -3 & \\
0 & 14 & 7 & 7 & \\
0 & 2 & 1 & 1 & \\
0 & 0 & 0 & 0 & \\
\end{pmatrix}
=
\]
\[
=
\begin{pmatrix}
-1 & -5 & -1 & -3 & \\
0 & 0 & 0 & 0 & \\
0 & 2 & 1 & 1 & \\
0 & 0 & 0 & 0 & \\
\end{pmatrix}
=
\begin{pmatrix}
-1 & -5 & -1 & -3 & \\
0 & 2 & 1 & 1 & \\
0 & 0 & 0 & 0 & \\
0 & 0 & 0 & 0 & \\
\end{pmatrix}
=
\]
\begin{equation*}
\label{fcr_l2}
=
\begin{pmatrix}
1 & 5 & 1 & 3 & \\
0 & 2 & 1 & 1 & \\
\end{pmatrix}
\end{equation*}
А значит базисными являются векторы $b_1$ и $b_2$, $\text{dim } L_2 = 2$
\item 
Найдем базис и размерность $U = L_1 + L_2$, для этого запишем векторы из $L_1$ и $L_2$ в одну матрицу:
\[
\begin{pmatrix}
2 & -5 & 16 & 19 & 9 & -17 & -22 & -4 & \\
-1 & 0 & -3 & -2 & -2 & 4 & 5 & 1 & \\
-5 & -1 & -13 & -7 & -9 & 11 & 19 & 1 & \\
-3 & -4 & -1 & 6 & -2 & 6 & 6 & 2 & \\
\end{pmatrix}
=
\]
\[
=
\begin{pmatrix}
2 & -5 & 16 & 19 & 9 & -17 & -22 & -4 & \\
-1 & 0 & -3 & -2 & -2 & 4 & 5 & 1 & \\
-5 & -1 & -13 & -7 & -9 & 11 & 19 & 1 & \\
-1 & -9 & 15 & 25 & 7 & -11 & -16 & -2 & \\
\end{pmatrix}
=
\]
\[
=
\begin{pmatrix}
2 & -5 & 16 & 19 & 9 & -17 & -22 & -4 & \\
-1 & 0 & -3 & -2 & -2 & 4 & 5 & 1 & \\
-5 & -1 & -13 & -7 & -9 & 11 & 19 & 1 & \\
0 & -9 & 18 & 27 & 9 & -15 & -21 & -3 & \\
\end{pmatrix}
=
\]
\[
=
\begin{pmatrix}
0 & -5 & 10 & 15 & 5 & -9 & -12 & -2 & \\
-1 & 0 & -3 & -2 & -2 & 4 & 5 & 1 & \\
-5 & -1 & -13 & -7 & -9 & 11 & 19 & 1 & \\
0 & -9 & 18 & 27 & 9 & -15 & -21 & -3 & \\
\end{pmatrix}
=
\]
\[
=
\begin{pmatrix}
0 & -5 & 10 & 15 & 5 & -9 & -12 & -2 & \\
-1 & 0 & -3 & -2 & -2 & 4 & 5 & 1 & \\
0 & -1 & 2 & 3 & 1 & -9 & -6 & -4 & \\
0 & -9 & 18 & 27 & 9 & -15 & -21 & -3 & \\
\end{pmatrix}
=
\]
\[
=
\begin{pmatrix}
-1 & 0 & -3 & -2 & -2 & 4 & 5 & 1 & \\
0 & -5 & 10 & 15 & 5 & -9 & -12 & -2 & \\
0 & -1 & 2 & 3 & 1 & -9 & -6 & -4 & \\
0 & -9 & 18 & 27 & 9 & -15 & -21 & -3 & \\
\end{pmatrix}
=
\]
\[
=
\begin{pmatrix}
-1 & 0 & -3 & -2 & -2 & 4 & 5 & 1 & \\
0 & 0 & 0 & 0 & 0 & 36 & 18 & 18 & \\
0 & -1 & 2 & 3 & 1 & -9 & -6 & -4 & \\
0 & -9 & 18 & 27 & 9 & -15 & -21 & -3 & \\
\end{pmatrix}
=
\]
\[
=
\begin{pmatrix}
-1 & 0 & -3 & -2 & -2 & 4 & 5 & 1 & \\
0 & 0 & 0 & 0 & 0 & 36 & 18 & 18 & \\
0 & -1 & 2 & 3 & 1 & -9 & -6 & -4 & \\
0 & 0 & 0 & 0 & 0 & 66 & 33 & 33 & \\
\end{pmatrix}
=
\]
\[
=
\begin{pmatrix}
-1 & 0 & -3 & -2 & -2 & 4 & 5 & 1 & \\
0 & -1 & 2 & 3 & 1 & -9 & -6 & -4 & \\
0 & 0 & 0 & 0 & 0 & 36 & 18 & 18 & \\
0 & 0 & 0 & 0 & 0 & 66 & 33 & 33 & \\
\end{pmatrix}
=
\]
\[
=
\begin{pmatrix}
-1 & 0 & -3 & -2 & -2 & 4 & 5 & 1 & \\
0 & -1 & 2 & 3 & 1 & -9 & -6 & -4 & \\
0 & 0 & 0 & 0 & 0 & 36 & 18 & 18 & \\
0 & 0 & 0 & 0 & 0 & -6 & -3 & -3 & \\
\end{pmatrix}
=
\]
\[
=
\begin{pmatrix}
-1 & 0 & -3 & -2 & -2 & 4 & 5 & 1 & \\
0 & -1 & 2 & 3 & 1 & -9 & -6 & -4 & \\
0 & 0 & 0 & 0 & 0 & 0 & 0 & 0 & \\
0 & 0 & 0 & 0 & 0 & -6 & -3 & -3 & \\
\end{pmatrix}
=
\]
\[
=
\begin{pmatrix}
1 & 0 & 3 & 2 & 2 & -4 & -5 & -1 & \\
0 & 1 & -2 & -3 & -1 & 9 & 6 & 4 & \\
0 & 0 & 0 & 0 & 0 & 6 & 3 & 3 & \\
\end{pmatrix}
\]
А значит базис составляют векторы $a_1, a_2$ и $b_2$, $\text{dim } U = 3$
\item Размерность $W = L_1 \cap L_2$ будет равна 2 + 2 - 3 = 1. Тогда посчитаем базис W:
\begin{enumerate}
\item
Найдем ФСР $L_1$:
\[
\begin{pmatrix}
2 & -1 & -5 & -3 & \\
-5 & 0 & -1 & -4 & \\
16 & -3 & -13 & -1 & \\
19 & -2 & -7 & 6 & \\
\end{pmatrix}
=
\begin{pmatrix}
2 & -1 & -5 & -3 & \\
-5 & 0 & -1 & -4 & \\
1 & -3 & -16 & -13 & \\
19 & -2 & -7 & 6 & \\
\end{pmatrix}
=
\]
\[
=
\begin{pmatrix}
0 & 5 & 27 & 23 & \\
-5 & 0 & -1 & -4 & \\
1 & -3 & -16 & -13 & \\
19 & -2 & -7 & 6 & \\
\end{pmatrix}
=
\begin{pmatrix}
0 & 5 & 27 & 23 & \\
0 & -15 & -81 & -69 & \\
1 & -3 & -16 & -13 & \\
19 & -2 & -7 & 6 & \\
\end{pmatrix}
=
\]
\[
=
\begin{pmatrix}
0 & 5 & 27 & 23 & \\
0 & -15 & -81 & -69 & \\
1 & -3 & -16 & -13 & \\
0 & 55 & 297 & 253 & \\
\end{pmatrix}
=
\begin{pmatrix}
1 & -3 & -16 & -13 & \\
0 & 0 & 0 & 0 & \\
0 & 5 & 27 & 23 & \\
0 & 55 & 297 & 253 & \\
\end{pmatrix}
=
\]
\[
=
\begin{pmatrix}
1 & -3 & -16 & -13 & \\
0 & 0 & 0 & 0 & \\
0 & 5 & 27 & 23 & \\
0 & 0 & 0 & 0 & \\
\end{pmatrix}
=
\]
\[
=
\begin{pmatrix}
1 & -3 & -16 & -13 & \\
0 & 5 & 27 & 23 & \\
\end{pmatrix}
=
\begin{pmatrix}
1 & -3 & -16 & -13 & \\
0 & 1 & \frac{27}{5}& \frac{23}{5} & \\
\end{pmatrix}
=
\]
\[
=
\begin{pmatrix}
1 & 0 & \frac{1}{5} & \frac{4}{5} & \\
0 & 1 & \frac{27}{5}& \frac{23}{5} & \\
\end{pmatrix}
\]
\[
\begin{cases}
x_1 = -\frac{1}{5}x_3 - \frac{4}{5}x_4 \\
x_2 = -\frac{27}{5}x_3 - \frac{23}{5}x_4
\end{cases}
\]
Отсюда получаем ФСР:
\[
\begin{pmatrix}
-1 & -27 & 5 & 0 \\
-4 & -23 & 0 & 5 \\
\end{pmatrix}
\]
\item 
Найдем ФСР $L_2$:
\[
\begin{pmatrix}
9 & -2 & -9 & -2 & \\
-17 & 4 & 11 & 6 & \\
-22 & 5 & 19 & 6 & \\
-4 & 1 & 1 & 2 & \\
\end{pmatrix}
=
\begin{pmatrix}
9 & -2 & -9 & -2 & \\
1 & 0 & -7 & 2 & \\
-22 & 5 & 19 & 6 & \\
-4 & 1 & 1 & 2 & \\
\end{pmatrix}
=
\]
\[
=
\begin{pmatrix}
0 & -2 & 54 & -20 & \\
1 & 0 & -7 & 2 & \\
-22 & 5 & 19 & 6 & \\
-4 & 1 & 1 & 2 & \\
\end{pmatrix}
=
\begin{pmatrix}
0 & -2 & 54 & -20 & \\
1 & 0 & -7 & 2 & \\
-2 & 0 & 14 & -4 & \\
-4 & 1 & 1 & 2 & \\
\end{pmatrix}
=
\]
\[
=
\begin{pmatrix}
0 & -2 & 54 & -20 & \\
1 & 0 & -7 & 2 & \\
0 & 0 & 0 & 0 & \\
-4 & 1 & 1 & 2 & \\
\end{pmatrix}
=
\begin{pmatrix}
0 & -2 & 54 & -20 & \\
1 & 0 & -7 & 2 & \\
0 & 0 & 0 & 0 & \\
0 & 1 & -27 & 10 & \\
\end{pmatrix}
=
\]
\[
=
\begin{pmatrix}
1 & 0 & -7 & 2 & \\
0 & -2 & 54 & -20 & \\
0 & 0 & 0 & 0 & \\
0 & 1 & -27 & 10 & \\
\end{pmatrix}
=
\begin{pmatrix}
1 & 0 & -7 & 2 & \\
0 & -2 & 54 & -20 & \\
0 & 1 & -27 & 10 & \\
0 & 0 & 0 & 0 & \\
\end{pmatrix}
=
\]
\[
=
\begin{pmatrix}
1 & 0 & -7 & 2 & \\
0 & 0 & 0 & 0 & \\
0 & 1 & -27 & 10 & \\
0 & 0 & 0 & 0 & \\
\end{pmatrix}
=
\begin{pmatrix}
1 & 0 & -7 & 2 & \\
0 & 1 & -27 & 10 & \\
\end{pmatrix}
\]
Получаем систему уравнений:
\[
\begin{cases}
x_1 - 7x_3 + 2x_4 = 0 \\
x_2 - 27x_3 + 10x_4 = 0\\
\end{cases}
\]
Отсюда получаем ФСР:
\[
\begin{pmatrix}
7 & 27 & 1 & 0 \\
-2 & -10 & 0 & 1 \\
\end{pmatrix}
\]
\item
Размерность по формуле будет равна $2 + 2 = 3 + \text{dim } W$, $\text{dim } W  = 1$.  Теперь найдем базис пересечения, объединив ФСР $L_1$ и $L_2$ в одну систему (ну и запишем в виде матрицы соотвественно):
\[
\begin{pmatrix}
-1 & -27 & 5 & 0 & \\
-4 & -23 & 0 & 5 & \\
7 & 27 & 1 & 0 & \\
-2 & -10 & 0 & 1 & \\
\end{pmatrix}
=
\begin{pmatrix}
-1 & -27 & 5 & 0 & \\
0 & 85 & -20 & 5 & \\
7 & 27 & 1 & 0 & \\
-2 & -10 & 0 & 1 & \\
\end{pmatrix}
=
\]
\[
=
\begin{pmatrix}
-1 & -27 & 5 & 0 & \\
0 & 85 & -20 & 5 & \\
0 & -162 & 36 & 0 & \\
-2 & -10 & 0 & 1 & \\
\end{pmatrix}
=
\begin{pmatrix}
-1 & -27 & 5 & 0 & \\
0 & 85 & -20 & 5 & \\
0 & -162 & 36 & 0 & \\
0 & 44 & -10 & 1 & \\
\end{pmatrix}
=
\]
\[
=
\begin{pmatrix}
-1 & -27 & 5 & 0 & \\
0 & 85 & -20 & 5 & \\
0 & 14 & -4 & 4 & \\
0 & 44 & -10 & 1 & \\
\end{pmatrix}
=
\begin{pmatrix}
-1 & -27 & 5 & 0 & \\
0 & 1 & 4 & -19 & \\
0 & 14 & -4 & 4 & \\
0 & 44 & -10 & 1 & \\
\end{pmatrix}
=
\]
\[
=
\begin{pmatrix}
-1 & -27 & 5 & 0 & \\
0 & 1 & 4 & -19 & \\
0 & 0 & -60 & 270 & \\
0 & 44 & -10 & 1 & \\
\end{pmatrix}
=
\begin{pmatrix}
-1 & -27 & 5 & 0 & \\
0 & 1 & 4 & -19 & \\
0 & 0 & -60 & 270 & \\
0 & 0 & -186 & 837 & \\
\end{pmatrix}
=
\]
\[
=
\begin{pmatrix}
-1 & -27 & 5 & 0 & \\
0 & 1 & 4 & -19 & \\
0 & 0 & -60 & 270 & \\
0 & 0 & -6 & 27 & \\
\end{pmatrix}
=
\begin{pmatrix}
-1 & -27 & 5 & 0 & \\
0 & 1 & 4 & -19 & \\
0 & 0 & 0 & 0 & \\
0 & 0 & -6 & 27 & \\
\end{pmatrix}
=
\]
\[
=
\begin{pmatrix}
-1 & -27 & 5 & 0 & \\
0 & 1 & 4 & -19 & \\
0 & 0 & -6 & 27 & \\
\end{pmatrix}
=
\begin{pmatrix}
-1 & 0 & 113 & -513 & \\
0 & 1 & 4 & -19 & \\
0 & 0 & -6 & 27 & \\
\end{pmatrix}
=
\]
\[
=
\begin{pmatrix}
-1 & 0 & -1 & 0 & \\
0 & 1 & 4 & -19 & \\
0 & 0 & -6 & 27 & \\
\end{pmatrix}
=
\begin{pmatrix}
1 & 0 & 1 & 0 & \\
0 & 1 & -2 & 8 & \\
0 & 0 & -6 & 27 & \\
\end{pmatrix} 
=
\]
\[
\begin{pmatrix}
1 & 0 & 1 & 0 & \\
0 & 1 & -2 & 8 & \\
0 & 0 & 1 & -\frac{9}{2} & \\
\end{pmatrix}
=
\begin{pmatrix}
1 & 0 &  0& \frac{9}{2}& \\
0 & 1 & 0& -1 & \\
0 & 0 & 1 & -\frac{9}{2} & \\
\end{pmatrix}
\]
\[
\begin{cases}
x_1 + \frac{9}{2}x_4 = 0 \\
x_2 - x_4 = 0 \\
x_3 - \frac{9}{2}x_4 = 0
\end{cases}
\]
Получается следующий вектор:
\[
\begin{pmatrix}
-\frac{9}{2} & 1 & \frac{9}{2} & 1\\
\end{pmatrix}
=
\begin{pmatrix}
-9& 2 & 9 & 2\\
\end{pmatrix}
\]
\[
\text{dim } W = 1
\]
\end{enumerate}
 \end{itemize} 
{\large \begin{center}
\textbf{Ответ:} 
\begin{itemize}
\item
Базис $L_1$: $a_1, \; a_2$
\item
Базис $L_2$: $b_1, \; b_2$
\item
Базис $U$: $a_1, \; a_2, \; b_2$
\item
Размерности:
\[
\text{dim } L_1 = 2,\;
\text{dim } L_2 = 2,\;
\text{dim } U = 3,\;
\text{dim } W = 1\;
\]
\end{itemize}
\end{center}}
\end{document}



