\documentclass[a4paper,12pt]{article}

%%% Работа с русским языком
\usepackage{cmap}					% поиск в PDF
\usepackage{mathtext} 				% русские буквы в формулах
\usepackage[T2A]{fontenc}			% кодировка
\usepackage[utf8]{inputenc}			% кодировка исходного текста
\usepackage[english,russian]{babel}	% локализация и переносы
\usepackage{xcolor}
\usepackage{hyperref}
 % Цвета для гиперссылок
\definecolor{linkcolor}{HTML}{799B03} % цвет ссылок
\definecolor{urlcolor}{HTML}{799B03} % цвет гиперссылок

\hypersetup{pdfstartview=FitH,  linkcolor=linkcolor,urlcolor=urlcolor, colorlinks=true}

%%% Дополнительная работа с математикой
\usepackage{amsfonts,amssymb,amsthm,mathtools} % AMS
\usepackage{amsmath}
\usepackage{icomma} % "Умная" запятая: $0,2$ --- число, $0, 2$ --- перечисление

%% Номера формул
%\mathtoolsset{showonlyrefs=true} % Показывать номера только у тех формул, на которые есть \eqref{} в тексте.

%% Шрифты
\usepackage{euscript}	 % Шрифт Евклид
\usepackage{mathrsfs} % Красивый матшрифт

%% Свои команды
\DeclareMathOperator{\sgn}{\mathop{sgn}}

%% Перенос знаков в формулах (по Львовскому)
\newcommand*{\hm}[1]{#1\nobreak\discretionary{}
{\hbox{$\mathsurround=0pt #1$}}{}}
% графика
\usepackage{graphicx}
\graphicspath{{pictures/}}
\DeclareGraphicsExtensions{.pdf,.png,.jpg}
\author{Бурмашев Григорий, БПМИ-208}
\title{Матан, дз -- 2}
\date{\today}
\begin{document}
\maketitle 
\section*{Номер 1}
Исследовать семейство функций $f(x, y)$ на равномерную по $y$ сходимость на указанном множестве:
\[
f(x, y) = \frac{x \arctg(xy)}{x + 1 }, y \in \left(0, + \infty \right), x \rightarrow 0+
\]
Заметим:
\[
\lim_{x \rightarrow 0+} \frac{x \arctg(xy)}{x + 1} = 0
\]
Предположим, что сходится к $g(y) = 0$, рассмотрим:
\[
\underset{(0, +\infty)}{\sup} | f - g | \overset{?}{\underset{x \rightarrow 0+}{\longrightarrow}} 0
\]
Берем частную производную:
\[
\frac{\partial f}{\partial y} = \frac{x^2}{(x + 1)(x^2y^2 + 1)}
\]
Частная производная больше нуля, функция возрастает, значит супремум будет при $y \rightarrow \infty$. Получаем:
\[
\lim_{
\begin{matrix}
x \rightarrow 0+ \\
y \rightarrow +\infty 
\end{matrix}} 
\underset{y \in (0, +\infty)}{\sup} \left|  \frac{x \arctg(xy)}{x + 1 } - 0\right|  = 
\lim_{
\begin{matrix}
x \rightarrow 0+ \\
y \rightarrow +\infty 
\end{matrix}} 
\left(
\frac{x \arctg(xy)}{x + 1 }
\right) =
\]
\[
=
\begin{bmatrix}
\arctg(xy) \text{ ограничен} \\
\frac{x}{x + 1} \underset{x \rightarrow 0+}{\longrightarrow} 0
\end{bmatrix} = 
0 
\]
Предел равен нулю, а значит равномерная сходимость есть.
\begin{center}
\textbf{Ответ: } равномерная сходимость есть
\end{center}
\clearpage

\section*{Номер 2}
Исследовать семейство функций $f(x, y)$ на равномерную по $y$ сходимость на указанном множестве:
\[
f(x, y) = \frac{x \arctg(xy)}{x + 1 }, y \in \left(0, + \infty \right), x \rightarrow +\infty
\]
Заметим:
\[
\lim_{x \rightarrow + \infty} \frac{x \arctg(xy)}{x + 1} = \lim_{x \rightarrow + \infty} \frac{\frac{xy}{x^2y^2 + 1} + \arctg(xy)}{0 + 1}  = \frac{\pi}{2}
\]
Всё аналогично первому номеру, только теперь $g(y) = \frac{\pi}{2}$ и супремум работает по другому. Функция у нас возрастает (выяснили в первом номере). При устремлении $x \rightarrow \infty, y \rightarrow \infty$ мы получаем в пределе $\frac{\pi}{2}$. Тогда $| f(x, y) - \frac{\pi}{2}|$ будет минимален (а мы хотим его максимизировать по определению). Из-за этого понимаем, что супремум будет слева (т.е при $y \rightarrow 0$). Смотрим:
\[
\lim_{
\begin{matrix}
x \rightarrow +\infty \\
y \rightarrow 0
\end{matrix}}
\underset{y \in (0, +\infty)}{\sup} \left|  \frac{x \arctg(xy)}{x + 1 } - \frac{\pi}{2}\right|  = 
\lim_{
\begin{matrix}
x \rightarrow +\infty \\
y \rightarrow 0
\end{matrix}}
\left|
  \frac{x \arctg(xy)}{x + 1 } - \frac{\pi}{2}
 \right| =
\]
\[
=
\begin{bmatrix}
\arctg (xy) \underset{y \rightarrow 0}{\rightarrow}0 \\
\frac{x}{x + 1} \underset{x \rightarrow +\infty}{\rightarrow} 1
\end{bmatrix} = 
 \frac{\pi}{2} \neq 0
\]
Получили \textbf{не} ноль, значит равномерной сходимости нет
\begin{center}
\textbf{Ответ: } равномерной сходимости нет
\end{center}
\clearpage

\section*{Номер 3}
Исследовать несобственный интеграл на равномерную сходимость (по определению):
\[
\int\limits_1^{+\infty} \frac{pdx}{1 + p^2x^2}, p \in [1, p_0], p_0 > 1
\]
Нас интересует предел:
\[
\lim_{A \rightarrow + \infty} \; \underset{[1, p_0]}{\sup} \left|
\int\limits_A^{+\infty}  \frac{pdx}{1 + p^2x^2}
 \right| = 0?
\]
Считаем:
\[
\int\limits_A^{+\infty}  \frac{pdx}{1 + p^2x^2} = \arctg(px) \Bigg|^{+\infty}_{A} = \frac{\pi}{2} - \arctg(pA)
\]
Функция убывает, супремум будет достигаться слева. Теперь смотрим предел:
\[
\lim_{A \rightarrow +\infty}
\underset{[1, p_0]}{\sup} \left|
 \frac{\pi}{2} - \arctg(pA) \right| =
 \lim_{A \rightarrow +\infty} \left|
 \frac{\pi}{2} - \arctg(A) \right| =
0
\]
Получили 0, значит есть равномерная сходимость
\begin{center}
\textbf{Ответ: } равномерная сходимость есть
\end{center}
\clearpage
\section*{Номер 4}
Исследовать несобственный интеграл на равномерную сходимость (по определению):
\[
\int\limits_1^{+\infty} \frac{pdx}{1 + p^2x^2}, p \in [0, p_0], p_0 > 0
\]
Аналогично предыдущему номеру нас интересует предел:
\[
\lim_{A \rightarrow + \infty} \; \underset{[0, p_0]}{\sup} \left|
\int\limits_A^{+\infty}  \frac{pdx}{1 + p^2x^2}
 \right| = 0?
\]
Интеграл все тот же, считаем сразу:
\[
\lim_{A \rightarrow +\infty}
\underset{[0, p_0]}{\sup} \left|
 \frac{\pi}{2} - \arctg(pA) \right| =
 \lim_{
\begin{matrix}
A \rightarrow +\infty \\
p \rightarrow 0
\end{matrix}} \left|
 \frac{\pi}{2} - \arctg(pA) \right| =
\frac{\pi}{2} \neq 0
\]
Получили \textbf{не} ноль, значит равномерной сходимости нет
\begin{center}
\textbf{Ответ: } равномерной сходимости нет
\end{center}
\end{document}
