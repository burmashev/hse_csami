\documentclass[a4paper,12pt]{article}

\usepackage[left=2cm,right=2cm,
    top=4cm,bottom=4cm,bindingoffset=0cm]{geometry}

%%% Работа с русским языком
\usepackage{cmap}					% поиск в PDF
\usepackage{mathtext} 				% русские буквы в формулах
\usepackage[T2A]{fontenc}			% кодировка
\usepackage[utf8]{inputenc}			% кодировка исходного текста
\usepackage[english,russian]{babel}	% локализация и переносы
\usepackage{xcolor}
\usepackage{hyperref}
 % Цвета для гиперссылок
\definecolor{linkcolor}{HTML}{799B03} % цвет ссылок
\definecolor{urlcolor}{HTML}{799B03} % цвет гиперссылок

\hypersetup{pdfstartview=FitH,  linkcolor=linkcolor,urlcolor=urlcolor, colorlinks=true}

%%% Дополнительная работа с математикой
\usepackage{amsfonts,amssymb,amsthm,mathtools} % AMS
\usepackage{amsmath}
\usepackage{icomma} % "Умная" запятая: $0,2$ --- число, $0, 2$ --- перечисление

%% Номера формул
%\mathtoolsset{showonlyrefs=true} % Показывать номера только у тех формул, на которые есть \eqref{} в тексте.

%% Шрифты
\usepackage{euscript}	 % Шрифт Евклид
\usepackage{mathrsfs} % Красивый матшрифт

%% Свои команды
\DeclareMathOperator{\sgn}{\mathop{sgn}}

%% Перенос знаков в формулах (по Львовскому)
\newcommand*{\hm}[1]{#1\nobreak\discretionary{}
{\hbox{$\mathsurround=0pt #1$}}{}}
% графика
\usepackage{graphicx}
\graphicspath{{pictures/}}
\DeclareGraphicsExtensions{.pdf,.png,.jpg}
\author{Бурмашев Григорий, БПМИ-208}
\title{Матан, дз -- 1}
\date{\today}
\begin{document}
\maketitle
\section*{1.}
\begin{equation*}
\begin{gathered}
\sum_{n = 3}^{\infty} \left( 2 \sqrt[5]{n - 2} +3 \sqrt[5]{n} - 5 \sqrt[5]{n + 1} \right) \\
\end{gathered}
\end{equation*}
Находим частичную сумму:
\[
S_n =  2\sum_{k = 3}^{n}  \sqrt[5]{k - 2} + 3\sum_{k = 3}^{n} \sqrt[5]{k}- 5\sum_{k = 3}^{n} \sqrt[5]{k + 1} 
\]
\[
= 2 \sum_{k = 1}^{n - 2} \sqrt[5]{k} + 3 \sum_{k = 3}^{n} \sqrt[5]{k} - 5\sum_{k = 4}^{n + 1} \sqrt[5]{k} = 2 \sqrt[5]{1} + 2 \sqrt[5]{2} + 5 \sqrt[5]{3} - 2 \sqrt[5]{n - 1} - 2 \sqrt[5]{n}  - 5 \sqrt[5]{n + 1} \underset{n \rightarrow \infty}{\longrightarrow} - \infty
\]
\begin{center}
\textbf{Ряд расходится} 
\end{center}
\subsection*{2.}
\[
\sum_{n = 1}^{\infty} n \sin \frac{n + 1}{n^2 + 2}
\]
Проверяем по необходимому условию сходимости ряда:


Синус можем разложить в $x + \overline{o}(x)$, т.к аргумент синуса стремится к нулю ($n^2$ в знаменателе сильнее) при $n \rightarrow \infty$
\[
a_n = n \left(\frac{n + 1}{n^2 + 2} + \overline{o} \left[\frac{n + 1}{n^2 + 2} \right] \right) = \frac{n^2 + n}{n^2 + 2} + \overline{o}(1) = \frac{1 + \frac{1}{n}}{1 + \frac{2}{n^2}} \underset{n \rightarrow \infty}{\longrightarrow} 1 \neq 0 
\]
Необходимое условие не выполняется $\rightarrow$ ряд \textbf{расходится}
\begin{center}
\textbf{Ч.Т.Д} 
\end{center}
\end{document}
