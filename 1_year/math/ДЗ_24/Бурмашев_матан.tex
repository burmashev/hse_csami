\documentclass[a4paper,12pt]{article}

%%% Работа с русским языком
\usepackage{cmap}					% поиск в PDF
\usepackage{mathtext} 				% русские буквы в формулах
\usepackage[T2A]{fontenc}			% кодировка
\usepackage[utf8]{inputenc}			% кодировка исходного текста
\usepackage[english,russian]{babel}	% локализация и переносы
\usepackage{xcolor}
\usepackage{hyperref}
 % Цвета для гиперссылок
\definecolor{linkcolor}{HTML}{799B03} % цвет ссылок
\definecolor{urlcolor}{HTML}{799B03} % цвет гиперссылок

\hypersetup{pdfstartview=FitH,  linkcolor=linkcolor,urlcolor=urlcolor, colorlinks=true}

%%% Дополнительная работа с математикой
\usepackage{amsfonts,amssymb,amsthm,mathtools} % AMS
\usepackage{amsmath}
\usepackage{icomma} % "Умная" запятая: $0,2$ --- число, $0, 2$ --- перечисление

%% Номера формул
%\mathtoolsset{showonlyrefs=true} % Показывать номера только у тех формул, на которые есть \eqref{} в тексте.

%% Шрифты
\usepackage{euscript}	 % Шрифт Евклид
\usepackage{mathrsfs} % Красивый матшрифт

%% Свои команды
\DeclareMathOperator{\sgn}{\mathop{sgn}}

%% Перенос знаков в формулах (по Львовскому)
\newcommand*{\hm}[1]{#1\nobreak\discretionary{}
{\hbox{$\mathsurround=0pt #1$}}{}}
% графика
\usepackage{graphicx}
\graphicspath{{pictures/}}
\DeclareGraphicsExtensions{.pdf,.png,.jpg}
\author{Бурмашев Григорий, БПМИ-208}
\title{}
\date{\today}
\begin{document}
\begin{center}
Бурмашев Григорий. 208. Матан -- 14
\end{center}
\begin{center}
Ну я даже мемы вставлять не буду, прекол какой-то
\end{center}
\section*{Номер 12}
\subsection*{а)}
\[
x = u \cos v , y = u \sin v 
\]
\[
A =
\begin{pmatrix}
\frac{\sigma x}{\sigma u} & \frac{\sigma x}{\sigma v} \\
\frac{\sigma y}{\sigma u } &
\frac{\sigma y}{\sigma v}
\end{pmatrix} = 
\begin{pmatrix}
\cos v & -u \sin v\\
\sin v & u \cos v
\end{pmatrix}
\]
\subsection*{b)}
\[
x = u v w, y = uv - uvw, z = v - uv
\]
\[
A = 
\begin{pmatrix}
\frac{\sigma x}{\sigma u} & \frac{\sigma x}{\sigma v} & \frac{\sigma x}{\sigma w} \\
\frac{\sigma y}{\sigma u} & \frac{\sigma y}{\sigma v} &  \frac{\sigma y}{\sigma w} \\
\frac{\sigma z}{\sigma u} & \frac{\sigma z}{\sigma v}  &\frac{\sigma z}{\sigma w}
\end{pmatrix} = 
\begin{pmatrix}
vw & uw & uv \\v - vw & u - uw& -uv \\
 -v & 1 - u & 0 
\end{pmatrix}
\]
\end{document}
