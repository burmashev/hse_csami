\documentclass[a4paper,12pt, fleqn]{article}

%%% Работа с русским языком
\usepackage{cmap}					% поиск в PDF
\usepackage{mathtext} 				% русские буквы в формулах
\usepackage[T2A]{fontenc}			% кодировка
\usepackage[utf8]{inputenc}			% кодировка исходного текста
\usepackage[english,russian]{babel}	% локализация и переносы
\usepackage{xcolor}
\usepackage{hyperref}
 % Цвета для гиперссылок
\definecolor{linkcolor}{HTML}{799B03} % цвет ссылок
\definecolor{urlcolor}{HTML}{799B03} % цвет гиперссылок

\hypersetup{pdfstartview=FitH,  linkcolor=linkcolor,urlcolor=urlcolor, colorlinks=true}

%%% Дополнительная работа с математикой
\usepackage{amsfonts,amssymb,amsthm,mathtools} % AMS
\usepackage{amsmath}
\usepackage{icomma} % "Умная" запятая: $0,2$ --- число, $0, 2$ --- перечисление

%% Номера формул
%\mathtoolsset{showonlyrefs=true} % Показывать номера только у тех формул, на которые есть \eqref{} в тексте.

%% Шрифты
\usepackage{euscript}	 % Шрифт Евклид
\usepackage{mathrsfs} % Красивый матшрифт

%% Свои команды
\DeclareMathOperator{\sgn}{\mathop{sgn}}

%% Перенос знаков в формулах (по Львовскому)
\newcommand*{\hm}[1]{#1\nobreak\discretionary{}
{\hbox{$\mathsurround=0pt #1$}}{}}
% графика
\usepackage{graphicx}
\graphicspath{{pictures/}}
\DeclareGraphicsExtensions{.pdf,.png,.jpg}
\author{Бурмашев Григорий, БПМИ-208}
\title{}
\date{\today}
\begin{document}
\begin{center}
Бурмашев Григорий. Алгебра -- 3. 208
\end{center}
\section*{Номер 1}
\[
\mathbb{Z}_{2} \times \mathbb{Z}_{10} \times \mathbb{Z}_{25}
\]
Ну собственно  делаем 1 в 1 как на семинаре. Рассмотрим все варианты:
\begin{itemize}
\item Порядок 2:
\[
\begin{cases}
(2a, 2b, 2c) = 0 \\
(a, b, c) \neq 0
\end{cases}
\]
\[
2a = 0 \rightarrow \text{ 2 варианта (0, 1)}
\]
\[
2b = 0 \rightarrow \text{ 2 варианта (0, 5)}
\]
\[
2c = 0 \rightarrow \text{ 1 вариант (0)}
\]
Всего 4 варианта 2x = 0, но вариант x = 0 не подходит, остается 3 элемента порядка 2
\item Порядок 5:
\[
\begin{cases}
(5a, 5b, 5c ) = 0 \\
(a, b, c) \neq 0
\end{cases}
\]
\[
5a = 0 \rightarrow \text{ 1 вариант (0)}
\]
\[
5b = 0 \rightarrow \text{ 5 вариантов (0, 2, 4, 6, 8)}
\]
\[
5c = 0 \rightarrow \text{ 5 вариантов (0, 5, 10, 15, 20)}
\]
Всего 25 вариантов, но вариант x = 0 не подходит, остается 24 элемента порядка 5 
\item Порядок 10:
\[
\begin{cases}
(10a, 10b, 10c ) = 0 \\
(a, b, c) \neq 0
\end{cases}
\]
\[
10a = 0 \rightarrow \text{ 2 варианта (0, 1)}
\]
\[
10b = 0 \rightarrow \text{ 10 вариантов (0, 1, 2, \ldots,9)}
\]
\[
10c = 0 \rightarrow \text{ 5 вариантов (0, 5, 10, 15, 20)}
\]
Всего 100 вариантов, нам не подходят $x = 0, x = 3$ и $x = 5$, т.е остается  $100 - 3 - 24 -1 = 72$ элементов порядка 10
\clearpage
\item Порядок 25:
\[
\begin{cases}
(25a, 25b, 25c ) = 0 \\
(5a, 5b, 5c) \neq 0
\end{cases}
\]
\[
25a = 0 \rightarrow \text{ 1 вариант (0)}
\]
\[
25b = 0 \rightarrow \text{ 5 вариантов (0, 2, 4, 6, 8)}
\]
\[
25c = 0 \rightarrow \text{ 25 вариантов (0, 1, \ldots, 24)}
\]
Всего 125 вариантов, нам не подходят $x = 0$ и $x = 5$ , т.е  остается $125 - 24 - 1 = 100$ элементов порядка 25
\end{itemize}
\begin{center}
\textbf{Ответ: } $3, 24, 72, 100$
\end{center}
\section*{Номер 2}
Заметим, что $90 = 2 \cdot 3^2 \cdot 5$, тогда делаем точь в точь аналогично семинару:
\[
\mathbb{Z}_2 \times \mathbb{Z}_3 \times \mathbb{Z}_3 \times \mathbb{Z}_5 = A \; \leftarrow  \text{ подходит нам}
\]
\[
\mathbb{Z}_2 \times \mathbb{Z}_9 \times \mathbb{Z}_5 \cong \mathbb{Z}_{90} \; \leftarrow \text{ циклическая группа}  
\]
\[
\]
Теперь смотрим на подгруппы:
\begin{itemize}
\item Порядок $3$,  т.е $\mathbb{Z}_3$. Интересуют циклические подгруппы порядка 3, такие подгруппы порождаются элементами порядка 6, найдем число элементов:
\[
x = (a, b, c, d) 
\]
\[
3a = 0 \rightarrow \text{ 1 вариант (0)}
\]
\[
3b = 0 \rightarrow \text{ 3 варианта (0, 1, 2)}
\]
\[
3c= 0 \rightarrow \text{ 3 варианта (0, 1, 2)}
\]
\[
3d = 0 \rightarrow \text{ 1 вариант (0)}
\]
Всего 9 вариантов, нам не подходит $x = 0$, значит остается ${9 - 1 = 8}$ элементов порядка 3.

Подгруппа порядка 3 $ \approx\mathbb{Z}_3 = \{0, 1, 2\}$. 

ord : 1, 3, 3 (по порядку как записаны в множестве). Два элемента имеют порядок 3. Итого число подгрупп порядка 3 будет $\frac{8}{2} = 4$
\item  Порядок 15, 15 = $5 \cdot 3$, т.е $\mathbb{Z}_3 \times \mathbb{Z}_5 \cong \mathbb{Z}_{15}$. Интересуют циклические подгруппы порядка 15, такие подгруппы порождаются элементами порядка 16, найдем число элементов:
\[
x = (a, b, c, d) 
\]
\[
15a = 0 \rightarrow \text{ 1 вариант (0)}
\]
\[
15b = 0 \rightarrow \text{ 3 варианта (0, 1, 2)}
\]
\[
15c= 0 \rightarrow \text{ 3 вариант (0, 1, 2)}
\]
\[
15d = 0 \rightarrow \text{ 5 вариантов (0, 1, 2, 3)}
\]
Всего 45 вариантов, нам не подходит $x = 0$, а также варианты $x = 3$ и $x = 5$, посмотрим для $x = 5$: 
\[
5a = 0 \rightarrow \text{ 1 вариант (0)}
\]
\[
5b = 0 \rightarrow \text{ 1 вариант (0)}
\]
\[
5c = 0 \rightarrow \text{ 1 вариант (0)}
\]
\[
5d = 0 \rightarrow \text{ 5 вариантов (0, 1, 2, 3)}
\]

Значит остается $45 - 5 - 8  = 32$ элемента

Подгруппа порядка 15 $\approx \mathbb{Z}_{15} = \{0, 1, 2, 3, 4, 5, 6, 7, 8, 9, 10, 11, 12, 13, 14\}$.

ord : 1, 15, 15, 5, 15, 3, 5, 15, 15, 5, 3, 15, 5, 15, 15 (по порядку как записаны в множестве). 8 элементов имеют порядок 15. Итого число подгрупп порядка 15 будет $\frac{32}{8} = 4$
\end{itemize}
\begin{center}
\textbf{Ответ: } 4, 4
\end{center}
\clearpage
\section*{Номер 3}
Кажется, что стоит в тупую порасписывать прямое произведение по следствию Лагранжа (точнее теоремы 4) с лекции, чтобы хоть какие-то оценки на ответ увидеть:
\[
\mathbb{Z}_{15} \times \mathbb{Z}_{18} \times \mathbb{Z}_{20} \cong \mathbb{Z}_{5} \times \mathbb{Z}_{3} \times \mathbb{Z}_{9} \times \mathbb{Z}_{2} \times \mathbb{Z}_{5} \times \mathbb{Z}_{4}
\]
Чтобы минимизировать $n$, нужно объединить 5, 3, 2 в 30, а 9, 5, 4 в 180 и получить $n = 2$, т.е:
\[
\mathbb{Z}_{15} \times \mathbb{Z}_{18} \times \mathbb{Z}_{20} \cong \mathbb{Z}_{30} \times \mathbb{Z}_{180}
\]
Получили ответ 2, большие $n$ нас очев уже не интересуют, из меньших остается только 1. Поэтому нужно найти строгое док-во невозможности $n = 1$. Для этого может пригодится предложение 6 из лекции, т.е критерий цикличности. 

Пойдем от обратного, пусть $n = 1$ -- ответ. Тогда наша группа должна быть изоморфна какой-то \textbf{одной} циклической группе, а значит и она сама должна являться циклической. Проверим по предложению 6 цикличность, должно выполняться следующее (пусть наша группа есть A):
\[
\text{exp } A = |A|
\]
Считаем:
\[
\text{exp } A = \text{НОК} \{\;ord(a)\; | \;a \in A \} = \text{НОК} (15, 18, 20) = 180
\]
\[
|A| = \big| \text{по замечанию 7.2 с лекции} \big| = 15 \cdot 18 \cdot 20 = 5400
\]
Получили, что $\text{exp } A \neq |A| \rightarrow$ группа не циклична, а значит мы приходим к \textbf{противоречию} и ответ действительно 2.
\begin{center}
\textbf{Ответ: } $ n = 2 $
\end{center}

\end{document}
