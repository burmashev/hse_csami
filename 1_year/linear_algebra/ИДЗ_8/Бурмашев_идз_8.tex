\documentclass[a4paper,12pt]{article}

%%% Работа с русским языком
\usepackage{cmap}					% поиск в PDF
\usepackage{mathtext} 				% русские буквы в формулах
\usepackage[T2A]{fontenc}			% кодировка
\usepackage[utf8]{inputenc}			% кодировка исходного текста
\usepackage[english,russian]{babel}	% локализация и переносы
\usepackage{xcolor}
\usepackage{hyperref}
 % Цвета для гиперссылок
\definecolor{linkcolor}{HTML}{799B03} % цвет ссылок
\definecolor{urlcolor}{HTML}{799B03} % цвет гиперссылок

\hypersetup{pdfstartview=FitH,  linkcolor=linkcolor,urlcolor=urlcolor, colorlinks=true}

%%% Дополнительная работа с математикой
\usepackage{amsfonts,amssymb,amsthm,mathtools} % AMS
\usepackage{amsmath}
\usepackage{icomma} % "Умная" запятая: $0,2$ --- число, $0, 2$ --- перечисление

%% Номера формул
%\mathtoolsset{showonlyrefs=true} % Показывать номера только у тех формул, на которые есть \eqref{} в тексте.

%% Шрифты
\usepackage{euscript}	 % Шрифт Евклид
\usepackage{mathrsfs} % Красивый матшрифт

%% Свои команды
\DeclareMathOperator{\sgn}{\mathop{sgn}}

%% Перенос знаков в формулах (по Львовскому)
\newcommand*{\hm}[1]{#1\nobreak\discretionary{}
{\hbox{$\mathsurround=0pt #1$}}{}}
% графика
\usepackage{graphicx}
\graphicspath{{pictures/}}
\DeclareGraphicsExtensions{.pdf,.png,.jpg}
\author{Бурмашев Григорий, БПМИ-208}
\title{}
\date{\today}
\begin{document}
\begin{center}
Бурмашев Григорий. 208. Идз -- 8 
\end{center}

\section*{Номер 1}
\[
A = \begin{pmatrix}
5 & -3 & -3 \\
14 & -14 & -17 \\
-8 & 10 & 13
\end{pmatrix}
\]

Ищем собственные значения, для этого считаем det$(A - \lambda E$):
\[
\begin{vmatrix}
5 - \lambda& -3 & -3 \\
14 & -14 - \lambda & -17 \\
-8 & 10 & 13 - \lambda \\
\end{vmatrix} =
-\lambda^3 + 4\lambda^2 - \lambda -6 = -(\lambda + 1) (\lambda - 2) (\lambda -3 )
\]
Получили три собственных значения : $\lambda = -1, \lambda = 2, \lambda = 3$. 

Все алгебраические кратности равны 1
\\\\
Теперь ищем базисы собственных подпространств:

\begin{itemize}
\item $\lambda = -1 $:

\[
\begin{pmatrix}
6 & -3 & -3 & \\
14 & -13 & -17 & \\
-8 & 10 & 14 & \\
\end{pmatrix}
=
\begin{pmatrix}
6 & -3 & -3 \\
2 & -7 & -11  \\
-8 & 10 & 14  \\
\end{pmatrix}
=
\]
\[
=
\begin{pmatrix}
0 & 18 & 30  \\
2 & -7 & -11  \\
-8 & 10 & 14  \\
\end{pmatrix}
=
\begin{pmatrix}
0 & 18 & 30 \\
2 & -7 & -11 \\
0 & -18 & -30  \\
\end{pmatrix}
=
\]
\[
=
\begin{pmatrix}
2 & -7 & -11\\
0 & 3& 5  \\
\end{pmatrix}
=
\begin{pmatrix}
1 & 0 & \frac13 \\
0 & 1& \frac53  \\
\end{pmatrix}
\]
Получаем ФСР:
\[
\begin{pmatrix}
-\frac13 \\-\frac53 \\ 1
\end{pmatrix}
\]

\[
g_{\lambda} = 1
\]
\clearpage
\item $\lambda = 2$:
\[
\begin{pmatrix}
1 & -1 & -1  \\
14 & -16 & -17 \\
-8 & 10 & 11 \\
\end{pmatrix}
=
\begin{pmatrix}
1 & -1 & -1 \\
0 & -2 & -3  \\
-8 & 10 & 11  \\
\end{pmatrix}
=
\]
\[
=
\begin{pmatrix}
1 & -1 & -1  \\
0 & 2 &3 \\
\end{pmatrix}
=
\begin{pmatrix}
1 & 0 & \frac12 \\
0 & 0 & \frac32\\
\end{pmatrix}
\]
Получаем ФСР:
\[
\begin{pmatrix}
-\frac12 \\ -\frac32 \\ 1
\end{pmatrix}
\]
\[
g_{\lambda} = 1
\]
\item $\lambda = 3$:
\[
\begin{pmatrix}
2 & -3 & -3 \\
14 & -17 & -17  \\
-8 & 10 & 10 \\
\end{pmatrix}
=
\begin{pmatrix}
2 & -3 & -3 \\
0 & 4 & 4 \\
-8 & 10 & 10 \\
\end{pmatrix}
=
\]
\[
=
\begin{pmatrix}
2 & -3 & -3 \\
0 & 4 & 4 \\
0 & -2 & -2  \\
\end{pmatrix}
=
\begin{pmatrix}
1 & 0 & 0  \\
0 & 1 & 1  \\
\end{pmatrix}
\]
Получаем ФСР:
\[
\begin{pmatrix}
0 \\ -1 \\ 1
\end{pmatrix}
\]
\[
g_{\lambda} = 1
\]
\end{itemize}
Получили, что $a_{\lambda} = g_{\lambda}$ для всех получившихся собственных значений, а значит $\varphi$ диагонализуем
\clearpage
Фиксируем базис:
\[
e_1' = 
\begin{pmatrix}
-\frac13 \\-\frac53 \\ 1
\end{pmatrix}
\]
\[
e_2' = 
\begin{pmatrix}
-\frac12 \\ -\frac32 \\ 1
\end{pmatrix}
\]
\[
e_3' =
\begin{pmatrix}
0 \\ -1 \\ 1
\end{pmatrix}
\]

Матрица перехода C:
\[
C = \begin{pmatrix}
-\frac13 & -\frac12 & 0 \\
-\frac53 & -\frac32 & -1 \\
1 & 1 & 1
\end{pmatrix}
\]
Тогда знаем, что матрица в новом базисе будет задаваться формулой:
\[
A' = C^{-1}AC 
\]
Cчитаем A':
\[
C^{-1} = \begin{pmatrix}
3 & -3 & -3 \\
-4 & 2 & 2 \\
1 & 1 & 2
\end{pmatrix}
\]
\[
A' = \begin{pmatrix}
3 & -3 & -3 \\
-4 & 2 & 2 \\
1 & 1 & 2
\end{pmatrix} \cdot \begin{pmatrix}
5 & -3 & -3 \\
14 & -14 & -17 \\
-8 & 10 & 13
\end{pmatrix} \cdot \begin{pmatrix}
-\frac13 & -\frac12 & 0 \\
-\frac53 & -\frac32 & -1 \\
1 & 1 & 1
\end{pmatrix} =
\]
\[
=
\left(\begin{matrix}
-3 & 3 & 3 \\
-8 & 4 & 4 \\
3 & 3 & 6
\end{matrix}\right) \cdot \begin{pmatrix}
-\frac13 & -\frac12 & 0 \\
-\frac53 & -\frac32 & -1 \\
1 & 1 & 1
\end{pmatrix} = \left(\begin{matrix}
-1 & 0 & 0 \\
0 & 2 & 0 \\
0 & 0 & 3
\end{matrix}\right)
\]
\clearpage
\begin{center}
\textbf{Ответ: } 

$\lambda = -1$, базис соотв.подпространства:
\[
\begin{pmatrix}
-\frac13 \\-\frac53 \\ 1
\end{pmatrix}
\]
$\lambda = 2$, базис соотв.подпространства: 
\[
\begin{pmatrix}
-\frac12 \\ -\frac32 \\ 1
\end{pmatrix}
\]
$\lambda = 3$, базис соотв.подпространства:  
\[
\begin{pmatrix}
0 \\ -1 \\ 1
\end{pmatrix}
\]

$\varphi$ диагонализуем

базис:
\[
e_1' = 
\begin{pmatrix}
-\frac13 \\-\frac53 \\ 1
\end{pmatrix},
e_2' = 
\begin{pmatrix}
-\frac12 \\ -\frac32 \\ 1
\end{pmatrix},
e_3' =
\begin{pmatrix}
0 \\ -1 \\ 1
\end{pmatrix}
\]
матрица:
\[
A' = 
\left(\begin{matrix}
-1 & 0 & 0 \\
0 & 2 & 0 \\
0 & 0 & 3
\end{matrix}\right)
\]
\end{center}
\clearpage
\section*{Номер 2}
\[
A = \begin{pmatrix}
1 & -1 & -1 \\
5 & 6 & 4 \\
-1 & -1 & 1 \\
\end{pmatrix}
\]
Аналогично предыдущему номеру:
\[
\begin{vmatrix}
1 - \lambda & -1 & -1 \\
5 & 6 - \lambda  & 4 \\
-1 & -1 & 1 - \lambda \\
\end{vmatrix} = -\lambda^3 + 8 \lambda^2 - 21\lambda + 18 = -(\lambda - 3)^2 \cdot (\lambda - 2)
\]
Получили два собственных значения : $\lambda = 3, \lambda = 2$. 

Причем у $\lambda  = 3$ : $a_{\lambda} = 2$

Теперь ищем базисы собственных подпространств:
\begin{itemize}
\item $\lambda = 3$:
\[
\begin{pmatrix}
-2 & -1 & -1 \\
5 & 3 & 4 \\
-1 & -1 & -2 \\
\end{pmatrix}
=
\begin{pmatrix}
0 & 1 & 3 \\
5 & 3 & 4 \\
-1 & -1 & -2 \\
\end{pmatrix}
=
\]
\[
=
\begin{pmatrix}
0 & 1 & 3 \\
0 & -2 & -6  \\
-1 & -1 & -2  \\
\end{pmatrix}
=
\begin{pmatrix}
1 & 0 & -1  \\
0 & 1 & 3 \\
\end{pmatrix}
\]
Получаем ФСР:
\[
\begin{pmatrix}
1 \\ -3 \\ 1
\end{pmatrix}
\]
\[
g_{\lambda} = 1  \neq a_{\lambda}
\]
Значит критерий диагонализуемости не выполнен и диагонализовать мы не сможем.
\item $\lambda = 2$:
\[
\begin{pmatrix}
-1 & -1 & -1 \\
5 & 4 & 4 \\
-1 & -1 & -1 \\
\end{pmatrix}
=
\begin{pmatrix}
-1 & -1 & -1 \\
0 & -1 & -1  \\
\end{pmatrix}
=
\]
\[
=
\begin{pmatrix}
1 & 0 & 0  \\
0 & 1 & 1 \\
\end{pmatrix}
\]
Получаем ФСР:
\[
\begin{pmatrix}
0 \\ -1 \\ 1
\end{pmatrix}
\]
\end{itemize}
\begin{center}
\textbf{Ответ: } 

$\lambda = 3$, базис соотв.подпространства: 
\[
\begin{pmatrix}
1 \\ -3 \\ 1
\end{pmatrix}
\]
$\lambda = 2$, базис соотв.подпространства:  
\[
\begin{pmatrix}
0 \\ -1 \\ 1
\end{pmatrix}
\]
\end{center}
\clearpage
\section*{Номер 3}
\[
Q(x_1, x_2, x_3) = 14x_1^2 + 11x_2^2 + 11x_3^3 - 8x_1x_2 - 8x_1x_3 - 2x_2x_3
\]

Выпишем матрицу квадратичной формы:
\[
A = \begin{pmatrix}
14 & -4& -4 \\
-4 & 11 & -1 \\
-4 & -1 & 11
\end{pmatrix}
\]
Ищем собственные значения:
\[
\begin{vmatrix}
14 - \lambda& -4& -4 \\
-4 & 11 - \lambda & -1 \\
-4 & -1 & 11 - \lambda 
\end{vmatrix} = -\lambda^3 + 36\lambda^2 - 396\lambda + 1296 = -(\lambda - 6)(\lambda  -12)(\lambda - 18)
\]
Получили три собственных значения : $\lambda = 6, \lambda = 12, \lambda = 18$

Все алгебраические кратности равны 1
\\\\
Считаем:
\begin{itemize}
\item $\lambda = 6$:
\[
\begin{pmatrix}
8 & -4 & -4  \\
-4 & 5 & -1  \\
-4 & -1 & 5  \\
\end{pmatrix}
=
\begin{pmatrix}
0 & 6 & -6  \\
-4 & 5 & -1  \\
-4 & -1 & 5  \\
\end{pmatrix}
=
\]
\[
=
\begin{pmatrix}
0 & 1 & -1 \\
0 & 6 & -6 \\
-4 & -1 & 5  \\
\end{pmatrix}
=
\begin{pmatrix}
1 & 0 & -1 \\
0 & 1 & -1 \\
\end{pmatrix}
=
\]
Получаем ФСР:
\[
\begin{pmatrix}
1 \\ 1 \\ 1
\end{pmatrix}
\]
\item $\lambda = 12$:
\[
\begin{pmatrix}
2& -4& -4 \\
-4 & -1 & -1 \\
-4 & -1 & -1 \\
\end{pmatrix}
=
\begin{pmatrix}
1& -2& -2 \\
-4 & -1 & -1 \\
\end{pmatrix}
=
\]
\[
=
\begin{pmatrix}
1& 0& 0 \\
0 & 1 & 1 \\
\end{pmatrix}
\]
Получаем ФСР:
\[
\begin{pmatrix}
0 \\ -1 \\ 1
\end{pmatrix}
\]
\item $\lambda = 18$:
\[
\begin{pmatrix}
-4 & -4 & -4  \\
-4 & -7 & -1  \\
-4 & -1 & -7  \\
\end{pmatrix}
=
\begin{pmatrix}
1 & 1 & 1  \\
0 & -3 & 3  \\
-4 & -1 & -7  \\
\end{pmatrix}
=
\]
\[
=
\begin{pmatrix}
1 & 1 & 1  \\
0 & 1 & -1 \\
0 & 1 & -1 \\
\end{pmatrix}
=
\begin{pmatrix}
1 & 1 & 1 \\
0 & -1 & 1  \\
\end{pmatrix}
=
\]
\[
=
\begin{pmatrix}
1 & 2 & 0  \\
0 & 1 & -1  \\
\end{pmatrix}
=
\]
Получаем ФСР:
\[
\begin{pmatrix}
-2 \\ 1 \\ 1
\end{pmatrix}
\]
\end{itemize}
Получили, что $a_{\lambda} = g_{\lambda}$ для всех получившихся собственных значений, а значит $\varphi$ диагонализуем. Тогда положим:
\[
v_1 = 
\begin{pmatrix}
1 \\ 1 \\ 1
\end{pmatrix}
\]
\[
v_2 = 
\begin{pmatrix}
0 \\ -1 \\ 1
\end{pmatrix}
\]
\[
v_3 = 
\begin{pmatrix}
-2 \\ 1 \\ 1
\end{pmatrix}
\]
Теперь ортонормируем:
\[
e_1' = \frac{v_1}{|v_1|} = \begin{pmatrix}
\frac{1}{\sqrt{3}} \\ \frac{1}{\sqrt{3}} \\ \frac{1}{\sqrt{3}}
\end{pmatrix}
\]
\[
e_2' = \frac{v_2}{|v_2|} = \begin{pmatrix}
0 \\ \frac{-1}{\sqrt{2}} \\ \frac{1}{\sqrt{2}}
\end{pmatrix}
\]
\[
e_3' =  \frac{v_3}{|v_3|} = \begin{pmatrix}
\frac{-2}{\sqrt{6}} \\ \frac{1}{\sqrt{6}}  \\ \frac{1}{\sqrt{6}} 
\end{pmatrix}
\]
Матрица перехода от исходного базиса к $e'$:
\[
C = \begin{pmatrix}
\frac{1}{\sqrt{3}} & 0 & -\frac{2}{\sqrt{6}} \\
\frac{1}{\sqrt{3}}  & -\frac{1}{\sqrt{2}} & \frac{1}{\sqrt{6}}\\
\frac{1}{\sqrt{3}}  & \frac{1}{\sqrt{2}} & \frac{1}{\sqrt{6}}
\end{pmatrix}
\]
Матрица кв.формы:
\[
A' = C^{-1} A C = \begin{pmatrix}
6 & 0 & 0 \\
0 & 12 & 0 \\
0 & 0 & 18
\end{pmatrix}
\]
Выражение старых координат через новые:
\[
\begin{cases}
x_1 = \frac{x_1'}{\sqrt{2}}- \frac{2x_3'}{\sqrt{6}} \\
x_2 = \frac{x_1' }{\sqrt{2}}- \frac{x_2'}{\sqrt{2}}+\frac{ x_3'}{\sqrt{6}} \\
x_3 = \frac{x_1'}{\sqrt{2}} + \frac{x_2'}{\sqrt{2}}+ \frac{x_3'}{\sqrt{6}}
\end{cases}
\]
Cама квадратичная форма в новом базисе:
\[
Q(x_1', x_2', x_3') = 6(x_1')^2 + 12(x_2')^2 + 18(x_3')^2
\]

\begin{center}
\textbf{Ответ: } 

выражение координат:
\[
\begin{cases}
x_1 = \frac{x_1'}{\sqrt{2}}- \frac{2x_3'}{\sqrt{6}} \\
x_2 = \frac{x_1' }{\sqrt{2}}- \frac{x_2'}{\sqrt{2}}+\frac{ x_3'}{\sqrt{6}} \\
x_3 = \frac{x_1'}{\sqrt{2}} + \frac{x_2'}{\sqrt{2}}+ \frac{x_3'}{\sqrt{6}}
\end{cases}
\]
квадратичная форма:
\[
Q(x_1', x_2', x_3') = 6(x_1')^2 + 12(x_2')^2 + 18(x_3')^2
\]
\end{center}
\clearpage
\section*{Номер 4}
\begin{center}
[решаю по аналогии с семинаром 205 под номером 30]
\end{center}
\[
A = \left(\begin{matrix}
\frac{-1}{9} & \frac{-4}{9} & \frac{-8}{9} \\
\frac{8}{9} & \frac{-4}{9} & \frac{1}{9} \\
\frac{4}{9} & \frac{7}{9} & \frac{-4}{9}
\end{matrix}\right)
\]
Из конспекта знаем, что существует базис, в котором матрица будет выглядеть как:
\[
\begin{pmatrix}
\Pi(\alpha) & 0 \\
0 & \lambda_0 
\end{pmatrix}, \Pi(\alpha) = \begin{pmatrix}
\cos \alpha & - \sin \alpha \\
\sin \alpha & \cos \alpha
\end{pmatrix}, \lambda_0 \in \{1, -1\}
\]
Видим, что $A \neq A^T$, тогда ищем собственные значения среди -1 и 1:
\begin{itemize}
\item $\lambda = -1$:
\[
\left(\begin{matrix}
-\frac{1}{9} - (-1) & \frac{-4}{9} & \frac{-8}{9} \\
\frac{8}{9} & -\frac{4}{9} - (-1)& \frac{1}{9} \\
\frac{4}{9} & \frac{7}{9} & -\frac{4}{9} - (-1) 
\end{matrix}\right) = \left(\begin{matrix}
\frac89 & \frac{-4}{9} & \frac{-8}{9} \\
\frac{8}{9} & \frac59& \frac{1}{9} \\
\frac{4}{9} & \frac{7}{9} & \frac59
\end{matrix}\right)  =
\]
\[
=\left(\begin{matrix}
\frac{8}{9} & \frac{-4}{9} & \frac{-8}{9} \\
0 & 1 & 1 \\
\frac{4}{9} & \frac{7}{9} & \frac{5}{9}
\end{matrix}\right) = \left(\begin{matrix}
\frac{8}{9} & \frac{-4}{9} & \frac{-8}{9} \\
0 & 1 & 1 \\
0 & 1 & 1
\end{matrix}\right) = 
\]
\[
=
\left(\begin{matrix}
2 & -1 & -2 \\
0 & 1 & 1 \\
\end{matrix}\right)
=
\begin{pmatrix}
1 & 0 & -\frac12 \\0 & 1 & 1
\end{pmatrix}
\]
\end{itemize}
Видим, что ранг уменьшился и $\lambda = -1$ нам подходит, тогда имеем собственный вектор $\left(\frac12, -1, 1\right)$. Теперь положим:
\[
e_3 = \frac23 \cdot \begin{pmatrix}
\frac12 \\ -1 \\ 1
\end{pmatrix}
\]
Тогда:
\[
e_3^\perp = \langle \begin{pmatrix}
0 \\ 1 \\ 1
\end{pmatrix}, \begin{pmatrix}
-2 \\ 1 \\ 2
\end{pmatrix}\rangle
\]
Т.к $(e_3, (0, 1, 1)) = 0$, $(e_3, (-2, 1, 2)) = 0$.
\clearpage
Пусть:
\[
v_1 = \begin{pmatrix}
0 \\ 1 \\ 1
\end{pmatrix}
\]
\[
v_2 = \begin{pmatrix}
-2 \\ 1 \\ 2
\end{pmatrix}
\]
Ортогонализуем:
\[
u_1 =  \begin{pmatrix}
0 \\ 1 \\ 1
\end{pmatrix}
\]
\[
u_2 = v_2 - \frac{(v_2, u_1)}{(u_1, u_1)} \cdot u_1 = \begin{pmatrix}
-2 \\ 1 \\ 2
\end{pmatrix} - \frac{3}{2} \cdot \begin{pmatrix} 0 \\ 1 \\ 1 \end{pmatrix} = \begin{pmatrix} -2 \\ -\frac12 \\ \frac12 \end{pmatrix}
\]
А теперь ортонормируем:
\[
e_1 = \frac{u_1}{|u_1|} = \frac{1}{\sqrt{2}} \cdot \begin{pmatrix}
0 \\ 1 \\ 1
\end{pmatrix}
\]
\[
e_2 = \frac{u_2}{|u_2|} = \frac{\sqrt{2}}{3} \cdot \begin{pmatrix} -2 \\ -\frac12 \\ \frac12 \end{pmatrix}
\]
Теперь считаем:
\[
\varphi(e_1) = A e_1 = \frac{1}{9} \cdot \begin{pmatrix}
-1 & -4 & -8 \\ 8 & -4 & 1 \\ 4 & 7 & -4 
\end{pmatrix} \cdot \frac{1}{\sqrt{2}} \cdot \begin{pmatrix}
0 \\ 1 \\ 1
\end{pmatrix} = \frac{1}{9 \sqrt{2}}\left(\begin{matrix}
-12 \\
-3 \\
3
\end{matrix}\right) = \frac{1}{3 \sqrt{2}} \begin{pmatrix}
-4 \\ -1 \\1 
\end{pmatrix}
\]
Знаем, что: 
\[
\varphi(e_1) = \cos \alpha \cdot e_1 + \sin \alpha \cdot e_2
\]
А значит:
\[
\cos \alpha = (\varphi(e_1), e_1) = 0
\]
\[
\sin \alpha = (\varphi(e_1), e_2) = 1
\] 
По итогу:

в базисе $(e_1, e_2, e_3)$ матрица будет иметь канонический вид, а именно:
\[
\begin{pmatrix}
0 & -1 & 0 \\
1 & 0 & 0 \\
0 & 0 & -1 \\
\end{pmatrix}
\]
\clearpage
\begin{center}
\textbf{Ответ: } 

базис:
\[
e_1 = \frac{1}{\sqrt{2}} \cdot \begin{pmatrix}
0 \\ 1 \\ 1
\end{pmatrix}, 
e_2 = \frac{\sqrt{2}}{3} \cdot \begin{pmatrix} -2 \\ -\frac12 \\ \frac12 \end{pmatrix},
e_3 = \frac23  \cdot \begin{pmatrix}
\frac12 \\ -1 \\ 1
\end{pmatrix}
\]
матрица:
\[
\begin{pmatrix}
0 & -1 & 0 \\
1 & 0 & 0 \\
0 & 0 & -1 \\
\end{pmatrix}
\]
осью будет являться вектор $e_3$

угол поворота : $\alpha = \arccos(0) = \frac{\pi}{2}$
\end{center}
\end{document}
