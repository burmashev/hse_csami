\documentclass[a4paper,12pt]{article}

%%% Работа с русским языком
\usepackage{cmap}					% поиск в PDF
\usepackage{mathtext} 				% русские буквы в формулах
\usepackage[T2A]{fontenc}			% кодировка
\usepackage[utf8]{inputenc}			% кодировка исходного текста
\usepackage[english,russian]{babel}	% локализация и переносы
\usepackage{xcolor}
\usepackage{hyperref}
 % Цвета для гиперссылок
\definecolor{linkcolor}{HTML}{00FFFF} % цвет ссылок
\definecolor{urlcolor}{HTML}{4682B4} % цвет гиперссылок

\hypersetup{pdfstartview=FitH,  linkcolor=linkcolor,urlcolor=urlcolor, colorlinks=true}

%%% Дополнительная работа с математикой
\usepackage{amsfonts,amssymb,amsthm,mathtools} % AMS
\usepackage{amsmath}
\usepackage{icomma} % "Умная" запятая: $0,2$ --- число, $0, 2$ --- перечисление

%% Номера формул
%\mathtoolsset{showonlyrefs=true} % Показывать номера только у тех формул, на которые есть \eqref{} в тексте.

%% Шрифты
\usepackage{euscript}	 % Шрифт Евклид
\usepackage{mathrsfs} % Красивый матшрифт

%% Свои команды
\DeclareMathOperator{\sgn}{\mathop{sgn}}

\usepackage{enumerate}
%% Перенос знаков в формулах (по Львовскому)
\newcommand*{\hm}[1]{#1\nobreak\discretionary{}
{\hbox{$\mathsurround=0pt #1$}}{}}
% графика
\usepackage{graphicx}
\graphicspath{{picture/}}
\DeclareGraphicsExtensions{.pdf,.png,.jpg}
\author{Бурмаzшев Григорий, 208. \href{https://teleg.run/burmashev}{@burmashev}}
\title{Дискретная математика. Коллок -- 1. Определения и задачи по ним.}

\begin{document}
\begin{center}
Бурмашев Григорий. Дискра -- 13
\end{center}
\subsection*{Номер 1}
От противного:
\\\\
Пусть НОД$(a, bc) \neq 1$. Тогда найдется такое простое $q \neq 1$, что:

$a$ делится на $q$ 

 $b$ или $c$ тоже делится на $q$ 
\\
Оно найдется, потому что либо q это и есть сам НОД (в случае если НОД простой), либо это один из простых множителей НОД.
\\\\
Тогда либо общий делитель$(a, b)$ = $q$, либо общий делитель$(a, c)$ = $q$, а значит один из двух НОД уже точно не будет равен 1 и мы получаем противоречие.
\subsection*{Номер 2}
НОД$(74, 47)$ = НОД$(27, 47)$ = НОД$(27, 20)$ = НОД$(7, 20)$ = НОД$(7, 6)$ =  НОД$(1, 6)$ = 1

Тогда воспользуемся расширенным алгоритмом евклида:

\[
74x +  47y = 1
\]
\[
1 = 1 \cdot 1 + 0 \cdot 6 = (7 -6) \cdot 1 + 0 \cdot 6 = 7 \cdot 1 - 6 \cdot 1 = 7 \cdot 1 - 20 \cdot 1- 7 \cdot 2 = 7 \cdot 3 - 20 \cdot 1 =
\]
\[
= 27 \cdot 3 - 20 \cdot 3 - 20 \cdot 1 = 27 \cdot 3 - 20 \cdot 4 = 27 \cdot 3- 47 \cdot 4 - 27 \cdot 4 = 27 \cdot 7 - 47 \cdot 4 = 
\]
\[
= 74 \cdot 7 - 47 \cdot 7 - 47 \cdot 4 = 74 \cdot 7 - 47 \cdot 11
\]
Получается:
\[
74 \cdot 7 + 47 \cdot (-11) = 1
\]
Тогда для (учитывая НОД = 1):
\[
74x + 47y = 0
\]
\[
x = 47t
\]
\[
y = -74t
\]
Тогда общее решение:
\[
x = 2900 \cdot 7 + 47t, \; y = 2900 \cdot (-11) - 74t, \; t \in \mathbb{Z}
\]
Но нужно учитывать, что у нас спрашивают существование решения в неотрицательных целых числах.

x и y находятся в зависимости: при увеличении икса игрик уменьшается. Но при x = 0 игрик получается нецелым ($t = -431,9, \; y = 2900 \cdot (-11) - 74 \cdot (-431,9)$), а значит решений в целых неотрицательных числах не существует. 
\begin{center}
\textbf{Ответ:} не существует
\end{center}
\newpage
\subsection*{Номер 3}
Поскольку a четно и при этом не делится на 4, то его можно представить как $a = 2 \cdot x$. Причем $x$ -- обязательно нечетное число. Можно рассмотреть множество всех делителей числа x. Пусть их n штук, тогдаOF это множество $x_1, x_2, x_3, \ldots x_n$. Причем $x_i$ -- нечетное(т.к $x$ -- нечетное). Они же и являются всеми нечетными делителями числа $a$  (т.к второй множитель в разложении числа $a$ -- это четное число 2 и соотвественно других нечетных делителей мы не найдем). В таком случае всеми четными делителями числа a будут числа $2 \cdot x_1, 2 \cdot x_2, \ldots 2 \cdot x_n$. Их тоже n штук и они все четные. Итого мы получили, что четных делителей столько же, сколько и нечетных. 
\subsection*{Номер 4}
Cтепень тройки должна оканчиваться на 0001. Другими словами, остаток от деления на 10000 должен быть равен 1. Мы знаем теорему Эйлера, применим её для $n = 10000$:
\[
3^{\varphi(10000)}  \equiv 1 \; (\text{mod} \; 10000)
\]
Условие про взаимную простоту соблюдается, т.к 3 и 10000 -- взаимно простые.
\begin{center}
\textbf{Ответ:} да, существует
\end{center}
\subsection*{Номер 5}
\[
p^2 - 1 = (p-1)(p+1)
\]
p -- простое $\rightarrow$ $p-1$ и $p+1$ являются четными. 
Поскольку числа идут друг за другом, то одно из них делится на 4, по итогу $(p-1)(p+1)$ делится на 8 ($4 \cdot 2$). 

Из трех чисел идущих подряд друг за другом одно делится на 3. У нас это числа $p-1, p, p+1$. $p$ точно не делится на три, т.к оно простое, значит либо $p-1$, либо $p+1$ делится на три, но тогда и $(p-1)(p+1)$ тоже делится на три.

Суммируя эти два факта, мы получаем, что $p^2 - 1= (p-1)(p+1)$ делится на 24, т.к $24 = 3 \cdot 8$)
\subsection*{Номер 6}
Если $2^{n!} -1 $ делится на n, то его остаток при делении на n должен быть равен нулю, т.е:
\[
2^{n!} -1 \equiv 0 (\text{mod}\; n)
\]
А тогда:
\[
2^{n!} \equiv 1 (\text{mod} \; n)
\]
Это очень похоже внешне на использование теоремы Эйлера. Поскольку n -- целое, то n! будет делится на $\varphi (n)$, ведь $\varphi(n)$ есть количество остатков по модулю $n$, взаимно простых с $n$, что точно меньше чем $n$. Тогда можно представить его как $\varphi(n) \cdot z$ , где $z$ -- какое-то целое число. Тогда можно использовать теорему Эйлера. Пусть $2^k = x$, тогда:
\[
x^{\varphi (n)} \equiv 1 (\text{mod} \; n) 
\] 
Условие про взаимную простоту выполняется, т.к НОД($x, n$) = 1, что следует из нечетности n. Таким образом, доказано то, что от нас требовалось.
\subsection*{Номер 7}
Можно сложить дроби следующим образом:
\[
\frac{1}{1} + \frac{1}{p-1} = \frac{1 + p -1 }{1 \cdot (p-1)} = \frac{p}{1 \cdot (p-1)}
\]
\[
\frac{1}{2} + \frac{1}{p-2} = \frac{2 + p - 2}{2(p-2)} = \frac{p}{2(p-2)}
\]
Т.е первую дробь с последней, вторую с предпоследней и так далее до конца. Это возможно сделать, поскольку p есть простое число, а значит p-1 четное и всего слагаемых будет четное количество штук. Дроби попарно сложатся в вид (где n точно меньше,чем $p$ и $n$ -- целое):
\[
\frac{p}{n(p-n)}
\]
Если сложить все эти дроби, то числитель в итоге будет кратен p (т.к при приведении к общему знаменателю мы все еще сможем вынести p за одну большую скобку), при этом знаменатель не будет кратен p (т.к все они меньше чем p) и p не сократится. А значит по итогу вся дробь будет кратна p, что и требовалось доказать.

\newpage
\subsection*{Номер 8}
Нужно доказать, что $a, b, c, d, e,$f по отдельности делятся на 11, тогда их произведение будет делиться на $11^6$. 

Если предположить, что  a не делится на 11, тогда по малой теореме Ферма (т.к 11 -- простое число):
\[
a^{11 - 1} \equiv 1 (\text{mod } 11)
\]
Аналогично для $b, c, d, e, f$. 

Тогда в случае, если хотя бы одно из них не делится на 11, их сумма не будет давать остаток 0 при делении на 11 (он уже будет как минимум 1), что противоречит условию задачи $\rightarrow$ $a, b, c, d, e, f$ делятся на 11. А значит их произведение делится на $11^6$
\end{document}