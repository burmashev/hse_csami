\documentclass[a4paper,12pt]{article}

%%% Работа с русским языком
\usepackage{cmap}					% поиск в PDF
\usepackage{mathtext} 				% русские буквы в формулах
\usepackage[T2A]{fontenc}			% кодировка
\usepackage[utf8]{inputenc}			% кодировка исходного текста
\usepackage[english,russian]{babel}	% локализация и переносы
\usepackage{xcolor}
\usepackage{hyperref}
 % Цвета для гиперссылок
\definecolor{linkcolor}{HTML}{799B03} % цвет ссылок
\definecolor{urlcolor}{HTML}{799B03} % цвет гиперссылок

\hypersetup{pdfstartview=FitH,  linkcolor=linkcolor,urlcolor=urlcolor, colorlinks=true}

%%% Дополнительная работа с математикой
\usepackage{amsfonts,amssymb,amsthm,mathtools} % AMS
\usepackage{amsmath}
\usepackage{icomma} % "Умная" запятая: $0,2$ --- число, $0, 2$ --- перечисление

%% Номера формул
%\mathtoolsset{showonlyrefs=true} % Показывать номера только у тех формул, на которые есть \eqref{} в тексте.

%% Шрифты
\usepackage{euscript}	 % Шрифт Евклид
\usepackage{mathrsfs} % Красивый матшрифт

%% Свои команды
\DeclareMathOperator{\sgn}{\mathop{sgn}}

%% Перенос знаков в формулах (по Львовскому)
\newcommand*{\hm}[1]{#1\nobreak\discretionary{}
{\hbox{$\mathsurround=0pt #1$}}{}}
% графика
\usepackage{graphicx}
\graphicspath{{pictures/}}
\DeclareGraphicsExtensions{.pdf,.png,.jpg}
\author{Бурмашев Григорий, БПМИ-208}
\title{Матан, дз -- 6}
\date{\today}
\begin{document}
\maketitle
\section*{Номер 1}
\[
\int\limits_0^{\infty} \frac{\sqrt[3]{x}}{(4 + x^2)^5} dx
=
\frac{1}{4^5 }
\int\limits_0^{\infty} \frac{\sqrt[3]{x}}{(1 + \frac{x^2}{4})^5} dx
=
\begin{bmatrix}
\frac{x^2}{4} = t \\ 
x = 2\sqrt{t} \\ 
dt = \frac{x}{2} dx \\
dx = \frac{dt}{\sqrt{t}}
\end{bmatrix}
=
\frac{1}{4^5 }
\int\limits_0^{\infty}
\frac{\sqrt[3]{2\sqrt{t}}}{(1 + t)^5}
\frac{dt}{\sqrt{t}}
=
\frac{\sqrt[3]{2} }{4^5 }
\int\limits_0^{\infty}
\frac{\sqrt[6]{t}}{(1 + t)^5} \frac{dt}{\sqrt{t}}
=
\]
\[
=
\frac{\sqrt[3]{2} }{4^5 }
\int\limits_0^{\infty}
\frac{t^{-\frac{1}{3}}}{(1 + t)^5} dt = (\times)
\]
Получили вид $B(p, q)$, но нужно найти сами $p$ и $q$:
\[
\begin{cases}
p - 1 = -\frac{1}{3} \\
p + q = 5
\end{cases}
\sim 
\begin{cases}
p  = \frac{2}{3} \\
q = 5 - \frac{2}{3} = \frac{13}{3}
\end{cases}
\]
Тогда:
\[
(\times) = \frac{\sqrt[3]{2} }{4^5 } \cdot \frac{\Gamma \left(\frac{2}{3}\right)\left(\frac{13}{3}\right)}{\left(5\right)}
=
 \frac{\sqrt[3]{2} }{4^5 } \cdot \frac{\Gamma \left(\frac{2}{3}\right)
\frac{10}{3} \cdot \frac{7}{3} \cdot \frac{4}{3} \cdot \frac{1}{3}\Gamma \left(\frac{1}{3}\right)}
{4!}
= 
\begin{bmatrix}
\Gamma \left(\frac{2}{3}\right)\Gamma \left(\frac{1}{3}\right)= \frac{\pi}{\sin \frac{\pi}{3}}
\end{bmatrix}
=
\]
\[
=
\frac{\sqrt[3]{2} }{4^5 } \cdot 
\frac{
\frac{10}{3} \cdot \frac{7}{3} \cdot \frac{4}{3} \cdot \frac{1}{3} \frac{\pi}{\sin \frac{\pi}{3}}}{4!}
=
\]
\[
=
\frac{\sqrt[3]{2} }{4^5 } \cdot \frac{\pi}{\sin \frac{\pi}{3}} \cdot \frac{1}{4!} 
\cdot \frac{10 \cdot 7 \cdot 4}{3^4}  = \frac{\sqrt[3]{2} }{4^5 } \cdot \frac{\pi}{\sin \frac{\pi}{3}}
\cdot \frac{5 \cdot 7}{81 \cdot 3} 
\]
\begin{center}
\textbf{Ответ: } 
\[
\frac{\sqrt[3]{2} }{4^5 } \cdot \frac{\pi}{\sin \frac{\pi}{3}}
\cdot \frac{35}{243} 
\]
\end{center}
\clearpage
\section*{Номер 2}
\[
\int\limits_0^a x^2 \sqrt{a^2 - x^2} dx, \; (a > 0)
\]
\[
\int\limits_0^a x^2 \sqrt{a^2 - x^2} dx = 
\int\limits_0^a x^2 a\sqrt{1 - \left(\frac{x}{a}\right)^2} dx = 
\begin{bmatrix}
\left(\frac{x}{a}\right)^2 = t \\
x = a \sqrt{t} \\
dt = \frac{2x}{a^2} dx \\
dx = \frac{a}{2\sqrt{t}} dt
\end{bmatrix}
=
\]
\[
=
\int\limits_0^a  a^3 t \sqrt{1 - t} \frac{a}{2\sqrt{t}} dt = 
\frac{a^4}{2} \int\limits_0^a \sqrt{t} \sqrt{1 - t}  dt = (\times)
\]
Получили опять $B(p, q)$, найдем коэффы:
\[
\begin{cases}
p - 1 = \frac{1}{2} \\
q -1 = \frac{1}{2} \\
\end{cases}
\sim
\begin{cases}
p = \frac{3}{2} \\
q= \frac{3}{2} \\ 
p + q = 3
\end{cases}
\]
\[
(\times) = 
\frac{a^4}{2} \frac{\Gamma \left(\frac{3}{2}\right) \Gamma \left(\frac{3}{2}\right)}{\Gamma(3)}
=
\frac{a^4}{2} \cdot \frac{\frac{1}{2} \Gamma \left(\frac{1}{2}\right) \frac{1}{2} \Gamma \left(\frac{1}{2}\right)}{2!} = \frac{a^4}{2} \cdot \frac{1}{2^3} \cdot \Gamma \left(\frac{1}{2}\right) \Gamma \left(\frac{1}{2}\right)
=
\frac{a^4}{2^4} \cdot \pi
\]
\begin{center}
\textbf{Ответ: } 
\[
\frac{a^4}{16} \cdot \pi
\]
\end{center}
\clearpage
\section*{Номер 3}
\[
\int\limits_0^{+\infty} x^p e^{-x^q} dx
\]
Около нуля экспонента ведет себя как единичка, значит:
\[
f \sim x^p = \frac{1}{x^{-p}} \rightarrow -p < 1 \rightarrow p > -1
\]
На бесконечности экспонента все сьест и:
\[
|f| \leq e^{\frac{-x^q}{2}}, q > 0
\]
Такой интеграл будет сходится. При $q = 0$ получаем интеграл:
\[
\int\limits_0^{+\infty} x^p dx 
\]
Ну а этот интеграл расходится. Заменяем $x^q = t$ и получам:
\[
\int\limits_0^{+\infty} x^p e^{-x^q} dx = 
\begin{bmatrix}
x^q = t \\
x = t^{\frac{1}{q}}\\
dt = q x^{q-1} dx \\
dx = \frac{1}{q x^{q-1}}dt
\end{bmatrix}
= 
\int\limits_0^{+\infty} 
t^{\frac{p}{q}} e^{-t} \frac{1}{qt^{\frac{q-1}{q}}} dt = 
\frac{1}{q} \int\limits_0^{+\infty} t^{\frac{p}{q}} \cdot e^{-t} \cdot q^{\frac{1-q}{q}} dt
=
\]
\[
=
\frac{1}{q} \int\limits_0^{+\infty} e^{-t} \cdot t^{\frac{p-q+1}{q}dt} = 
\frac{1}{q} \Gamma \left( \frac{p + 1}{q}\right)
\]
\begin{center}
\textbf{Ответ: } 
\[
\frac{1}{q} \Gamma \left( \frac{p + 1}{q}\right), \; \; p > -1, \; q > 0
\]
\end{center}
\clearpage
\section*{Номер 4}
\[
\int\limits_0^{\frac{\pi}{2}} (\tg x)^p dx 
\]
Тангенс мешает, избавимся от него заменой:
\[
\int\limits_0^{\frac{\pi}{2}} (\tg x)^p dx  =
\begin{bmatrix}
\tg x = t \\
dt = \frac{1}{\cos^2 x} dx \\
dx = \frac{1}{1 + t^2} dt
\end{bmatrix}
=
\int\limits_0^{\infty} \frac{t^p}{1+t^2} dt
\]
Получили номер 1 в 1 аналогичный номеру 5 с семинара, только в знаменателе вместо $q$ двойка. Тогда смотрим.
Около нуля:
\[
f \sim t^p = \frac{1}{x^{-p}} \rightarrow p > - 1
\]
На бесконечности:
\[
f \sim x^{p - 2} \rightarrow p - 2 < -1 \rightarrow p < 1
\] 
Итого:
\[
-1 < p < 1 
\]
Заменим $x^2 = u$:
\[
\int\limits_0^{\infty} \frac{t^p}{1+t^2} dt = \begin{bmatrix}
t^2 = u \\
t = \sqrt{u}\\
2tdt = du \\
dt = \frac{1}{u} dt\\
\end{bmatrix}
=
\int\limits_0^{\infty} \frac{u^{\frac{p}{2}}}{1+u} \cdot \frac{1}{2} \cdot u^{-\frac{1}{2}} du = \frac{1}{2} \int\limits_0^{\infty} \frac{u^{\frac{p-1}{2}}}{1+u} du = \frac{1}{2} \frac{\Gamma \left(\frac{p+1}{2}\right) \cdot \Gamma \left(1 - \frac{p + 1}{2}\right)}{\Gamma(1)} = 
\]
\[
=
\frac{\pi}{2 \sin \left(\pi \frac{p + 1}{2}\right)}
\]
\begin{center}
\textbf{Ответ: } 
\[
\frac{\pi}{2 \sin \left(\pi \frac{p + 1}{2}\right)}, -1 < p < 1
\]
\end{center}

\end{document}
