\documentclass[a4paper,12pt]{article}

%%% Работа с русским языком
\usepackage{cmap}					% поиск в PDF
\usepackage{mathtext} 				% русские буквы в формулах
\usepackage[T2A]{fontenc}			% кодировка
\usepackage[utf8]{inputenc}			% кодировка исходного текста
\usepackage[english,russian]{babel}	% локализация и переносы
\usepackage{xcolor}
\usepackage{hyperref}
 % Цвета для гиперссылок
\definecolor{linkcolor}{HTML}{799B03} % цвет ссылок
\definecolor{urlcolor}{HTML}{799B03} % цвет гиперссылок

\hypersetup{pdfstartview=FitH,  linkcolor=linkcolor,urlcolor=urlcolor, colorlinks=true}

%%% Дополнительная работа с математикой
\usepackage{amsfonts,amssymb,amsthm,mathtools} % AMS
\usepackage{amsmath}
\usepackage{icomma} % "Умная" запятая: $0,2$ --- число, $0, 2$ --- перечисление

%% Номера формул
%\mathtoolsset{showonlyrefs=true} % Показывать номера только у тех формул, на которые есть \eqref{} в тексте.

%% Шрифты
\usepackage{euscript}	 % Шрифт Евклид
\usepackage{mathrsfs} % Красивый матшрифт

%% Свои команды
\DeclareMathOperator{\sgn}{\mathop{sgn}}

%% Перенос знаков в формулах (по Львовскому)
\newcommand*{\hm}[1]{#1\nobreak\discretionary{}
{\hbox{$\mathsurround=0pt #1$}}{}}
% графика
\usepackage{graphicx}
\graphicspath{{pictures/}}
\DeclareGraphicsExtensions{.pdf,.png,.jpg}
\author{Бурмашев Григорий, БПМИ-208}
\title{}
\date{\today}
\begin{document}
\begin{center}
Бурмашев Григорий. 208. Матан -- 15
\end{center}
\section*{Номер 10}
\subsection*{a)}
\[
f(x, y) = \ln(x + y^2)
\]
\[
\frac{\sigma f}{\sigma x} = \frac{1}{x + y^2}
\]
\[
\frac{\sigma f}{\sigma y} = \frac{2y}{x + y^2}
\]
\[
\frac{\sigma^2f}{\sigma x^2} = -\frac{1}{(x + y^2)^2}
\]
\[
\frac{\sigma^2 f}{\sigma y^2} = \frac{2(x - y^2)}{(x+y^2)^2}
\]
\[
\frac{\sigma^2 f}{\sigma x \sigma y} = -\frac{2y}{(x+y^2)^2}
\]
\[
\frac{\sigma^2 f}{\sigma y \sigma x} = -\frac{2y}{(x+y^2)^2}
\]
\clearpage
\subsection*{b)}
\[
f(x, y, z) = \sin (xy + z^2)x
\]
\[
\frac{\sigma f}{\sigma x} = y \cos (xy + z^2) 
\]
\[
\frac{\sigma f}{\sigma y} = x \cos (xy + z^2)
\]
\[
\frac{\sigma f}{\sigma z} = 2z \cos (xy + z^2)
\]
\[
\frac{\sigma^2 f}{\sigma x^2} = -y^2 \sin(xy + z^2) 
\]
\[
\frac{\sigma^2 f}{\sigma y^2} = 
-x^2 \sin (xy + z^2)
\]
\[
\frac{\sigma^2f}{\sigma z^2} = 2 (\cos(x y + z^2) - 2 z^2 \sin(x y + z^2))
\]
\[
\frac{\sigma^2f}{\sigma y \sigma x} = \cos (xy + z^2) - xy \sin (xy + z^2)
\]
\[
\frac{\sigma^2f}{\sigma z \sigma x} = -2 yz \sin (xy + z^2)
\]
\[
\frac{\sigma^2f}{\sigma x \sigma y} = \cos (xy + z^2) - xy \sin(xy + z^2)
\]
\[
\frac{\sigma^2f}{\sigma z \sigma y} = - 2xz \sin(xy + z^2)
\]
\[
\frac{\sigma^2f}{\sigma x \sigma z} = -2yz \sin (xy + z^2)
\]
\[
\frac{\sigma^2f}{\sigma y \sigma z} = - 2xz \sin(xy + z^2)
\]
\clearpage
\section*{Номер 11}
\subsection*{a)}
\[
f(x, y) = x \ln (xy), \; \text{найти } \frac{\sigma^3 f}{\sigma x^2 \sigma y}
\]
\[
\frac{\sigma f}{\sigma y} = \frac{x}{y}
\]
\[
\frac{\sigma^2 f}{\sigma x \sigma y} = \frac{1}{y}
\]
\[
\frac{\sigma^3 f}{\sigma x^2 \sigma y} = \frac{1}{y} = 0
\]
\begin{center}
\textbf{Ответ: } 0
\end{center}
\subsection*{b)}
\[
f(x, y, z) = \sin (xy + z^2), \; \text{найти } \frac{\sigma^3 f}{\sigma x \sigma y \sigma z}
\]
\[
\frac{\sigma f}{\sigma z} = 2z \cos (xy + z^2)
\]
\[
\frac{\sigma^2 f}{\sigma y \sigma z} = -2xz \sin (xy + z^2)
\]
\[
\frac{\sigma^3 f}{\sigma x \sigma y \sigma z} = -2 z (x y \cos(x y + z^2) + \sin(x y + z^2))
\]
\begin{center}
\textbf{Ответ: } 
\[
-2 z (x y \cos(x y + z^2) + \sin(x y + z^2))
\]
\end{center}
\end{document}
