\documentclass[a4paper,12pt]{article}

%%% Работа с русским языком
\usepackage{cmap}					% поиск в PDF
\usepackage{mathtext} 				% русские буквы в формулах
\usepackage[T2A]{fontenc}			% кодировка
\usepackage[utf8]{inputenc}			% кодировка исходного текста
\usepackage[english,russian]{babel}	% локализация и переносы
\usepackage{xcolor}
\usepackage{hyperref}
 % Цвета для гиперссылок
\definecolor{linkcolor}{HTML}{799B03} % цвет ссылок
\definecolor{urlcolor}{HTML}{799B03} % цвет гиперссылок

\hypersetup{pdfstartview=FitH,  linkcolor=linkcolor,urlcolor=urlcolor, colorlinks=true}

%%% Дополнительная работа с математикой
\usepackage{amsfonts,amssymb,amsthm,mathtools} % AMS
\usepackage{amsmath}
\usepackage{icomma} % "Умная" запятая: $0,2$ --- число, $0, 2$ --- перечисление

%% Номера формул
%\mathtoolsset{showonlyrefs=true} % Показывать номера только у тех формул, на которые есть \eqref{} в тексте.

%% Шрифты
\usepackage{euscript}	 % Шрифт Евклид
\usepackage{mathrsfs} % Красивый матшрифт

%% Свои команды
\DeclareMathOperator{\sgn}{\mathop{sgn}}

\newcommand*{\hm}[1]{#1\nobreak\discretionary{}
{\hbox{$\mathsurround=0pt #1$}}{}}
% графика
\usepackage{graphicx}
\graphicspath{{pictures/}}
\DeclareGraphicsExtensions{.pdf,.png,.jpg}
\author{Бурмашев Григорий}
\title{Матанализ  - 4}
\date{\today}
\begin{document}
\begin{large}
\begin{center}
Бурмашев Григорий.  208. Матан. Д/з - 4
\end{center}
\end{large}
\newpage
\section*{Задача № 10}
Применяя критерий Коши, установите сходимость или расходимость рядов:
\subsection*{ а)}
\[
\sum\limits_{n=1}^{\infty} \frac{\cos(2^n)}{n^2}
\]
По критерию Коши:
\[
\forall \; \varepsilon > 0 \; \exists \; N(\varepsilon) \; \forall \; n, m > N(\varepsilon):
\]
\[
|S_n - S_m| < \varepsilon
\]
Тогда:
\[
cos(2^n) \in [-1; 1]
\]
\[
|\frac{cos(2^{m+1})}{(m+1)^2} + \frac{cos(2^{m+2})}{(m+2)^2} + \ldots + \frac{cos(2^n)}{n^2}| \leq \frac{1}{(m+1)^2} + \frac{1}{(m+2)^2} + \ldots + \frac{1}{n^2} \leq 
\]
\[
\leq \frac{1}{m(m+1)} + \frac{1}{(m+1)(m+2)} + \ldots + \frac{1}{n(n-1)} =  
\]
\[
=
\frac{1}{m} - \frac{1}{m+1} + \frac{1}{m+1} - \frac{1}{m-2} + \frac{1}{m+2} + \ldots +\frac{1}{n-1} - \frac{1}{n} = 
\]
\[
=
\frac{1}{m} - \frac{1}{n} \leq \frac{1}{m}
\leq \varepsilon
\]
Значит:
\[
m \geq \frac{1}{\varepsilon}
\]
\[
N(\varepsilon) = \left[\frac{1}{\varepsilon}\right] + 1 
\] 
\begin{center}
$\rightarrow$ ряд сходится
\end{center}
\begin{center}
\textbf{Ответ:}  ряд сходится
\end{center}
\subsection*{ б)}
\[
\sum\limits_{n=1}^{\infty} \frac{\sin(\log_2 n)}{n}
\]
\[
\sin(\log_2 n) \in [-1;1] \rightarrow \lim_{n \rightarrow \infty} \frac{\sin(\log_2 n)}{n}  = 0
\]
По критерию Коши:
\[
\forall \; \varepsilon > 0 \; \exists \; N(\varepsilon) \; \forall \; n, m > N(\varepsilon):
\]
\[
|S_n - S_m| < \varepsilon
\]
Тогда:
NONE
\section*{Задача № 11}
Установите сходимость или расходимость рядов следующих рядов, а в случае расходимости вычислите суммы:
\subsection*{ а)}
\[
\sum\limits_{n=1}^{\infty} \frac{1}{n(n+1)(n+2)zzœ}
\]
\end{document}