\documentclass[a4paper,12pt]{article}

%%% Работа с русским языком
\usepackage{cmap}					% поиск в PDF
\usepackage{mathtext} 				% русские буквы в формулах
\usepackage[T2A]{fontenc}			% кодировка
\usepackage[utf8]{inputenc}			% кодировка исходного текста
\usepackage[english,russian]{babel}	% локализация и переносы
\usepackage{xcolor}
\usepackage{hyperref}
 % Цвета для гиперссылок
\definecolor{linkcolor}{HTML}{799B03} % цвет ссылок
\definecolor{urlcolor}{HTML}{799B03} % цвет гиперссылок

\hypersetup{pdfstartview=FitH,  linkcolor=linkcolor,urlcolor=urlcolor, colorlinks=true}

%%% Дополнительная работа с математикой
\usepackage{amsfonts,amssymb,amsthm,mathtools} % AMS
\usepackage{amsmath}
\usepackage{icomma} % "Умная" запятая: $0,2$ --- число, $0, 2$ --- перечисление

%% Номера формул
%\mathtoolsset{showonlyrefs=true} % Показывать номера только у тех формул, на которые есть \eqref{} в тексте.

%% Шрифты
\usepackage{euscript}	 % Шрифт Евклид
\usepackage{mathrsfs} % Красивый матшрифт

%% Свои команды
\DeclareMathOperator{\sgn}{\mathop{sgn}}

%% Перенос знаков в формулах (по Львовскому)
\newcommand*{\hm}[1]{#1\nobreak\discretionary{}
{\hbox{$\mathsurround=0pt #1$}}{}}
% графика
\usepackage{graphicx}
\graphicspath{{pictures/}}
\DeclareGraphicsExtensions{.pdf,.png,.jpg}
\author{Бурмашев Григорий, БПМИ-208}
\title{}
\date{\today}
\begin{document}
\begin{center}
Бурмашев Григорий, 208.  Матан -- 12
\end{center}
\begin{center}
\includegraphics[scale=0.4]{1.jpg}
\end{center}
\begin{center}
Тех получился немножко ублюдошным, но я не придумал как его сделать красивым, сорри
\end{center}
\clearpage
\section*{ Номер 8}
p > 0 по условию задачи
\subsection*{d)}
\[
\int_{0}^{\infty} \frac{ \cos x}{x^p} \, dx = \int_{0}^{1} \frac{ \cos x}{x^p} \, dx  + \int_{1}^{\infty} \frac{ \cos x}{x^p} \, dx 
\]
\begin{enumerate}
\item
\[
\int_{0}^{1} \frac{ \cos x}{x^p} \, dx
\]
\[
0 \leq \frac{\cos x}{x^p} \sim \frac{2 - x^2}{2 \cdot x^p} = \frac{1}{x^p} + \frac{1}{2x^{p-2}}
\]
Эквивалентность беру из следующих соображений:
\[
1 - \cos x \sim \frac{1}{2} x^2 \text{  (знаем)}
\]
\[
2 - x^2 \sim 2\cos x
\]
\[
\cos x \sim \frac{2 - x^2}{2}
\]
Сходится при p < 1
\item
\[
\int_{1}^{\infty} \frac{ \cos x}{x^p} \, dx
\]
Будем использовать признак Дирихле:
\\\\
Во первых:
\[
\int_{1}^{x} \cos t \, dt = \sin t \bigg|_1^x
\]
\[
|\sin x  - \sin 1| \leq 2 
\]
Имеем ограниченность
\\\\
Во вторых:
\\
$\frac{1}{x^p} $ -- монотонна и $\lim_{x \rightarrow \infty} \frac{1}{x^p} = 0 $ (p > 0 по условию)
\\\\
Выполнены оба условия, значит интеграл сходится
\end{enumerate}
Из пунктов 1 и 2 имеем, что $\int_{0}^{\infty} \frac{ \cos x}{x^p} \, dx$  сходится при $p \in (0, 1)$
{\Large \begin{center}
\textbf{Ответ: } сходится при $p \in (0, 1)$
\end{center}}
\subsection*{f)}
\[
\int_0^{\infty} \frac{\ln (1+x)}{x^p} \, dx = \int_0^{1} \frac{\ln (1+x)}{x^p} \, dx + \int_1^{\infty} \frac{\ln (1+x)}{x^p} \, dx
\]
\begin{enumerate}
\item
\[
0 \leq  \frac{\ln (1+x)}{x^p} \sim \frac{x}{x^p} = \frac{1}{x^{p-1}}
\]
Сходится при p < 2
\item
\[
\int_1^{\infty} \frac{\ln (1+x)}{x^p} \, dx
\]
Зажмем функцию:
\[
0 \leq \frac{1}{x^p} \leq \frac{\ln (1+x)}{x^p} \leq \frac{1}{x^{p - \varepsilon}}
\]
Мы знаем, что $\frac{1}{x^p}$ расходится при $p \leq 1$, а $\frac{1}{x^{p - \varepsilon}}$ сходится при p > 1 + $\varepsilon (> 0)$ 

Отсюда получаем ограничение на сходимость : $(1, \infty)$
\end{enumerate}
Из пунктов 1 и 2 имеем, что $\int_0^{\infty} \frac{\ln (1+x)}{x^p} \, dx$  сходится при $p \in (1, 2)$
{\Large \begin{center}
\textbf{Ответ: } сходится при $p \in (1, 2)$
\end{center}}
\clearpage
\subsection*{g)}
\[
\int_0^{\infty} \cos \left(x^3\right) \, dx = \int_0^{\infty} \frac{\cos t}{3 \sqrt[3]{t^2}} \, dt = \frac{1}{3} \cdot \left( \int_0^{1} \frac{\cos t}{\sqrt[3]{t^2}} \, dt  + \int_1^{\infty} \frac{\cos t}{ \sqrt[3]{t^2}} \, dt  \right)
\]
\begin{enumerate}
\item 
\[
\int_0^{1} \frac{\cos t}{\sqrt[3]{t^2}} \, dt = \int_0^{1} \frac{\cos t}{ t^{\frac{2}{3}}} \, dt
\]
\[
0 \leq \frac{\cos t}{ t^{\frac{2}{3}}} \sim \frac{2-t^2}{t^{\frac{2}{3}}} \text{( а это сходится)}
\]
Значит $\int_0^{1} \frac{\cos t}{\sqrt[3]{t^2}} \, dt$ сходится
\item
\[
\int_1^{\infty} \frac{\cos t}{ t^{\frac{2}{3}}} \, dt
\]
Будем использовать признак Дирихле:
\\\\
Во первых:
\[
\int_{1}^{t} \cos u \, du = \sin u \bigg|_1^t
\]
\[
|\sin t  - \sin 1| \leq 2 
\]
Имеем ограниченность
\\\\
Во вторых:
\\
$\frac{1}{t^{\frac{2}{3}}} $ -- монотонна и $\lim_{t \rightarrow \infty} \frac{1}{t^{\frac{2}{3}}} = 0 $
\\\\
Выполнены оба условия, значит интеграл сходится
\end{enumerate}
Из пунктов 1 и 2 следует сходимость $\int_0^{\infty} \cos \left(x^3\right) \, dx$
{\Large \begin{center}
\textbf{Ответ: } сходится
\end{center}
\clearpage
\section*{ Номер 9}}
\subsection*{a)}
\[
\frac{d}{dx} \int_{\sin x}^{\cos x} \cos \left( \pi t^2\right) \, dt = \frac{d}{dx} \left(F(t) \bigg|_{\sin x}^{\cos x} \right) = \frac{d}{dx} \left(F(\cos x) - F (\sin x)\right) = F'(\cos x) - F'(\sin x)=
\]
\[
= - \sin x \cos (\pi \cos^2 x) - \cos x \cos (\pi \sin^2 x)
\]
{\Large \begin{center}
\textbf{Ответ: } 
\[
- \sin x \cos (\pi \cos^2 x) - \cos x \cos (\pi \sin^2 x)
\]
\end{center}}
\subsection*{b)}
\[
\lim_{x \rightarrow +\infty} \frac{\left(\int_0^x e^{u^2}\right)^2}{\int_0^x e^{2u^2} \, du} = \frac{[\infty]}{[\infty]}
\]
Применяем Лопиталя:
\[
\frac{d}{dx} \left(\int_0^x e^{u^2}\right)^2 = 2 \left(F(u) \bigg|_0^x \right) = 2e^{x^2}
\]
\[
\frac{d}{dx} \int_0^x e^{2u^2} \, du = e^{2x^2}
\]
По итогу получаем:
\[
\lim_{x \rightarrow +\infty} \frac{\left(\int_0^x e^{u^2}\right)^2}{\int_0^x e^{2u^2} \, du} = \lim_{x \rightarrow +\infty} \frac{2e^{x^2}}{e^{2x^2}} = \lim_{x \rightarrow +\infty} 2 e^{-x^2} = 0
\] 
{\Large \begin{center}
\textbf{Ответ: }  0
\end{center}}
\end{document}
