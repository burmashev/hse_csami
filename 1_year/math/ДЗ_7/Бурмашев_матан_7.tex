\documentclass[a4paper,12pt]{article}

%%% Работа с русским языком
\usepackage{cmap}					% поиск в PDF
\usepackage{mathtext} 				% русские буквы в формулах
\usepackage[T2A]{fontenc}			% кодировка
\usepackage[utf8]{inputenc}			% кодировка исходного текста
\usepackage[english,russian]{babel}	% локализация и переносы
\usepackage{xcolor}
\usepackage{hyperref}
 % Цвета для гиперссылок
\definecolor{linkcolor}{HTML}{799B03} % цвет ссылок
\definecolor{urlcolor}{HTML}{799B03} % цвет гиперссылок

\hypersetup{pdfstartview=FitH,  linkcolor=linkcolor,urlcolor=urlcolor, colorlinks=true}

%%% Дополнительная работа с математикой
\usepackage{amsfonts,amssymb,amsthm,mathtools} % AMS
\usepackage{amsmath}
\usepackage{icomma} % "Умная" запятая: $0,2$ --- число, $0, 2$ --- перечисление

%% Номера формул
%\mathtoolsset{showonlyrefs=true} % Показывать номера только у тех формул, на которые есть \eqref{} в тексте.

%% Шрифты
\usepackage{euscript}	 % Шрифт Евклид
\usepackage{mathrsfs} % Красивый матшрифт

%% Свои команды
\DeclareMathOperator{\sgn}{\mathop{sgn}}

%% Перенос знаков в формулах (по Львовскому)
\newcommand*{\hm}[1]{#1\nobreak\discretionary{}
{\hbox{$\mathsurround=0pt #1$}}{}}
% графика
\usepackage{graphicx}
\graphicspath{{pictures/}}
\DeclareGraphicsExtensions{.pdf,.png,.jpg}
\author{Бурмашев Григорий, БПМИ-208}
\title{Линал. ИДЗ - 2 Вариант 2.}
\date{\today}
\begin{document}
\begin{center}
Бурмашев Григорий.  208. Матан. Д/з -- 7
\end{center}
\begin{center}
\includegraphics[scale=0.3]{123.jpg}
\end{center}
\newpage
\section*{Номер 9}
\subsection*{а)}
\[
\lim_{x \rightarrow 1} \frac{\sin \frac{\pi x}{2}}{x} = \frac{\sin \frac{\pi }{2}}{1} = \frac{1}{1} = 1
\]
\begin{center}
\textbf{Ответ: } 1
\end{center}

\subsection*{b)}
\[
\lim_{x \rightarrow 0} \frac{x - \sin 2x}{x + \sin 3x} =  \lim_{x \rightarrow 0} \frac{1 - \frac{\sin 2x}{x}}{1 + \frac{\sin 3x}{x}} = \lim_{x \rightarrow 0} \frac{1 - \frac{2 \cdot \sin 2x}{2 \cdot x}}{1 + \frac{3 \cdot \sin 3x}{3 \cdot x}} = \frac{1 - 2}{1 + 3} = - \frac{1}{4}
\]
\begin{center}
\textbf{Ответ: } $-\frac{1}{4}$
\end{center}

\subsection*{c)}
\[
\lim_{x \rightarrow 0} \frac{tg x + tg2x + \ldots + tgnx}{arctg x} = \lim_{x \rightarrow 0} \frac{\frac{\sin x}{\cos x} + \frac{\sin 2x}{\cos 2x} +\ldots + \frac{\sin nx}{\cos nx}}{arctg x} = 
\]
\[
= \lim_{x \rightarrow 0} \frac{\frac{x \cdot \sin x}{x \cdot \cos x} + \frac{2x \cdot \sin 2x}{2x \cdot \cos 2x}+ \ldots \frac{nx \cdot \sin nx}{nx \cdot \cos nx}}{arctg x} = \lim_{x \rightarrow 0} \frac{\frac{x}{\cos x} + \frac{2x}{\cos 2x} + \ldots + \frac{nx}{\cos nx}}{arctg x} =
\]
\[
= \lim_{x \rightarrow 0} \frac{x \cdot \left( \frac{1}{\cos x}  + \frac{2}{\cos 2x} + \ldots + \frac{n}{\cos nx} 
\right)}{arctg x } = \lim_{x \rightarrow 0} \frac{x}{arctg x } \cdot \lim_{x \rightarrow 0} \left( \frac{1}{\cos x}  + \frac{2}{\cos 2x}+ \ldots + \frac{n}{\cos nx}  \right) =
\]
$\# $ $\lim\limits_{x \rightarrow 0} \frac{x}{arctg x}  = 1$, выводили на семинаре.
\[
= 1 \cdot \lim_{x \rightarrow 0} \left( \frac{1}{\cos x}  + \frac{2}{\cos 2x}+ \ldots + \frac{n}{\cos nx}  \right) = 1 + 2 + 3 + \ldots + n = \frac{n(n+1)}{2}
\]
\begin{center}
\textbf{Ответ: } $\frac{n(n+1)}{2}$
\end{center}

\subsection*{d)}
\[
\lim_{x \rightarrow 0} \frac{tgx - \sin x}{\sin^3 x} = \lim_{x \rightarrow 0} \frac{\frac{\sin x(1 - \cos x)}{\cos x}}{\sin ^3 x} = \lim_{x \rightarrow 0} \frac{1 - \cos x}{\sin^ 2 x \cdot \cos x} = \lim_{x \rightarrow 0} \frac{\sin^2 \frac{x}{2}}{2\sin^2 \frac{x}{2} \cos^2\frac{x}{2} \cdot \cos x} =
\]
\[
= \lim_{x \rightarrow 0} \frac{1}{2\cos^2 \frac{x}{2} \cdot \cos x} = \frac{1}{2 \cdot 1 \cdot 1} = \frac{1}{2}
\]
\begin{center}
\textbf{Ответ: } $\frac{1}{2}$
\end{center}

\subsection*{e)}
\[
\lim_{x \rightarrow 0} \frac{\cos x - \cos 3x}{x^2} = \lim_{x \rightarrow 0} \frac{-2 \cdot \sin 2x \cdot (-\sin x)}{x^2} = \lim_{x \rightarrow 0} \frac{2\sin 2x \sin x}{x^2} =
\]
\[
= \lim_{x \rightarrow 0} \frac{2\sin 2x}{x} = \lim_{x \rightarrow 0} \frac{2 \cdot 2 \sin 2x}{ 2x} = \frac{4}{1} = 4
\]
\begin{center}
\textbf{Ответ: } $4$
\end{center}

\subsection*{f)}
\[
\lim_{x \rightarrow a} \frac{\sin x - \sin a}{x - a} = \lim_{x \rightarrow a} \frac{2\sin \left(\frac{x-a}{2} \right) cos  \left(\frac{x+a}{2} \right) }{x-a} 
\]
Пусть:
\[
y = x - a 
\]
Тогда:
\[
x \rightarrow a \equiv y \rightarrow 0
\]
\[
\lim_{y \rightarrow 0} \frac{2\sin \left(\frac{y}{2}\right) cos \left(\frac{y+2a}{2}\right)}{y} =  \lim_{y \rightarrow 0} \frac{\sin \left( \frac{y}{2} \right) \cos \left( \frac{y+2a}{2} \right )}{\frac{y}{2}} = \lim_{y \rightarrow 0} \cos \left(\frac{y+2a}{2}\right)= \cos \frac{2a}{2}=  \cos a
\]
\begin{center}
\textbf{Ответ: } $\cos a$
\end{center}

\subsection*{g)} 
\[
\lim_{x \rightarrow 0} \frac{\cos (a+2x) - 2 \cos (a +x) + cos a}{x^2} = \lim_{x \rightarrow 0} \frac{2\cos \left(\frac{2a+2x}{2}\right)\cos x - 2\cos (a + x)}{x^2} =
\]
\[
= \lim_{x \rightarrow 0} \frac{2\cos(a+x) \cos x - 2\cos (a+x)}{x^2} = \lim_{x \rightarrow 0} \frac{2\cos (a+x) \cdot (\ cos x - 1)}{x^2} =
\]
\[
= \lim_{x \rightarrow 0} - \frac{2\sin^2 \left( \frac{x}{2} \right) 2 \cos (a + x)}{x^2} = \lim_{x \rightarrow 0}- \frac{\cos(a +x) \sin \left(\frac{x}{2}\right) \sin \left(\frac{x}{2}\right)}{\frac{x}{2} \cdot \frac{x}{2}} =
\]
\[
= \lim_{x \rightarrow 0} - \cos (a  + x) = -\cos a
\]
\begin{center}
\textbf{Ответ: } $- \cos a$
\end{center}
\newpage
\[
U^{'} R > \frac{1}{n} \sum_{i = 1}^{n} x_i
\]

\end{document}