\documentclass[a4paper,12pt]{article}

%%% Работа с русским языком
\usepackage{cmap}					% поиск в PDF
\usepackage{mathtext} 				% русские буквы в формулах
\usepackage[T2A]{fontenc}			% кодировка
\usepackage[utf8]{inputenc}			% кодировка исходного текста
\usepackage[english,russian]{babel}	% локализация и переносы
\usepackage{xcolor}
\usepackage{hyperref}
 % Цвета для гиперссылок
\definecolor{linkcolor}{HTML}{799B03} % цвет ссылок
\definecolor{urlcolor}{HTML}{799B03} % цвет гиперссылок

\hypersetup{pdfstartview=FitH,  linkcolor=linkcolor,urlcolor=urlcolor, colorlinks=true}

%%% Дополнительная работа с математикой
\usepackage{amsfonts,amssymb,amsthm,mathtools} % AMS
\usepackage{amsmath}
\usepackage{icomma} % "Умная" запятая: $0,2$ --- число, $0, 2$ --- перечисление

%% Номера формул
%\mathtoolsset{showonlyrefs=true} % Показывать номера только у тех формул, на которые есть \eqref{} в тексте.

%% Шрифты
\usepackage{euscript}	 % Шрифт Евклид
\usepackage{mathrsfs} % Красивый матшрифт

%% Свои команды
\DeclareMathOperator{\sgn}{\mathop{sgn}}

%% Перенос знаков в формулах (по Львовскому)
\newcommand*{\hm}[1]{#1\nobreak\discretionary{}
{\hbox{$\mathsurround=0pt #1$}}{}}
% графика
\usepackage{graphicx}
\graphicspath{{pictures/}}
\DeclareGraphicsExtensions{.pdf,.png,.jpg}
\author{Бурмашев Григорий, БПМИ-208}
\title{Математические структуры, контрольная работа, вариант H}
\date{\today}
\begin{document}
\maketitle
\clearpage
\section*{Номер 1}
Пускай:
\begin{itemize}
\item a -- счастливый билет
\item b -- хорошо подготовился
\item c -- экзамен успешно сдал
\end{itemize}
Тогда наше выражение в таком языке будет иметь вид:
\[
(((a \vee b) \rightarrow c) \wedge (\neg b \rightarrow \neg c)) \rightarrow (\neg c \rightarrow \neg a)
\]
Проверим, может ли следствие быть ложным, это единственный случай, когда вся формула станет ложной, т.е мы хотим:
\[
\begin{cases}
((a \vee b) \rightarrow c) \wedge (\neg b \rightarrow \neg c)) = 1 \\
 (\neg c \rightarrow \neg a)  = 0 
\end{cases}
\]
Из правой части однозначно получаем:
\[
\begin{cases}
((a \vee b) \rightarrow c) \wedge (\neg b \rightarrow \neg c)) = 1\\
 ( c \vee \neg a)  = 0 
\end{cases}
\]
\[
\begin{cases}
((a \vee b) \rightarrow c) \wedge (\neg b \rightarrow \neg c)) = 1\\
c = 0 \\ 
a = 1
\end{cases}
\]
Теперь подставим значения в первое выражение:
\[
((1 \vee b) \rightarrow 0) \wedge (\neg b \rightarrow \neg 0)) = 1
\]
\[
(1 \rightarrow 0) \wedge (\neg b \rightarrow 1) = 1
\]
В левой	 скобке однозначно 0, значит получаем:
\[
0 = 1
\]
Что неверно, приходим к противоречию, значит случая $1 \rightarrow 0$ не может быть, следовательно заключение является логическим следствием конъюнкции посылок
\begin{center}
\textbf{Ответ: } да, является
\end{center}
\section*{Номер 2}
\[
((q \wedge r) \rightarrow p) \rightarrow \neg (p \wedge q) 
\]
\[
\neg  (\neg (q \wedge r) \vee p)) \vee \neg (p \wedge q) 
\]
\[
\neg (\neg  q \vee \neg r \vee p) \vee (\neg p \vee \neg q)
\]
\[
(q \wedge r \wedge \neg p) \vee \neg p \vee \neg q
\]
\begin{center}
\textbf{Ответ: } $(q \wedge r \wedge \neg p) \vee \neg p \vee \neg q$
\end{center}
\clearpage
\section*{Номер 3}
\begin{center}
\includegraphics[scale=0.4]{3.jpg}
\end{center}
\clearpage
\section*{Номер 4}
Докажем, что не является:
\\\\
Пусть:
\[
M, x \not \models \neg (p \rightarrow q) \rightarrow (p \wedge \neg q) 
\]
Т.е хотим $\exists y : x \preceq  y$
\[
\begin{cases}
y \models \neg (p \rightarrow  q) \\
y \not \models (p \wedge \neg q)
\end{cases}
\]
Разберемся сначала со вторым выражением, хотим $\exists y' : y' \preceq  y$, что $y' \not \models p$ или $y' \not \models \neg q \equiv y' \models q$.
\\\\
Теперь для первого выражения хотим: $\exists z: y \preceq z$, что:
\[
\begin{cases}
z \models p \\
z \not \models q
\end{cases}
\]
Построили опровергающаю модель Крипке, но \textbf{не} получили в  ней противоречий, следовательно формула \textbf{не} является интуиционисткой тавтологией.
\begin{center}
\textbf{Ч.Т.Д} 
\end{center}
\clearpage
\section*{Номер 5}
\[
\neg \neg \neg p \rightarrow \neg p
\]
Пусть:
\[
M, x \not \models \neg \neg \neg p \rightarrow \neg p
\]
Тогда:
$\exists y: x \preceq y$, что:
\[
\begin{cases}
y \models \neg \neg \neg p \\
y \not \models p \rightarrow \bot
\end{cases}
\]
Для второго: $\exists z : y \preceq z$:
\[
z \models p
\]
Для первого:
\[
y \models \neg \neg p \rightarrow \bot
\]
\[
y \not \models \neg \neg p
\]
\[
y \not \models \neg p \rightarrow \bot
\]
Тобишь $\exists s : y \preceq s$:
\[
s \models \neg p
\]
Для $z : z \models p$, для $s: s \models \neg p$, посмотрим, будет ли противоречие.
\\\\
Заметим, что :
\[
p \rightarrow \neg \neg p \text{ есть и-тавтология }
\]
Следовательно замечаем:
\[
z \models \neg \neg p
\]
\textbf{Но} в $y$:
\[
y \not \models \neg \neg p
\]
Тогда:
\[
z \models \bot
\]
Получили \textbf{противоречие}, значит исходная формула является и-тавтологией
\begin{center}
\textbf{Ч.Т.Д} 
\end{center}
\end{document}
