\documentclass[a4paper,12pt]{article}
\usepackage[left=2cm,right=2cm,
    top=2cm,bottom=2cm,bindingoffset=0cm]{geometry}
%%% Работа с русским языком
\usepackage{cmap}					% поиск в PDF
\usepackage{mathtext} 				% русские буквы в формулах
\usepackage[T2A]{fontenc}			% кодировка
\usepackage[utf8]{inputenc}			% кодировка исходного текста
\usepackage[english,russian]{babel}	% локализация и переносы
\usepackage{xcolor}
\usepackage{hyperref}
 % Цвета для гиперссылок
\definecolor{linkcolor}{HTML}{799B03} % цвет ссылок
\definecolor{urlcolor}{HTML}{799B03} % цвет гиперссылок

\hypersetup{pdfstartview=FitH,  linkcolor=linkcolor,urlcolor=urlcolor, colorlinks=true}

%%% Дополнительная работа с математикой
\usepackage{amsfonts,amssymb,amsthm,mathtools} % AMS
\usepackage{amsmath}
\usepackage{icomma} % "Умная" запятая: $0,2$ --- число, $0, 2$ --- перечисление

%% Номера формул
%\mathtoolsset{showonlyrefs=true} % Показывать номера только у тех формул, на которые есть \eqref{} в тексте.

%% Шрифты
\usepackage{euscript}	 % Шрифт Евклид
\usepackage{mathrsfs} % Красивый матшрифт

%% Свои команды
\DeclareMathOperator{\sgn}{\mathop{sgn}}

%% Перенос знаков в формулах (по Львовскому)
\newcommand*{\hm}[1]{#1\nobreak\discretionary{}
{\hbox{$\mathsurround=0pt #1$}}{}}
% графика
\usepackage{graphicx}
\graphicspath{{pictures/}}
\DeclareGraphicsExtensions{.pdf,.png,.jpg}
\author{Бурмашев Григорий, БПМИ-208}
\title{Дифференциальные уравнения, дз -- 1}
\date{\today}
\begin{document}
\maketitle
\clearpage
\section*{Номер 1}
\subsection*{a)}
\[
(2xy^2 -y) d x + x dy = 0
\]
Начинаем решать стандартно
\[
(2xy^2 -y) d x = - xdy
\]
Теперь поделим обе части на $xdx$. Проверим случай $x = 0$, при таком $x$ получаем $0 = 0$, значит это решение
\[
2y^2 - \frac{y}{x} = - \frac{dy}{dx}
\]
\[
y' - \frac{y}{x} = -2y^2
\]
Получили уравнение Бернулли с $n = 2$. Делим на $y^2$. Проверим случай $y = 0$, при таком $y$ получаем $0 = 0$, значит это решение
\[
\frac{y'}{y^2} - \frac{1}{xy} = -2
\]
Делаем замену $z = \frac{1}{y^{n -1}} = \frac{1}{y}$. Тогда $dz = -\frac{dy}{y^2} = -z^2dy$, отсюда $dy =  -\frac{dz}{z^2}$. Подставляем замену
\[
-\frac{z^2}{z^2} \cdot \frac{dz}{dz} - \frac{z}{x}  = -2
\]
Меняем знак 
\[
\frac{dz}{dx} + \frac{z}{x} = 2
\]
Получили понятное нам уравнение, решаем его с помощью метода вариации постоянной
\\\\
Шаг 1
\[
\frac{dz}{dx} + \frac{z}{x} = 0
\]
\[
\frac{dz}{z} = -\frac{dx}{x}
\]
\[
\int \frac{dz}{z}  = - \int \frac{dx}{x}
\]
\[
\ln |z| = - \left(\ln |x| + C_0\right)
\]
\[
- \ln |z| =  \ln |x| + C_0
\]
\[
|z| = \frac{1}{|x| \cdot C_1}
\]
\[
z = \frac{C}{x}
\]
Шаг 2 (вариация переменной)
\[
z = \frac{C(x)}{x}
\]
\[
z' = \frac{xC'(x) - C(x)}{x^2}
\]
Подставляем
\[
\frac{xC'(x) - C(x)}{x^2} + \frac{C(x)}{x^2} = 2
\]
\[
\frac{xC'(x)}{x^2} = 2
\]
\[
C'(x) = 2x
\]
\[
C(x) = x^2 + D
\]
Возвращаемся к $z$
\[
z = \frac{C(x)}{x} = \frac{x^2 + D}{x}
\]
Избавляемся от замены
\[
y = \frac{1}{z} = \frac{x}{x^2 + D}
\]
\begin{center}
\textbf{Ответ: } 
\[
x = 0
\]
\[
y = 0
\]
\[
y = \frac{x}{x^2 + D}
\]
\end{center}
\clearpage
\subsection*{b)}
\[
y + y' \ln^2y = (x + 2 \ln y) y'
\] 
\[
y + \frac{dy}{dx} \ln^2 y = (x + 2 \ln y) \frac{dy}{dx}
\]
Домножим на  $dx$
\[
ydx + dy \ln^2 y = (x + 2\ln y) dy 
\]
\[
y dx + (\ln^2 y - x - 2\ln y) dy = 0
\]
Получили вид УПД. Проверяем производные:
\[
(y)^{'}_y = 1
\]
\[
(\ln^2 y - x - 2\ln y)^{'}_x = -1 
\]
\[
1 \neq -1
\]
Равенства нет, тогда подберем  $m(x, y)$ для общей теоремы. Нам подходит $\frac{1}{y^2}$.
\[
\frac{1}{y} dx + \frac{\ln^2 y - x - 2\ln y}{y^2} dy = 0
\] 
Снова проверяем производные
\[
\left( \frac{1}{y}  \right)_y' = -\frac{1}{y^2}
\]
\[
\left( \frac{\ln^2 y - x - 2\ln y}{y^2} \right)_x' = -\frac{1}{y^2}
\]
Получили равенство, теперь ищем такую функцию, что
\[
\begin{cases}
F_x'(x, y) = \frac{1}{y} \\
F_y'(x, y) = \frac{\ln^2 y - x - 2\ln y}{y^2}
\end{cases}
\]
Для этого проинтегрируем пока только одно соотношение, например первое
\[
F_x(x, y) = \int \frac{1}{y} dx = \frac{1}{y} \int dx = \frac{x}{y} + C(y)
\]
Теперь из второго найдем, чему равно это $C(y)$
\[
\left( \frac{x}{y} + C(y) \right)_y' = \frac{\ln^2 y - x - 2\ln y}{y^2}
\]
\[
C'(y) -\frac{x}{y^2} = \frac{\ln^2 y - x - 2\ln y}{y^2}
\]
\[
C'(y) = \frac{\ln^2 y - 2\ln y}{y^2}
\]
\[
C(y) = \int \frac{\ln^2 y - 2\ln y}{y^2} = \int \frac{\ln^2 y}{y^2} dy - 2 \int \frac{\ln y}{y^2} dy = 
-\frac{\ln^2 y + 2 \ln y + 2}{y} - 2 \cdot \left(- \frac{\ln y + 1}{y}\right) = - \frac{\ln^2 y}{y}
\]
Итак
\[
F(x, y) = \frac{x}{y} + C(y) = \frac{x - \ln^2 y}{y}
\]
\begin{center}
\textbf{Ответ: } 
\end{center}
\[
C=  \frac{x - \ln^2 y}{y}
\]
\clearpage
\section*{c)}
\[
\frac{y - xy'}{x + yy'} = 2
\]
\[
y - xy' - 2x - 2yy' = 0
\]
\[
y - 2x = (x + 2y)y'
\]
Делим на $x + 2y$. Проверим случай $x + 2y = 0, x = -2y$. При таком получаем $5y = 0$, значит это не решение
\[
\frac{y - 2x}{x + 2y} = y'
\]
Пусть $y = tx$, тогда $y'= x \frac{dt}{dx} + t$
\[
\frac{tx - 2x}{x + 2tx} = x \frac{dt}{dx} + t
\]
Делим на $x$ числитель и знаменатель левой части. Проверим случай $x = 0$, при таком получаем $0 = t$, значит это не решение
\[
\frac{t - 2}{1 + 2t} = x \frac{dt}{dx} + t
\]
\[
x \frac{dt}{dx} = \frac{t - 2}{1 + 2t}  - t
\]
\[
x \frac{dt}{dx} = \frac{t - 2 - t - 2t^2}{1 + 2t}  
\]
\[
x \frac{dt}{dx} = \frac{- 2 - 2t^2}{1 + 2t}  
\]
Делим на $x$
\[
\frac{dt}{dx} = \frac{- 2 (1+ t^2)}{(1 + 2t)x}  
\]
Делим на $\frac{1+t^2}{1+2t}$
\[
\frac{dt}{dx} \cdot \frac{1+2t}{1+t^2} = \frac{-2}{x}
\]
Домножим на $dx$
\[
dt \cdot \frac{1 + 2t}{1 + t^2} = -\frac{2}{x} dx
\]
Теперь интегрируем
\[
\int \frac{1 + 2t}{1 + t^2}  dt = \int -\frac{2}{x} dx = - 2 \ln |x| + C_0
\]
Посчитаем отдельно интеграл слева
\[
\int \frac{1 + 2t}{1 + t^2}  dt = \int \frac{dt}{1 +t^2} + 2\int \frac{t}{1 + t^2} dt =\arctg(t) +  \ln (1 + t^2) + C_1
\]
Итого
\[
\arctg(t) + \ln (1 + t^2) + C_1 = - 2 \ln |x| + C_0
\]
Вспоминаем про $y = tx$, отсюда $t = \frac{y}{x}$, также перенесим все налево
\[
\arctg\left(\frac{y}{x}\right) + \ln \left( 1 + \frac{y^2}{x^2} \right) + 2 \ln |x| + C = 0
\]
\begin{center}
\textbf{Ответ: } 
\[
\arctg\left(\frac{y}{x}\right) + \ln \left( 1 + \frac{y^2}{x^2}\right)  + 2 \ln |x| + C = 0
\]
\end{center}
\clearpage
\section*{d)}
\[
(\sin x + y)dy + (y \cos x - x^2) dx = 0
\]
Видим вид УПД. Проверяяем производные
\[
\left( \sin x + y \right)_x' = \cos x
\]
\[
\left( y \cos x - x^2 \right)_y' =  \cos x
\]
Получили равенство, теперь ищем такую функцию, что
\[
\begin{cases}
F_x'(x, y) = y \cos x - x^2 \\
F_y'(x, y) = \sin x + y 
\end{cases}
\]
Для этого проинтегрируем пока только одно соотношение, например первое
\[
F_x(x, y) = \int (y \cos x - x^2 )dx = y \sin x - \frac{x^3}{3} + C(y)
\]
Теперь из второго найдем, чему равно это $C(y)$
\[
\left( y \sin x - \frac{x^3}{3} + C(y) \right)_y'= \sin x + y
\]
\[
\sin x + C'(y)  = \sin x + y
\]
\[
C'(y) = y
\]
Отсюда
\[
C(y) = \frac{y^2}{2}
\]
Итак
\[
F(x, y) = y \sin x - \frac{x^3}{3} + C(y) = y \sin x - \frac{x^3}{3} + \frac{y^2}{2}
\]
\begin{center}
\textbf{Ответ: } 
\[
C =  y \sin x - \frac{x^3}{3} + \frac{y^2}{2}
\]
\end{center}
\clearpage

\subsection*{e)}
\[
y' = \frac{x}{y} e^{2x} + y
\]
\[
y' - y = \frac{x}{y} e^{2x}
\]
\[
\frac{dy}{dx} - y =x e^{2x} \frac{1}{y} 
\]
Получили уравнение Бернулли с $n = -1$. Домножим уравнение на $y$
\[
\frac{ydy}{dx} - y^2 =x e^{2x}
\]
Делаем замену $z = \frac{1}{y^{n - 1}} = \frac{1}{y^{-2}} = y^2$. Тогда $dz = 2ydy$, отсюда $dy = \frac{dz}{2y}$. Подставляем замену
\[
\frac{y\frac{dz}{2y}}{dx} - z = xe^{2x}
\]
\[
\frac{dz}{2dx} - z = xe^{2x}
\]
Делим на $e^{2x}$
\[
\frac{dz}{2e^{2x}dx} - \frac{z}{e^{2x}} = x
\]
Домножим на 2
\[
\frac{dz}{e^{2x}dx} - 2\frac{z}{e^{2x}} = 2x
\]
\[
e^{-2x}\frac{dz}{dx} - 2e^{-2x}z = 2x
\]
Заметим, что $-2e^{-2x}$ есть $(e^{-2x})'$
\[
e^{-2x}\frac{dz}{dx} - (e^{-2x})'z = 2x
\]
\[
e^{-2x}\frac{dz}{dx} +  z\frac{d(e^{-2x})}{dx}= 2x
\]
\[
e^{-2x}(z)' +  (e^{-2x})'z= 2x
\]
Имеем вид $f g' + f'g = (fg)'$. Получаем
\[
\frac{d(e^{-2x} z)}{dx} = 2x
\]
Интегрируем с двух сторон
\[
\int \left( \frac{d(e^{-2x} z}{dx})\right) dx = \int 2x dx
\]
\[
e^{-2x} z = x^2 + C_0
\]
Вспоминаем про замену
\[
e^{-2x} y^2  = x^2 + C_0
\]
\[
y^2  = e^{2x} \cdot (x^2 + C_0)
\]
\[
y = \pm e^{x} \cdot \sqrt{x^2 + C}
\]
\begin{center}
\textbf{Ответ: } 
\[
y = \pm e^{x} \cdot \sqrt{x^2 + C}
\]
\end{center}
\clearpage

\subsection*{f)}
\[
xy' = x \sqrt{y - x^2} + 2y
\]
Пусть $y = tx^2, y' =  2xt + x^2t' $. Проверим случай $x = 0 $, при таком получаем $0 = 2y$, значит это не решение
\[
x(2xt + x^2t') = x \sqrt{tx^2- x^2} + 2x^2t
\]
\[
x(2xt + x^2t') = x \sqrt{tx^2- x^2} + 2x^2t
\]
\[
2x^2t  + x^3t' = x \sqrt{x^2 (t - 1)} + 2x^2 t
\]
\[
x^3t' = x^2 \sqrt{(t - 1)}
\]
Делим на $x^3$
\[
xt' = \sqrt{t - 1}
\]
\[
 \frac{dt}{dx} = \frac{\sqrt{t - 1}}{x}
\]
Делим на $\sqrt{t - 1}$. Проверим случай $t = 1, y = x^2$, при таком получаем $0 = 0$, значит это решение
\[
 \frac{dt}{\sqrt{t-1}dx} = \frac{1}{x}
\]
Берем интеграл
\[
 \int \left( \frac{dt}{\sqrt{t-1}dx} \right)dx=\int  \frac{1}{x} dx
\]
\[
2 \sqrt{t - 1} = \ln|x| + C
\]
Вытаскиваем $t$ из под корня
\[
4(t -1) = (\ln|x| + C)^2
\]
\[
4t - 4 = \ln|x|^2 + 2\ln|x| \cdot C + C^2
\]
\[
t = \frac{ \ln|x|^2 + 2\ln|x| \cdot C + C^2 + 4 }{4}
\]
Вспоминаем про нашу замену, $y = tx^2$, отсюда $t = \frac{y}{x^2}$
\[
 \frac{y}{x^2}=  \frac{ \ln^2 |x| + 2\ln|x| \cdot C + C^2 + 4 }{4}
\]
\[
y = \frac{x^2( \ln^2 |x| + 2\ln|x| \cdot C + C^2 + 4) }{4}
\]
\begin{center}
\textbf{Ответ: } 
\[
y = x^2
\]
\[
y = \frac{x^2( \ln^2 |x| + 2\ln|x| \cdot C + C^2 + 4) }{4}
\]
\end{center}
\clearpage
\subsection*{g)}
\[
xy'(\ln y - \ln x)  = y 
\]
Поделим на $x$. 
\[
y' ( \ln \frac{y}{x}) = \frac{y}{x}
\]
\[
y' = \frac{\frac{y}{x}}{( \ln \frac{y}{x})}
\]
Введем замену $y = tx, y' = t'x + t$
\[
t'x + t = \frac{t}{\ln t}
\]
Перекинем все с $x$ налево
\[
\frac{dt}{dx} x = \frac{t}{\ln t} - t
\]
\[
\frac{dt}{dx} = \frac{t - t\ln t}{x\ln t}
\]
\[
\frac{dx}{dt} = \frac{x\ln t}{ t - t\ln t}
\]
\[
dx = \frac{x dt \ln t}{ t - t\ln t}
\]
\[
\frac{dx}{x} = \frac{\ln t dt}{t - t  \ln t}
\]
Интегрируем
\[
\int \frac{dx}{x} =  \ln(x) + C  = \int \frac{\ln t dt}{t - t  \ln t}
\]
Считаем интеграл справа отдельно
\[
 \int \frac{\ln t dt}{t - t  \ln t} = \begin{bmatrix}
u = t - t \ln (t) \\
du = - \ln t dt
\end{bmatrix} = - \int \frac{1}{u} du  = - \ln(u) = - \ln (t - \ln t)
\]
Получаем
\[
\ln(x) + C  = - \ln (t - \ln t)
\]
\[
C = - \ln (t - \ln t) - \ln x
\]
Вспоминаем про замену $y = tx$, $t = \frac{y}{x}$
\[
C = - \ln \left(\frac{y}{x}- \ln \frac{y}{x}\right) -  \ln x
\]
\begin{center}
\textbf{Ответ: } 
\[
C = - \ln \left(\frac{y}{x}- \ln \frac{y}{x}\right) - \ln x
\]
\end{center}
\clearpage
\subsection*{i)}
\[
y' - 8x \sqrt{y} = \frac{4xy}{x^2 - 1}
\]
Пусть $z = \sqrt{y}$, тогда $dz = \frac{1}{2\sqrt{y}}dy$. Подставим замену
\[
\frac{dy}{dx} - 8x \sqrt{y} = \frac{4xy}{x^2 - 1}
\]
\[
\frac{2\sqrt{y} dz}{dx} - 8x \sqrt{y} = \frac{4xy}{x^2 - 1}
\]
\[
\frac{2\sqrt{y} dz}{dx} = \frac{4xy}{x^2 - 1} + 8x \sqrt{y}
\]
Поделим на $\sqrt{y}$. Проверим случай $y = 0$, при таком $y$ получаем $0 = 0$, значит это решение
\[
\frac{2dz}{dx} = 8x + \frac{4xz}{x^2 - 1}
\]
Избавимся от двойки, поделим на 2
\[
\frac{dz}{dx} = 4x + \frac{2xz}{x^2 - 1}
\]
\[
\frac{dz}{dx}  - \frac{2xz}{x^2 - 1} = 4x
\]
Получили понятное нам уравнение, решаем его с помощью метода вариации постоянной
\\\\
Шаг 1
\[
\frac{dz}{dx} - \frac{2xz}{x^2 - 1} = 0
\]
\[
\frac{dz}{dx} = \frac{2xz}{x^2 - 1}
\]
\[
\frac{dz}{z} = \frac{2x dx}{x^2 - 1}
\]
Интегрируем
\[
\int \frac{dz}{z} = 2\int \frac{x dx}{x^2 - 1}
\]
\[
\ln |z| = \ln |x^2 -1| + C_0
\]
\[
\ln|z| = \ln |C_1 (x^2 - 1) |
\]
\[
|z| = |C_1(x^2 -1)|
\]
\[
z = C \cdot (x^2 - 1)
\]
Шаг 2 (вариация переменной) 
\[
z = C(x)  \cdot (x^2 - 1)
\]
\[
z' = C'(x) (x^2 - 1) + C(x) 2x
\]
Подставляем
\[
\frac{dz}{dx}  - \frac{2xz}{x^2 - 1} = 4x = 
C'(x) (x^2 - 1) + C(x) 2x - \frac{2x\left(C(x)  \cdot (x^2 - 1)\right)}{x^2-1} = 4x
\]
\[
C'(x)(x^2 - 1) = 4x
\]
\[
C'(x) = \frac{4x}{x^2 - 1}
\]
\[
C(x) = 2 \ln |x^2 - 1| + D
\]
Возвращаемся к $z$
\[
z = (2 \ln|x^2 - 1| + D)(x^2 - 1)
\]
Избавляемся от замены
\[
\sqrt{y} = (2 \ln|x^2 - 1| + D)(x^2 - 1)
\]
\[
y = \left( (2 \ln|x^2 - 1| + D)(x^2 - 1) \right)^2
\]
\begin{center}
\textbf{Ответ: } 
\[
y = 0
\]
\[
y = \left( (2 \ln|x^2 - 1| + D)(x^2 - 1) \right)^2
\]
\end{center}
\clearpage
\section*{Номер 2}
\[
y' = y^2 - 2e^x y + e^{2x} + e^x
\]
Известно частное решение $y = e^x$, проверим
\[
e^x = e^{2x} - 2e^{2x} + e^{2x} + e^x
\]
\[
0 = 0
\]
Введем новую неизвестную функцию $z$, чтобы
\[
y = z + e^x 
\]
\[
dy = dz + e^x dx
\]
Подставляем в исходное уравнение
\[
\frac{dz + e^x dx}{dx} =(z + e^x)^2 - 2e^x(z + e^x) + e^{2x} + e^x
\]
\[
\frac{dz}{dx} + e^x =(z + e^x)^2 - 2e^x(z + e^x) + e^{2x} + e^x
\]
\[
\frac{dz}{dx}  = (z + e^x)^2 - 2e^x(z + e^x) + e^{2x}
\]
\[
\frac{dz}{dx}  = z^2 + 2ze^x + e^{2x} - 2e^xz - 2e^{2x} + e^{2x}
\]
\[
\frac{dz}{dx} = z^2 
\]
\[
dz = z^2 dx
\]
Поделим на $z^2$. Решение $z = 0$  это $y = e^x$, мы уже на него смотрели и его знаем
\[
\frac{dz}{z^2} = dx
\]
Интегрируем 
\[
\int \frac{dz}{z^2} = \int  dx
\]
\[
-\frac{1}{z} = x + C
\]
\[
-z = \frac{1}{x + C}
\]
\[
z = - \frac{1}{x + C}
\]
Вспоминаем про замену $y = z + e^x$, отсюда $z = y - e^x$
\[
y - e^x = - \frac{1}{x + C}
\]
\[
y = e^x - \frac{1}{x + C}
\]
\begin{center}
\textbf{Ответ: } 
\[
y = e^x - \frac{1}{x + C}
\]
\[
y = e^x
\]
\end{center}
 \end{document}
