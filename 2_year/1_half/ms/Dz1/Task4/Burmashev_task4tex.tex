\documentclass[a4paper,12pt]{article}

%%% Работа с русским языком
\usepackage{cmap}					% поиск в PDF
\usepackage{mathtext} 				% русские буквы в формулах
\usepackage[T2A]{fontenc}			% кодировка
\usepackage[utf8]{inputenc}			% кодировка исходного текста
\usepackage[english,russian]{babel}	% локализация и переносы
\usepackage{xcolor}
\usepackage{hyperref}
 % Цвета для гиперссылок
\definecolor{linkcolor}{HTML}{799B03} % цвет ссылок
\definecolor{urlcolor}{HTML}{799B03} % цвет гиперссылок

\hypersetup{pdfstartview=FitH,  linkcolor=linkcolor,urlcolor=urlcolor, colorlinks=true}

%%% Дополнительная работа с математикой
\usepackage{amsfonts,amssymb,amsthm,mathtools} % AMS
\usepackage{amsmath}
\usepackage{icomma} % "Умная" запятая: $0,2$ --- число, $0, 2$ --- перечисление

%% Номера формул
%\mathtoolsset{showonlyrefs=true} % Показывать номера только у тех формул, на которые есть \eqref{} в тексте.

%% Шрифты
\usepackage{euscript}	 % Шрифт Евклид
\usepackage{mathrsfs} % Красивый матшрифт

%% Свои команды
\DeclareMathOperator{\sgn}{\mathop{sgn}}

%% Перенос знаков в формулах (по Львовскому)
\newcommand*{\hm}[1]{#1\nobreak\discretionary{}
{\hbox{$\mathsurround=0pt #1$}}{}}
% графика
\usepackage{graphicx}
\graphicspath{{pictures/}}
\DeclareGraphicsExtensions{.pdf,.png,.jpg}
\author{Бурмашев Григорий, БПМИ-208}
\title{}
\date{\today}
\begin{document}
\section*{Задание 4}
Можем заметить, что начиная с 18 достаточно легко собирать цифры из 4к и 7к:
\[
18 = 4 + 7 + 7
\]
\[
19 = 4 + 4 + 4 + 7
\]
\[
20 = 4 + 4 +4 +4 +4 
\]
\[
21 = 7 + 7 + 7
\]
\[
22 = 4 + 4 + 7 + 7 
\]
Теперь воспользуемся методом математической индукии и докажем, что все цифры далее можно собрать из 4к и 7к. Назовем наши числа как $Q$, тогда базой будет являться $Q_{18}, Q_{19}, Q_{20}, Q_{21}, Q_{22}$.

Шаг индукции:

Пусть верно для $Q_n, Q_{n + 1}, Q_{n + 2}, Q_{n + 3}$. Тогда получаем (прибавлением числа 4):
\[
Q_n \rightarrow Q_{n + 4}
\]
\[
Q_{n + 1} \rightarrow Q_{n + 5}
\]
\[
Q_{n + 2} \rightarrow Q_{n + 6}
\]
\[
Q_{n + 3} \rightarrow Q_{n + 7}
\]
\begin{center}
\textbf{Ч.Т.Д} 
\end{center}
Теперь покажем с другой стороны, что меньше 17 не может быть (т.к $k + n$  и $n$ хотя бы 1, то рассматриваем не с 18, а с 17), для этого воспользуемся полным перебором всех комбинаций (таким образом, что при прибавлении 4 или 7 мы либо выйдем за верхнюю границу, либо попадем в уже рассмотренный вариант):
\[
16 = 4 + 4 +4 +4 
\]
\[
15 = 4 + 4 + 7
\]
\[
14 = 7 + 7
\]
\[
13 - \text{мимо }
\]
\[
12 = 4 + 4 + 4
\]
\[
11 = 4 + 7
\]
\[
10 - \text{мимо }
\]
\[
9 - \text{мимо }
\]
\[
8 = 4 + 4
\]
\[
7 = 7
\]
\[
4 = 4
\]
Прибавляя к любому из вариантов 4 или 7, мы либо выскочим за пределы 17, либо попадем в уже рассмотренное число $\rightarrow$ составить меньше 17 не получится, а значит 17 -- это ответ
\begin{center}
\textbf{Ответ: } 17
\end{center}
\end{document}
