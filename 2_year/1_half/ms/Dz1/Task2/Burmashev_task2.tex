\documentclass[a4paper,12pt]{article}

%%% Работа с русским языком
\usepackage{cmap}					% поиск в PDF
\usepackage{mathtext} 				% русские буквы в формулах
\usepackage[T2A]{fontenc}			% кодировка
\usepackage[utf8]{inputenc}			% кодировка исходного текста
\usepackage[english,russian]{babel}	% локализация и переносы
\usepackage{xcolor}
\usepackage{hyperref}
 % Цвета для гиперссылок
\definecolor{linkcolor}{HTML}{799B03} % цвет ссылок
\definecolor{urlcolor}{HTML}{799B03} % цвет гиперссылок

\hypersetup{pdfstartview=FitH,  linkcolor=linkcolor,urlcolor=urlcolor, colorlinks=true}

%%% Дополнительная работа с математикой
\usepackage{amsfonts,amssymb,amsthm,mathtools} % AMS
\usepackage{amsmath}
\usepackage{icomma} % "Умная" запятая: $0,2$ --- число, $0, 2$ --- перечисление

%% Номера формул
%\mathtoolsset{showonlyrefs=true} % Показывать номера только у тех формул, на которые есть \eqref{} в тексте.

%% Шрифты
\usepackage{euscript}	 % Шрифт Евклид
\usepackage{mathrsfs} % Красивый матшрифт

%% Свои команды
\DeclareMathOperator{\sgn}{\mathop{sgn}}

%% Перенос знаков в формулах (по Львовскому)
\newcommand*{\hm}[1]{#1\nobreak\discretionary{}
{\hbox{$\mathsurround=0pt #1$}}{}}
% графика
\usepackage{graphicx}
\graphicspath{{pictures/}}
\DeclareGraphicsExtensions{.pdf,.png,.jpg}
\author{Бурмашев Григорий, БПМИ-208}
\title{}
\date{\today}
\begin{document}
\section*{Задание 2}
Для малых $n$ можем понять на пальцах. Если взять 1 прямую, плоскость поделится ровно на 2 части. Если взять 2 прямые, то возможно 3 случая:
\begin{itemize}
\item Прямые совпадают:

возвращаемся к случаю 1 прямой

\item Прямые параллельны (т.е никогда не пересекутся):

плоскость, очевидно, разобьется на три части

\item Прямые пересекаются:

плоскость разобьется на 4 части (тоже достаточно очевидно)
\end{itemize}

Нас интересует случай \textbf{максимального разбиения}, поэтому прямые нужно пересекать. Если к двум пересекающимся прямым добавить третью, которая будет пересекать их в новой точке, то мы получим 3 новые области и суммарно 7 областей.
Можем наблюдать закономерность:
\[
P_1 = 2
\]
\[
P_2 = 4
\]
\[
P_3 = 7
\]
Можно предположить (ибо похоже), что искомой нами формулой будет $P_n = S_n  + 1 =  1 + \frac{n(n+1)}{2}$
\\\\
Докажем это методом мат.индукции (база уже доказана выше)

Пусть верно для $n$ прямых, тогда для $n + 1$ прямой:

Если наша новая прямая пересечет предыдущие в $q$ разных точках, то добавится $q + 1$ область. Чтобы максимизировать это $q$, нужно провести прямую через новые точки пересечения (т.е не проходить через предыдущие), в таком случае $q = n$. Тогда получаем:
\[
 P_{n} + (n + 1) = 1 + \frac{n(n+1)}{2} + n + 1 = 1 + \frac{(n+1)(n+2)}{2} = P_{n + 1}
\]
\begin{center}
\textbf{Ч.Т.Д } 
\end{center}
\begin{center}
\textbf{Ответ: } $P_n = S_n  + 1 =  1 + \frac{n(n+1)}{2}$
\end{center}
\end{document}
