\documentclass[a4paper,12pt]{article}

%%% Работа с русским языком
\usepackage{cmap}					% поиск в PDF
\usepackage{mathtext} 				% русские буквы в формулах
\usepackage[T2A]{fontenc}			% кодировка
\usepackage[utf8]{inputenc}			% кодировка исходного текста
\usepackage[english,russian]{babel}	% локализация и переносы
\usepackage{xcolor}
\usepackage{hyperref}
 % Цвета для гиперссылок
\definecolor{linkcolor}{HTML}{799B03} % цвет ссылок
\definecolor{urlcolor}{HTML}{799B03} % цвет гиперссылок

\hypersetup{pdfstartview=FitH,  linkcolor=linkcolor,urlcolor=urlcolor, colorlinks=true}

%%% Дополнительная работа с математикой
\usepackage{amsfonts,amssymb,amsthm,mathtools} % AMS
\usepackage{amsmath}
\usepackage{icomma} % "Умная" запятая: $0,2$ --- число, $0, 2$ --- перечисление

%% Номера формул
%\mathtoolsset{showonlyrefs=true} % Показывать номера только у тех формул, на которые есть \eqref{} в тексте.

%% Шрифты
\usepackage{euscript}	 % Шрифт Евклид
\usepackage{mathrsfs} % Красивый матшрифт

%% Свои команды
\DeclareMathOperator{\sgn}{\mathop{sgn}}

%% Перенос знаков в формулах (по Львовскому)
\newcommand*{\hm}[1]{#1\nobreak\discretionary{}
{\hbox{$\mathsurround=0pt #1$}}{}}
% графика
\usepackage{graphicx}
\graphicspath{{pictures/}}
\DeclareGraphicsExtensions{.pdf,.png,.jpg}
\author{Бурмашев Григорий, БПМИ-208}
\title{ТВиМС, дз -- 4}
\date{\today}
\begin{document}
\maketitle 
\clearpage
\section*{Номер 8}
Будем решать с помощью формулы Байеса, для этого введем события:
\begin{center}
$A_n$ -- преподаватель получает работу студента $n$
\end{center}
\begin{center}
$B$ -- студент решил три задачи
\end{center}
Нас интересует вероятность:
\begin{center}
$P(A_n | B) $
\end{center}
Тогда по формуле:
\[
P(A_n | B) = \frac{P(B | A_n) \cdot P(A_n)}{P(B)}
\]
Теперь считаем все нужные нам вероятности:
\begin{center}
$P(A_n) = \frac{1}{3}$, т.к всего 3 студента
\end{center}
\begin{center}
$P(B|A_n)$ -- решено три задачи n-ым студентом
\end{center}
Нам нужно расставить 3 верно решенных задачи на 4 позиции, т.е у нас будет 4 множителя, каждый из которых имеет вид (т.к все задачи одинаковые и порядок нам не важен) [под решением подразумеваю $n$-го студента]:
\[
P(\text{решена неверно}) \cdot P(\text{верно})^3
\]
При этом:
\[
P(\text{решена неверно}) = 1 - P(\text{верно})
\]
Тогда:
\[
P(B|A_1) = \frac{1 }{4} \cdot \left(\frac{3}{4} \right)^3 \cdot 4 = \frac{27}{64}
\]
\[
P(B|A_2) = \frac{1}{2} \cdot \left(\frac{1}{2}\right)^3 \cdot 4 =  \frac{1}{16} \cdot 4 = \frac{1}{4} = \frac{16}{64}\]
\[
P(B|A_3) = \frac{3}{4} \cdot \left(\frac{1}{4}\right)^3 \cdot 4 = 3 \cdot \frac{1}{64} = \frac{3}{64}
\]
Теперь можем заметить, что $P(B)$ не зависит от выбора студента и для всех студентов одинакова, т.е при сравнении вероятностей она нас не интересует, вероятность $P(A_n)$ у всех студентов тоже одинаковая, значит наибольшая вероятность будет при наибольшем множителе $P(B|A_n)$, это:
\[
P(B|A_1) = \frac{27}{64} > \frac{16}{64} > \frac{3}{64}
\]
\begin{center}
\textbf{Ответ: } работа скорее всего принадлежит первому студенту
\end{center}
\clearpage
\section*{Номер 9}
Введем аналогичные предыдущей задаче события:
\begin{center}
$A_n$ -- $n$-e заболевание \\
$B$ - 3 раза из 4 тест положителен
\end{center}
Тогда:
\[
P(A_n | B) = \frac{P(B | A_n) \cdot P(A_n)}{P(B)}
\]
Решение в таком случае (почти) аналогичное задаче номер 8:
\begin{center}
$P(B|A_n)$ -- получили 3 положительных теста при условии болезни $A_n$
\end{center}
Аналогично расставляем 3 верных теста на 4 позиции, при этом вероятности верных и неверных тестов мы знаем из условия (верные $0.1$ для $A_1$, $0.2$ для $A_2$ и $0.8$ для $A_3$)
\[
P(B|A_1) =(0.1)^3 \cdot 0.9 \cdot 4 = \frac{1}{1000} \cdot \frac{9}{10} \cdot 4 = \frac{9 \cdot 4}{10000} = \frac{9}{2500}
\]
\[
P(B|A_2) = (0.2)^3 \cdot 0.8 \cdot 4 = \frac{1}{125} \cdot \frac{4}{5} \cdot 4 = \frac{16}{625}
\]
\[
P(B|A_3) = (0.8)^3 \cdot 0.2 \cdot 4 = \frac{64}{125} \cdot \frac{1}{5} \cdot 4 = \frac{256}{625}
\]
По аналогии с предыдущей задачей видим, что вероятность третьей болезни выше остальных, но нам нужно не это, считаем $P(B)$:
\[
P(B) = P(B|A_1) \cdot P(A_1) + P(B|A_2) \cdot P(A_2) + P(B|A_3) \cdot P(A_3)\]
Знаем:
\[
P(A_n) = p_n \text{ (дано в условии)}
\]
Тогда:
\[
P(B) = \frac{9}{2500} \cdot \frac12+ \frac{16}{625}\cdot \frac13 + \frac{256}{625} \cdot \frac16 = \frac{9}{5000} + \frac{16}{1875} + \frac{128}{1875} = \frac{393}{5000}
\]
Теперь считаем:
\[
P(A_1|B) = \frac{\frac{9}{2500} \cdot \frac{1}{2}}{\frac{393}{5000}} = \frac{3}{131}
\] 
\[
P(A_2|B) = \frac{\frac{16}{625} \cdot \frac{1}{3}}{\frac{393}{5000}} = \frac{128}{1179}
\]
\[
P(A_3|B) = \frac{\frac{256}{625} \cdot \frac{1}{6}}{\frac{393}{5000}} = \frac{1024}{1179}
\]
\begin{center}
\textbf{Ответ: } вероятность первого заболевания $\frac{3}{131}$, второго $\frac{128}{1179}$, третьего $\frac{1024}{1179}$
\end{center}
\clearpage
\section*{Номер 10}
Введем события:
\begin{center}
A -- человек здоровый \\
B -- человек богатый \\
C -- человек умный
\end{center}
Из условия получаем, что:
\[
P(A) > \frac12 \text{ (больше половины) }
\]
\[
P(B) > \frac12 \text{ (больше половины) }
\]
\[
P(C) \neq 0 \text{ (есть хотя бы один умный) } 
\]
Используем формулу независимых событий:
\[
P(A \cap C) = P(A) \cdot P(C) > \frac{1}{2} \cdot P(C)
\]
\[
P(B \cap C) = P(B) \cdot P(C) > \frac{1}{2} \cdot P(C)
\]
К тому же заметим, что:
\[
A \cap C \subseteq C
\]
\[
B \cap C \subseteq C
\]
Отсюда:
\[
X = (A \cap C) \cup (B \cap C)  \subseteq C
\]
Другими словами:
\[
P(X) \leq P(C) 
\]
Посмотрим на $P(X)$:
\[
P(X) = P(A \cap C) \left[> \frac12 P(C)\right]+ P(B \cap C)\left[> \frac12 P(C)\right] - P(A \cap B \cap C) \leq P(C)
\]
Из $> \frac12 P(C)$ у двух множителей получаем, что:
\[
\left[ > P(C)\right] - P(A \cap B \cap C) \leq P(C)
\]
\[
\left[ > P(C)\right] - P(C) \leq P(A \cap B \cap C)
\]
\[
0 < P(A \cap B \cap C) =
\] 
\[
=
P( \text{гражданин здоровый и богатый и умный})
 \]
\[
\rightarrow \text{ такой гражданин есть }
\]
\begin{center}
\textbf{Ч.Т.Д} 
\end{center}
\end{document}
