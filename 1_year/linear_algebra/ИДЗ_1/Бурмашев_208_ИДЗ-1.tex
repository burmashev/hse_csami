\documentclass[a4paper,12pt]{article}

%%% Работа с русским языком
\usepackage{cmap}					% поиск в PDF
\usepackage{mathtext} 				% русские буквы в формулах
\usepackage[T2A]{fontenc}			% кодировка
\usepackage[utf8]{inputenc}			% кодировка исходного текста
\usepackage[english,russian]{babel}	% локализация и переносы
\usepackage{xcolor}
\usepackage{hyperref}
 % Цвета для гиперссылок
\definecolor{linkcolor}{HTML}{799B03} % цвет ссылок
\definecolor{urlcolor}{HTML}{799B03} % цвет гиперссылок

\hypersetup{pdfstartview=FitH,  linkcolor=linkcolor,urlcolor=urlcolor, colorlinks=true}

%%% Дополнительная работа с математикой
\usepackage{amsfonts,amssymb,amsthm,mathtools} % AMS
\usepackage{amsmath}
\usepackage{icomma} % "Умная" запятая: $0,2$ --- число, $0, 2$ --- перечисление

%% Номера формул
%\mathtoolsset{showonlyrefs=true} % Показывать номера только у тех формул, на которые есть \eqref{} в тексте.

%% Шрифты
\usepackage{euscript}	 % Шрифт Евклид
\usepackage{mathrsfs} % Красивый матшрифт

%% Свои команды
\DeclareMathOperator{\sgn}{\mathop{sgn}}

%% Перенос знаков в формулах (по Львовскому)
\newcommand*{\hm}[1]{#1\nobreak\discretionary{}
{\hbox{$\mathsurround=0pt #1$}}{}}
% графика
\usepackage{graphicx}
\graphicspath{{pictures/}}
\DeclareGraphicsExtensions{.pdf,.png,.jpg}
\author{Бурмашев Григорий, БПМИ-208}
\title{Линал. ИДЗ - 1. Вариант 2.}
\date{\today}
\begin{document}
\maketitle
\section*{1.} Вычислить матрицу (по данным матрицам в условии)
\[
-2D \cdot A \cdot A^T + tr(B^T \cdot B) \cdot (A-B) \cdot (A^T + B^T) - 6C^2 - 12C \cdot D - 6D^2
\]
\\
Вычислим по порядку:
\[
A^T = \begin{pmatrix}
5  & -3\\
-6 & -4\\
1 & 4\\
\end{pmatrix}
\]
\[
B^T = \begin{pmatrix}
-6 & 3\\
1 & -3\\
-4 & 7
\end{pmatrix}
\]
\[
-2D = -2 \cdot
\begin{pmatrix}
4 & 7\\
6& -3 \\
\end{pmatrix}
= 
\begin{pmatrix}
-8 & 14\\
-12 & 6\\
\end{pmatrix}
\]
\[
-2D \cdot A =\begin{pmatrix}
-8 & 14\\
-12 & 6\\
\end{pmatrix} \cdot
\begin{pmatrix}
5 & -6 & 1\\
-3 & -4 & 4\\
\end{pmatrix} = 
\begin{pmatrix}
2 & 104 & -64 \\
-78 & 48 & 12\\
\end{pmatrix}
\]
\[
X = -2D \cdot A \cdot A^T = \begin{pmatrix}
2 & 104 & -64 \\
-78 & 48 & 12\\
\end{pmatrix} \cdot
\begin{pmatrix}
5  & -3\\
-6 & -4\\
1 & 4\\
\end{pmatrix} = 
\begin{pmatrix}
-678 & -678\\
-666 & 90\\
\end{pmatrix}
\]

\[
B ^ T \cdot B = 
\begin{pmatrix}
-6 & 3\\
1 & -3\\
-4 & 7\\
\end{pmatrix}
\cdot 
\begin{pmatrix}
-6 & 1 & -4\\
3  & -3 & 7
\end{pmatrix}
=
\begin{pmatrix}
6 \cdot 6 + 3\cdot 3 & -6 - 3\cdot 3& 6 \cdot 4 + 3\cdot 7\\
-6 - 3\cdot 3 & 1 + 3\cdot 3 & -4 - 3 \cdot 7\\
4 \cdot  + 7 \cdot 3 & -4 - 7 \cdot 3 & 4 \cdot 4 + 7 \cdot 7\\
\end{pmatrix} = 
\]
\[=
\begin{pmatrix}
45 & -15 & 45\\
-15 & 10 & -25\\
45 & -25 & 65\\
\end{pmatrix}
\]
\[
tr(B^T \cdot B) = 45 + 10 + 65 = 120
\]
\[
A - B = 
\begin{pmatrix}
5 + 6 & -6 -1 & 1 + 4\\
-3 -3 & -4 + 3 & 4 - 7\\
\end{pmatrix}
=
\begin{pmatrix}
11 & -7 & 5\\
-6 & -1 & -3\\
\end{pmatrix}
\]
\[
A^T + B^T = 
\begin{pmatrix}
5 - 6 & -3 + 3\\
-6 + 1 & -4 - 3\\
1  -4 & 4 + 7\\
\end{pmatrix} = 
\begin{pmatrix}
-1 & 0\\
-5 & -7\\
-3 & 11\\
\end{pmatrix}
\]
\[
(A -B) \cdot (A^T + B^T) = \begin{pmatrix}
11 & -7 & 5\\
-6 & -1 & -3\\
\end{pmatrix} \cdot
\begin{pmatrix}
-1 & 0\\-5 & -7 \\ -3 & 11\\
\end{pmatrix} = 
\begin{pmatrix}
9 & 104\\20 & -26\\
\end{pmatrix}
\] 
\[
Y =tr(B^T \cdot B) \cdot (A -B) \cdot (A^T + B^T) = 
120 \cdot \begin{pmatrix}
9 & 104 \\ 20 & -26\\
\end{pmatrix} = 
\begin{pmatrix}
1080 & 12480\\ 2400 & -3120\\
\end{pmatrix}
\]
\[
6C^2 = 6 \cdot \begin{pmatrix}
28 & -54 \\ -18 & 37\\
\end{pmatrix} = \begin{pmatrix}
168 & -324 \\ -108  & 222\\
\end{pmatrix}
\]
\[
C \cdot D = \begin{pmatrix}
4 & -6 \\ -2 & 5 
\end{pmatrix} \cdot 
\begin{pmatrix}
4 & 7 \\ 6 & -3\\
\end{pmatrix}  = \begin{pmatrix}
-20 & 46\\ 22 & -29
\end{pmatrix}
\]
\[
12C \cdot D = 12 \cdot \begin{pmatrix}
-20 & 46 \\ 22 & -29
\end{pmatrix} = \begin{pmatrix}
-240 & 552 \\ 264 & -348\\
\end{pmatrix}
\]
\[
6C^2  + 12C \cdot D =\begin{pmatrix}
-72 & 228\\ 156 & -126\\
\end{pmatrix}
\]
\[
6D^2 = 6 \cdot \begin{pmatrix}
58 & 7 \\ 6 & 51\\
\end{pmatrix} = \begin{pmatrix}
348 & 42 \\ 36 & 306\\
\end{pmatrix}
\]
\[
Z = 6C^2 + 12C \cdot D + 6D^2 = \begin{pmatrix}
276 & 270 \\ 192 & 180\\
\end{pmatrix}
\]
\[
X + Y = \begin{pmatrix}
-679 + 1080 & -678 + 12480 \\ -666 + 2400 & 90 - 3120\\
\end{pmatrix} = \begin{pmatrix}
402 & 11802 \\ 1734 & -3030\\
\end{pmatrix}
\]
\[
X + Y - Z = \begin{pmatrix}
402- 276 & 11802- 270 \\ 1734 - 192 & -3030 - 180\\
\end{pmatrix} = 
\begin{pmatrix}
126& 11532 \\ 1542 & -3210
\end{pmatrix}
\]
\begin{large}
\begin{center}
\textbf{Ответ:} $\begin{pmatrix}
126 & 11532\\ 1542 & -3210\\
\end{pmatrix} $
\end{center}
\end{large}
\section*{2.} 
Решите каждую из приведённых ниже систем линейных уравнений методом Гаусса. Если система совместна, то выпишите её общее решение и укажите одно частное решение
\subsection*{(а)}
\[
\begin{pmatrix}
5 & -2 & 0 & -11 & \vrule & -5 & \\
-8 & 4 & -4 & 16 & \vrule & 12 & \\
8 & 1 & -21 & -26 & \vrule & 13 & \\
10 & -7 & 15 & -16 & \vrule & -25 \\
\end{pmatrix} =
\begin{pmatrix}
1 & 0 & -2 & -3 & \vrule & 1 & \\
-4 & 2 & -2 & 8 & \vrule & 6 & \\
0 & 5 & -25 & -10 & \vrule & 25 & \\
0 & -3 & 15 & 6& \vrule & -15 \\
\end{pmatrix}  = 
\]
\[=
\begin{pmatrix}
1 & 0 & -2 & -3 & \vrule & 1 & \\
0 & -1 & 5& 2 & \vrule & -5 & \\
0 & 5 & -25& -10 & \vrule & 25 & \\
0 & -3 & 15 & 6& \vrule & -15 \\
\end{pmatrix}  =
\begin{pmatrix}
1 & 0 & -2 & -3 & \vrule & 1 & \\
0 & -1 & 5& 2& \vrule &-5 & \\
0 & 0 & 0 & 0 & \vrule & 0 & \\
0 & 0 & 0& 0& \vrule & 0 \\
\end{pmatrix} 
\]
\\\\
Вернемся к системе уравнений:
\[
\begin{cases}
x_1 - 2x_3 - 3x_4 = 1 \\
-x_2 +5x_3 + 2x_4 = -5\\
x_3 - \text{произвольное}\\
x_4 - \text{произвольное}\\
\end{cases}
\]
\[
\begin{cases}
x_1 = 1 + 2x_3 + 3x_4\\
x_2 = 5  +5x_3 + 2x_4\\
x_3 - \text{произвольное}\\
x_4 - \text{произвольное}\\
\end{cases}
\]\\
\begin{large}
\begin{center}
\textbf{Ответ:} \\
общее решение:
\[
\begin{cases}
x_1 = 1 + 2x_3 + 3x_4\\
x_2 = 5  +5x_3 + 2x_4\\
x_3 - \text{произвольное}\\
x_4 - \text{произвольное}\\
\end{cases}
\]
одно частное решение:
\[
\begin{cases}
x_1 = 9\\
x_2 = 14\\
x_3 = 1\\
x_4 = 2\\
\end{cases}
\]
\end{center}
\end{large}
\subsection*{(б)}
\[
\begin{pmatrix}
5 & -2 & 0 & -11 & \vrule & -5 & \\
-8 & 4 & -4 & 16 & \vrule & 0 & \\
8 & 1 & -21 & -26 & \vrule & -9 & \\
10 & -7 & 15 & -16 & \vrule & 0\\
\end{pmatrix} =
\begin{pmatrix}
1 & 0 & -2 & -3 & \vrule & -5 & \\
-4 & 2 & -2 & 8 & \vrule & 0 & \\
0 & 5 & -25 & -10 & \vrule & -9 & \\
0 & -3 & 15 & 6& \vrule & 10 \\
\end{pmatrix}  = 
\]
\[=
\begin{pmatrix}
1 & 0 & -2 & -3 & \vrule & -5 & \\
0& -1 & 5 & 2 & \vrule & 10 & \\
0 & 5 & -25 & -10 & \vrule & -9 & \\
0 & -3 & 15 & 6& \vrule & 10 \\
\end{pmatrix}  = 
\begin{pmatrix}
1 & 0 & -2 & -3 & \vrule & -5 & \\
0& -1 & 5 & 2 & \vrule & 10 & \\
0 & 0 & 0 & 0& \vrule & 41 & \\
0 & -3 & 15 & 6& \vrule & 10 \\
\end{pmatrix}  
\]
\\
В третьей строке матрицы:
\[
0x_1 + 0x_2 + 0x_3 + 0x_4 = 41
\]
\[
0 = 41
\]
\begin{center}
\textbf{противоречие}, значит система несовместна \\
\end{center}
\begin{large}
\begin{center}
\textbf{Ответ:}  система несовместна
\end{center}
\end{large}
\section*{3.}
$AX = XA$, пусть:
\[
X = \begin{pmatrix}
a & 0 & b \\
c & d & e \\
0 & 0 & f\\
\end{pmatrix}
\]
Тогда:
\[
\begin{pmatrix}
10 & 0 & 2 \\
-1 & -4 & 3\\
0 & 0 & 0 \\
\end{pmatrix}
\cdot
 \begin{pmatrix}
a & 0 & b \\
c & d & e \\
0 & 0 & f\\
\end{pmatrix} = 
 \begin{pmatrix}
a & 0 & b \\
c & d & e \\
0 & 0 & f\\
\end{pmatrix} 
\cdot
\begin{pmatrix}
10 & 0 & 2 \\
-1 & -4 & 3\\
0 & 0 & 0 \\
\end{pmatrix}
\]
Умножим матрицы:
\[
\begin{pmatrix}
10a & 0 & 10b+2f\\
-a-4c& -4d & -b + 3f - 4e\\
0 & 0 & 0 &
\end{pmatrix}
=
\begin{pmatrix}
10a & 0 & 2a\\
-10c-d& -4d & 2c+3d\\
0 & 0 & 0 &
\end{pmatrix}
\]
Чтобы эти матрицы были равны, должна быть верна система:
\[
\begin{cases}
10b + 2f = 2a\\10c - d = -a -4c\\
2c + 3d = -b + 3f - 4e\\
\end{cases}
\]
\[
\begin{cases}
5b + f = a\\
14c = d - a\\
2c + 3d = -b + 3f - 4e\\
\end{cases}
\]
Мы знаем, что:
\[
a = 5b + f
\]
Выразим c:
\[
14c = d - 5v - f
\]
\[
c = \frac{d-5b-f}{14}
\]
Выразим d:
\[
\frac{d-5b-f}{7} + 3d = -b + 3f - 4e
\]
\[
22d = 22f - 2b - 28e
\]
\[
d = \frac{11f-b-14e}{11}
\]
Значит матрица X имеет вид:
\[
\begin{pmatrix}
5b+f & 0 & b\\
\frac{d-5b-f}{14} & \frac{11f-b-14e}{11} & e\\
0 & 0 & f\\
\end{pmatrix}
\]
где $b, e, f$ -- произвольные числа
\begin{large}
\begin{center}
\textbf{Ответ:}
$
\begin{pmatrix}
5b+f & 0 & b\\
\frac{d-5b-f}{14} & \frac{11f-b-14e}{11} & e\\
0 & 0 & f\\
\end{pmatrix} 
$
\end{center}
\begin{center}
где $b, e, f$ -- произвольные 
\end{center}
\end{large}
\section*{4.}
Определите число решений системы в зависимости от a  и b
\begin{large}
\[
\begin{pmatrix}
a & -2 & -5 & \vrule &7\\
2 & 0 & b & \vrule & -9\\
0 & -1 & -2 & \vrule & 2\\
\end{pmatrix} =
\begin{pmatrix}
a & 0& -1 & \vrule &3\\
2 & 0 & b & \vrule & -9\\
0 & 1 & 2 & \vrule & -2\\
\end{pmatrix} = 
\begin{pmatrix}
1 & 0 & \frac{b}{2} & \vrule & -\frac{9}{2}\\
a & 0& -1 & \vrule &3\\
0 & 1 & 2 & \vrule & -2\\
\end{pmatrix} =
\]
\[
= 
\begin{pmatrix}
1 & 0 & \frac{b}{2} & \vrule & -\frac{9}{2}\\
0 & 0& -1-\frac{ab}{2} & \vrule &3 + \frac{9a}{2}\\
0 & 1 & 2 & \vrule & -2\\
\end{pmatrix} =
\begin{pmatrix}
1 & 0 & \frac{b}{2} & \vrule & -\frac{9}{2}\\
0 & 0& -1-\frac{ab}{2} & \vrule &3 + \frac{9a}{2}\\
0 & 1 & 2 & \vrule & -2\\
\end{pmatrix} =
\]
\[
=
\begin{pmatrix}
1 & 0 & \frac{b}{2} & \vrule & -\frac{9}{2}\\
0 & 1 & 2 & \vrule & -2\\
0 & 0& \frac{-2-ab}{2} & \vrule & \frac{6+9a}{2}\\
\end{pmatrix}
\]
\begin{itemize}
\item Если  $-2 - ab = 0$:
\[
0x + 0y + 0z = \frac{6+9a}{2}
\]
\textbf{1)} Если $\frac{6+9a}{2} \neq 0$, т.е  $3a+2 \neq 0$, то мы получаем противоречие (cлева ноль, а справа -- что-то, не равное нулю)

\textbf{2)} Если $\frac{6+9a}{2} = 0 $, т.е $3a+2 = 0$, то:
\[
\begin{cases}
a = -\frac{2}{3}\\
b = 3\\
\end{cases}
\]
\[
\begin{pmatrix}
1 & 0 & \frac{3}{2} & \vrule & -\frac{9}{2}\\
0 & 1 & 2 & \vrule & -2\\
0 & 0& 0& \vrule & 0\\
\end{pmatrix}
\]
\[
\begin{cases}
x =- \frac{9}{2} - \frac{3}{2} \cdot z\\
y = -2 - 2 \cdot z\\
z - \text{произвольное}
\end{cases}
\]
\item Если $-2 - ab\neq 0$:
\[
\begin{pmatrix}
1 & 0 & \frac{b}{2} & \vrule & -\frac{9}{2}\\
0 & 1 & 2 & \vrule & -2\\
0 & 0& \frac{-2 -ab}{2} & \vrule & \frac{6+9a}{2}\\
\end{pmatrix} =
\begin{pmatrix}
1 & 0 & \frac{b}{2} & \vrule & -\frac{9}{2}\\
0 & 1 & 2 & \vrule & -2\\
0 & 0& 1 & \vrule & \frac{6+9a}{-2-ab}\\
\end{pmatrix} =
\begin{pmatrix}
1 & 0 & \frac{b}{2} & \vrule & -\frac{9}{2}\\
0 & 1 & 0& \vrule & \frac{-18+2a-8}{-2-ab}\\
0 & 0& 1 & \vrule & \frac{6+9a}{-2-ab}\\
\end{pmatrix} =
\]
\[
=
\begin{pmatrix}
1 & 0 & 0 & \vrule & \frac{9-3b}{-2-ab}\\
0 & 1 & 0& \vrule & \frac{-18+2a-8}{-2-ab}\\
0 & 0& 1 & \vrule & \frac{6+9a}{-2-ab}\\
\end{pmatrix}
\]
\[
\begin{cases}
x =\frac{9-3b}{-2-ab}\\
y = \frac{-18+2a-8}{-2-ab}\\
z = \frac{6+9a}{-2-ab}\\
\end{cases}
\]
\end{itemize}
\begin{center}
\textbf{Ответ:}\\
Если $ab \neq -2$, то система совместна и имеет решение:
\[
\begin{cases}
x =\frac{9-3b}{-2-ab}\\
y = \frac{-18+2a-8}{-2-ab}\\
z = \frac{6+9a}{-2-ab}\\
\end{cases}
\]
\\
Если $ab = -2$, то есть два случая:\\
\end{center}
\begin{center}
\textbf{1)}
если $a \neq -\frac{2}{3}$, то cистема несовместна
\\
\textbf{2)} если $a = -\frac{2}{3}$, то система совместна и имеет бесконечно число решений:
\[
\begin{cases}
x =- \frac{9}{2} - \frac{3}{2} \cdot z\\
y = -2 - 2 \cdot z\\
z - \text{произвольное}
\end{cases}
\]
\end{center}
\end{large}
\section*{5.}
Найдите матрицу, обратную данной
\[
\begin{pmatrix}
1 & -2 & -1 & 0 & \vrule & 1 & 0 & 0 & 0\\
4 & -4 &-1 & -1 & \vrule & 0 & 1  & 0 &0\\
-1 & -14 & -1 & 3 & \vrule & 0 & 0 & 1 & 0\\
-1 & -3 & 0 & 1 & \vrule & 0 & 0 & 0 & 1\\
\end{pmatrix} =
\begin{pmatrix}
1 & -2 & -1 & 0 & \vrule & 1 & 0 & 0 & 0\\
0 & 4&3& -1 & \vrule & -4& 1  & 0 &0\\
0 & -16 & -2& 3 & \vrule & 1 & 0 & 1 & 0\\
0 & -5 & -1 & 1 & \vrule & 1 & 0 & 0 & 1\\
\end{pmatrix} =
\]
\[
= 
\begin{pmatrix}
1 & -2 & -1 & 0 & \vrule & 1 & 0 & 0 & 0\\
0 &1&-2& 0 & \vrule & 3& 1  & 0 &-1\\
0 & -16 & -2& 3 & \vrule & 1 & 0 & 1 & 0\\
0 & -5 & -1 & 1 & \vrule & 1 & 0 & 0 & 1\\
\end{pmatrix} =
\begin{pmatrix}
1 & 0 & -5& 0 & \vrule & 7& -2& 0 & -2\\
0 & 1&-2& 0& \vrule & 3& -1  & 0 &-1\\
0 & 0& -34& 3 & \vrule & 49& -16 & 1 & -16\\
0 & 0& -11 & 0 & \vrule & -16& 5 & 0 & 4\\
\end{pmatrix} =
\]
\[
=
\begin{pmatrix}
1 & 0 & -5& 0 & \vrule & 7& -2& 0 & -2\\
0 & 1&-2& 0& \vrule & 3& -1  & 0 &-1\\
0 & 0& 1& 0 & \vrule & -1& 1 & -1& -4\\
0 & 0& -11 & 0 & \vrule & -16& 5 & 0 & 4\\
\end{pmatrix} =
\begin{pmatrix}
1 & 0 & 0& 0 & \vrule & 2& 3& -5 & -18\\
0 & 1&0& 0& \vrule & 1& 1  & -2&7\\
0 & 0& 1& 0 & \vrule & -1& 1 & -1& -4\\
0 & 0& 0& -1& \vrule & -5& -6 & 11 & -40\\
\end{pmatrix} =
\]
\[
=
\begin{pmatrix}
1 & 0 & 0& 0 & \vrule & 2& 3& -5 & -18\\
0 & 1&0& 0& \vrule & 1& 1  & -2&7\\
0 & 0& 1& 0 & \vrule & -1& 1 & -1& -4\\
0 & 0& 0& 1& \vrule & 5& 6 & -11 & 40\\
\end{pmatrix} 
\]
\begin{large}
\begin{center}
\textbf{Ответ:} $\begin{pmatrix}
 2& 3& -5 & -18\\
 1& 1  & -2&7\\
-1& 1 & -1& -4\\
 5& 6 & -11 & 40\\
\end{pmatrix} $
\end{center}
\end{large}

\end{document}