\documentclass[a4paper,12pt]{article}

%%% Работа с русским языком
\usepackage{cmap}					% поиск в PDF
\usepackage{mathtext} 				% русские буквы в формулах
\usepackage[T2A]{fontenc}			% кодировка
\usepackage[utf8]{inputenc}			% кодировка исходного текста
\usepackage[english,russian]{babel}	% локализация и переносы
\usepackage{xcolor}
\usepackage{hyperref}
 % Цвета для гиперссылок
\definecolor{linkcolor}{HTML}{799B03} % цвет ссылок
\definecolor{urlcolor}{HTML}{799B03} % цвет гиперссылок

\hypersetup{pdfstartview=FitH,  linkcolor=linkcolor,urlcolor=urlcolor, colorlinks=true}

%%% Дополнительная работа с математикой
\usepackage{amsfonts,amssymb,amsthm,mathtools} % AMS
\usepackage{amsmath}
\usepackage{icomma} % "Умная" запятая: $0,2$ --- число, $0, 2$ --- перечисление

%% Номера формул
%\mathtoolsset{showonlyrefs=true} % Показывать номера только у тех формул, на которые есть \eqref{} в тексте.

%% Шрифты
\usepackage{euscript}	 % Шрифт Евклид
\usepackage{mathrsfs} % Красивый матшрифт

%% Свои команды
\DeclareMathOperator{\sgn}{\mathop{sgn}}

%% Перенос знаков в формулах (по Львовскому)
\newcommand*{\hm}[1]{#1\nobreak\discretionary{}
{\hbox{$\mathsurround=0pt #1$}}{}}
% графика
\usepackage{graphicx}
\graphicspath{{pictures/}}
\DeclareGraphicsExtensions{.pdf,.png,.jpg}
\author{Бурмашев Григорий}
\title{Дискра - 1}
\date{\today}
\begin{document}
\begin{center}
Бурмашев Гриша. 208. Дискра - 3
\end{center}
\section*{1}
Кабинет Нумеролога:\\\\
В октябре дни с 1 по 31. \\Числа, в которые входит 0 : отрезок $[1, 9]$, 10, 20, 30

(Однозначные он пишет как 01, 02 $\ldots$)
\\\\
В ноябре дни с 1 по 30. \\Числа, в которые входит 1: 1, отрезок $[10, 19]$, 21
\\\\
В декабре дни с 1 по 31. \\Числа, в которые входит 2: 2, 12, отрезок $[20, 29]$
\\\\
Итого: 12 + 12 + 12 = 36
\\\\
\textbf{Ответ:} 36
\section*{2}
30 студентов по 10 аудиториям:
\\\\
У каждого студента есть выбор из 10 аудиторий,  и он может выбрать любую. Логично, что каждый из 30 студентов может выбрать любую аудиторию.  Тогда по правилу произведения:
\[
10 \times 10 \times 10 \times \ldots \times 10_{30} = 10^{30}
\]
Всего есть $10^{30}$ способов распределить 30 студентов по 10 аудиториям
\\\\
\textbf{Ответ:} $10^{30}$
\section*{3}
4 - значные числа, где хотя бы 1 цифра 7:
\\\\
Посчитаем общее число 4 - значных чисел:

На каждую из позиций можно поставить любое число от 0 до 10, кроме первой (т.к число не может начинаться с нуля) Значит для первой позиции 9 вариантов,  для остальных - 10. Тогда:
\[
9 * 10 * 10 * 10  = 9000
\]
Вычтем из <<всего>>  те числа, которые нам не подходят

4-х значные числа, в которых нет 7:
\[
8 * 9 * 9 * 9 = 5832
\]

Разность:
\[
9000 - 5832 = 3168
\]
\textbf{Ответ:} 3168
\section*{4}
Докажите, что:
\[
(A_1 \setminus A_2) \times (B_1 \setminus B_2) \subseteq (A_1 \times B_1) \setminus (A_2 \times B_2)
\]
Пусть у нас есть упорядоченная пара элементов:
\[
(a;b)
\]
Т.е по определению декартового произведения:
\[
(a;b) \in (A_1 \setminus A_2) \times (B_1 \setminus B_2) 
\]
\[
(a \in A_1 \setminus A_2) \cap (b \in B_1 \setminus B_2)
\]
\[
(a \in A_1 \cap a \notin A_2) \cap (b \in B_1 \cap b \notin B_2)
\]
Можно поменять элементы местами:
\[
(a \in A_1 \cap b \in B_1) \cap (a \notin A_2 \cap b \notin B_2) 
\]
Значит, что
\[
(a \in A_1 \cap b \in B_1) \rightarrow (a;b) \in (A_1 \times B_1)
\]
\[(a \notin A_2 \cap b \notin B_2) \rightarrow (a;b) \notin (A_2 \times B_2)
\]
\[
(a;b) \in (A_1 \times B_1) \setminus (A_2 \times B_2)
\]
Тогда:
\[
(a \in A_1 \cap b \in B_1) \cap (a \notin A_2 \cap b \notin B_2) \subseteq  (a \in A_1 \cap b \in B_2) \cap (a \notin A_2 \cup b \notin B_2)
\]
Пусть:
\[
F = (a \notin A_2 \cap b \notin B_2)
\]
\[
G = (a \notin A_2 \cup b \notin B_2)
\]
В G мы имеем случаи, когда выполняется либо $a \notin A_2$, либо $b \notin B_2$, либо и то, и другое (всего 3 варианта). А F является одним из этих трех вариантов (выполняется и то, и другое) Т.е F является подмножеством G
\\\\
Можно точно сказать, что:
\[
(a \notin A_2 \cap b \notin B_2) \subseteq (a \notin A_2 \cup b \notin B_2)
\]
Значит верно, что:
\[
(a \in A_1 \cap b \in B_1) \cap (a \notin A_2 \cap b \notin B_2) \subseteq  (a \in A_1 \cap b \in B_2) \cap (a \notin A_2 \cup b \notin B_2)
\]
\[
(A_1 \setminus A_2) \times (B_1 \setminus B_2) \subseteq (A_1 \times B_1) \setminus (A_2 \times B_2)
\]
\begin{center}
\textbf{Ч.Т.Д}
\end{center}
\section*{5}
Равенство возможно, если слева и справа в исходной записи будет $\varnothing$ (т.е $(A_1 \setminus A_2) \times (B_1 \setminus B_2) =(A_1 \times B_1) \setminus (A_2 \times B_2)  = \varnothing$)
\\\\
Это возможно, если $A_1 \subseteq A_2$ и  $B_1 \subseteq B_2$. 
\\\\Тогда:
\[
A_1 \setminus A_2 = \varnothing, \; \; 
B_1 \setminus B_2 = \varnothing \rightarrow
\]
\[
\rightarrow  (A_1 \setminus A_2) \times (B_1 \setminus B_2) = \varnothing
\]
А также:
\[
A_1 \times B_1 \subseteq A_2 \times B_2 \rightarrow
\]
\[
\rightarrow (A_1 \times B_1) \setminus (A_2 \times B_2) = \varnothing
\]
Тогда:
\[
(A_1 \setminus A_2) \times (B_1 \setminus B_2) =(A_1 \times B_1) \setminus (A_2 \times B_2)  = \varnothing
\]
\begin{center}
\textbf{Равенство выполняется}
\end{center}
Равенство также будет, если посмотреть на $(a \notin A_2 \cap b \notin B_2) \subseteq (a \notin A_2 \cup b \notin B_2)$ и превратить 3 возможных случая G (из № 4) в один. Это получится, если $B_2 = A_2 = \varnothing$ \\
Тогда:
\[
A_1 \setminus A_2 = A_1 \setminus \varnothing = A_1
\]
\[
B_1 \setminus B_2 = B_1 \setminus \varnothing = B_1
\]
\[
A_2 \times B_2 = \varnothing
\]
\[
(A_1 \setminus A_2) \times (B_1 \setminus B_2) = A_1 \times B_1
\]
\[
(A_1 \times B_1) \setminus (A_2 \times B_2) = (A_1 \times B_1) \setminus \varnothing = A_1 \times B_1
\]
\[
A_1 \times B_1 = A_1 \times B_1
\]
\begin{center}
\textbf{Равенство выполяется}
\end{center}
\textbf{Ответ:} $ ((A_1 \subseteq A_2) \wedge (B_1 \subseteq B_2)) \vee (A_2 = B_2 = \varnothing)$
\section*{6}
Двоичные слова длины n без двух нулей подряд:
\\\\
Попробуем составить рекуррентную последовательность  для двоичных слов, чтобы доказать, что их число связано с Фибоначчи (т.к числа Фибоначчи есть рекуррентое соотношение)
\\\\
Пусть $A_n$ -  число двоичных слов длины n, заканчивающихся на 1 (например: слово  10101)
\\
Пусть $B_n$ - число двоичных слов длины n, заканчивающихся  на 0 (например: слово 1011010)
\\\\
Двоичное слово, которое заканчивается на 1, мы можем получить как прибавлением единицы к слову, которое заканчивается на 0, так и к слову, которое заканчивается на 1 (т.к ограничений для единиц у нас нет). Т.е:
\[
A_{n+1} = B_{n} + A_{n}
\]
А двоичное слово, которое заканчивается на 0, мы можем получить только прибавлением нуля к слову, которое заканчивается на 1 (иначе мы нарушим условие отсутствия двух нулей подряд).  Т.е:
\[
B_{n+1}= A_{n}
\]
Тогда общее число двоичных слов длины n будет равно:
\[
A_n  + B_n  = B_{n-1} + A_{n-1} + A_{n-1}
\]
Тогда воспользуемся методом математической индукции:
\begin{itemize}
\item База: n = 2 

Двоичные последовательности длины 2: 01, 10, 11

\[
F_{n+1} = F_3 = 3
\]
\[
3 = 3
\]
\begin{center}
Верно
\end{center}
\item База: n = 3

Двоичные последовательности длины 3: 111, 110, 101, 010, 011, 

\[
F_{n+1} = F_4 = 5
\]
\[
5 = 5
\]
\begin{center}
Верно
\end{center}
\item  Переход: пусть верно для $n$ и $n-1$:\\ ($ A_{n} + B_{n} =  F_{n+1} $, \;$A_{n-1} + B_{n-1} = F_n$), докажем, что это верно и для $n+1$. Т.е:
\[
A_{n+1} + B_{n+1} = F_{(n+1) +1}
\]
\[
A_{n+1} + B_{n+1} = F_{n+1} + F_{n}
\]
\[
A_{n+1} + B_{n+1} = A_{n} + B_{n} + A_{n-1} + B_{n-1}
\]
\[
A{n} + B_{n} + A_{n} = A_{n} + B_{n} + A_{n-1} + B_{n-1}
\]
\[
A_{n} + B_{n} + A_{n-1} + B_{n-1} = A_{n} + B_{n} + A_{n-1} + B_{n-1}
\]
\[
0 = 0
\]
\begin{center}
\textbf{Ч.Т.Д}
\end{center}
\end{itemize}
\section*{7}
Каких чисел больше среди первых 10 миллионов целых неотрицательных чисел: тех, в десятичной записи которых есть цифра 1, или тех, в десятичной записи которых этой цифры нет?
\\\\
Посчитаем кол-во чисел, в которых нет единицы:


На каждую из позиций можно поставить любую цифру от 0 до 9, кроме единицы (т.е 9 вариантов). Если в начале стоит 0 - значит у нас просто НЕ семизначное число. Но нужно вычесть случай, когда на всех позициях стоят  нули (т.к 0000000 не является числом) По правилу произведения:
\[
9 * 9 * 9 * 9 * 9 * 9 * 9 - 1= 9^7 - 1 = 4782969 - 1 = 4782968
\]
Тогда кол-во чисел, в которых единица есть:
\[
10000000 - 4782968 = 5217032
\]
\[
5217032 > 4782968
\]
Значит чисел, в которых единица есть, больше, чем чисел, в которых её нет\\\\
\textbf{Ответ:} больше тех чисел, в десятичной записи которых есть цифра 1
\section*{8}
Сколькими способами можно закрасить клетки таблицы $3 \times 4$
так, чтобы незакрашенные клетки
содержали или верхний ряд, или нижний ряд, или две средних вертикали?
\\\\
Посчитаем отдельно каждый из случаев, а потом найдем их объединение
\\\\
Все возможные случаи, когда верхний ряд - незакрашен (A):

Всего у нас $3 \times 4 = 12 $ клеток.  4 из них (верхний ряд) мы не закрашиваем. Остается $12 - 4 = 8$ клеток. Мы имеем 8 позиций и на каждую из позиций есть 2 случая (раскрасить или нет). Тогда по правилу произведения: \\
\begin{center}
\begin{tabular}{|c|c|c|c|}
\hline
1 &1  &1  & 1\\
\hline
 2& 2 &2  & 2 \\
\hline
 2&2  & 2  & 2 \\
\hline
\end{tabular}
\end{center}
\[
|A| = 2 ^ 8 = 256
\]
\\\\
Все возможные случаи, когда нижний ряд - незакрашен (B):

Количество аналогично случаю для верхнего ряда.  У нас 4 клетки в нижнем ряду незакрашены, а оставшиеся 8 мы можем как закрасить, так и нет:
\begin{center}
\begin{tabular}{|c|c|c|c|}
\hline
1 &1  &1  & 1\\
\hline
 2& 2 &2  & 2 \\
\hline
 2&2  & 2  & 2 \\
\hline
\end{tabular}
\end{center}
\[
|B| = 2^8 = 256
\]
\\\\
Все возможные случаи, когда незакрашены вертикали (C):

2 средних вертикали занимают  $3 \times 2 = 6$ клеток. У нас остается $12-6 = 6 $ свободных клеток, которые мы можем как закрасить, так и нет:


\begin{center}
\begin{tabular}{|c|c|c|c|}
\hline
2&1&1 & 2\\
\hline
 2& 1&1& 2\\
\hline
 2&1 &1& 2 \\
\hline
\end{tabular}
\end{center}
\[
|C| = 2^6 = 64
\]

Мы знаем формулу включений-исключений, она гласит:
\[
|A \cup B \cup C| = |A| \cup |B| \cup |C| - |A \cap B| - |A \cap C| - |C \cap B| + |A \cap B \cap C|
\]
\[
|A \cap B| = 2^4 \text{ (верхний+нижний незакрашены, остается 4 клетки)}
\]
\[
|A \cap C| = 2^4 \text{ (верхний ряд + вертикали незакрашены, остается 4 клетки)}
\]
\[
|C \cap B| = 2^4 \text{ (нижний ряд + вертикали незакрашены, остается 4 клетки)}
\]
\[
|A \cap B \cap C| = 2^2 \text{ (верхний, нижний ряд + вертикали незакрашены, остается 2 клетки)}
\]
Подставляем значения в формулу:
\[
|A \cup B \cup C| = 256 + 256 + 64 - 16 - 16 - 16 + 4 = 580 - 48 = 532 \text{ способа} 
\]
\textbf{Ответ:} 532 
\section*{9}
 
Пускай:

a - повар

b - медик

c - пилот

d - астроном 
\\\\
По условию:
\[
|a| = |b| = |c| = |d| = 6
\]
\[
|a \cap b \cap c \cap d|  = 1
\]
\[
|a \cap b| = |a \cap c| = |a \cap d| = |b \cap c| = |b \cap d| = |c \cap d| = 4
\]
\[
|a \cap b \cap c| = |a \cap b \cap d| = |b \cap c \cap d| = 2
\]
Тогда по формуле включений - исключений для 4-х множеств:
\[
|a \cup b \cup c \cup d| = 6 + 6 + 6 + 6 - 4 * 6 + 2 * 4 - 1 = 8 - 1 = 7 
\]
Значит, всего у нас 7 людей.

Можно заметить, что поваров, медиков и пилотов одновременно у нас по формуле включений - исключений для двух множеств:
\[
|a \cup b \cup c| = 6 + 6 + 6 -4 - 4 - 4 + 2 = 8 \text{ человек}
\]
Но у нас всего 7 человек.
Мы видим противоречие, значит тех.задание невыполнимо\\\\
\textbf{Ответ:} нет
\section*{10}
Подслово 011 в двоичных словах длины 9
\\\\
Т.к найти напрямую слова, содержащие последовательности 011 очень сложно (как мне показалось), то намного проще из общего числа слов вычесть те слова, где 011 не встречается
\\

Двоичных слов длины 9 у нас:
\[
2^9 = 512 
\]

Найдем слова, в которых нет последовательности 011:

Пусть для n таких слов $F(n)$, тогда:
\begin{itemize}
\item Если на n-й позиции стоит 1, то мы можем на (n-1) позицию поставить либо 1, либо 0. 

Если ставим 1, тогда больше нулей у нас быть не может и это просто n единичек ($111\ldots 1_n$)

Если ставим 0, то получаем $F_{n-2}$ слов (длина слова как-бы уменьшается и мы рассматриваем количество для n - 2)
\item Если на n-й позиции стоит 0, то мы можем на (n-1) позицию поставить что угодно, при этом мы не ограничены в выборе и получаем рекурренто $F_{n-1}$ слов

Таким образом, для длины n мы имеем $F_n = 1 + F_{n-2} + F_{n-1}$
\\\\
Но это зависимость не работает для случая, когда длина меньше трех, т.к $ n > 0$ и двоичных слов c 011 длины меньше трех вообще нет. 
\\\\
Если n = 1:

Слов $2^1 = 2$
\\\\
Если n = 2:

Слов  $ 2^2 = 4$
\\\\
Т.е:
\[
F_n = 1 + F_{n-2} + F_{n-1}, \; n \geq 3
\]
(P.S Я не очень хорошо чувствую и понимаю, нужно ли здесь доказывать верность формулы по мат.индукции, но скорее склоняюсь к тому, что не нужно, поэтому был бы рад услышать уточнения)\\\\
Посчитаем для n = 9:
\[
F_3 = 1 + 2 + 4 = 7
\]
\[
F_4 = 1 + 4 + 7 = 12
\]
\[
F_5 = 1 + 7 + 12 = 20
\]
\[
F_6 = 1 + 12 + 20 = 33
\]
\[
F_7 = 1 + 20 + 33 = 54
\]
\[
F_8 = 1 + 33 + 54 = 88
\]
\[
F_9 = 1 + 54 + 88 = 143
\]
Тогда получаем:
\[
512 - 143 = 369 \text{ слов}
\]
\textbf{Ответ:} 369
\end{itemize}
\end{document}