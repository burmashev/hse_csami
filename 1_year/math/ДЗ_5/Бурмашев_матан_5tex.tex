\documentclass[a4paper,12pt]{article}

%%% Работа с русским языком
\usepackage{cmap}					% поиск в PDF
\usepackage{mathtext} 				% русские буквы в формулах
\usepackage[T2A]{fontenc}			% кодировка
\usepackage[utf8]{inputenc}			% кодировка исходного текста
\usepackage[english,russian]{babel}	% локализация и переносы
\usepackage{xcolor}
\usepackage{hyperref}
 % Цвета для гиперссылок
\definecolor{linkcolor}{HTML}{799B03} % цвет ссылок
\definecolor{urlcolor}{HTML}{799B03} % цвет гиперссылок

\hypersetup{pdfstartview=FitH,  linkcolor=linkcolor,urlcolor=urlcolor, colorlinks=true}

%%% Дополнительная работа с математикой
\usepackage{amsfonts,amssymb,amsthm,mathtools} % AMS
\usepackage{amsmath}
\usepackage{icomma} % "Умная" запятая: $0,2$ --- число, $0, 2$ --- перечисление

%% Номера формул
%\mathtoolsset{showonlyrefs=true} % Показывать номера только у тех формул, на которые есть \eqref{} в тексте.

%% Шрифты
\usepackage{euscript}	 % Шрифт Евклид
\usepackage{mathrsfs} % Красивый матшрифт

%% Свои команды
\DeclareMathOperator{\sgn}{\mathop{sgn}}
\DeclareMathOperator*\lowlim{\underline{lim}}
\DeclareMathOperator*\uplim{\overline{lim}}

\newcommand*{\hm}[1]{#1\nobreak\discretionary{}
{\hbox{$\mathsurround=0pt #1$}}{}}
% графика
\usepackage{graphicx}
\graphicspath{{pictures/}}
\DeclareGraphicsExtensions{.pdf,.png,.jpg}
\author{Бурмашев Григорий}
\title{Матан, коллок - 1 }
\date{\today}
\begin{document}
\begin{center}
Бурмашев Григорий.  208. Матан. Д/з - 5
\end{center}
\begin{center}
\includegraphics[scale=0.4]{jz.jpg}
\end{center}
\newpage
\section*{Задача № 12}
Найти:
\[
\lowlim_{n \rightarrow \infty} a_n
\]
\[
\uplim_{n \rightarrow \infty} a_n
\]
\subsection*{а)}
\[
a_n = \frac{(-1)^n}{n} + \frac{1+(-1)^n}{2}
\]
Пускай n = 2k (т.е четное), тогда:
\[
a_n = \frac{(-1)^{2k}}{2k} + \frac{1+(-1)^{2k}}{2} = \frac{1}{2k} + 1
\]
\[
\lim_{k \rightarrow \infty} (1 + \frac{1}{2k}) = 1
\]
А значит:
\[
\uplim_{n \rightarrow \infty} a_n = 1
\]
Пускай n = 2k+1 (т.е нечетное), тогда:
\[
a_n = \frac{(-1)^{2k+1}}{2k+1} + \frac{1+(-1)^{2k+1}}{2} = -\frac{1}{2k+1} + \frac{0}{2} = -\frac{1}{2k+1}
\]
\[
\lim_{k \rightarrow \infty}a_n (-\frac{1}{2k+1}) = 0
\]
А значит:
\[
\lowlim_{n \rightarrow \infty} a_n = 0
\]
\begin{center}
\textbf{Ответ:}
\end{center}
\[
\uplim_{n \rightarrow \infty} a_n = 1
\]
\[
\lowlim_{n \rightarrow \infty} a_n = 0
\]
\subsection*{б) }
\[
a_n = \frac{n}{n+1} \sin^2{\frac{n\pi}{4}}
\]
Мы знаем, что:
\[
\sin^2{\frac{n\pi}{4}} \in [0, 1]
\]
Рассмотрим случай, когда $\sin^2{\frac{n\pi}{4}}  = 0$:
\[
\frac{n\pi}{4} = \pi k \rightarrow n = 4k
\]
Тогда:
\[
a_n = \frac{4k}{4k+1} \cdot 0  = 0 
\]
\[
\lim_{k \rightarrow \infty} \frac{4k}{4k+1} \cdot 0  = 0
\]
А значит:
\[
\lowlim_{n \rightarrow \infty} a_n = 0
\]
Теперь рассмотрим случай, когда $\sin^2{\frac{n\pi}{4}}  = 1$:
\[
\frac{n\pi}{4} = \frac{\pi k}{2} + \pi k \rightarrow n = 4k + 2
\]
Тогда:
\[
a_n = \frac{4k+2}{4k+3} \cdot 1 
\]
\[
\lim_{k \rightarrow \infty} \frac{4k+2}{4k+3} \cdot 1  = 1
\]
А значит:
\[
\uplim_{n \rightarrow \infty} a_n = 1
\]
\begin{center}
\textbf{Ответ:}
\end{center}
\[
\uplim_{n \rightarrow \infty} a_n = 1
\]
\[
\lowlim_{n \rightarrow \infty} a_n = 0
\]
\subsection*{c) }
\[
a_n = 1 + 2 \cdot (-1)^{n+1} + 3 \cdot (-1)^{\frac{n(n-1)}{2}}
\]
Рассмотрим случай, когда
$\frac{n(n-1)}{2}$ четно, тогда n = 4k:
\[
1 + 2 \cdot (-1)^{4k+1} + 3 \cdot (-1)^{\frac{4k(4k-1)}{2}}
\]
\[
1 + 2 \cdot (-1) + 3 \cdot 1 = 2 
\]
Рассмотрим оставшиеся n, пусть n = 4k + 1:
\[
1 + 2 \cdot (-1)^{4k+2} + 3 \cdot (-1)^{\frac{(4k+1)(4k)}{2}}
\]
\[
1 + 2 \cdot 1 + 3 \cdot 1 = 6
\]
Пусть n = 4k + 2:
\[
1 + 2 \cdot (-1)^{4k+3} + 3 \cdot (-1)^{\frac{(4k+2)(4k+1)}{2}}
\]
\[
1 + 2 \cdot (-1) + 3 \cdot (-1) = -4 
\]
Пусть n = 4k + 3:
\[
1 + 2 \cdot (-1)^{4k+4} + 3 \cdot (-1)^{\frac{(4k+3)(4k+2)}{2}}
\]
\[
1 + 2 \cdot 1 + 3 \cdot (-1) = 0
\]
Мы рассмотрели все возможные значения n.
\begin{center}
\textbf{Ответ:}
\end{center}
\[
\uplim_{n \rightarrow \infty} a_n = 6
\]
\[
\lowlim_{n \rightarrow \infty} a_n = -4
\]
\end{document}