\documentclass[a4paper,12pt]{article}

%%% Работа с русским языком
\usepackage{cmap}					% поиск в PDF
\usepackage{mathtext} 				% русские буквы в формулах
\usepackage[T2A]{fontenc}			% кодировка
\usepackage[utf8]{inputenc}			% кодировка исходного текста
\usepackage[english,russian]{babel}	% локализация и переносы
\usepackage{xcolor}
\usepackage{hyperref}
 % Цвета для гиперссылок
\definecolor{linkcolor}{HTML}{799B03} % цвет ссылок
\definecolor{urlcolor}{HTML}{799B03} % цвет гиперссылок

\hypersetup{pdfstartview=FitH,  linkcolor=linkcolor,urlcolor=urlcolor, colorlinks=true}

%%% Дополнительная работа с математикой
\usepackage{amsfonts,amssymb,amsthm,mathtools} % AMS
\usepackage{amsmath}
\usepackage{icomma} % "Умная" запятая: $0,2$ --- число, $0, 2$ --- перечисление

%% Номера формул
%\mathtoolsset{showonlyrefs=true} % Показывать номера только у тех формул, на которые есть \eqref{} в тексте.

%% Шрифты
\usepackage{euscript}	 % Шрифт Евклид
\usepackage{mathrsfs} % Красивый матшрифт

%% Свои команды
\DeclareMathOperator{\sgn}{\mathop{sgn}}

%% Перенос знаков в формулах (по Львовскому)
\newcommand*{\hm}[1]{#1\nobreak\discretionary{}
{\hbox{$\mathsurround=0pt #1$}}{}}
% графика
\usepackage{graphicx}
\graphicspath{{pictures/}}
\DeclareGraphicsExtensions{.pdf,.png,.jpg}
\author{Бурмашев Григорий, БПМИ-208}
\title{Матан, дз -- какое-то там}
\date{\today}
\begin{document}
\maketitle
\section*{Номер 1}
\[
\lim\limits_{n \rightarrow \infty} \frac{z^n}{1 + z^{2n}} = \frac{0}{1 + 0} = 0
\]
\begin{center}
\textbf{Ответ: } 
\[
0
\]
\end{center}
\subsection*{b)}
\[
\lim\limits_{z \rightarrow 0 }  \frac{z \cdot Re(z) }{|z|}
\]
\[
\frac{Re(z) }{|z|} \in [-1;1] \rightarrow \lim\limits_{z \rightarrow 0 }  \frac{z \cdot Re(z) }{|z|} = \lim\limits_{z \rightarrow 0 }  z = 0
\]
\begin{center}
\textbf{Ответ: } 0
\end{center}
\clearpage
\section*{Номер 2}
\[
\sin z = \frac{e^{iz} - e^{-iz}}{2i} = \frac{e^{ix}e^{-y} - e^{-ix} e^y}{2i} = \frac{-e^{-y} \cos x \cdot  i + e^{-y} \sin x + e^y \cos x \cdot i + e^y \sin x}{2}  =
\]
\[
=
\frac{e^{-y} sin x + e^{y} \sin x  + (-e^{-y} \cos x + e^y \cos x)i }{2}
\]
Тогда кладем:
\[
\begin{cases}
u = \frac{e^{-y} sin x + e^{y} \sin x }{2}  \\
v = \frac{(-e^{-y} \cos x + e^y \cos x)}{2}
\end{cases}
\]
Отсюда:
\[
\begin{cases}
(u)'_x = \frac{e^{-y} \cos x + e^y \cos x}{2} \\
(u)'_u = \frac{-e^{-y} \sin x + e^y \sin x}{2}
\end{cases} \sim \begin{cases}
(u)'_x = v'_y \\
(u)'_y = (-v)'_x
\end{cases}
\]
Условие Коши Римана выполнено, значит функция $\mathbb{C}$-- дифференцируема и производная равна $w' = \cos z$
\clearpage
\section*{Номер 3}
Заметим, что:
\[
u = \sqrt{|xy} \\
v = 0
\]
Смотрим производные в $z = 0$:
\[
u'_x(0, 0) = \lim_{x \rightarrow 0} \frac{u(x, 0) - u(0, 0)}{x - 0} = 0 = v'_y (0, 0)
\]
\[
u'_y(0, 0) = \lim_{y \rightarrow 0} \frac{u(0, y) - u(0, 0)}{y - 0} = 0 = -v'_x(0, 0)
\]
Тобишь:
\[
\begin{cases}
u'_x = v'_y \\
u'_y = -v'_x 
\end{cases}
\]
Доказали, что выполнено условия Коши-Римана.
\begin{center}
\textbf{Ч.Т.Д} 
\end{center}
 Теперь покажем, что $f'(0)$ не существует:
Смотрим:
\[ 
f'(0) = \lim\limits_{z \rightarrow 0 } \frac{f(z) - f(0)}{z - 0 } = 
\lim_{\begin{matrix}
x \rightarrow 0 \\
y \rightarrow 0
\end{matrix}}  \frac{\sqrt{|xy|} - 0}{x + iy} = 
\]
Переходим к полярным координатам:
\[
=
\lim_{r \rightarrow 0} \frac{r \sqrt{|\sin \varphi \cos \varphi |}}{r(\cos \varphi + i \sin\varphi )} = 
\lim_{r \rightarrow 0} \frac{\sqrt{|\sin \varphi \cos \varphi |}}{(\cos\varphi + i \sin\varphi )}
\] 
Видим, что предел от $r$ не зависит, значит предела нет, значит производной в точке $f'(0)$ не существует 
\begin{center}
\textbf{Ч.Т.Д} 
\end{center}
\clearpage
\section*{Номер 4}
\[
u = Re(f) = 3x^2 - 4xy - 3y^2
\]
Используем Коши-Римана, получаем систему:
\[
\begin{cases}
u'_x = 6x - 4y = v'_y \\
u'_y = -4x -6y = -v'_x
\end{cases}
\]
Так что интегрируем:
\[
v = \int (6x -4y) dy = 6xy - 2y^2 + \varphi(x)
\]
\[
v'_x = 4x + 6y = 6y + \varphi'(x)
\]
Получили:
\[
\varphi'(x) = 4x 
\]
\[
\varphi(x) = \int (4x) dx = 2x^2 + C 
\]
Т.е итого:
\[
v = 6xy - 2y^2 + 2x^2 + C
\]
А теперь константу находим из условия, что$f(1) = 3 + 2i$:
\[
f(1) = u(1) + iv(1) = 3 + 2i  \rightarrow C = 0
\]
Так что:
\[
f = 3x^2 - 4xy - 3y^2+ i(6xy - 2y^2 + 2x^2)
\] 
\begin{center}
\textbf{Ответ: } 
\[
f = 3x^2 - 4xy - 3y^2+ i(6xy - 2y^2 + 2x^2)
\]
\end{center}
\clearpage
\section*{Номер 5}
Делаем аналогично предыдущему номеру:
\[
v = 
\text{Im} (f) = \ln(x^2 +y^2) + x - 2y
\]
По Коши-Риману:
\[
\begin{cases}
v'_y = \frac{2y}{x^2 + y^2} - 2 = u'_x \\
-v'_x = - \frac{2x}{x^2 + y^2}  -1 = u'_y 
\end{cases}
\]
Так что интегрируем:
\[
u = \int \left(
\frac{2y}{x^2 + y^2} - 2 
\right) dx  = 2 \arctg \frac{x}{y} - 2x + \varphi(y) 
\]
\[
u'_y = - \frac{2x}{x^2 + y^2}  -1 = -\frac{2x}{x^2 + y^2} + \varphi'(y) 
\]
Получили:
\[
\varphi'(y)  = -1
\]
\[
\varphi(y) = -y + C
\]
Т.е итого:
\[
u = 2\arctg \frac{x}{y} - 2x - y + C
\]
Находим константу:
\[
f(i) = 2 \arctg 0 - 1 + C = i( \ln(1) -2)  = -2i \rightarrow C = 1
\]
Так что:
\[
f = 2 \arctg \frac{x}{y} - 2x - y + 1 + i (\ln(x^2 + y^2) + x - 2y) 
\]
\begin{center}
\textbf{Ответ: } 
\[
f = 2 \arctg \frac{x}{y} - 2x - y + 1 + i (\ln(x^2 + y^2) + x - 2y) 
\]
\end{center}
\end{document}
