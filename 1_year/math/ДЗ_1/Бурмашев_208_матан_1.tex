\documentclass[a4p aper,12pt]{article}

%%% Работа с русским языком
\usepackage{cmap}					% поиск в PDF
\usepackage{mathtext} 				% русские буквы в формулах
\usepackage[T2A]{fontenc}			% кодировка
\usepackage[utf8]{inputenc}			% кодировка исходного текста
\usepackage[english,russian]{babel}	% локализация и переносы

%%% Дополнительная работа с математикой
\usepackage{amsfonts,amssymb,amsthm,mathtools} % AMS
\usepackage{amsmath}
\usepackage{icomma} % "Умная" запятая: $0,2$ --- число, $0, 2$ --- перечисление

%% Номера формул
%\mathtoolsset{showonlyrefs=true} % Показывать номера только у тех формул, на которые есть \eqref{} в тексте.

%% Шрифты
\usepackage{euscript}	 % Шрифт Евклид
\usepackage{mathrsfs} % Красивый матшрифт

%% Свои команды
\DeclareMathOperator{\sgn}{\mathop{sgn}}

%% Перенос знаков в формулах (по Львовскому)
\newcommand*{\hm}[1]{#1\nobreak\discretionary{}
{\hbox{$\mathsurround=0pt #1$}}{}}

\author{Бурмашев Григорий}
\title{Матанализ  -1}
\date{\today}
\begin{document}
\begin{center}
Бурмашев Григорий, 208. Матан. Д/з - 1
\end{center}
\section{Задача 8}
Вычислите сумму:
\subsection*{a)}
\[S = \frac{1}{1\times5} + \frac{1}{5\times9} + \ldots + \frac{1}{(4n-3)(4n+1)}\]
Можно заметить, что:\\
\begin{eqnarray*}
\begin{gathered}
\left(\frac{1}{4n-3} - \frac{1}{4n+1}\right)\times \frac{1}{4} = \frac{1}{4(4n-3)} - \frac{1}{4(4n+1)} = \frac{4n+1 - (4n-3)}{4(4n-3)(4n+1)} =\\ = \frac{4}{4(4n-3)(4n+1)} = \frac{1}{(4n-3)(4n+1)}
\end{gathered}
\end{eqnarray*}
Представим всю нашу сумму в виде таких разностей, умноженных на $\frac{1}{4}$
Тогда:
\begin{equation*}
\begin{gathered}
(\frac{1}{1} - \frac{1}{5}) \times \frac{1}{4} + (\frac{1}{5} - \frac{1}{9}) \times \frac{1}{4} + \ldots + (\frac{1}{4n-3} - \frac{1}{4n+1}) \times \frac{1}{4}
\end{gathered}
\end{equation*}
Вынесем общий множитель $\frac{1}{4}$ за скобки:
\begin{equation*}
\begin{gathered}
\frac{1}{4} \times \left(\frac{1}{1}- \frac{1}{5} + \frac{1}{5} - \frac{1}{9} + \frac{1}{9}  +\ldots- \frac{1}{4n-3} +\frac{1}{4n-3} - \frac{1}{4n+1}\right)
\end{gathered}
\end{equation*}
Все множители внутри скобок, кроме первого и последнего, сокращаются:
\begin{equation*}
\begin{gathered}
\frac{1}{4} \times \left(\frac{1}{1} -\frac{1}{4n+1}\right)\\\\
\frac{1}{4} -\frac{1}{4(4n+1)}\\\\
\frac{4n+1-1}{4(4n+1)}\\\\
\frac{n}{4n+1}
\end{gathered}
\end{equation*}
{\textbf{Ответ:} $ \frac{n}{4n+1}$
\subsection*{б)}
\[S =  \frac{1}{2} + \frac{3}{2^2} + \ldots + \frac{2n-1}{2^n}\]\\

Представим сумму в виде $ S = 2S - S$, чтобы избавиться от "$\ldots$":
\begin{equation*}
\begin{gathered}
S =  2\times(\frac{1}{2} + \frac{3}{2^2} + \ldots + \frac{2n-1}{2^n})\; -  \\
-\; (\frac{1}{2} + \frac{3}{2^2} + \ldots + \frac{2n-1}{2^n})\\\\
S = \frac{2}{2} + \frac{3}{2} + \frac{5}{2^2} + \ldots   + \frac{2n-1}{2^{n-1}} \;- \\
-\;\frac{1}{2} - \frac{3}{2^2} - \frac{5}{2^3} - \ldots - \frac{2n-1}{2^n}\\\\
\text{Вычтем попарно множители (кроме первого и последнего) и получим:}\\
S = 1 + 1 + \frac{1}{2} + \frac{1}{2^2} + \frac{1}{2^3} + \ldots + \frac{1}{2^{n-1}} - \frac{2n-1}{2^n}\\
\text{Применим формулу суммы геометрической последовательности для ряда } \frac{1}{2} + \frac{1}{2^2} + \ldots + \\ +\frac{1}{2^{n-1}}:\\\\
S = 2 + 1 - \frac{1}{2^{n-1}} - \frac{2n-1}{2^n}\\\\
S = 3 - \frac{2}{2^n} - \frac{2n-1}{2^n}\\
S = 3 - \frac{2n-3}{2^n}\\\\
S = \frac{3\times2^n -2n -3}{2^n}
\end{gathered}
\end{equation*}
{\large \textbf{Ответ:} $\frac{3\times2^n -2n -3}{2^n}$}

\section{Задача 9}
Применяя метод математической индукции, докажите, что:
\subsection*{а)}
\[ 1^2+2^2+ 3^2 + \ldots + n^2 = \frac{n(n+1)(2n+1)}{6}
\]
\textbf{База:}

Пусть $n = 1$, тогда:
\begin{equation*}
\begin{gathered}
1^2 = \frac{1(1+1)(2\times1+1)}{6} \\\\
1 = \frac{6}{6}\\\\
1 = 1
\end{gathered}
\end{equation*}
 \begin{center}
\textbf{Верно}
\end{center}
\textbf{Переход:}

Докажем, что это верно для $n+1$:
\begin{equation*}
\begin{gathered}
1^2+2^2+3^2+\ldots+n^2+(n+1)^2 = \frac{(n+1)(n+1+1)(2(n+1)+1)}{6}\\\\
\frac{n(n+1)(2n+1)}{6}+(n+1)^2 =  \frac{(n+1)(n+1+1)(2(n+1)+1)}{6}\\\\
\frac{n(n+1)(2n+1)+6(n+1)^2}{6}  =  \frac{(n+1)(n+1+1)(2(n+1)+1)}{6}\\\\
(n^2+n)(2n+1)+6n^2+12n+6 = (n+1)(n+2)(2n+3)\\\\
2n^3+n^2+2n^2+n+6n^2+12n +6 = (n^2+3n+2)(2n+3)\\\\
2n^3+9n^2+13n+6=2n^3+6n^2+4n+3n^2+9n+6\\\\
2n^3+9n^2+13n+6=2n^3+9n^2+13n+6
\end{gathered}
\end{equation*}
\begin{center}
\textbf{Ч.Т.Д}
\end{center}
\subsection*{б)}
\[
\sqrt{n}< \frac{1}{\sqrt1}+\frac{1}{\sqrt{2}} + \ldots + \frac{1}{\sqrt{n}} < 2\sqrt{n}, n\geq2
\]
\textbf{База:}

Пусть n = 2, тогда:
\begin{equation*}
\begin{gathered}
\sqrt{2} < 1 + \frac{1}{\sqrt{2}} < 2\sqrt{2} \;\;\;\; | \times \sqrt{2}\\\\
2 < \sqrt{2} + 1 < 4\\\\
1 < \sqrt{2} < 3\\\\
1 < 2 < 9
\end{gathered}
\end{equation*}
\begin{center}
\textbf{Верно}
\end{center}
\textbf{Переход:}

Докажем, что это верно для $n+1$:
\begin{equation*}
\begin{gathered}
\sqrt{n+1} < 1 + \frac{1}{\sqrt{2}} +\ldots + \frac{1}{\sqrt{n}} + \frac{1}{\sqrt{n+1}} < 2\sqrt{n+1}\\\\
\text{Сначала докажем нижнюю границу:}\\
\sqrt{n+1} < 1 + \frac{1}{\sqrt{2}} +\ldots + \frac{1}{\sqrt{n}} + \frac{1}{\sqrt{n+1}}\\\\
\text{Т.к мы считаем, что неравенство выполняется для n, то прибавим к нему} \frac{1}{\sqrt{n+1}}\\\\
\sqrt{n} + \frac{1}{\sqrt{n+1}}< \frac{1}{\sqrt1}+\frac{1}{\sqrt{2}} + \ldots + \frac{1}{\sqrt{n}} + \frac{1}{\sqrt{n+1}}\\\\
\text{Нужно доказать, что:}\\\\
\sqrt{n+1} < \sqrt{n} + \frac{1}{\sqrt{n+1}}\\
\sqrt{n+1} < \frac{\sqrt{n}\times\sqrt{n+1} +1}{\sqrt{n+1}}\\\\
n+1 < \sqrt{n} \times \sqrt{n+1} + 1\\
n < \sqrt{n} \times \sqrt{n+1}\\
n^2 < n(n+1)\\
n^2 < n^2 + n\\
0 < n\\
\text{Т.к } n \geq 2 \text{,то неравенство выполняется}\\
\text{\textbf{ч.т.д}}\\\\
\text{Теперь, аналогично (прибавляя к исходному неравенству) } \frac{1}{\sqrt{n+1}} \\\text{ докажем верхнюю границу:}\\\\
\end{gathered}
\end{equation*}
\begin{equation*}
\begin{gathered}
2\sqrt{n} + \frac{1}{\sqrt{n+1}} < 2\sqrt{n+1}\\
\frac{2\sqrt{n(n+1)} +1}{\sqrt{n+1}} < 2\sqrt{n+1}\\\\
2\sqrt{n(n+1)} +1 <2n+2\\
2\sqrt{n^2+n} < 2n + 1\\
\sqrt{n^2+n} < n + 0.5\\
n^2 + n < n^2  + n +0.25\\
0 < 0.25\\
\text{\textbf{ч.т.д}}
\end{gathered}
\end{equation*}
Мы доказали как верхнюю, так и нижнюю границу, значит все неравенство верно.\\
\begin{center}
\textbf{\textbf{Ч.Т.Д}}
\end{center}
\end{document}