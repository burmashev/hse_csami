\documentclass[a4paper,12pt]{article}

%%% Работа с русским языком
\usepackage{cmap}					% поиск в PDF
\usepackage{mathtext} 				% русские буквы в формулах
\usepackage[T2A]{fontenc}			% кодировка
\usepackage[utf8]{inputenc}			% кодировка исходного текста
\usepackage[english,russian]{babel}	% локализация и переносы
\usepackage{xcolor}
\usepackage{hyperref}
 % Цвета для гиперссылок
\definecolor{linkcolor}{HTML}{799B03} % цвет ссылок
\definecolor{urlcolor}{HTML}{799B03} % цвет гиперссылок

\hypersetup{pdfstartview=FitH,  linkcolor=linkcolor,urlcolor=urlcolor, colorlinks=true}

%%% Дополнительная работа с математикой
\usepackage{amsfonts,amssymb,amsthm,mathtools} % AMS
\usepackage{amsmath}
\usepackage{icomma} % "Умная" запятая: $0,2$ --- число, $0, 2$ --- перечисление

%% Номера формул
%\mathtoolsset{showonlyrefs=true} % Показывать номера только у тех формул, на которые есть \eqref{} в тексте.

%% Шрифты
\usepackage{euscript}	 % Шрифт Евклид
\usepackage{mathrsfs} % Красивый матшрифт

%% Свои команды
\DeclareMathOperator{\sgn}{\mathop{sgn}}

%% Перенос знаков в формулах (по Львовскому)
\newcommand*{\hm}[1]{#1\nobreak\discretionary{}
{\hbox{$\mathsurround=0pt #1$}}{}}
% графика
\usepackage{graphicx}
\graphicspath{{pictures/}}
\DeclareGraphicsExtensions{.pdf,.png,.jpg}
\author{Бурмашев Григорий, БПМИ-208}
\title{Матан, дз -- 5}
\date{\today}
\begin{document}
\maketitle 
\clearpage
\section*{Номер 1}
\[
\sum_{n = 1}^{\infty} \frac{1}{x^2 + nx + n^2}, D = (0, \infty)
\]
Воспользуемся признаком Вейерштрасса:
\begin{flushleft}
\includegraphics[scale=0.3]{1.png}
\end{flushleft}
Заметим, что $D = (0, \infty)$, $n$ идет от 1 до $\infty$, отсюда делаем вывод, что $x^2 + nx + n^2$ в знаменателе будет положительным, а значит:
\[
\frac{1}{x^2 + nx + n^2} < \frac{1}{n^2}
\]
Но поскольку все члены нашего ряда положительные $(x > 0, n >= 1)$, то выполняется:
\[
\Big| \frac{1}{x^2 + nx + n^2} \Big| < \frac{1}{n^2} \; \forall x \in D = (0, \infty)
\]
При этом заметим, что ряд $\sum \frac{1}{n^2}$ сходится (очевидно), а значит по признаку Вейерштрасса наш исходный ряд сходится абсолютно и равномерно 
\begin{center}
\textbf{Ч.Т.Д} 
\end{center}
\clearpage
\section*{Номер 2}
\[
\sum_{n = 1}^{\infty} \frac{x^n}{1 + x^{2n}}, \; D = (-1, 1)
\]
Предположим, что ряд сходится равномерно на $(-1, 1)$, тогда он должен сходиться равномерно и на $[-1, 1]$ (аналогично подобной задаче с семинара), \textbf{но} в граничной точке $x = 1$ мы получаем:
\[
\sum_{n = 1}^{\infty} \frac{1^n}{1 + 1^n} =\sum_{n = 1}^{\infty} \frac{1}{2}
\]
Видим, что такой ряд будет расходится $\rightarrow$ исходный ряд сходится не равномерно
\begin{center}
\textbf{Ч.Т.Д}  
\end{center}

\end{document}
