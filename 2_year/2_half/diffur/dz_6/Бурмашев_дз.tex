\documentclass[a4paper,12pt]{article}

%%% Работа с русским языком
\usepackage{cmap}					% поиск в PDF
\usepackage{mathtext} 				% русские буквы в формулах
\usepackage[T2A]{fontenc}			% кодировка
\usepackage[utf8]{inputenc}			% кодировка исходного текста
\usepackage[english,russian]{babel}	% локализация и переносы
\usepackage{xcolor}
\usepackage{hyperref}
 % Цвета для гиперссылок
\definecolor{linkcolor}{HTML}{799B03} % цвет ссылок
\definecolor{urlcolor}{HTML}{799B03} % цвет гиперссылок

\hypersetup{pdfstartview=FitH,  linkcolor=linkcolor,urlcolor=urlcolor, colorlinks=true}

%%% Дополнительная работа с математикой
\usepackage{amsfonts,amssymb,amsthm,mathtools} % AMS
\usepackage{amsmath}
\usepackage{icomma} % "Умная" запятая: $0,2$ --- число, $0, 2$ --- перечисление

%% Номера формул
%\mathtoolsset{showonlyrefs=true} % Показывать номера только у тех формул, на которые есть \eqref{} в тексте.

%% Шрифты
\usepackage{euscript}	 % Шрифт Евклид
\usepackage{mathrsfs} % Красивый матшрифт

%% Свои команды
\DeclareMathOperator{\sgn}{\mathop{sgn}}

%% Перенос знаков в формулах (по Львовскому)
\newcommand*{\hm}[1]{#1\nobreak\discretionary{}
{\hbox{$\mathsurround=0pt #1$}}{}}
% графика
\usepackage{graphicx}
\graphicspath{{pictures/}}
\DeclareGraphicsExtensions{.pdf,.png,.jpg}
\author{Бурмашев Григорий, БПМИ-208}
\title{Диффуры, дз -- 6}
\date{\today}
\begin{document}
\maketitle
[Завал по дедлайнам, на диффуры особо не оставалось времени :(]
\clearpage
\section*{Номер 4}
\[
\begin{cases}
x' = xz \\
y' = x + yz \\
z' = -z^2
\end{cases}
\]
Решаем по аналогии с семинаром. Поделим 1 и 3 ($z > 0$, все супер):
\[
\frac{dx}{dz} = \frac{xz}{-z^2} = -\frac{x}{z}
\]
\[
dx \cdot z = -dz \cdot x
\]
\[
dx \cdot z + dz \cdot x = 0
\]
\[
d(xz) = 0
\]
Первый интеграл:
\[
xz = C_1
\]
На семе решали при $x > 0$, но тут такого ограничения к сожалению нет, поэтому потом отдельно посмотрим случай $x = 0$:
\[
x' = C_1 
\]
\[
x = C_1 t + C_2
\]
\[
z = \frac{C_1}{x} = \frac{C_1}{C_1t + C_2}
\]
Подставим $x$ и $z$ во второе уравнение:
\[
y' = C_1 t + C_2 + y \cdot \frac{C_1}{C_1t + C_2}
\]
Ну а это линейное уравнение. Решим его, начнем с однородного:
\[
\frac{dy}{dt} = y \cdot \frac{C_1}{C_1 t + C_2}
\]
\[
\frac{dy}{y} = \frac{C_1}{C_1 t + C_2} dt
\]
Тогда интеграл:
\[
\ln |y| = \ln |C_1 t + C_2| + D
\]
\[
y = D(C_1 t + C_2)
\]
Метод вариации постоянной:
\[
y' = D(t)' (C_1 t + C_2) + D(t) C_1
\]
В уравнении:
\[
D(t)' (C_1 t + C_2) + D(t) C_1 = C_1t + C_2 + \frac{C_1 D(t)(C_1 t + C_2)}{C_1t + C_2}
\]
Отсюда:
\[
D(t)' (C_1t + C_2) = C_1 t + C_2
\]
\[
D(t)' = 1
\]
\[
D(t) = t + C_3
\]
А решение:
\[
y = (t + C_3) ( C_1t + C_2)
\]
Терь посмотрим при $x = 0$, тогда система примет вид:
\[
\begin{cases}
x' = 0\\
y' = yz \\
z' = -z^2
\end{cases}
\]
Пусть $ y \neq 0$:
\[
\frac{dy}{dz} = \frac{dy}{dz} = -\frac{y}{z}
\]
\[
d(yz) = 0
\]
\[
yz = C_1
\]
\[
z = \frac{C_1}{y}
\]
\[
y' = y \frac{C_1}{y} 
\]
\[
y' = C_1
\]
Отсюда:
\[
y = C_1 t + C_2
\]
При $y = 0$:
\[
z = \frac{1}{C_1 + t}
\]
Вроде все.
\begin{center}
\textbf{Ответ: } 
\\
Если $x = 0, y = 0$:
\[
\begin{cases}
x = 0 \\
y = 0 \\
z = \frac{1}{C_1 + t}
\end{cases}
\]
Если $x = 0$:
\[
\begin{cases}
x = 0 \\
y = C_1t + C_2 \\
z = \frac{C_1}{C_1 t + C_2}
\end{cases}
\]
Иначе:
\[
\begin{cases}
x = C_1 t + C_2 \\
y = (t + C_3)(C_1t + C_2) \\
z = \frac{C_1}{C_1 t + C_2}
\end{cases}
\]
\end{center}
\end{document}
