\documentclass[a4paper,12pt]{article}

%%% Работа с русским языком
\usepackage{cmap}					% поиск в PDF
\usepackage{mathtext} 				% русские буквы в формулах
\usepackage[T2A]{fontenc}			% кодировка
\usepackage[utf8]{inputenc}			% кодировка исходного текста
\usepackage[english,russian]{babel}	% локализация и переносы
\usepackage{xcolor}
\usepackage{hyperref}
 % Цвета для гиперссылок
\definecolor{linkcolor}{HTML}{799B03} % цвет ссылок
\definecolor{urlcolor}{HTML}{799B03} % цвет гиперссылок

\hypersetup{pdfstartview=FitH,  linkcolor=linkcolor,urlcolor=urlcolor, colorlinks=true}

%%% Дополнительная работа с математикой
\usepackage{amsfonts,amssymb,amsthm,mathtools} % AMS
\usepackage{amsmath}
\usepackage{icomma} % "Умная" запятая: $0,2$ --- число, $0, 2$ --- перечисление

%% Номера формул
%\mathtoolsset{showonlyrefs=true} % Показывать номера только у тех формул, на которые есть \eqref{} в тексте.

%% Шрифты
\usepackage{euscript}	 % Шрифт Евклид
\usepackage{mathrsfs} % Красивый матшрифт

%% Свои команды
\DeclareMathOperator{\sgn}{\mathop{sgn}}

%% Перенос знаков в формулах (по Львовскому)
\newcommand*{\hm}[1]{#1\nobreak\discretionary{}
{\hbox{$\mathsurround=0pt #1$}}{}}
% графика
\usepackage{graphicx}
\graphicspath{{pictures/}}
\DeclareGraphicsExtensions{.pdf,.png,.jpg}
\author{Бурмашев Григорий, БПМИ-208}
\title{Матан, дз -- 4}
\date{\today}
\begin{document}
\maketitle
\section*{Номер 1}
\[
f(x) = 
\begin{cases}
e^{-ax},& x \geq 0 \\
0, &x < 0
\end{cases}, \text{ где } \; a > 0 
\]
Ищем прямое преобразование Фурье:
\[
\stackrel{\wedge}{f}(y) = \frac{1}{\sqrt{2\pi}} \int\limits_{-\infty}^{\infty} f(x) e^{-ixy} dx = \frac{1}{\sqrt{2\pi}} \int\limits_{0}^{\infty} e^{-ax} e^{-ixy} dx = \frac{1}{\sqrt{2\pi}} \int\limits_{0}^{\infty} e^{-ax - ixy} dx =
\]  
\[
=
\frac{1}{2\sqrt{\pi}} \int\limits_{0}^{\infty} e^{x(-a- iy)} dx  = \frac{1}{\sqrt{2\pi}} \cdot \frac{1}{-a-iy} \cdot \int_{0}^{+\infty} e^u du = \frac{1}{\sqrt{2\pi}} \cdot \frac{e^{-x(a + iy)}}{-a-iy}  \Bigg|_0^{\infty} = \frac{1}{\sqrt{2\pi}} \cdot \frac{1}{a + iy }
\] 
Теперь обратное:
\[
\stackrel{\vee}{f} (y)= \frac{1}{\sqrt{2\pi}}  \int\limits_{-\infty}^{+\infty} \stackrel{\wedge}{f}(x) e^{ixy} dy = \frac{1}{\sqrt{2\pi}}  \int\limits_{-\infty}^{+\infty}  \frac{1}{\sqrt{2\pi}} \cdot \frac{1}{a + iy } e^{ixy} dy = \frac{1}{\sqrt{2\pi}}  \int\limits_{-\infty}^{+\infty}  \frac{e^{ixy} }{(a + iy)} dy = 
\]
\[
 = \frac{1}{\sqrt{2\pi}}  \int\limits_{-\infty}^{+\infty}  \frac{1 }{(a + iy)}  \cdot \frac{a - iy}{a - iy}\cdot \left((\cos (xy) + i \sin (xy)\right)dy = \frac{1}{\sqrt{2\pi}}  \int\limits_{-\infty}^{+\infty}  \frac{a - iy}{a^2 + y^2}\cdot \left((\cos (xy) + i \sin (xy)\right)dy  = 
\]
\[
 = 
\frac{1}{\sqrt{2\pi}}  \int\limits_{-\infty}^{+\infty}  \frac{a \cos (xy) +  y \sin (xy) }{a^2 + y^2} dy + \frac{1}{\sqrt{2\pi}} \int\limits_{-\infty}^{+\infty}  \frac{i(a \sin (xy) -y \cos (xy) )}{a^2 + y^2} dy
\]
Правый элемент обнуляется в силу нечетности, получаем ответ:
\begin{center}
\textbf{Ответ: } 
\[
\frac{1}{\sqrt{2\pi}}  \int\limits_{-\infty}^{+\infty}  \frac{a \cos (xy) +  y \sin (xy) }{a^2 + y^2} dy 
\]
\end{center}
\clearpage
\section*{Номер 2}
\[
f(x) = \frac{\sin x}{x}
\]
Ищем:
\[
\stackrel{\wedge}{f}(x)  =\sqrt{ \frac{2}{\pi}} \int\limits_0^{\infty} \frac{\sin x}{x}  \cdot cos (xy) dx =
\]
\[
=
 \frac{1}{\sqrt{2\pi}} \int\limits_0^{\infty}  \frac{\sin (x ( 1- y))}{x} dx + \frac{1}{\sqrt{2\pi}} \int\limits_0^{\infty} \frac{\sin(x(1 + y))}{x} dx =
\]
\[
=
 \frac{1}{\sqrt{2\pi}}  \left(
\frac{\pi}{2} \sgn(1 -y) + \frac{\pi}{2} \cdot \sgn( 1 +y )
\right)
\] 
\begin{center}
\textbf{Ответ: } 
\[
 \frac{1}{\sqrt{2\pi}}  \left(
\frac{\pi}{2} \sgn(1 -y) + \frac{\pi}{2} \cdot \sgn( 1 +y )
\right)
\]
\end{center}
\end{document}
