\documentclass[a4paper,12pt]{article}

%%% Работа с русским языком
\usepackage{cmap}					% поиск в PDF
\usepackage{mathtext} 				% русские буквы в формулах
\usepackage[T2A]{fontenc}			% кодировка
\usepackage[utf8]{inputenc}			% кодировка исходного текста
\usepackage[english,russian]{babel}	% локализация и переносы
\usepackage{xcolor}
\usepackage{hyperref}
 % Цвета для гиперссылок
\definecolor{linkcolor}{HTML}{799B03} % цвет ссылок
\definecolor{urlcolor}{HTML}{799B03} % цвет гиперссылок

\hypersetup{pdfstartview=FitH,  linkcolor=linkcolor,urlcolor=urlcolor, colorlinks=true}

%%% Дополнительная работа с математикой
\usepackage{amsfonts,amssymb,amsthm,mathtools} % AMS
\usepackage{amsmath}
\usepackage{icomma} % "Умная" запятая: $0,2$ --- число, $0, 2$ --- перечисление

%% Номера формул
%\mathtoolsset{showonlyrefs=true} % Показывать номера только у тех формул, на которые есть \eqref{} в тексте.

%% Шрифты
\usepackage{euscript}	 % Шрифт Евклид
\usepackage{mathrsfs} % Красивый матшрифт

%% Свои команды
\DeclareMathOperator{\sgn}{\mathop{sgn}}

%% Перенос знаков в формулах (по Львовскому)
\newcommand*{\hm}[1]{#1\nobreak\discretionary{}
{\hbox{$\mathsurround=0pt #1$}}{}}
% графика
\usepackage{graphicx}
\graphicspath{{pictures/}}
\DeclareGraphicsExtensions{.pdf,.png,.jpg}
\author{Бурмашев Григорий, БПМИ-208}
\title{}
\date{\today}
\begin{document}
\begin{center}
Бурмашев Григорий. 208. Матан -- 7 
\end{center}
 \begin{center}
\includegraphics[scale=0.34]{1.jpg}
\end{center}
\clearpage
\section*{Номер 6}
\subsection*{3)}
\[
\int \frac{x^2 + 3x -2}{(x-1)(x^2+x+1)^2} \, dx
\]
По методу Остроградского
\[
\int \frac{x^2 + 3x -2}{(x-1)(x^2+x+1)^2} \, dx = \frac{ax+b}{x^2+x+1} + \int \frac{\text{что-то}}{(x-1)(x^2+x+1)} \,dx =
\]
\[
=
 \frac{ax+b}{x^2+x+1} + \int  \left[ \frac{\text{с}}{(x-1)}  + \frac{dx + e}{x^2+x+1} \right] \,dx = 
\]
Продифференцируем:
\[
\frac{x^2+3x-2}{(x-1)(x^2+x+1)^2} = \frac{a(x^2+x+1) -(2x+1)(ax+b)}{(x^2+x+1)^2} + \frac{c}{x-1} + \frac{dx+e}{x^2+x+1}
\]
Тогда:
\[
x^2 +3x -2 =
\]
\[
=
 (x-1)\left[ a(x^2+x+1) - (2x+1)(ax+b) \right]  + 
\]
\[
+ c(x^2 + x +1)^2 + (dx+e)(x-1)(x^2+x+1) = 
\]
\[
= 
-ax^3 + ax^2 + ax - a - 2bx^2 + bx + b + cx^4 + 2cx^3 + 3cx^2 + 2cx + c + dx^4 -dx + ex^3 - e = 
\]
\[
= 
(c+d)x^4 + (-2+2c+e)x^3 + (a-2b+3c)x^2 + (a +b + 2c -d)x + (-a + b + c - e) 
\]
Получаем систему:
\[
\begin{cases}
c  + d = 0 \\ 
-a + 2c + e = 0 \\
a -2b + 3c = 1 \\
a + b + 2c - d = 3\\
-a + b + c - e = -2
\end{cases}
\]
Надо (к сожалению) её решать, приведем к матричному виду:
\[
\begin{pmatrix}
0 & 0 & 1 & 1 & 0 & \vrule &0 & \\
0 & -2 & 5 & 0 & 1 & \vrule &1 & \\
1 & -2 & 3 & 0 & 0 & \vrule &1 & \\
1 & 1 & 2 & -1 & 0 & \vrule &3 & \\
-1 & 1 & 1 & 0 & -1 & \vrule &-2 & \\
\end{pmatrix}
=
\begin{pmatrix}
0 & 0 & 1 & 1 & 0 & \vrule &0 & \\
0 & -2 & 5 & 0 & 1 & \vrule &1 & \\
0 & -3 & 1 & 1 & 0 & \vrule &-2 & \\
1 & 1 & 2 & -1 & 0 & \vrule &3 & \\
-1 & 1 & 1 & 0 & -1 & \vrule &-2 & \\
\end{pmatrix}
=
\]
\[
=
\begin{pmatrix}
0 & 0 & 1 & 1 & 0 & \vrule &0 & \\
0 & -2 & 5 & 0 & 1 & \vrule &1 & \\
0 & -3 & 1 & 1 & 0 & \vrule &-2 & \\
1 & 1 & 2 & -1 & 0 & \vrule &3 & \\
0 & 2 & 3 & -1 & -1 & \vrule &1 & \\
\end{pmatrix}
=
\]
\[
=
\begin{pmatrix}
1 & 1 & 2 & -1 & 0 & \vrule &3 & \\
0 & 0 & 8 & -1 & 0 & \vrule &2 & \\
0 & -1 & 4 & 0 & -1 & \vrule &-1 & \\
0 & 0 & 1 & 1 & 0 & \vrule &0 & \\
0 & 2 & 3 & -1 & -1 & \vrule &1 & \\
\end{pmatrix}
=
\]
\[
=
\begin{pmatrix}
1 & 1 & 2 & -1 & 0 & \vrule &3 & \\
0 & 1 & -4 & 0 & 1 & \vrule &1 & \\
0 & 0 & 0 & -9 & 0 & \vrule &2 & \\
0 & 0 & 1 & 1 & 0 & \vrule &0 & \\
0 & 0 & 11 & -1 & -3 & \vrule &-1 & \\
\end{pmatrix}
=
\begin{pmatrix}
1 & 1 & 2 & -1 & 0 & \vrule &3 & \\
0 & 1 & -4 & 0 & 1 & \vrule &1 & \\
0 & 0 & 0 & -9 & 0 & \vrule &2 & \\
0 & 0 & 1 & 1 & 0 & \vrule &0 & \\
0 & 0 & 0 & -12 & -3 & \vrule &-1 & \\
\end{pmatrix}
=
\]
\[
=
\begin{pmatrix}
1 & 1 & 2 & -1 & 0 & \vrule &3 & \\
0 & 1 & -4 & 0 & 1 & \vrule &1 & \\
0 & 0 & 1 & 1 & 0 & \vrule &0 & \\
0 & 0 & 0 & -9 & 0 & \vrule &2 & \\
0 & 0 & 0 & -3 & -3 & \vrule &-3 & \\
\end{pmatrix}
=
\begin{pmatrix}
1 & 1 & 2 & -1 & 0 & \vrule &3 & \\
0 & 1 & -4 & 0 & 1 & \vrule &1 & \\
0 & 0 & 1 & 1 & 0 & \vrule &0 & \\
0 & 0 & 0 & 0 & 9 & \vrule &11 & \\
0 & 0 & 0 & 1 & 1 & \vrule &1 & \\
\end{pmatrix}
=
\]
\[
=
\begin{pmatrix}
1 & 1 & 2 & -1 & 0 & \vrule &3 & \\
0 & 1 & -4 & 0 & 1 & \vrule &1 & \\
0 & 0 & 1 & 1 & 0 & \vrule &0 & \\
0 & 0 & 0 & 1 & 1 & \vrule &1 & \\
0 & 0 & 0 & 0 & 9 & \vrule &11 & \\
\end{pmatrix}
=
\begin{pmatrix}
1 & 0 & 0 & 0 & 0 & \vrule &\frac{5}{3} \\
0 & 1 & 0& 0 & 0& \vrule &\frac{2}{3} \\
0 & 0 & 1 & 0 & 0 & \vrule &\frac{2}{9} \\
0 & 0 & 0 & 1 & 0 & \vrule &-\frac{2}{9} \\
0 & 0 & 0 & 0 & 1& \vrule &\frac{11}{9} \\
\end{pmatrix}
\]
А значит:
\[
\begin{cases}
a = \frac{5}{3}\\
b = \frac{2}{3} \\
c = \frac{2}{9} \\
d =  -\frac{2}{9} \\
e = \frac{11}{9} \\
\end{cases}
\]
\[
 \frac{ax+b}{x^2+x+1} + \int  \left[ \frac{c}{(x-1)}  + \frac{dx + e}{x^2+x+1} \right] \,dx = 
\]
\[
= 
 \frac{\frac{5}{3}x+\frac{2}{3}}{x^2+x+1} + \int  \left[ \frac{ \frac{2}{9}}{(x-1)}  + \frac{-\frac{2}{9}x + \frac{11}{9}}{x^2+x+1} \right] \,dx
\]
Посчитаем интеграл по отдельности:
\[
\int  \left[ \frac{ \frac{2}{9}}{(x-1)}  + \frac{-\frac{2}{9}x + \frac{11}{9}}{x^2+x+1} \right] \,dx = \frac{2}{9} \int \frac{1}{x-1} \, dx  - \frac{1}{9} \int \frac{2x-11}{x^2 +x + 1} \, dx  = 
\]
\[
= \frac{2}{9} \ln|x-1| - \frac{1}{9}\int \frac{2x - 11}{x^2 +x + 1} \, dx
\]
\[
\int \frac{2x - 11}{x^2 + x + 1} \, dx = \int \frac{(2x+1) - 12 }{x^2 + x + 1} \, dx  = \int \frac{2x+1}{x^2 + x + 1} \, dx  - \int \frac{12}{x^2 + x +1} \, dx =
\]
\[
=
\frac{2}{9} \ln |x-1| - \frac{1}{9} \int \frac{2x-11}{x^2+x+1} \, dx
\]
\[
\int \frac{2x-11}{x^2 + x +1} \, dx = \int \frac{(2x+1) - 12}{x^2 + x + 1} \, dx = \int \frac{2x+1}{x^2 + x + 1} \, dx - \int \frac{12}{x^2 + x + 1} \, dx = 
\]
\[
=
\int \frac{(2x+1)}{(x^2+x+1)(2x+1)} \, d(x^2 + x + 1) -12 \int \frac{1}{(x + \frac{1}{2})^2 + \frac{3}{4}} \, dx = 
\]

\[
=
\ln (x^2 + x + 1) - 12 \int \frac{1}{(x + \frac{1}{2})^2 + \frac{3}{4}} \, dx = 
\]
\[
= \ln (x^2 + x + 1) - \frac{12}{\sqrt{\frac{3}{4}}} \cdot \arctan \left( \frac{x+ \frac{1}{2}}{\sqrt{\frac{3}{4}}}\right) + C = 
\]
\[
= 
\ln (x^2 + x + 1) + \frac{20}{\sqrt{3}} \arctan \left( \frac{2x+1}{\sqrt{3}}\right) + C\]
Тогда возвращаемся к исходному выражению:
\[
 \frac{\frac{5}{3}x+\frac{2}{3}}{x^2+x+1} +  \frac{2}{9} \ln|x-1| - \frac{1}{9} \ln (x^2 + x + 1) + \frac{8 \sqrt{3}}{9} \arctan \left( \frac{2x+1}{\sqrt{3}}\right) + C
\]
\begin{center}
\textbf{Ответ: } 
\[
 \frac{\frac{5}{3}x+\frac{2}{3}}{x^2+x+1} +  \frac{2}{9} \ln|x-1| - \frac{1}{9} \ln (x^2 + x + 1) +\frac{ 8 \sqrt{3}}{9} \arctan \left( \frac{2x+1}{\sqrt{3}}\right) + C
\]
\end{center}
\subsection*{4)}
\[
\int \frac{dx}{x(x^3+1)^2}
\]
По методу Остроградского:
\[
\int \frac{dx}{x(x^3+1)^2} = \frac{ax^2 +bx + c}{x^3 + 1} + \int \frac{\text{что-то}}{x(x^3+1)} \,dx = \frac{ax^2 + bx + c}{x^3 +1} + \int \left( \frac{d}{x} + \frac{ex^2 + fx + g}{x^3 + 1}\right) \, dx 
\]
Продифференцируем:
\[
\frac{1}{x(x^3 + 1)^2} = \frac{(2ax + b)(x^3 +1) - (3x^2)(ax^2 + bx +c)}{(x^3 + 1)^2} + \frac{d}{x} + \frac{ex^2 +fx +g}{x^3 + 1} 
\]
\[
1 = x((ax+b)(x^3+1) - 3x^2(ax^2+bx+c) + d(x^3+1)^2 + (ex^2+fx+g)(x^2+1)x) = 
\]
\[
= -ax^5 + 2ax - 2x^4 + bx - 3ex^3 + dx^6 + 2dx^3 + d + ex^6 + ex^3 + fx^5 + fx^2 + gx^4+ gx
\]
\[
\begin{cases}
d + e = 0 \\
-a + f = 0 \\
-2b + g = 0 \\
-3c + 2d + e = 0 \\
f = 0\\
2a + b + g = 0 \\
d = 1\\
\end{cases}
\]
\[
\begin{cases}
e = -1 \\
 a = 0 \\
c = \frac{1}{3}  \\
d = 1\\
f = 0 \\
b = 0 \\
 g = 0 \\
\end{cases}
\]
Тогда:
\[
\int \frac{dx}{x(x^3 + 1)^2} = \frac{1}{3(x^3 + 1)} + \int \left(\frac{1}{x} - \frac{x^2}{x^3 + 1}\right) \, dx
\]
Посчитаем интеграл отдельно:
\[
\int \left( \frac{1}{x} - \frac{x^2}{x^3 + 1}\right) \, dx = \int \frac{dx}{x} - \int \frac{x^2}{x^3 + 1} \, dx = 
\]
\[
= \ln |x| - \frac{1}{3} \int \frac{x^2}{(x^3+1)x^2} \, d(x^3 + 1) = \ln |x| - \frac{1}{3} \int \frac{d(x^3 + 1)}{x^3 + 1} = 
\]
\[
= \ln |x| - \frac{1}{3} \ln |x^3 + 1| + C
\]
\[
\int \frac{dx}{x(x^3+1)} =  \frac{1}{3(x^3 + 1)} + \ln |x| - \frac{1}{3} \ln |x^3 + 1| + C
\]
\begin{center}
\textbf{Ответ: } 
\[
 \frac{1}{3(x^3 + 1)} + \ln |x| - \frac{1}{3} \ln |x^3 + 1| + C
\]
\end{center}
\section*{Номер 7}
\subsection*{1)}
\[
\int \frac{dx}{3 + \sin x}
\]
Универсальная триг.подстановка:
\[
\tan \frac{x}{2} = t
\]
\[
\sin x = \frac{ 2 \tan \frac{x}{2}}{1 + \tan^2 \frac{x}{2}} = \frac{2t}{1 + t^2}
\]
\[
dx = \frac{2dt}{1 + t^2}
\]
Тогда получаем:
\[
\int \frac{dx}{3 + \sin x} =  \int \frac{2dt}{(3 + \frac{2t}{1 + t^2})(1+t^2)} = 
\]
\[
=\int  \frac{1+t^2}{3(1+t^2) + 2t} \cdot \frac{2dt}{1+t^2} = 2 \int \frac{dt}{3t^2 + 2t + 3} = \frac{2}{3} \int \frac{dt}{t^2 + \frac{2}{3}t + 1} = 
\]
\[
=
\frac{2}{3} \int \frac{dt}{(t^2 + \frac{1}{3})^2 + \frac{8}{9}}   = \frac{3}{4} \cdot \int \frac{dt}{\left( \frac{3t+1}{2\sqrt{2}}\right)^2 + 1}= \frac{\arctan \left(\frac{3 \tan \frac{x}{2} + 1}{2\sqrt{2}}\right)}{\sqrt{2}} + C
\]
\begin{center}
\textbf{Ответ: } \[ \frac{\arctan \left(\frac{3 \tan \frac{x}{2} + 1}{2\sqrt{2}}\right)}{\sqrt{2}} + C\]
\end{center}
\end{document}
