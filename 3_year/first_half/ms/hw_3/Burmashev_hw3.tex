\documentclass[a4paper,12pt]{article}

%%% Работа с русским языком
\usepackage{cmap}					% поиск в PDF
\usepackage{mathtext} 				% русские буквы в формулах
\usepackage[T2A]{fontenc}			% кодировка
\usepackage[utf8]{inputenc}			% кодировка исходного текста
\usepackage[english,russian]{babel}	% локализация и переносы
\usepackage{xcolor}
\usepackage{hyperref}
 % Цвета для гиперссылок
\definecolor{linkcolor}{HTML}{799B03} % цвет ссылок
\definecolor{urlcolor}{HTML}{799B03} % цвет гиперссылок

\hypersetup{pdfstartview=FitH,  linkcolor=linkcolor,urlcolor=urlcolor, colorlinks=true}

%%% Дополнительная работа с математикой
\usepackage{amsfonts,amssymb,amsthm,mathtools} % AMS
\usepackage{amsmath}
\usepackage{icomma} % "Умная" запятая: $0,2$ --- число, $0, 2$ --- перечисление

%% Номера формул
%\mathtoolsset{showonlyrefs=true} % Показывать номера только у тех формул, на которые есть \eqref{} в тексте.

%% Шрифты
\usepackage{euscript}	 % Шрифт Евклид
\usepackage{mathrsfs} % Красивый матшрифт

%% Свои команды
\DeclareMathOperator{\sgn}{\mathop{sgn}}

%% Перенос знаков в формулах (по Львовскому)
\newcommand*{\hm}[1]{#1\nobreak\discretionary{}
{\hbox{$\mathsurround=0pt #1$}}{}}
% графика
\usepackage{graphicx}
\graphicspath{{pictures/}}
\DeclareGraphicsExtensions{.pdf,.png,.jpg}
\author{Бурмашев Григорий, БПМИ-208}
\title{Математические структуры, дз -- 3}
\date{\today}
\begin{document}
\maketitle
\clearpage
\section*{Номер 1}
Является ли интуиционисткой тавтологией следующая формула:
\[
(
	(
	\neg \neg p
	\rightarrow 
	p
	)
\rightarrow
	(
	p \vee \neg p
	)
)
\rightarrow
	(
	\neg p \vee \neg \neg p
	)
)?
\]
Пусть:
\[
M, x \not \models
(
	(
	\neg \neg p
	\rightarrow 
	p
	)
\rightarrow
	(
	p \vee \neg p
	)
)
\rightarrow
	(
	\neg p \vee \neg \neg p
	)
)
\]
Тогда $\exists y : x \preceq  y$, что:
\[
\begin{cases}
	y\models (\neg \neg p \to p) \to (p \vee \neg p) \\
	y \not \models (\neg p \vee \neg \neg p)
\end{cases}
\]
Разберемся со вторым, хотим:
\[
\begin{cases}
y \not \models \neg p \\
y \not \models \neg \neg p
\end{cases}
\sim
\begin{cases}
y \not \models p  \rightarrow  \bot \\
y \not \models \neg p \rightarrow \bot
\end{cases}
\]
Т.е $\exists z , z' : y \preceq  z, y \preceq z'$, что:
\[
z \models p
\]
\[
z' \models \neg p
\]
Теперь разберемся с первым, из полученного выше получаем $(\times)$:
\[
y \not \models (\neg \neg p \rightarrow p) 
\]
Но тогда мы не сможем получить:
\[
\begin{cases}
y \models (\neg \neg p \rightarrow p) \\
y \not \models (p \vee \neg p)
\end{cases}
\]
По итогу из $(\times)$ получаем:
\[
y\models (\neg \neg p \to p) \to (p \vee \neg p)
\]
А значит исходная формула \textbf{не} является интуиционисткой тавтологией
\begin{center}
\textbf{Ответ: } \textbf{не} является интуиционисткой тавтологией
\end{center}
\end{document}
