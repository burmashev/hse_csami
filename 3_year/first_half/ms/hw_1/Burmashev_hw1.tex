\documentclass[a4paper,12pt]{article}

%%% Работа с русским языком
\usepackage{cmap}					% поиск в PDF
\usepackage{mathtext} 				% русские буквы в формулах
\usepackage[T2A]{fontenc}			% кодировка
\usepackage[utf8]{inputenc}			% кодировка исходного текста
\usepackage[english,russian]{babel}	% локализация и переносы
\usepackage{xcolor}
\usepackage{hyperref}
 % Цвета для гиперссылок
\definecolor{linkcolor}{HTML}{799B03} % цвет ссылок
\definecolor{urlcolor}{HTML}{799B03} % цвет гиперссылок

\hypersetup{pdfstartview=FitH,  linkcolor=linkcolor,urlcolor=urlcolor, colorlinks=true}

%%% Дополнительная работа с математикой
\usepackage{amsfonts,amssymb,amsthm,mathtools} % AMS
\usepackage{amsmath}
\usepackage{icomma} % "Умная" запятая: $0,2$ --- число, $0, 2$ --- перечисление

%% Номера формул
%\mathtoolsset{showonlyrefs=true} % Показывать номера только у тех формул, на которые есть \eqref{} в тексте.

%% Шрифты
\usepackage{euscript}	 % Шрифт Евклид
\usepackage{mathrsfs} % Красивый матшрифт

%% Свои команды
\DeclareMathOperator{\sgn}{\mathop{sgn}}

%% Перенос знаков в формулах (по Львовскому)
\newcommand*{\hm}[1]{#1\nobreak\discretionary{}
{\hbox{$\mathsurround=0pt #1$}}{}}
% графика
\usepackage{graphicx}
\graphicspath{{pictures/}}
\DeclareGraphicsExtensions{.pdf,.png,.jpg}
\author{Бурмашев Григорий, БПМИ-208}
\title{Математические структуры, дз -- 1}
\date{28 сентября 2022 г.}
\begin{document}
\maketitle
\section*{Номер 1}
Приведите к ДНФ формулу:
\[
(p \vee q ) \to (p \vee \neg r)
\]
\\
Решаем, для начала избавимся от $\to$:
\[
(p \vee q ) \to (p \vee \neg r) \equiv \neg (p \vee q) \vee (p \vee \neg r) \equiv
\]
Теперь избавимся от отрицания:
\[
\equiv 
(\neg p \wedge \neg q) \vee (p \vee \neg r) 
\equiv
(\neg p \wedge \neg q) \vee p \vee \neg r
\equiv 
\]
Теперь используем дистрибутивность и "сократим" скобку, которая всегда равна 1 $(\neg p \vee p)$:
\[
\equiv
(\neg p \vee p) \wedge (\neg q \vee p) \vee \neg r
\equiv
\neg q  \vee p \vee \neg r
\]
\begin{center}
\textbf{Ответ: } $ \neg q  \vee p \vee \neg r $
\end{center}
\clearpage
\section*{Номер 2}
Докажите, что следующая формула является тавтологией для любого $ n > 0 $:
\[
\bigwedge\limits_{i=1}^{n+1} \bigvee\limits_{j=1}^{n} p_{ij} \to \bigvee\limits_{j=1}^{n}\bigvee\limits_{\begin{matrix}i_1, i_2 = 1 \\ i_1 < i_2 \end{matrix}}^{n+1} (p_{i_1j} \wedge p_{i_2j})
\]
Доказательство:
\\\

Чтобы показать, что эта формула -- тавтология, нужно понять, что она всегда равна 1. Нам нужно проверить случай $1 \to 0$,  поскольку только этот случай дает 0.  

Если абсолютно все наши переменные ложны, то мы получаем $0 \rightarrow 0 $, что равно 1.

Посмотрим, как выглядит левое выражение:
\[
(p_{11} \vee p_{12} \vee \ldots \vee p_{1n}) \wedge \ldots \wedge (p_{n + 1, 1} \vee \ldots \vee p_{n + 1, n})
\]

Теперь предположим, что левая часть выражения дает 1. Простыми словами это означает, что в каждой скобке найдется хотя бы один элемент, который не равен нулю. Т.е для любого $i$  от $1$ до $n + 1$ мы найдем $p_{ij} = 1$. С другой стороны заметим, что тогда для $i_1 < i_2$ найдется такое $j$, что $(p_{i_1 j} \wedge p_{i_2 j }) = 1$ (вытекает из рассуждений выше). Следовательно хотя бы одна скобка даст нам 1 в правом выражении, а значит все правое выражение тоже даст нам 1, следовательно случай $ 1 \to 0 $ получить мы никак не можем, а значит формула -- тавтология.

\begin{center}
\textbf{Ч.Т.Д} 
\end{center}
\end{document}
