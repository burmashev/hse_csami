\documentclass[a4paper,12pt]{article}

%%% Работа с русским языком
\usepackage{cmap}					% поиск в PDF
\usepackage{mathtext} 				% русские буквы в формулах
\usepackage[T2A]{fontenc}			% кодировка
\usepackage[utf8]{inputenc}			% кодировка исходного текста
\usepackage[english,russian]{babel}	% локализация и переносы

%%% Дополнительная работа с математикой
\usepackage{amsfonts,amssymb,amsthm,mathtools} % AMS
\usepackage{amsmath}
\usepackage{icomma} % "Умная" запятая: $0,2$ --- число, $0, 2$ --- перечисление

%% Номера формул
%\mathtoolsset{showonlyrefs=true} % Показывать номера только у тех формул, на которые есть \eqref{} в тексте.

%% Шрифты
\usepackage{euscript}	 % Шрифт Евклид
\usepackage{mathrsfs} % Красивый матшрифт

%% Свои команды
\DeclareMathOperator{\sgn}{\mathop{sgn}}

%% Перенос знаков в формулах (по Львовскому)
\newcommand*{\hm}[1]{#1\nobreak\discretionary{}
{\hbox{$\mathsurround=0pt #1$}}{}}

\author{Бурмашев Григорий}
\title{Дискра-1}
\date{\today}
\begin{document}
\begin{center}
Бурмашев Григорий, 208. Дискра-1
\end{center}
\section*{1.}
Доказать, что:\[A \vee B \equiv \neg B \rightarrow A \] 

Построим таблицу истинности:\\\\
\begin{tabular}{|c|c|c|c|c|}
\hline
 $A$& $B $& $ \neg B $&   $A \vee B$ & $\neg B \rightarrow A$ \\
\hline
 0&  0&  1&0 & 0 \\
\hline
 0& 1 & 0 & 1& 1\\
\hline
 1& 0 & 1 & 1 &1\\
\hline
1 & 1 &0 & 1&1\\
\hline 
\end{tabular}\\\\
Заметим, что при любых A, B выражения эквивалентны.\\\\
 \textbf{Ч.Т.Д}
\section*{2.} 
Выразить  $A \triangleleft B $, которое ложно, если A ложно, а B истинно, в остальных случаях оно истинно:\\

Это выражение вида: \[ A \vee \neg B \]

Построим таблицу истинности:\\\\
\begin{tabular}{|c|c|c|c|}
\hline
$ A $ & $ B $ & $ \neg B $ & $ A \vee \neg B $ \\
\hline
0 & 0 & 1 & 1 \\
\hline
0 & 1 & 0 & 0 \\
\hline
1 & 0 & 1 & 1 \\
\hline
1 & 1 & 0 & 1\\
\hline
\end{tabular}\\\\
Из таблицы истинности видно, что это выражение полностью соотвествует условию задачи.\\

Ответ: $ A \vee \neg B $
\section*{3.}
Ассоциативна ли импликация? Другими словами, равносильны ли высказывания:
\[A \rightarrow (B \rightarrow C) \text{ и } (A \rightarrow B) \rightarrow C \]\\

Рассмотрим случай, когда A = 0, B = 0, C = 0. $(0 \rightarrow (0 \rightarrow 0))$. При указанных значениях  A, B, C левое выражение принимает истину. (Из лжи следует истина). А правое - принимает ложь (Из истины следует ложь) $ ((0\rightarrow 0) \rightarrow 0) $. Значит, эти высказывания \textbf{не} равносильны \\

Ответ: \textbf{нет}
\section*{4.}
Выполняется ли дистрибутивность для конъюнкции относительно импликации? Другими словами, равносильны ли высказывания:

\[A \wedge (B \rightarrow C) \text{ и }(A \wedge B) \rightarrow (A \wedge C) \]\\

При A = 0, B = 0, C = 0 мы получим, что $ A \wedge (B \rightarrow C)$ ложно (т.к конъюнкция лжи с чем угодно есть ложь), а $(A \wedge B) \rightarrow (A \wedge C) $ истинно (т.к из лжи следует ложь есть истина). Значит, эти высказывания \textbf{не} равносильны\\

Ответ: \textbf{нет}
\section*{5.}
Выполняется ли дистрибутивность для импликации относительно импликации? Другими словами, равносильны ли высказывания:
\[A \rightarrow (B \rightarrow C) \text{ и } (A \rightarrow B) \rightarrow (A \rightarrow C)\]\\

Рассмотрим случай, когда высказывания ложны. Чтобы левое высказывание было ложно, нужно, чтобы A = 1, B = 1, C = 0. ($ 1 \rightarrow(1 \rightarrow 0)$ Во всех остальных случаях оно будет истинным. При таких значениях A, B, C правое высказывание также принимает ложь ($ (1  \rightarrow 1) \rightarrow (1 \rightarrow 0)$ Аналогично первому высказыванию, при остальных значениях A, B, C  второе высказывание истинно. Значит, эти высказывания равносильны.\\

Ответ: \textbf{да}
\section*{6.}
Доказать, что:
\[
|x+y| \leq |x| + |y|
\]\\

Рассмотрим все возможные случаи:\\
Если $x, y \geq  0$:
\begin{equation*}
\begin{gathered}
|x| = x \\
|y| = y\\
x + y \leq x + y
\end{gathered}
\end{equation*}
Если $ x > 0$, $ y < 0$:
\begin{equation*}
\begin{gathered}
|x| = x \\
|y| = -y\\
x - y \leq x - y
\end{gathered}
\end{equation*}
Если $ x < 0 $, $ y >	 0 $:
\begin{equation*}
\begin{gathered}
|x| = -x \\
|y| = y\\
-x + y \leq -x + y
\end{gathered}
\end{equation*}
Если $ x < 0 $, $ y > 0 $:
\begin{equation*}
\begin{gathered}
|x| = -x \\
|y| = -y\\
-x - y \leq -x - y
\end{gathered}
\end{equation*}
\textbf{Ч.Т.Д}
\section*{7.}
Доказать, что  $\forall (a, b, n) > 0$:

\[ \text{Из } (a \times b = n) \rightarrow (a \leq \sqrt{n}) \vee (b \leq \sqrt{n}) \]\\

Воспользуемся законом контрпозиции:\\
\[ (a > \sqrt{n}) \wedge  (b > \sqrt{n}) \rightarrow (a \times b \neq n)\]

Если: 
\begin{equation*}
 \begin{cases}
   a > \sqrt{n} 
   \\
   b > \sqrt{n}
 \end{cases}
\end{equation*}

То:
\[ a \times b > n\]

Тогда абсолютно точно:
\[ a \times b \neq n\]

\textbf{Ч.Т.Д}
\section*{8.}
Доказать, что $ \forall \; x,y,z,w \in \mathbb{Z} $:
\[ x^2 + y^2 + z^2 = w^2\] 
$ A \equiv B $, где:

A - <<w чётное>>
B - <<все числа x, y, z чётные>>\\

Рассмотрим случай,  когда  A = 1, B = 1, тогда:

чётное$^2$ + чётное$^2$  + чётное$^2$ = чётное$^2$

Это высказывание \textbf{истино}. Чётное число в квадрате чётно, cумма трех чётных чисел также чётна.\\ 

Рассмотрим случай, когда A = 0, B = 0, тогда:

нечётное$^2$ + нечётное$^2$ + нечётное$^2$ = нечётное$^2$

Это высказывание также \textbf{истино}.  Потому что нечётное число в квадрате нечётно, а сумма трех нечётных цифр также нечётна.\\

Значит, $A \equiv B $ 

\textbf{Ч.Т.Д}
\section*{9.}
Пусть $x = \sqrt{10}$, а $ y = \log_{\sqrt{10}}5 $\\\\ $x,\;y$ --- иррациональные. Тогда по основному логарифмическому тождеству:
\[ x ^ y = \sqrt{10}^{log_{\sqrt10}5} = 5\]

При возведении иррационального числа в иррациональную степень мы получили рациональное число.

\textbf{Ч.Т.Д}


\end{document}