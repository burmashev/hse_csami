\documentclass[a4paper,12pt]{article}

\usepackage[left=4cm,right=4cm,
    top=4cm,bottom=4cm,bindingoffset=0cm]{geometry}
%%% Работа с русским языком
\usepackage{cmap}					% поиск в PDF
\usepackage{mathtext} 				% русские буквы в формулах
\usepackage[T2A]{fontenc}			% кодировка
\usepackage[utf8]{inputenc}			% кодировка исходного текста
\usepackage[english,russian]{babel}	% локализация и переносы
\usepackage{xcolor}
\usepackage{hyperref}
 % Цвета для гиперссылок
\definecolor{linkcolor}{HTML}{799B03} % цвет ссылок
\definecolor{urlcolor}{HTML}{799B03} % цвет гиперссылок

\hypersetup{pdfstartview=FitH,  linkcolor=linkcolor,urlcolor=urlcolor, colorlinks=true}

%%% Дополнительная работа с математикой
\usepackage{amsfonts,amssymb,amsthm,mathtools} % AMS
\usepackage{amsmath}
\usepackage{icomma} % "Умная" запятая: $0,2$ --- число, $0, 2$ --- перечисление

%% Номера формул
%\mathtoolsset{showonlyrefs=true} % Показывать номера только у тех формул, на которые есть \eqref{} в тексте.

%% Шрифты
\usepackage{euscript}	 % Шрифт Евклид
\usepackage{mathrsfs} % Красивый матшрифт

%% Свои команды
\DeclareMathOperator{\sgn}{\mathop{sgn}}

%% Перенос знаков в формулах (по Львовскому)
\newcommand*{\hm}[1]{#1\nobreak\discretionary{}
{\hbox{$\mathsurround=0pt #1$}}{}}
% графика
\usepackage{graphicx}
\graphicspath{{pictures/}}
\DeclareGraphicsExtensions{.pdf,.png,.jpg}
\author{Бурмашев Григорий, БПМИ-208}
\title{}
\date{\today}
\begin{document}
\begin{center}
Бурмашев Григорий. 208. Алгебра -- 7
\end{center}
\section*{Номер 1}
\[
I = (x^2y + 2z^2 (= f_1), y^2 - yz = (f_2))
\]
\[
g_1 = x^3z^3 + 3xyz^3
\]
\[
g_2 = x^3y^2z + 2xy^2z^2
\]
Решаем стандартным алгоритмом:
\\\\
\textbf{1)} Cтроим базис Гребнера:
\[
S(f_1, f_2) = yf_1 - x^2f_2 = x^2y^2 + 2yz^2 - (x^2y^2 - x^2yz) =2yz^2+x^2yz \overset{f_1}{\rightarrow} 2yz^2 - 2z^3 = 2(yz^2 - z^3)
\]
Дальше не редуцируется, значит добавляем $f_3 = yz^2 - z^3$
\[
S(f_1, f_3) = z^2f_1 - x^2 f_3 = x^2yz^2 + 2z^4 - x^2yz^2 + x^2z^3 = x^2z^3+2z^4
\]
Дальше не редуцируется, значит добавляем $f_4 = x^2z^3 + 2z^4$
\[
S(f_1, f_4) = z^3 f_1 - yf_4 = 2z^5 - 2yz^4 \overset{f_3(2z^2)}{\rightarrow} 0
\]
\[
S(f_2, f_3) = z^2 f_2 - y f_3 = y^2z^2 - yz^3 - (y^2z^2 - yz^3) = -yz^3 + yz^3 = 0
\]
\[
S(f_2, f_4) = x^2z^3 f_2 - y^2 f_4 = -x^2 y z^4 - 2 y^2 z^4 \overset{f_4(yz)}{\rightarrow}-2 y^2 z^4 + 2 y z^5 \overset{f_3(yz^2)}{\rightarrow} 0
\]
\[
S(f_3, f_4) = x^2z f_3 - yf_4 = -x^2 z^4 - 2 y z^4 \overset{f_4(z)}{\rightarrow} 2z^5 -2yz^4 \overset{f_3(2z^2)}{\rightarrow} 0
\]
Проверили для всех f, алгоритм построения базиса закончен, получили базис : $(f_1, f_2, f_3, f_4)$
\[
f_1 = x^2y + 2z^2
\]
\[
f_2 = y^2 - yz
\]
\[
f_3 = yz^2 - z^3
\]
\[
f_4 = x^2z^3 + 2z^4
\]
\clearpage
\textbf{2)} Редуцируем $g_1$ и $g_2$ относительно полученного базиса:
\begin{itemize}
\item $g_1$
\[
x^3z^3 + 3xyz^3 \overset{f_4(x)}{\rightarrow} 3xyz^3 - 2xz^4 \overset{f_3(3xz)}{\rightarrow} xz^4
\]
Далее не редуцируется, значит $g_1$ не принадлежит идеалу $I$
\item $g_2$
\[
x^3y^2z + 2xy^2z^2 \overset{f_1(xyz)}{\rightarrow} 2 x y^2 z^2 - 2 x y z^3 \overset{f_3(2xy)}{\rightarrow} 0
\]
Значит $g_2$ принадлежит идеалу $I$
\end{itemize}
\begin{center}
\textbf{Ответ: } $g_1$ не принадлежит, $g_2$ принадлежит
\end{center}
\clearpage
\section*{Номер 2}
\[
(xy + 2yz, x - y^2, yz^2 - y) \subseteq \mathbb{R}[x, y, z]
\]
\[
z > x > y
\]
Делаем замену для удобства [это было гигантской ошибкой]:
\[
z = x', x = y', y = z'
\]
\[
(y'z'+ 2z'x', y' - z'^2, z'x'^2 - z')
\]
\[
( 2x'z' + y'z' (=f_1), y' - z'^2(=f_2), x'^2z' - z'(=f_3))
\]
Теперь строим базис Гребнера:
\[
S(f_1, f_2) = y'f_1 - 2x'z'f_2 = 2x'z'^3 + y'^2z' \stackrel{f_1(z'^2)}{\rightarrow} y'^2z' - y'z'^3 \stackrel{f_2(y'z')}{\rightarrow} 0
\]
\[
S(f_1, f_3) = x'f_1 - 2f_3 = x'y'z'+ 2z' \stackrel{f_2(x'z')}{\rightarrow} x'z'^3 + 2z' \stackrel{f_1(\frac{1}{2}z'^2)}{\rightarrow} 2z' -\frac{1}{2}y'z'^3 \stackrel{f_2(\frac{1}{2}z'^3)}{\rightarrow} - \frac{1}{2}(z'^5 - 4z') 
\]
Далее редуцировать не можем, поэтому поставим $f_4 = z'^5 - 4z'$
\[
S(f_1, f_4) = z'^4f_1 -2x'z'f_4 =  8x'z' + y'z'^5 \stackrel{f_1(4)}{\rightarrow} y'z'^5 - 4y'z' \stackrel{f_4(y')}{\rightarrow} 0 
\]
\[
S(f_2, f_3) : \text{ т.к }gcd(L(f_2), L(f_3)) = 1 \text{ то } S(f_2, f_3) \leadsto 0
\]
\[
S(f_2, f_4): \text{ т.к }gcd(L(f_2), L(f_4)) = 1 \text{ то } S(f_2, f_4) \leadsto 0
\]
\[
S(f_3, f_4) = z'^4 f_3 - x'^2f_4  = 4x'^2z' - z'^5 \stackrel{f_3(4)}{\rightarrow} 4z' - z'^5 \stackrel{f_4}{\rightarrow} 0
\]
Получаем базис Гребнера из $(f_1, f_2, f_3, f_4)$:
\[
f_1 = 2x'z' + y'z'
\]
\[
f_2 = y' - z'^2
\]
\[
f_3 = x'^2z' - z'
\]
\[
f_4 = z'^5 - 4z'
\]
Видим, что $L(f_3) = x'^2z'$ делится на $L(f_1) = 2x'z'$, значит мы можем $f_3$ выкинуть и искать минимальный редуцированный базис Грёбнера по $(f_1, f_2, f_4)$:
\[
(2x'z' + y'z', y' -z'^2, z'^5 - 4z')
\]
\begin{center}
$f_1 -  z'\cdot f_2$:
\end{center}
\[
(2x'z'+z'^3, y'- z'^2, z'^5 - 4z')
\]
\begin{center}
\clearpage
избавляемся от коэффов:
\end{center}
\[
(x'z' + \frac{1}{2}z'^3, y' - z'^2, z'^5 - 4z')
\]
\begin{center}
теперь возвращаемся к замене:
\end{center}
\[
(yz + \frac{1}{2}y^3, x - y^2, y^5 - 4y)
\]
\begin{center}
\textbf{Ответ: } 
\[
(yz + \frac{1}{2}y^3, x - y^2, y^5 - 4y)
\]
\end{center}
\clearpage
\section*{Номер 3}
\[
I = (x^2y + 2xz + z^2, y^2z - 2z)
\]
\begin{itemize}
\item Ищем для $\mathbb{R}[x, y]$:

Делаем как на семинаре, задаем порядок $z > x > y$ и ищем базис Гребнера:
\[
S(f_1, f_2) = y^2f_1 - zf_2 = x^2y^3 + 2zxy^2 + 2z^2 \stackrel{f_1(2)}{\rightarrow} x^2y^3 + 2zxy^2 -2x^2y - 4xz \stackrel{f_2(2x)}{\rightarrow} x^2y^3 -2x^2y
\]
Далее не можем, поэтому пусть $f_3 = x^2y^3 -2x^2y$
\[
S(f_1, f_3) : \text{ т.к }gcd(L(f_1), L(f_3)) = 1 \text{ то } S(f_1, f_3) \leadsto 0
\]
\[
S(f_2, f_3) = x^2y f_2 - zf_3 = -2zx^2y + 2zx^2y = 0
\]
Все проверили, значит $(f_1, f_2, f_3)$ -- базис Грёбнера. Теперь для пересечения берем те $f$, которые зависят только от $x$ и $y$, это:
\[
f_3 = x^2y^3 -2x^2y
\]
\item Ищем для $\mathbb{R}[x, z]$:

Задаем порядок $ y > z > x$ [мне сказали, что так проще]

\[
S(f_1, f_2) = yzf_1 - x^2f_2 = 2 x^2 z + 2 x y z^2 + y z^3 \stackrel{}{\rightarrow}
\]
Дальше не можем, ставим $f_3 =2 x^2 z + 2 x y z^2 + y z^3$
\[
S(f_1, f_3) = z^3f_1 - x^2f_3 = -2 x^4 z - 2 x^3 y z^2 + 2 x z^4 + z^5 \stackrel{f_1(2xz^2)}{\rightarrow} -2 x^4 z + 4 x^2 z^3 + 4 x z^4 + z^5
\]

Дальше не можем, ставим $f_4 = -2 x^4 z + 4 x^2 z^3 + 4 x z^4 + z^5$
\[
S(f_1, f_4) : \text{ т.к }gcd(L(f_1), L(f_4)) = 1 \text{ то } S(f_1, f_4) \leadsto 0
\]
\[
S(f_2, f_3) = z^2f_2 - yf_3 = -2 x^2 y z - 2 x y^2 z^2 - 2 z^3 \stackrel{f_2(2xz)}{\rightarrow} -2 x^2 y z - 4 x z^2 - 2 z^3 \stackrel{f_1(2z)}{\rightarrow} 0 
\]
\[
S(f_2, f_4) = z^4f_2 - y^2f_4 = 2 x^4 y^2 z - 4 x^2 y^2 z^3 - 4 x y^2 z^4 - 2 z^5 \stackrel{f_2(4xz^3)}{\rightarrow} \]
\[
\stackrel{f_2(4xz^3)}{\rightarrow}
2 x^4 y^2 z - 4 x^2 y^2 z^3 - 8 x z^4 - 2 z^5 = 
\]
\[
= -4 x^4 z + 2 x^4 y^2 z + 8 x^2 z^3 - 4 x^2 y^2 z^3 \stackrel{f_2(4x^2z^2)}{\rightarrow} -4 x^4 z + 2 x^4 y^2 z \stackrel{f_2(2x^4)}{\rightarrow} 0 
\]
\[
S(f_3, f_4) = z^2f_3 - yf_4 = 2 x^4 y z + 2 x^2 z^3 - 4 x^2 y z^3 - 2 x y z^4 \stackrel{f_3(2xz)}{\rightarrow} 2 x^4 y z + 4 x^3 z^2 + 2 x^2 z^3 \stackrel{f_1(2x^2z)}{\rightarrow} 0
\]
Получили базис из $(f_1, f_2, f_3, f_4)$. Теперь для пересечения берем те f, которые зависят только от $x$ и $z$, т.е:
\[
f_4 = -2 x^4 z + 4 x^2 z^3 + 4 x z^4 + z^5
\]
[пока считал я умер, если где-то ошибься, сорри :()]
\end{itemize}
\begin{center}
\textbf{Ответ: } 

для $\mathbb{R}[x, y]:$
\[
x^2y^3 -2x^2y
\]

для $\mathbb{R}[x, z]:$
\[
 -2 x^4 z + 4 x^2 z^3 + 4 x z^4 + z^5
\]
\end{center}
\end{document}