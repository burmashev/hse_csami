\documentclass[a4paper,12pt]{article}

%%% Работа с русским языком
\usepackage{cmap}					% поиск в PDF
\usepackage{mathtext} 				% русские буквы в формулах
\usepackage[T2A]{fontenc}			% кодировка
\usepackage[utf8]{inputenc}			% кодировка исходного текста
\usepackage[english,russian]{babel}	% локализация и переносы
\usepackage{xcolor}
\usepackage{hyperref}
 % Цвета для гиперссылок
\definecolor{linkcolor}{HTML}{799B03} % цвет ссылок
\definecolor{urlcolor}{HTML}{799B03} % цвет гиперссылок

\hypersetup{pdfstartview=FitH,  linkcolor=linkcolor,urlcolor=urlcolor, colorlinks=true}

%%% Дополнительная работа с математикой
\usepackage{amsfonts,amssymb,amsthm,mathtools} % AMS
\usepackage{amsmath}
\usepackage{icomma} % "Умная" запятая: $0,2$ --- число, $0, 2$ --- перечисление

%% Номера формул
%\mathtoolsset{showonlyrefs=true} % Показывать номера только у тех формул, на которые есть \eqref{} в тексте.

%% Шрифты
\usepackage{euscript}	 % Шрифт Евклид
\usepackage{mathrsfs} % Красивый матшрифт

%% Свои команды
\DeclareMathOperator{\sgn}{\mathop{sgn}}

%% Перенос знаков в формулах (по Львовскому)
\newcommand*{\hm}[1]{#1\nobreak\discretionary{}
{\hbox{$\mathsurround=0pt #1$}}{}}
% графика
\usepackage{graphicx}
\graphicspath{{pictures/}}
\DeclareGraphicsExtensions{.pdf,.png,.jpg}
\author{Бурмашев Григорий, БПМИ-208}
\title{}
\date{\today}
\begin{document}
\begin{center}
Бурмашев Григорий. Дискра -- 20
\end{center}
\section*{Номер 1}
\subsection*{а)}
\[
\{\neg, \rightarrow \}
\]
Проверим по теореме Поста на принадлежность к классам:
\[
\begin{matrix}
x_1 & x_2 & f_1  = \neg x_1 &f_2 = x_1  \rightarrow x_ 2 \\
0 & 0 & 1 & 1 \\
0 & 1 & 1 & 1\\
1 & 0 & 0 & 0 \\
1 & 1 & 0  & 1 \\
\end{matrix}
\]
\begin{enumerate}
\item $T_1$
\[f_1 (1) = 0; f_1 \notin T_1
\]
\item $T_0$
\[
f_1(0) = 1; f_1 \notin T_0
\]
\[
f_2(0, 0) = 1; f_2 \notin T_0
\]
\item $M$

Рассмотрим наборы $(0, 1)$  и $(1, 0)$. $(0, 1) < (1, 0)$. 

При этом $f_2(0, 1) = 1 > f_2(1, 0) = 0$. Значит $f_2 \notin M$

\item $L$

В многочлене Жегалкина $f_2$ имеет вид $x_1 \odot x_1x_2$, значит $f_2 \notin L$
\item $S$
$f_2 \notin S$, т.к $f_2(0, 0) \neq \neg f_2(1, 1), 1 \neq 0$
 \end{enumerate}
\begin{center}
\textbf{Ответ: } система полная
\end{center}
\subsection*{б)}
\end{document}
