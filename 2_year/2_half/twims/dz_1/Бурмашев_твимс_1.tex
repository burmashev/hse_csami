 \documentclass[a4paper,12pt]{article}

%%% Работа с русским языком
\usepackage{cmap}					% поиск в PDF
\usepackage{mathtext} 				% русские буквы в формулах
\usepackage[T2A]{fontenc}			% кодировка
\usepackage[utf8]{inputenc}			% кодировка исходного текста
\usepackage[english,russian]{babel}	% локализация и переносы
\usepackage{xcolor}
\usepackage{hyperref}
 % Цвета для гиперссылок
\definecolor{linkcolor}{HTML}{799B03} % цвет ссылок
\definecolor{urlcolor}{HTML}{799B03} % цвет гиперссылок

\hypersetup{pdfstartview=FitH,  linkcolor=linkcolor,urlcolor=urlcolor, colorlinks=true}

%%% Дополнительная работа с математикой
\usepackage{amsfonts,amssymb,amsthm,mathtools} % AMS
\usepackage{amsmath}
\usepackage{icomma} % "Умная" запятая: $0,2$ --- число, $0, 2$ --- перечисление

%% Номера формул
%\mathtoolsset{showonlyrefs=true} % Показывать номера только у тех формул, на которые есть \eqref{} в тексте.

%% Шрифты
\usepackage{euscript}	 % Шрифт Евклид
\usepackage{mathrsfs} % Красивый матшрифт

%% Свои команды
\DeclareMathOperator{\sgn}{\mathop{sgn}}

%% Перенос знаков в формулах (по Львовскому)
\newcommand*{\hm}[1]{#1\nobreak\discretionary{}
{\hbox{$\mathsurround=0pt #1$}}{}}
% графика
\usepackage{graphicx}
\graphicspath{{pictures/}}
\DeclareGraphicsExtensions{.pdf,.png,.jpg}
\author{Бурмашев Григорий, БПМИ-208}
\title{ТВиМС, дз -- 1}
\date{\today}
\begin{document}
\maketitle 
\section*{Номер 3б)}
\[
X_n \overset{P}{ \longrightarrow } X , Y_n \overset{P}{ \longrightarrow } Y
\]
По определению:
\[
X_n \overset{P}{ \longrightarrow } X \longleftrightarrow \forall \varepsilon > 0  \lim_{n \rightarrow \infty} P \left( |X_n - X| \geq \varepsilon \right) = 0
\]
\[
Y_n \overset{P}{ \longrightarrow } Y \longleftrightarrow \forall \varepsilon > 0  \lim_{n \rightarrow \infty} P \left( |Y_n - Y| \geq \varepsilon \right) = 0
\]
А хотим мы доказать:
\[
\forall \varepsilon > 0  \lim_{n \rightarrow \infty} P \left( |(X_n + Y_n)- (X + Y) | \geq \varepsilon \right) = 0
\]
Для этого заметим:
\[
|(X_n + Y_n) - (X + Y)| = |(X_n - X) + (Y_n - Y)| \leq |X_n - X| + |Y_n - Y|  
\rightarrow
\]
\[
\rightarrow 
P(|(X_n -X) + (Y_n - Y) | \geq \varepsilon) \leq P\left[ \left(|X_n-X| \geq \frac{\varepsilon}{2} \right) \cup \left(|Y_n - Y| \geq \frac{\varepsilon}{2} \right) \right] 
\leq 
\]
\[
\leq P \left( |X_n - X| \geq \frac{\varepsilon}{2} \right) + P \left( |Y_n - Y| \geq \frac{\varepsilon}{2} \right) \overset{n \rightarrow \infty }{\longrightarrow} 0 + 0 = 0
\]
Отсюда:   
\[
P(|(X_n + Y_n) - (X + Y) | \geq \varepsilon) \overset{n \rightarrow \infty }{\longrightarrow} 0
\]
Что по определению:
\[
X_n + Y_n \overset{P}{ \longrightarrow } X + Y 
\]
\begin{center}
\textbf{Ч.Т.Д} 
\end{center}
\clearpage


\section*{Номер 10}
\[
m_n = \min \{X_1, \ldots X_n \} 
\]
Ищем распределение:
\[
F_{m_n}(t) = P(m_n \leq t) = 1 - P(m_n > t) 
\]
Теперь можем вернуться к определению $m_n$. Каждое из $X_i$ в нем имеет равномерное распределение на $[0, 1]$, а такое считать мы умеем:
\[
\forall i : F_{X_i}(t) = P(X_i \leq t) = t 
\]
Отсюда:
\[
P(X_i > t) = 1 - P(X_i \leq t) = 1 - t
\]
Мы хотим узнать $P(m_n > t)$. Т.к $m_n$ это минимум, то эта вероятность означает, что каждая из величин  $X_i$ должна быть больше $t$, отсюда просто получаем произведение вероятностей: $P(m_n > t) = (1-t)^n$.
Теперь возвращаемся:
\[
F_{m_n}(t) =  1 - P(m_n > t) =  1 - (1 - t)^n
\]
\begin{center}
\textbf{Ответ: } \[
F_{m_n}(t) = 1 - (1 - t)^n
\]
\end{center}
Распределение получили, теперь доказываем сходимость почти наверное к нулю. По определению хотим получить:
\[
P(\lim_{n \rightarrow \infty} m_n = 0) = 1?
\]
Т.к $\forall i \; : \; X_i \in [0, 1]$, то $m_n \in [0, 1]$. Тогда получаем:
\[
P(\lim_{n \rightarrow \infty} m_n = 0) = \int_0^1 \rho_{m_n} dt 
\] 
Считаем:
\[
\rho_{m_n} = (F_{m_n})' = (1 - (1 - t)^n)' = n(1 -t)^{n-1}
\]
Тогда:
\[
 \int_0^1 \rho_{m_n} dt  =  \int_0^1 n(1-t)^{n-1} = n \int_0^1 (1-t)^{n-1} = -(1-t)^n \Bigg|^1_0 = - (1 - 1)^n + (1 - 0)^n = 1
\]
Т.е:
\[
P(\lim_{n \rightarrow \infty} m_n = 0) = \int_0^1 \rho_{m_n} dt = 1
\]
\begin{center}
\textbf{Ч.Т.Д} 
\end{center}
\end{document}
