\documentclass[a4paper,12pt]{article}

%%% Работа с русским языком
\usepackage{cmap}					% поиск в PDF
\usepackage{mathtext} 				% русские буквы в формулах
\usepackage[T2A]{fontenc}			% кодировка
\usepackage[utf8]{inputenc}			% кодировка исходного текста
\usepackage[english,russian]{babel}	% локализация и переносы
\usepackage{xcolor}
\usepackage{hyperref}
 % Цвета для гиперссылок
\definecolor{linkcolor}{HTML}{799B03} % цвет ссылок
\definecolor{urlcolor}{HTML}{799B03} % цвет гиперссылок

\hypersetup{pdfstartview=FitH,  linkcolor=linkcolor,urlcolor=urlcolor, colorlinks=true}

%%% Дополнительная работа с математикой
\usepackage{amsfonts,amssymb,amsthm,mathtools} % AMS
\usepackage{amsmath}
\usepackage{icomma} % "Умная" запятая: $0,2$ --- число, $0, 2$ --- перечисление

%% Номера формул
%\mathtoolsset{showonlyrefs=true} % Показывать номера только у тех формул, на которые есть \eqref{} в тексте.

%% Шрифты
\usepackage{euscript}	 % Шрифт Евклид
\usepackage{mathrsfs} % Красивый матшрифт

%% Свои команды
\DeclareMathOperator{\sgn}{\mathop{sgn}}

%% Перенос знаков в формулах (по Львовскому)
\newcommand*{\hm}[1]{#1\nobreak\discretionary{}
{\hbox{$\mathsurround=0pt #1$}}{}}
% графика
\usepackage{graphicx}
\graphicspath{{pictures/}}
\DeclareGraphicsExtensions{.pdf,.png,.jpg}
\author{Бурмашев Григорий, БПМИ-208}
\title{Линал. ИДЗ - 2 Вариант 2.}
\date{\today}
\begin{document}
\begin{center}
Бурмашев Григорий. 208. Дискра -- 7
\end{center}
\section*{Номер 1}
Можно привести пример, когда это не выполняется:

Построим граф на 3 вершинах и соединим их так, чтобы получился треугольник.
Степень каждой вершины этого графа равна двум, при этом он не 2-раскрашиваемый (апример потому что в нем есть цикл длины 3, т.е нечетный).
\\\\
\textbf{Ответ:} нет

\section*{Номер 2}
Всего у нас 2n вершин. Граф является 2-раскрашиваемым, если длины всех циклов в нем четные. Но мы знаем, что в дереве вообще нет циклов. Следовательно дерево является 2-раскрашиваемым графом. Раскрасим наше дерево в 2 цвета. Как минимум в один из двух цветов будет покрашено n  вершин. Ибо если вершин каждого из цветов меньше n, то и всего у нас меньше 2n вершин, что противоречит условию. Т.е мы выберем n вершин одного цвета, что по определению и является независимым множеством (ни одна пара не соединена ребром) $\rightarrow$ мы сможем выбрать независимое множество из n вершин.

\section*{Номер 3}
\begin{itemize}
\item При n = 2:

Мы получаем квадрат, который невозможно расскрасить в 3 цвета. 1 из вершин квадрата будет соединена сразу с 3мя другими вершинами (2 соседа и 1 противоположная). При этом соседи тоже соединены друг с другом. Значит какой-то из трех цветов точно пересечется, т.к всего у нас 3 цвета и нам понадобится добавить 4й цвет.

\item Если n -- нечетно:

Мы сможем построить такой граф на 2 цветах. Четные вершины раскрасить в цвет A,нечетные -- в цвет B. Противоположные вершины будут разной четности и ребра соединят два разных цвета. Значит можно просто взять одну вершину и покрасить ее в цвет С. Сосед слева будет либо цвета A, либо цвета B. Аналогично с соседом справа и с противоположной вершиной.

\item Если n -- четное и не равно двум: 

Тогда возможен следующий вариант раскраски. Берем 3 цвета: A, B, C. Первые n вершин красим в цвета A, B по порядку (т.е A, B, A, B $\ldots$). n+1 вершину красим в цвет C. Вершины с n+1 по 2n-1 красим аналогично первым n вершинам (A, B, A, B, A, B, $\ldots$). Вершину 2n покрасим в C. Тогда первая вершина соединится с вершиной цвета C (n+1). n-1 вершина соединится с вершиной цвета С (2n). А оставшиеся вершины будут соединяться чередованием по аналогии с примером для нечетного n (вершина цвета A соединится с вершиной цвета B и наоборот).

\item n = 1 мы не рассматриваем, т.к в таком случае у нас будет всего 2 вершины и раскрасить их в 3 цвета с использованием всех цветов невозможно.
\end{itemize}
\textbf{Ответ:} при любых n, кроме 2 (и 1)


\section*{Номер 6}
У нас 10 чисел, значит у нас всего 10 позиций, куда мы их можем поставить. Всего у нас 9 способов поставить два числа рядом (1 и 2 позиции, 2 и 3 позиции, 3 и 4 позиции и т.д). Но мы можем менять минимум и максимум местами, т.к это два разных числа, следовательно количество способов умножается на два. После расстановки этих двух чисел у нас останется еще 8 позиций, на каждую из которых можно расставлять оставшиеся числа без повторений. Всего 8! способов.

Итого:

$8! \cdot 9 \cdot 2 = 9! \cdot 2 = 362880 \cdot 2 = 725760$
\\\\
\textbf{Ответ:} 725760

\section*{Номер 7}
С нуля число начинаться не может. А если ноль стоит не на первой позции, то цифры идут уже не в порядке возрастания. Значит у нас есть последовательность цифр:
\[
123456789
\]
Нам нужно посчитать все способы выбрать 4 цифры из этой последовательности. Из 9 цифр выбрать 4 можно:
\[
C^4_9 = \frac{9!}{4!\cdot5!} = \frac{6 \cdot 7 \cdot 8 \cdot 9}{ \cdot 2 \cdot 3 \cdot 4} = \frac{3024}{24} = 126
\]
\textbf{Ответ:} 126

\section*{Номер 8}
Чтобы четные числа шли в порядке возрастания, их нужно записать единственным способом:
\[
02468
\]
А чтобы нечетные числа шли в порядке убывания, их нужно также записать единственным способом:
\[
97531
\]
Всего у нас 10 чисел, а значит 10 позиций. Расставим наши нечетные числа: это будет число сочетаний из 10 по 5, т.е:
\[
C^5_{10} = \frac{10!}{5!\cdot5!} = 252
\]z
Ну а раз мы расставили наши нечетные числа, то для четных чисел остается всего 5 позиций, в которые они встают единственным образом (02468)
\\\\
\textbf{Ответ:} 252
\section*{Номер 9}
В n-угольнике у нас $\frac{n(n-3)}{2}$ диагоналей. Всего способов выбрать 2 различных диагонали у нас:
\[
\left( \frac{n(n-3)}{2} \cdot \left(  \frac{n(n-3)}{2} - 1 \right)  \right) : 2 = 
\]
\[
\frac{n(n-3)(n^2-3n-2)}{8}
\]
Но в условии сказано, что диагонали не должны пересекаться во внутрениих точках. Если 2 диагонали пересекаются, то их вершинки образуют четырехугольник. Значит из всех n вершин нужно удалить случаи сочетаний 4х вершинок,  т.е  $C_n^4$:
\[
C_n^4 = \frac{n!}{(n-4)! \cdot 4!} = \frac{n(n-1)(n-2)(n-3)}{4!} = \frac{n(n-1)(n-2)(n-3)}{24}
\]
Тогда итого диагоналей у нас:
\[
\frac{n(n-3)(n^2-3n-2)}{8} - \frac{n(n-1)(n-2)(n-3)}{24} =
\]
\[
= \frac{n(n-3)\left( 3(n^2-3n-2) - (n-1)(n-2) \right)}{24} = 
\]
\[
= \frac{n(n-3)2(n^2-3n-2)}{24} = \frac{n(n-3)(n^2-3n-2)}{12} 
\]
\textbf{Ответ:} $ \frac{n(n-3)(n^2-3n-2)}{12} $
\end{document}