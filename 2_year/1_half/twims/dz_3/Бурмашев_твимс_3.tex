\documentclass[a4paper,12pt]{article}

%%% Работа с русским языком
\usepackage{cmap}					% поиск в PDF
\usepackage{mathtext} 				% русские буквы в формулах
\usepackage[T2A]{fontenc}			% кодировка
\usepackage[utf8]{inputenc}			% кодировка исходного текста
\usepackage[english,russian]{babel}	% локализация и переносы
\usepackage{xcolor}
\usepackage{hyperref}
 % Цвета для гиперссылок
\definecolor{linkcolor}{HTML}{799B03} % цвет ссылок
\definecolor{urlcolor}{HTML}{799B03} % цвет гиперссылок

\hypersetup{pdfstartview=FitH,  linkcolor=linkcolor,urlcolor=urlcolor, colorlinks=true}

%%% Дополнительная работа с математикой
\usepackage{amsfonts,amssymb,amsthm,mathtools} % AMS
\usepackage{amsmath}
\usepackage{icomma} % "Умная" запятая: $0,2$ --- число, $0, 2$ --- перечисление

%% Номера формул
%\mathtoolsset{showonlyrefs=true} % Показывать номера только у тех формул, на которые есть \eqref{} в тексте.

%% Шрифты
\usepackage{euscript}	 % Шрифт Евклид
\usepackage{mathrsfs} % Красивый матшрифт

%% Свои команды
\DeclareMathOperator{\sgn}{\mathop{sgn}}

%% Перенос знаков в формулах (по Львовскому)
\newcommand*{\hm}[1]{#1\nobreak\discretionary{}
{\hbox{$\mathsurround=0pt #1$}}{}}
% графика
\usepackage{graphicx}
\graphicspath{{pictures/}}
\DeclareGraphicsExtensions{.pdf,.png,.jpg}
\author{Бурмашев Григорий, БПМИ-208}
\title{ТВиМС, дз - 3}
\date{\today}
\begin{document}
\maketitle
\clearpage
\section*{Задача 5}
\begin{center}
Пусть в -- выигрыш, п -- проигрыш (в игре) 
\end{center}
Рассмотрим оба варианта:
\begin{itemize}
\item Сильный - слабый - сильный:

P(2 выигрыша подряд) = P(пвв) + P(ввп) + P(ввв) = 
\[
= (1 - q)pq + qp(1-q) + qpq = 2pq - pq^2
\]

\item Слабый - сильный - слабый:

P(2 выигрыша подряд) =  P(пвв) + P(ввп) + P(ввв) = 
\[
(1 - p)qp + pq(1 - p) + pqp = 2pq - p^2 q
\]
\end{itemize}
Теперь выбираем, что лучше:
\[
2pq - pq^2   \; ? \; 2pq - p^2q
\]
\[
pq^2 \; ? \; p^2q
\]
Т.к $p > q$, то :$2pq - p^2q < 2pq - pq^2$
\begin{center}
\textbf{Ответ: } лучше сильный - слабый - сильный

вероятность выиграть :
\[
2pq - pq^2
\]
\end{center}
\clearpage
\section*{Задача 9}
\subsection*{N = 2:}
P(взять белый шар) = P(переложить белый и взять белый) + P(переложить черный и взять белый)
\[
P = \frac{a}{a + b} \cdot \frac{a + 1}{a + b + 1} + \frac{b}{a + b} + \frac{a}{a + b + 1 } = \frac{a^2 + ab + a}{(a +b + 1)(a + b)} = \frac{a}{a + b}
\]
\subsection*{N = 10: }
Кажется глупо и сложно решать напрямую, поэтому вероятно тут есть подвох и это можно решить через мат.индукцию (формула для $N = 2$ выполняется и для прочих $N$). База уже доказана пунктом выше, предположим, что это верно для $N$, посмотрим на $N + 1$ коробку:

Мы можем рассмотреть первые $N$ коробок как "условно" \; одну большую коробку, из которой вероятность вытянуть белый шар будет $\frac{a}{a+b}$, а вероятность вытянуть черный шар будет $\frac{b}{a + b}$, тогда рассуждения для $N + 1$ становятся аналогичными:
\[
P = \frac{a}{a + b} \cdot \frac{a + 1}{a + b + 1} + \frac{b}{a + b} + \frac{a}{a + b + 1 } = \frac{a^2 + ab + a}{(a +b + 1)(a + b)} = \frac{a}{a + b}
\]
\begin{center}
\textbf{Ответ: } $\frac{a}{a + b}$
\end{center}
\clearpage
\section*{Задача 10 }
По определению события A и B являются независимыми, если:
\[
P(A \cap B) = P(A) \cdot P(B)
\]
Также заметим:
\[
P(A \cap B) -- \text{делится и на 2, и на 5, т.е на 10}
\]
Рассмотрим для начала очевидные случаи $n$:
\begin{itemize}
\item $n \leq 4$:

При таком $n$ $P(A \cap B) = 0$ (делиться на 10 не может, т.к $\leq 4$), $P(B) = 0$ (делиться на 5 не может по аналогичной причине), отсюда получаем, что выполняется равенство для независимости, т.к:
\[
P(A \cap B) = 0 = P(A) \cdot 0 = P(A) \cdot P(B)
\]
\item 4 < n < 10:

При таком $n$ $P(A \cap B)$ все еще равна 0, но вот $P(A)$ и $P(B)$ уже точно не равны нулю, т.к при таких $n$ можно вытянуть число, которое может делится на 2 (ну или на 5), отсюда получаем, что:
\[
P(A \cap B) = 0 \neq P(A) \cdot P(B)
\]
\item $n \geq 10$:

При таких $n$ все события уже могут случиться, и надо смотреть более внимательно, мы хотим:
\[
P(A \cap B) = P(A) \cdot P(B)
\]
Посчитаем в общем случае эти вероятности:

всего у нас $n$ чисел, для $P(A)$ мы выбираем среди них только четные, пусть $r_2$ -- остаток от деления $n$ на 2, четное будет каждое 2е число тогда:
\[
P(A) = \frac{\frac{n - r_2}{2}}{n} = \frac{n - r_2}{2n}
\]

Аналогично для $P(B)$ -- пусть $r_5$ -- остаток от деления $n$ на 5, делится на 5 каждое 5е число, тогда:
\[
P(B) = \frac{\frac{n - r_5}{5}}{n} = \frac{n - r_5}{5n}
\]
И для $P(A \cap B)$ -- деление на 10, берем $r_{10}$:
\[
P(A \cap B) = \frac{\frac{n - r_{10}}{10}}{n} = \frac{n - r_{10}}{10n}
\]
\end{itemize}
Теперь подставляем это в равенство для независимых событий и упрощаем:
\[
\frac{n - r_{10}}{10n} = \frac{n - r_2}{2n} \cdot \frac{n - r_{10}}{10n}
\]
\[
n(r_2 + r_5 - r_{10}) = r_2 \cdot r_5
\]
Рассмотрим всевозможные случаи:
\begin{itemize}
\item Пусть $r_2 + r_5 - r_{10} = 0$:

Тогда $r_2 \cdot r_5 = 0$, если $r_5 = 0$, тогда получаем, что $n$ кратно 5, подставляем $r_5 = 0$ и получаем, что $r_2 = r_{10}$, если $r_2 = 1$, то этот случай невозможен, т.к $n$ уже кратно 5 и $r_{10}$  либо 0, либо 5, отсюда получаем, что $r_2 = 0 = r_{10}$ и $n$ -- кратно 10. 

Если же $r_2 = 0$, тогда $r_5 = r_{10}$, и при этом n -- четно. Помимо только кратных 10 нам подходят еще числа, которые при делении на 10 дают остаток 2 или 4.


Итого 2 варианта : числа, кратные 10 и числа вида $10a + b$, где $a$ -- любое натуральное, $b \in \{2, 4\}$, можно объединить эти два случая в один вида $10a + c$, где $c \in \{0, 2, 4\}$
\item Пусть $r_2 + r_5 - r_{10} \neq 0$:

Тогда $n = \frac{r_2 \cdot r_5}{r_2 + r_5 - r_{10}}$, если $r_2 = 0$, тогда $n = 0$, у нас такого быть не может (множество начинается с 1), если же $r_2 \neq 0$, тогда $r_2 = 1$ (других остатков по модулю 2 нет), тогда:
\[
n = \frac{r_5}{1 + r_5 - r_{10}}
\]
Отчетливо видно, что $r_5 < 4$, тогда  точно $n \leq 4$, а этот случай мы уже рассмотрели выше
\end{itemize}
\begin{center}
Все случаи разобраны, получаем ответ:
\end{center}
\begin{center}
\textbf{Ответ: } $n \leq 4$,  $n = 10a + b$, где $a$ -- натуральное , $b \in \{0, 2, 4\}$
\end{center}
\end{document}
