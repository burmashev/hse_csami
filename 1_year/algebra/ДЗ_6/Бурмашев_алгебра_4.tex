\documentclass[a4paper,12pt]{article}

\usepackage[left=4cm,right=4cm,
    top=4cm,bottom=4cm,bindingoffset=0cm]{geometry}
%%% Работа с русским языком
\usepackage{cmap}					% поиск в PDF
\usepackage{mathtext} 				% русские буквы в формулах
\usepackage[T2A]{fontenc}			% кодировка
\usepackage[utf8]{inputenc}			% кодировка исходного текста
\usepackage[english,russian]{babel}	% локализация и переносы
\usepackage{xcolor}
\usepackage{hyperref}
 % Цвета для гиперссылок
\definecolor{linkcolor}{HTML}{799B03} % цвет ссылок
\definecolor{urlcolor}{HTML}{799B03} % цвет гиперссылок

\hypersetup{pdfstartview=FitH,  linkcolor=linkcolor,urlcolor=urlcolor, colorlinks=true}

%%% Дополнительная работа с математикой
\usepackage{amsfonts,amssymb,amsthm,mathtools} % AMS
\usepackage{amsmath}
\usepackage{icomma} % "Умная" запятая: $0,2$ --- число, $0, 2$ --- перечисление

%% Номера формул
%\mathtoolsset{showonlyrefs=true} % Показывать номера только у тех формул, на которые есть \eqref{} в тексте.

%% Шрифты
\usepackage{euscript}	 % Шрифт Евклид
\usepackage{mathrsfs} % Красивый матшрифт

%% Свои команды
\DeclareMathOperator{\sgn}{\mathop{sgn}}

%% Перенос знаков в формулах (по Львовскому)
\newcommand*{\hm}[1]{#1\nobreak\discretionary{}
{\hbox{$\mathsurround=0pt #1$}}{}}
% графика
\usepackage{graphicx}
\graphicspath{{pictures/}}
\DeclareGraphicsExtensions{.pdf,.png,.jpg}
\author{Бурмашев Григорий, БПМИ-208}
\title{}
\date{\today}
\begin{document}
\begin{center}
Бурмашев Григорий. Алгебра. Дз -- 6
\end{center}
\section*{Номер 1}
\begin{center}
Начинаем с $x_1^3x_2^2x_3$ и заканчиваем $x_1^3x_2x_3^2$. 
\end{center}
\begin{center}
Пусть A$ = x_1^3x_2^2x_3$, B $ = x_1^3x_2x_3^2$ для удобства,
т.е A --  начало, B -- конец
\end{center}

Во -- первых $x_1$  роли не играет, т.к у A и B степени при нем одинаковые, нас интересуют степени при $x_2$ и $x_3$. Можем сразу построить цепочку длины 2 из исходных одночленов, т.к степень при $x_2$ у A больше, а именно:
\[
x_1^3x_2^2x_3 \succ x_1^3x_2x_3^2
\]
Теперь, чтобы удлинять цепочку, нужно впихивать одночлены посередине. Менять степень при $x_1$ нельзя из написанного выше. Вставить одночлен вида $x_1^3x_2^2x_3^m, m > 1 $ тоже не получится, т.к у A степень при $x_3$ равна 1, а значит повышать ее с сохранением степени при $x_2$ не получится. Мы можем вставить одночлен вида $x_1^3x_2^2$, что позволит нам удлинить цепочку до 3 элементов. Остаются многочлены вида $x_1^3x_2x_3^n$. Значит мы сможем для любого сколь угодно большого $n \;(n > 2)$ построить цепочку следующего вида:
\[
x_1^3x_2^2x_3 \succ x_1^3x_2^2 \succ x_1^3x_2x_3^n \succ x_1^3x_2x_3^{n-1} \succ \ldots \succ x_1^3x_2x_3^3 \succ  x_1^3x_2x_3^2
\]
А значит минимальная длина цепочки будет 2, а максимальная будет соотвественно $n - 2 + 2$ = $n$. (на самом деле без разницы, как называть, суть в том, что это получится сделать для любого сколь угодно большого числа, при этом условие про невозможность построения бесконечной цепи нарушаться не будет, т.к число конечномерное)
\begin{center}
\textbf{Ответ: } длины от 2 до $n$ (заданного выше)
\end{center}
\clearpage
\section*{Номер 2}
\[
g = x_2^4x_3^5 + 2x_1x_2^4x_3 + x_1^2x_2^2
\]
\[
f = x_2^4x_3 - 2x_1x_2x_3^2 + x_1x_2^2
\]
Собственно считаем стандартным алгоритмом с лекции:
\[
L(f) =  x_1x_2^2
\]
\[
L(g) = x_1^2x_2^2
\]
Первый шаг:

\[
g \overset{f}{\rightarrow} g - x_1 \cdot f = x_2^4x_3^5 + 2x_1^2x_2x_3^2 + x_1x_2^4x_3 = g_1
\]
\[
L(g_1) = 2x_1^2x_2x_3^2
\]
\\\\
Второй шаг:

по $L(g_1)$ редуцировать не можем, тогда по $x_1x_2^4x_3$ ($-x_2^2x_3 \cdot f$)
\[
x_2^4x_3^5 + 2x_1^2x_2x_3^2 + x_1x_2^4x_3 \overset{f}{\rightarrow} -x_2^6x_3^2 + x_2^4x_3^5  + 2x_1x_2^3x_3^3 + 2x_1^2x_2x_3^2 = g_2
\]
\[
L(g_2) = 2x_1^2x_2x_3^2
\]
\\\\
Третий шаг:

тоже не можем, тогда редуцируем по $2x_1x_2^3x_3^3$ ($-2x_2x_3^3 \cdot f$ )
\[
-x_2^6x_3^2 + x_2^4x_3^5  + 2x_1x_2^3x_3^3 + 2x_1^2x_2x_3^2  \overset{f}{\rightarrow} -x_2^6x_3^2 - 2x_2^5x_3^4 + x_2^4x_3^5+ 2x_1^2x_2x_3^2 + 4x_1x_2^2x_3^5 = g_3
\]
\clearpage
\noindentЧетвертый шаг:

редуцируем по $4x_1x_2^2x_3^5$ ($-4x_3^5 \cdot f$)
\[
-x_2^6x_3^2 - 2x_2^5x_3^4 + x_2^4x_3^5+ 2x_1^2x_2x_3^2 + 4x_1x_2^2x_3^5 
\overset{f}{\rightarrow}
8x_1x_2x_3^7 - 4x_2^4x_3^6 + x_2^4x_3^5 - 2x_2^5x_3^4 +
\]
\[
+ 2x_1^2x_2x_3^2 - x_2^6x_3^2 = g_4
\]

\[
L(g_4) = 2x_1^2x_2x_3^2 
\]
По $L(g_4)$ не можем редуцировать, но и по всем остальным одночленам тоже (нет одночленов с $x_1$ и $x_2^2$), а значит это и есть остаток $g$ относительно системы $\{f\}$.
\begin{center}
\textbf{Ответ: } 
\[
8x_1x_2x_3^7 - 4x_2^4x_3^6 + x_2^4x_3^5 - 2x_2^5x_3^4 + 2x_1^2x_2x_3^2 - x_2^6x_3^2
\]
\end{center}
\clearpage
\section*{Номер 3}
\begin{center}
Будем доказывать по критерию Бухбергера.
\end{center}

Считаем S полиномы от всех пар $f$ и пытаемся их $\leadsto 0$
\begin{itemize}
\item $S(f_1, f_2)$ : lcm$(f_1, f_2) = 4x_1x_2x_3^2$ 
\[
S(f_1, f_2) = 2x_3^2 \cdot f_1- x_2 \cdot f_2 = 4x_2  + 2x_2x_3^4 - x_2^2x_3^3 + 8x_1x_3^3 
\]
\[
S(f_1, f_2) \overset{f_2 (2x_3)}{\rightarrow} -x_2^2x_3^3 + 4x_2 + 8x_3 \overset{f_3 (1)}{\rightarrow} 0
\]
\begin{center}
Выполняется
\end{center}
\item $S(f_2, f_3)$:  lcm$(f_2, f_3) = 4x_1x_2^2x_3^3$ 
\[
S(f_2, f_3) = x_2^2x_3 \cdot f_2 - 4x_1 \cdot f_3 = x_2^3x_3^4 - 4x_2^2x_3 + 32x_1x_3 +16 x_1x_2
\]
\[
S(f_2, f_3) \overset{f_1 (8)}{\rightarrow} x_2^3x_3^4 - 4x_1^2 x_3 - 8x_2x_3^2 \overset{f_3 (x_2x_3)}{\rightarrow} 0
\]
\begin{center}
Выполняется
\end{center}
\item $S(f_1, f_3)$:  lcm$(f_1, f_3) = 2x_1x_2^2x_3^3$ 
\[
S(f_1, f_3) = x_2x_3^3 \cdot f_1 - 2x_1 \cdot f_3 = x_2^2x_3^5 +4x_1x_2x_3^4 + 16x_1x_3 + 8x_1x_2
\]
\[
S(f_2, f_3) \overset{f_2 (x_2x_3^2)}{\rightarrow} 8x_1x_2 + 16x_1x_3 + 4x_2x_3^2 \overset{f_1 (4)}{\rightarrow} 0 
\]
\begin{center}
Выполняется
\end{center}
\end{itemize}
А значит выполнен критерий Бухбергера и множество $\{f_1, f_2, f_3\}$ является системой Грёбнера
\begin{center}
\textbf{Ответ: } да, является
\end{center}
\end{document}
