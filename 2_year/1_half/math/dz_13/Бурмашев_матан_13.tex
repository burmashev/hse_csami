\documentclass[a4paper,12pt]{article}

%%% Работа с русским языком
\usepackage{cmap}					% поиск в PDF
\usepackage{mathtext} 				% русские буквы в формулах
\usepackage[T2A]{fontenc}			% кодировка
\usepackage[utf8]{inputenc}			% кодировка исходного текста
\usepackage[english,russian]{babel}	% локализация и переносы
\usepackage{xcolor}
\usepackage{hyperref}
 % Цвета для гиперссылок
\definecolor{linkcolor}{HTML}{799B03} % цвет ссылок
\definecolor{urlcolor}{HTML}{799B03} % цвет гиперссылок

\hypersetup{pdfstartview=FitH,  linkcolor=linkcolor,urlcolor=urlcolor, colorlinks=true}

%%% Дополнительная работа с математикой
\usepackage{amsfonts,amssymb,amsthm,mathtools} % AMS
\usepackage{amsmath}
\usepackage{icomma} % "Умная" запятая: $0,2$ --- число, $0, 2$ --- перечисление

%% Номера формул
%\mathtoolsset{showonlyrefs=true} % Показывать номера только у тех формул, на которые есть \eqref{} в тексте.

%% Шрифты
\usepackage{euscript}	 % Шрифт Евклид
\usepackage{mathrsfs} % Красивый матшрифт

%% Свои команды
\DeclareMathOperator{\sgn}{\mathop{sgn}}

%% Перенос знаков в формулах (по Львовскому)
\newcommand*{\hm}[1]{#1\nobreak\discretionary{}
{\hbox{$\mathsurround=0pt #1$}}{}}
% графика
\usepackage{graphicx}
\graphicspath{{pictures/}}
\DeclareGraphicsExtensions{.pdf,.png,.jpg}
\author{Бурмашев Григорий, БПМИ-208}
\title{Матан, дз -- 13}
\date{\today}
\begin{document}
\maketitle
\section*{Номер 1}
\[
\iint\limits_{x^2 + y^2 \geq 1}  \frac{dxdy}{(x^2+y^2)^p}, \; p > 0
\]
Номер почти как 1й с семинаров, функция все также неотрицательная, а значит предел не зависит от выбора $D_n$. Проблема теперь не с точкой (0, 0), а с бесконечностью. Тогда для исчерпания (чтобы прижиматься к бесконечности) кладем $D_n$ как:
\[
D_n : 1 \leq x^2 + y^2 \leq n^2
\]
Теперь переходим к пределу:
\[
\iint\limits_{x^2 + y^2 \geq 1}  \frac{dxdy}{(x^2+y^2)^p} = \lim\limits_{n \rightarrow \infty} \iint\limits_{1 \leq x^2 + y^2 \leq n^2} \frac{dxdy}{(x^2+y^2)^p} = 
\]
Для удобства переходим в полярные координаты:
\[
\begin{cases}
x = r \cos \varphi \\
y = r \sin \varphi 
\end{cases}
\]
\[
=\lim\limits_{n \rightarrow \infty} \int_{1}^{2 \pi} d \varphi \int_1^{\sqrt{n^2}} \frac{ 1}{(r^2 \cos^2 \varphi + r^2 \sin^2 \varphi)^p} rdr = \lim\limits_{n \rightarrow \infty} \int_{1}^{2 \pi} d \varphi \int_1^n \frac{ 1}{(r^2 \cdot 1)^p} rdr  = 
\]
\[
= 
\lim\limits_{n \rightarrow \infty} \int_{1}^{2 \pi} d \varphi \int_1^n \frac{r}{r^{2p}} dr = 2 \pi  \cdot \lim_{n \rightarrow \infty} \int_1^{n} \frac{1}{r^{2p - 1}} dr = 2 \pi \int_1^{\infty} \frac{1}{r^{2p - 1}} dr
\]
Привели к понятному виду, $\frac{1}{r^{2p-1}}$ сходится при степени больше 1, т.е $2p - 1 > 1 \rightarrow $ $p > 1$, теперь вычислим интеграл:
\[
2 \pi \frac{r^{2-2n}}{2-2p} \Bigg|^{\infty}_{1} = \pi \cdot \frac{1}{p - 1}, \; p > 1
\]
Ну а при $p \leq 1$ имеем расходимость 
\clearpage
\section*{Номер 2}
\[
\iiint\limits_{D} \frac{dxdydz}{(x^2+y^2+z^2)^{\frac43}}, \; D : x^2 + y^2 \leq z^2, \; 0 \leq z \leq 1
\]
Функция неотрицательная, проблема здесь при $z = 0$, потому что в знаменателе получается ноль. Тогда введем $D_n$:
\[
D_n : D \cap \left\{(x, y, z) \Big| \frac{1}{n} \leq z \leq 1 \right\}
\]
Теперь переходим к пределу:
\[
\iiint\limits_{D} \frac{dxdydz}{(x^2+y^2+z^2)^{\frac43}} = \lim_{n \rightarrow \infty}  \iiint\limits_{D_n} \frac{dxdydz}{(x^2 + y^2+z^2)^{\frac43 }}
\]
Теперь можем перейти к цилиндрическим координатам:
\[
\begin{cases}
x = r \cos \varphi \\
y = r \sin \varphi \\
 z = h \\ 
r > 0,
\varphi \in [0, 2 \pi)
\end{cases}
\]
\[
D : r^2 \leq h^2 , 0 \leq h \leq 1
\]
\[
 \lim_{n \rightarrow \infty}  \iiint\limits_{D_n} \frac{dxdydz}{(x^2 + y^2+z^2)^{\frac43 }}= \lim_{n \rightarrow \infty} \int_0^{2\pi} d \varphi  \int_{\frac1n}^{1} dh \int_0^h \frac{1}{\left(r^2 + h^2\right)^{\frac43}} rdr  = 
\]
\[
= 2 \pi \lim_{n \rightarrow \infty} \int_{\frac1n}^{1} dh \int_0^h \frac{1}{\left(r^2 + h^2\right)^{\frac43}} rdr  = (\times)
\]
Посчитаем отдельно:
\[
\int \frac{1}{(r^2+h^2)^{\frac{4}{3}}} rdr = \begin{vmatrix}
u = r^2 + h^2 \\
du = 2r dr  \\
dr = \frac{du}{2r}
\end{vmatrix} = \int  \frac{r}{2 \cdot r \cdot u^{\frac43}} du  = \frac12 \int u^{-\frac43} du =  \frac{1}{2} \cdot \left(-\frac{3}{\sqrt[3]{u}}\right) = 
\]
\[
=
-\frac{3}{2 \sqrt[3]{u^2}} =-\frac{3}{2 \sqrt[3]{r^2 + h^2}}
\] 
\[
-\frac{3}{2 \sqrt[3]{r^2 + h^2}}\Bigg|^h_0 = -\frac{3}{2\sqrt[3]{2h^2}} - \left(- \frac{3}{2\sqrt[3]{h^2}}\right) = \frac{-3 + 3 \sqrt[3]{2}}{2 \sqrt[3]{2h^2} }
\]
Берем еще раз интеграл:
\[
\int \frac{-3 + 3 \sqrt[3]{2}}{2 \sqrt[3]{2h^2} } dh= -\frac12 \int \frac{3(1 - \sqrt[3]{2})}{\sqrt[3]{2} \cdot \sqrt[3]{h^2}}dh = -\frac{3(1 - \sqrt[3]{2})}{2\sqrt[3]{2}} \int \frac{1}{ \sqrt[3]{h^2}}dh  = -\frac{3(1- \sqrt[3]{2})}{2\sqrt[3]{2}} \cdot 3 \sqrt[3]{h} = 
 \]
\[
=
-\frac{9(1 - \sqrt[3]{2})}{2\sqrt[3]{2}} \cdot  \sqrt[3]{h}  = -\frac{9(1 - \sqrt[3]{2})}{2\sqrt[3]{2}}\cdot \frac{\sqrt[3]{4}}{\sqrt[3]{4}} \cdot  \sqrt[3]{h}  = -\frac{9\sqrt[3]{4} - 18}{4} \sqrt[3]{h}
\]
\[
-\frac{9\sqrt[3]{4} - 18}{4} \sqrt[3]{h} \Bigg|^1_{\frac1n} = -\frac{9\sqrt[3]{4} - 18}{4} + \frac{9\sqrt[3]{4} - 18}{4} \sqrt[3]{\frac{1}{n}} = \frac{9\sqrt[3]{4} - 18}{4} \left(\sqrt[3]{\frac1n} - 1 \right)
\]
 Теперь возвращаемся к пределу:
\[
(\times) = 2 \pi \lim_{n \rightarrow \infty} \frac{9\sqrt[3]{4} - 18}{4} \left(\sqrt[3]{\frac1n} - 1 \right) = 2 \pi \frac{9\sqrt[3]{4} - 18}{4} \lim_{n \rightarrow \infty} \left(\sqrt[3]{\frac1n} - 1 \right) = -2 \pi \frac{9\sqrt[3]{4} - 18}{4}=
\]
\[
=
-\pi \cdot \frac{9(\sqrt[3]{4} - 2)}{2}
\]
\begin{center}
\textbf{Ответ: } 
\[
-\pi \cdot \frac{9(\sqrt[3]{4} - 2)}{2}
\]
\end{center}
\end{document}
