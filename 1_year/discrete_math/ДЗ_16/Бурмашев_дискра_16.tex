Ч%\documentclass[a4paper,12pt]{article}

%%% Работа с русским языком
\usepackage{cmap}					% поиск в PDF
\usepackage{mathtext} 				% русские буквы в формулах
\usepackage[T2A]{fontenc}			% кодировка
\usepackage[utf8]{inputenc}			% кодировка исходного текста
\usepackage[english,russian]{babel}	% локализация и переносы
\usepackage{xcolor}
\usepackage{hyperref}
 % Цвета для гиперссылок
\definecolor{linkcolor}{HTML}{799B03} % цвет ссылок
\definecolor{urlcolor}{HTML}{799B03} % цвет гиперссылок

\hypersetup{pdfstartview=FitH,  linkcolor=linkcolor,urlcolor=urlcolor, colorlinks=true}

%%% Дополнительная работа с математикой
\usepackage{amsfonts,amssymb,amsthm,mathtools} % AMS
\usepackage{amsmath}
\usepackage{icomma} % "Умная" запятая: $0,2$ --- число, $0, 2$ --- перечисление

%% Номера формул
%\mathtoolsset{showonlyrefs=true} % Показывать номера только у тех формул, на которые есть \eqref{} в тексте.

%% Шрифты
\usepackage{euscript}	 % Шрифт Евклид
\usepackage{mathrsfs} % Красивый матшрифт

%% Свои команды
\DeclareMathOperator{\sgn}{\mathop{sgn}}

%% Перенос знаков в формулах (по Львовскому)
\newcommand*{\hm}[1]{#1\nobreak\discretionary{}
{\hbox{$\mathsurround=0pt #1$}}{}}
% графика
\usepackage{graphicx}
\graphicspath{{pictures/}}
\DeclareGraphicsExtensions{.pdf,.png,.jpg}
\author{Бурмашев Григорий, БПМИ-208}
\title{}
\date{\today}
\begin{document}
\begin{center}
Бурмашев Григорий, 208, Дискра -- 16
\end{center}
\section*{Номер 1}
\begin{equation*}
\begin{gathered}
\Omega = \{x : 0 < x < 101\} \\
|\Omega| = 100 \\
A = \{x : 0 < x < 101, \;x \vdots 3\} \\
|A| = 33 \\
B = \{x : 0 < x < 101, \; x \vdots 7\} \\
|B| = 14\\
A \cap B = \{x : 0 < x < 101, \; (x \vdots 3) \cap (x \vdots 7)\} \\
|A \cap B| = 4 \\
P(A|B) = \frac{P(A\cap B)}{P(B)} = \frac{\frac{4}{100}}{\frac{14}{100}} = \frac{2}{7}
\end{gathered}
\end{equation*}
\begin{center}
\textbf{Ответ: } $\frac{2}{7}$
\end{center}
\section*{Номер 2}
\begin{equation*}
\begin{gathered}
\Omega = \{ f : \{1 \ldots n\} \rightarrow \{1 \ldots n\}\} \\
|\Omega| = n^n \\
A = \{f : f(1) = 1\} \\
|A| = n^{n-1} \\
B = \{f : f \text{ является иньекцией}\} \\
|B| = n! \\
A \cap B = \{f : (f(1) = 1) \cap (f \text{ является иньекцией})\} \\
|A \cap B| = (n-1)!\\
P(A|B) = \frac{P(A\cap B)}{P(B)} = \frac{(n-1)! \cdot n^n}{n^n \cdot n!} = \frac{1}{n} \\
P(A) = \frac{n^{n-1}}{n^n} = \frac{1}{n} \\
P(A) = P(A|B) \rightarrow \text{ А и B независмы }
\end{gathered}
\end{equation*}
\section*{Номер 3}
\begin{equation*}
\begin{gathered}
A = \{\text{вероятность вытащить карту из 1 колоды}\} \\
B = \{\text{вероятность вытащить 8 червей либо из 1, либо из 2 колоды}\}\\
A \cap B = \{ \text{вытащить 8 червей из 1 колоды} \} \\
P(B) = \frac{2}{36+35}\\
P(A \cap B) = \frac{1}{71}\\
P(A|B) = \frac{P(A\cap B)}{P(B)} = \frac{\frac{1}{71}}{\frac{2}{71}} = \frac{1}{2}
\end{gathered}
\end{equation*}
\begin{center}
\textbf{Ответ: } $\frac{1}{2}$
\end{center}
\section*{Номер 4}
\begin{equation*}
\begin{gathered}
\frac{1}{10} - \text{ вероятность наличия ошибки в задаче} \\
\frac{4}{5} - \text{ вероятность правильного ответа ассистента} \\
\frac{3}{4} - \text{ вероятность правильного ответа лектора}\\
A  = \{\text{в задаче есть ошибка}\} \\
B = \{\text{по ассистенту ошибки нет, по лектору ошибка есть}\} \\
A \cap B = \{\text{в задаче  точно есть ошибка, при этом ассистент ошибся, а лектор был прав}\}
\end{gathered} 
\end{equation*}
Найдем вероятность B: 

если ошибка действительно есть, тогда ассистент был прав, а лектор ошибься. Если же ошибки не было, то ассистат ошибся, а лектор был прав, тогда вероятность:
\begin{equation*}
\begin{gathered}
P(B) = \frac{1}{10} \cdot (1 - \frac{4}{5}) \cdot \frac{3}{4} + \frac{9}{10} \cdot \frac{4}{5} \cdot ( 1 - \frac{3}{4}) = \frac{3}{200} + \frac{36}{100} = \frac{39}{100}
\end{gathered}
\end{equation*}
Тогда найдем вероятность $A \cap B$:
\begin{equation*}
\begin{gathered}
P(A \cap B) = \frac{1}{10} \cdot \frac{1}{5} \cdot \frac{3}{4}  = \frac{3}{200}
\end{gathered}
\end{equation*}
Тогда:
\[
P(A|B) = \frac{P(A \cap B)}{P(B)} = \frac{\frac{3}{200}}{\frac{39}{200}} = \frac{1}{13}
\]
\begin{center}
\textbf{Ответ: } $\frac{1}{13}$
\end{center}















\section*{Номер 5}
Для проверки независимости нужно посмотреть на вероятность:
\[
P(A|B \cap C) = \frac{P(A \cap B \cap C)}{P(B \cap C)}
\]
Из попарной независимости событий следует:
\[
P(A \cap B) = P(A) \cdot P(B)
\]
\[
P(B \cap C) = P(B) \cdot P(C)
\]
\[
P(A \cap C) = P(A) \cdot P(C)
\]
\[
P(A\cap B\cap C) = P(A) \cdot P(B) \cdot P(C)
\]
\[
P(A|(B \cap C)) = \frac{P(A \cap B \cap C)}{P(B \cap C)} = \frac{Pr(A) \cdot P(B) \cdot P(C)}{P(B) \cdot P(C)} = P(A)
\]
Значит события A и $B \cap C$ независимы
\section*{Номер 6}
Пускай:
\begin{equation*}
\begin{gathered}
\Omega = \{1, 2, 3, 4, 5, 6, 7, 8\} \\ 
|\Omega| = 8 \\
A  = \{\text{выбрать } (1, 6, 7, 8) \text{ из } \Omega \} \\
B = \{\text{выбрать } (1, 3, 7, 8) \text{ из } \Omega \} \\
C =  \{\text{выбрать } (1, 2, 4, 5) \text{ из } \Omega \} \\
P(A) = \frac{4}{8} = \frac{1}{2} \\
P(B) = \frac{4}{8} = \frac{1}{2}  \\
P(C) = \frac{4}{8} = \frac{1}{2}  \\
P(A) \cdot P(B) \cdot P(C) = \frac{1}{2^3} = \frac{1}{8} \\
A \cap B \cap C = \{\text{выбрать } (1) \text{ из } \Omega \} \\
P(A \cap B \cap C) = \frac{1}{8} \\
P(A \cap B \cap C) = P(A) \cdot P(B) \cdot P(C) = \frac{1}{8}
\end{gathered}
\end{equation*}
\begin{center}
\textbf{Ч.Т.Д}
\end{center}
\end{document} 
