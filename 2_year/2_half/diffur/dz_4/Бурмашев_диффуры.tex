\documentclass[a4paper,12pt]{article}

%%% Работа с русским языком
\usepackage{cmap}					% поиск в PDF
\usepackage{mathtext} 				% русские буквы в формулах
\usepackage[T2A]{fontenc}			% кодировка
\usepackage[utf8]{inputenc}			% кодировка исходного текста
\usepackage[english,russian]{babel}	% локализация и переносы
\usepackage{xcolor}
\usepackage{hyperref}
 % Цвета для гиперссылок
\definecolor{linkcolor}{HTML}{799B03} % цвет ссылок
\definecolor{urlcolor}{HTML}{799B03} % цвет гиперссылок

\hypersetup{pdfstartview=FitH,  linkcolor=linkcolor,urlcolor=urlcolor, colorlinks=true}

%%% Дополнительная работа с математикой
\usepackage{amsfonts,amssymb,amsthm,mathtools} % AMS
\usepackage{amsmath}
\usepackage{icomma} % "Умная" запятая: $0,2$ --- число, $0, 2$ --- перечисление

%% Номера формул
%\mathtoolsset{showonlyrefs=true} % Показывать номера только у тех формул, на которые есть \eqref{} в тексте.

%% Шрифты
\usepackage{euscript}	 % Шрифт Евклид
\usepackage{mathrsfs} % Красивый матшрифт

%% Свои команды
\DeclareMathOperator{\sgn}{\mathop{sgn}}

%% Перенос знаков в формулах (по Львовскому)
\newcommand*{\hm}[1]{#1\nobreak\discretionary{}
{\hbox{$\mathsurround=0pt #1$}}{}}
% графика
\usepackage{graphicx}
\graphicspath{{pictures/}}
\DeclareGraphicsExtensions{.pdf,.png,.jpg}
\author{Бурмашев Григорий, БПМИ-208}
\title{Диффуры, дз -- 4}
\date{\today}
\begin{document}
\maketitle
\clearpage
\section*{Номер 1}
\subsection*{a)}
\[
\begin{cases}
x' = 2x + y - 2z \\
y' = -x + z \\
z' = 2x + 2y - z
\end{cases}
\]
Ищем собственные векторы:
\[
\begin{vmatrix}
2 - \lambda & 1 & -2 \\
-1 & 0 - \lambda & 1 \\
2 & 2 & -1 - \lambda 
\end{vmatrix}
=
-\lambda^3 + \lambda^2 - \lambda + 1 = -(\lambda^2(\lambda - 1) + \lambda - 1) = -(\lambda - 1)(\lambda^2 + 1)
\]
Получаем корни (все кратности 1):
\[
\lambda_1 = 1, \lambda_2 = i, \lambda_3 = -i
\] 
Получили парное $\pm i$. По аналогии с семинаром, можем рассматривать только один из них, оставим i. Подставляем $\lambda_1$:
\[
\begin{pmatrix}
1& 1 & -2 \\
-1 & -1 & 1 \\
2 & 2 & -2
\end{pmatrix}
\cdot
\begin{pmatrix}
x \\
y \\ 
z
\end{pmatrix}
=
\begin{pmatrix}
0 \\
0 \\ 
0 \\
\end{pmatrix}
\]
\[
\begin{pmatrix}
1 & 1 & -2 & \vrule & 0 \\
-1 & -1 & 1 & \vrule & 0 \\
2 & 2 & -2 & \vrule & 0 \\
\end{pmatrix}
\sim 
\begin{pmatrix}
1 & 1 & 0 & \vrule & 0 \\
0 & 0 & -2 & \vrule & 0 \\
0 & 0 & 0 & \vrule & 0 \\
\end{pmatrix} 
\]
\[
\begin{cases}
x = - y\\
z = 0 \\
\end{cases}
\]
Тогда берем вектор:
\[
h_1 = 
\begin{pmatrix}
-1 \\
1 \\
0
\end{pmatrix}
\]
Подставляем $\lambda_2$:
\[
\begin{pmatrix}
2 - i& 1 & -2 \\
-1 & -i& 1 \\
2 & 2 & -1 - i
\end{pmatrix}
\cdot
\begin{pmatrix}
x \\
y \\ 
z
\end{pmatrix}
=
\begin{pmatrix}
0 \\
0 \\ 
0 \\
\end{pmatrix}
\]
\[
\begin{pmatrix}
2 - i& 1 & -2 & \vrule & 0  \\
-1 & -i& 1 & \vrule & 0 \\
2 & 2 & -1 - i & \vrule & 0
\end{pmatrix}
\sim
\begin{pmatrix}
1 & 0& -1 - \frac{i}{2} & \vrule & 0  \\
0 & 1& \frac{1}{2}& \vrule & 0 \\
0 & 0 & 0 & \vrule & 0
\end{pmatrix}
\]
\[
\begin{cases}
x + z(-1 - \frac{i}{2}) = 0 \\
y = - \frac{z}{2}
\end{cases}
\]
Тогда берем вектор:
\[
h_2 = 
\begin{pmatrix}
1 + \frac{i}{2} \\
- \frac{1}{2} \\
1
\end{pmatrix}
\]
Поработаем с комплексными корнями:
\[
x_{h_2} = \left(1 + \frac{i}{2}\right) \cdot e^{it}= e^{it} + \frac{i}{2}e^{it} = \cos t + i \sin t - \frac{\sin t}{2} + \frac{i}{2} \cos t
\]
\[
y_{h_2} = -\frac{1}{2} e^{it} = - \frac{\cos t}{2} - \frac{i}{2} \sin t
\]
\[
z_{h_2} = \cos t + i \sin t
\]
Итого:
\[
x = -1 \cdot C_1  \cdot e^{t} + C_2 \cdot \left(\cos t - \frac{\sin t}{2}\right) + C_3 \cdot \left(
\sin t + \frac{\cos t}{2}
\right)
\]
\[
y = 1 \cdot C_1 \cdot e^t + C_2 \cdot \left(
-\frac{\cos t}{2}
\right) 
+ C_3 \cdot \left(
-\frac{1}{2} \sin t
\right) 
\]
\[
z = 0 \cdot C_1 \cdot e^t + C_2 \cdot \left(
\cos t
\right) 
+ C_3 \cdot \left(
\sin t
\right) 
\]
\begin{center}
\textbf{Ответ: } 
\[
x = -1 \cdot C_1  \cdot e^{t} + C_2 \cdot \left(\cos t - \frac{\sin t}{2}\right) + C_3 \cdot \left(
\sin t + \frac{\cos t}{2}
\right)
\]
\[
y = 1 \cdot C_1 \cdot e^t + C_2 \cdot \left(
-\frac{\cos t}{2}
\right) 
+ C_3 \cdot \left(
-\frac{1}{2} \sin t
\right)
\]
\[
z = C_2 \cdot \left(
\cos t
\right) 
+ C_3 \cdot \left(
\sin t
\right)
\]
\end{center}
\clearpage
\subsection*{b)}
\[
\begin{cases}
x' = x - y - 4z \\
y' = -2x + 2y + 12z \\
z' = x - y - 5z
\end{cases}
\]
Ищем:
\[
\begin{vmatrix}
1 - \lambda & -1 & -4 \\
-2  & 2 -\lambda & 12 \\
1 & -1 & - 5 - \lambda\\
\end{vmatrix} = - \lambda^3 -2 \lambda^2 - \lambda = -\lambda \cdot (\lambda + 1)^2
\]
Получаем $\lambda_1 = 0$ кратности 1, $\lambda_2 = -1$ кратности 2. Считаем для $\lambda_1$:
\[
\begin{pmatrix}
1 & -1 & -4 & \vrule & 0 \\
-2  & 2  & 12 & \vrule & 0 \\
1 & -1 & - 5 & \vrule & 0 \\
\end{pmatrix} 
\]
\[
\begin{pmatrix}
1 & -1 & 0 & \vrule & 0 \\
0 & 0 & 1& \vrule & 0 \\
0 & 0& 0 & \vrule & 0 \\
\end{pmatrix} 
\]
\[
\begin{cases}
x = y \\
z = 0
\end{cases}
\]
Берем:
\[
h_1 = \begin{pmatrix}
1 \\ 1 \\ 0 
\end{pmatrix}
\]
Считаем для $\lambda = -1$:
\[
\begin{pmatrix}
2& -1 & -4 & \vrule & 0 \\
-2  & 3  & 12 & \vrule & 0 \\
1 & -1 & - 4 & \vrule & 0 \\
\end{pmatrix} \sim
\begin{pmatrix}
1 & 0 & 0 & \vrule & 0 \\
0 & 1& 4& \vrule & 0 \\
0 & 0& 0 & \vrule & 0 \\
\end{pmatrix} 
\]
\[
\begin{cases}
x = 0 \\
y = -4z
\end{cases}
\]
Берем:
\[
h_2 = \begin{pmatrix}
0 \\
-4 \\
1
\end{pmatrix}
\]
Ранг матрицы 2, значит хотим еще присоединенный вектор
\[
\begin{pmatrix}
2& -1 & -4 & \vrule & 0 \\
-2  & 3  & 12 & \vrule & -4 \\
1 & -1 & - 4 & \vrule & 1 \\
\end{pmatrix} 
\sim
\begin{pmatrix}
2& 0& 0 & \vrule & -2 \\
0  & 1 & 4& \vrule & -2 \\
0& 0& 0 & \vrule & 0\\
\end{pmatrix} 
\]
\[
\begin{cases}
x= -1\\
y + 4z = -2
\end{cases}
\]
Тогда берем:
\[
h_3 = 
\begin{pmatrix}
-1\\
-6\\
1
\end{pmatrix}
\]
Получаем решение:
\begin{center}
\textbf{Ответ: } 
\end{center}
\[
x = 1 \cdot C_1 \cdot e^{0t} +C_2 \cdot (0t -1)
\]
\[
y = 1 \cdot C_1 \cdot e^{0t} + C_2  \cdot(-4t - 6) - 4 \cdot C_3 \cdot e^{-t}
\]
\[
z = C_2 \cdot (t + 1)+  1 \cdot C_3 \cdot e^{-t}
\]
\clearpage
\subsection*{c)}
\[
\begin{cases}
x' = 2x + 6y - 15z \\
y' = x + y - 5z \\
z' = x + 2y -6z
\end{cases}
\]
Решаем:
\[
\begin{vmatrix}
2 - \lambda & 6 & -15 \\
1 & 1  - \lambda & -5 \\
1 & 2 & - 6  - \lambda \\
\end{vmatrix} = -\lambda^3 -3\lambda^2 -3\lambda - 1 = -(\lambda + 1)^3
\]
Получаем $\lambda_{1} = -1$ кратности 3. Смотрим:
\[
\begin{pmatrix}
3 & 6 & -15 & \vrule & 0 \\
1 & 2 & -5  & \vrule & 0 \\
1 & 2 & -5 & \vrule & 0 \\
\end{pmatrix} \sim 
\begin{pmatrix}
1 & 2 & -5 & \vrule & 0 \\
0& 0& 0 & \vrule & 0 \\
0& 0 & 0 & \vrule & 0 \\
\end{pmatrix} 
\]
\[
 x + 2y - 5z = 0
\]
Ранг матрицы равен 1, значит есть два собственных вектора. Но нам нужно найти тот c.в, у которого есть присоединенный, заметим из нашей матрицы, что мы хотим:
\[
\begin{cases}
3x + 6y - 15z = 3a \\
x + 2y -5z = a \\
x + 2y - 5z = a\\ 
\end{cases}
\]
Потребуем, чтобы вектор $(3a, a, a)$ был собственным вектором. Например, возьмем $h_1 = \begin{pmatrix}
3 \\ 1 \\ 1
\end{pmatrix}$. Теперь второй можно взять произвольно, лишь бы он был линейно независимым. Например $h_3 = \begin{pmatrix}
5 \\ 0 \\ 1
\end{pmatrix}$. Тогда сам присоединенный к $h_1$:
\[
x + 2y - 5z = 1
\]
Подходит:
\[
h_2
= 
\begin{pmatrix}
1 \\ 0 \\ 0 
\end{pmatrix}
\]
Тогда получаем решение:
\begin{center}
\textbf{Ответ: } 
\end{center}
\[
x = 3 \cdot C_1 \cdot e^{-t} + C_2 \cdot (3t + 1) \cdot e^{-t}+ 5 \cdot C_3 \cdot e^{-t}
\]
\[
y = 1 \cdot C_1 \cdot e^{-t}+ C_2 \cdot (t + 0) \cdot e^{-t} 
\]
\[
z =  1 \cdot C_1 \cdot e^{-t}+ C_2 \cdot (t + 0)  \cdot e^{-t}+  1 \cdot C_3 \cdot e^{-t}
\]
\clearpage
\subsection*{d)}
\[
\begin{cases}
x' = x + y - z \\
y' = -x + 4y - 2z \\
z' = -2x + 5y -2z
\end{cases}
\]
Считаем:
\[
\begin{vmatrix}
1 - \lambda& 1 & -1 \\
-1 & 4  - \lambda& -2 \\
-2 &5 & -2  - \lambda 
\end{vmatrix} = -\lambda^3 + 3 \lambda^2 -3 \lambda + 1 = -(\lambda - 1)^3
\]
Имеем $\lambda_1 = 1$ кратности 3. Смотрим:
\[
\begin{pmatrix}
0& 1 & -1 & \vrule & 0 \\
-1 & 3& -2 & \vrule & 0\\
-2 &5 & -3  & \vrule & 0 
\end{pmatrix}  \sim 
\begin{pmatrix}
1& 0& -1 & \vrule & 0 \\
0 & 1& -1 & \vrule & 0\\
0 &0& 0  & \vrule & 0 
\end{pmatrix} 
\]
Ранг матрицы 2, будет 1 собственный вектор, значит нужно еще будет взять присоединенных:
\[
\begin{cases}
x - y = 0 \\
y - z = 0 
\end{cases}
\]
\[
\begin{cases}
x = y \\
y = z
\end{cases}
\]
Берем вектор:
\[
h_1 = \begin{pmatrix}
1 \\ 1 \\ 1
\end{pmatrix}
\]
Ищем присоединенный:
\[
\begin{pmatrix}
0& 1 & -1 & \vrule & 1 \\
-1 & 3& -2 & \vrule & 1\\
-2 &5 & -3  & \vrule & 1
\end{pmatrix}  \sim 
\begin{pmatrix}
-1& 3 & -2 & \vrule & 1 \\
0 & 1& -1 & \vrule & 1\\
0 & 0 & 0& \vrule & 0
\end{pmatrix}  
\]
\[
\begin{cases}
-x + 3y - 2z = 1 \\
y - z = 1 \\
\end{cases}
\]
Возьмем тогда:
\[
h_2 = \begin{pmatrix}
2 \\ 1\\ 0
\end{pmatrix}
\]
Ищем второй присоединенный:
\[
\begin{pmatrix}
0& 1 & -1 & \vrule & 2\\
-1 & 3& -2 & \vrule & 1\\
-2 &5 & -3  & \vrule & 0
\end{pmatrix}  \sim 
\begin{pmatrix}
-1& 3 & -2& \vrule & 1 \\
0 & 1& -1 & \vrule & 2\\
0 & 0 & 0 & \vrule & 0
\end{pmatrix}  
\]
Отсюда:
\[
\begin{cases}
-x + 3y - 2z = 1 \\
y - z = 2 
\end{cases}
\]
Положим тогда:
\[
h_3 = 
\begin{pmatrix}
8 \\
3 \\
1
\end{pmatrix}
\]
Получаем решение:
\begin{center}
\textbf{Ответ: } 
\end{center}
\[
x = 1 \cdot C_1 \cdot e^t + C_2 \cdot \left(t + 2\right) \cdot e^t + C_3 \cdot \left(\frac{t^2}{2} \cdot 1 + t \cdot 2 + 8\right) \cdot e^t
\]
\[
y = 1 \cdot C_1 \cdot e^t + C_2 \cdot \left(t + 1\right) \cdot e^t + C_3 \cdot \left(\frac{t^2}{2} \cdot 1 + t \cdot 1 + 3\right) \cdot e^t 
\]
\[
z = 1 \cdot C_1 \cdot e^t + C_2 \cdot \left(t + 0\right) \cdot e^t + C_3 \cdot \left(\frac{t^2}{2} \cdot 1 + t \cdot 0+ 1\right) \cdot e^t
\]
\clearpage
\section*{Номер 2}
\[
\begin{cases}
x' = 3x -5y -2e^t \\
y' = x - y - e^t
\end{cases}
\]
Решаем сначала однородную систему:
\[
\begin{cases}
x' = 3x -5y \\
y' = x - y 
\end{cases}
\]
Смотрим:
\[
\begin{vmatrix}
3 - \lambda & -5 \\
1 & -1 - \lambda 
\end{vmatrix} = 
\lambda^2 -2 \lambda + 2 
\]
Хотим:
\[
\lambda^2 -2 \lambda + 2  = 0
\]
\[
(\lambda - 1)^2 = -1
\]
Отсюда корни:
\[
\lambda_{1, 2} = 1 \pm i
\]
Можем один откинуть, т.к они сгруппируются в итоге, найдем для $1 + i$:
\[
\begin{pmatrix}
2 -i & 5 & \vrule & 0 \\
1  & -2 - i & \vrule & 0 
\end{pmatrix} \sim 
\begin{pmatrix}
1 & -2 - i& \vrule & 0 \\
0& 0 & \vrule & 0 
\end{pmatrix} 
\]
\[
x - y(2 + i) = 0
\]
Тогда берем:
\[
h_1 = \begin{pmatrix}
2 + i  \\ 1
\end{pmatrix}
\]
Тогда:
\[
x_{h_1} = (2 + i) \cdot e^{(1 + i)t} = 2e^t \cdot e^{it} + ie^t \cdot e^{it} = e^t(2e^{it} + ie^it) = e^t \cdot (- \sin t + 2 \cos t + 2i \sin t + i \cos t)
\]
\[
y_{h_1} = e^{(1 +i)t} = e^t  \cos t + e^t  \cdot i \sin t 
\]
А отсюда:
\[
x = C_1 \cdot e^t(- \sin t + 2 \cos t) + C_2 \cdot e^t \cdot (2 \sin t + \cos t)
\]
\[
y = C_1 \cdot e^t \cos t + C_2 \cdot e^t \cdot \sin t
\]
Теперь решаем неоднородную часть:
\[
\begin{cases}
x' = 3x -5y  - 2e^t\\
y' = x - y  - e^t
\end{cases}
\]
Ищем решение в виде:
\[
\begin{cases}
x_1 = A \cdot e^{t} \\
y_1 = B \cdot e^t
\end{cases}
\]
Тогда:
\[
\begin{cases}
x_1' = A \cdot e^{t} \\
y_1'= B \cdot e^t
\end{cases}
\]
Подставляем:
\[
\begin{cases}
A \cdot e^{t}= 3A \cdot e^{t}-5B \cdot e^t -2e^t \\
B \cdot e^t= A \cdot e^{t} - B \cdot e^t - e^t
\end{cases}
\]
\[
\begin{cases}
A = 3A -5B  -2\\
B = A - B  - 1
\end{cases}
\]
\[
\begin{cases}
A = 1 \\
B = 0
\end{cases}
\]
А значит
\[
\begin{cases}
x_1 = e^t \\
y_1 = 0 
\end{cases}
\]
Итого:
\begin{center}
\textbf{Ответ: } 
\[
x = C_1 \cdot e^t(- \sin t + 2 \cos t) + C_2 \cdot e^t \cdot (2 \sin t + \cos t)+ C_3 \cdot e^t
\]
\[
y =  C_1 \cdot e^t \cdot \cos t + C_2 \cdot e^t \cdot \sin t + 0
\]
\end{center}
\clearpage
\section*{Номер 3}
\[
\begin{cases}
x' = -2x + y + t \ln t \\
y' = -4x + 2y + 2 t \ln t
\end{cases}
\]
Для начала решим однородное:
\[
\begin{cases}
x' = -2x + y \\
y' = -4x + 2y 
\end{cases}
\]
Смотрим:
\[
\begin{vmatrix}
-2 - \lambda & 1 \\
-4 & 2 - \lambda 
\end{vmatrix} = \lambda^2 
\]
Получаем $\lambda_1 = 0$ кратности 2. Теперь смотрим собственный вектор:
\[
\begin{pmatrix}
-2  & 1 & \vrule & 0 \\
-4 & 2 & \vrule & 0
\end{pmatrix}  \sim 
\begin{pmatrix}
2  & -1& \vrule & 0 \\
0 & 0& \vrule & 0
\end{pmatrix} 
\]
\[
2x - y =0 
\]
Берем:
\[
h_1 = \begin{pmatrix}
1 \\
2
\end{pmatrix}
\]
Ищем присоединенный:
\[
\begin{pmatrix}
-2  & 1 & \vrule & 1\\
-4 & 2 & \vrule & 2
\end{pmatrix} \sim 
\begin{pmatrix}
-2  & 1 & \vrule & 1\\
0 & 0& \vrule & 0
\end{pmatrix}
\]
\[
-2x + y = 1
\]
Берем:
\[
h_2 = \begin{pmatrix}
\frac{1}{2} \\ 2 
\end{pmatrix}
\]
Тогда:
\[
x = 1 \cdot C_1 \cdot e^{0t} + C_2 \cdot \left(t + \frac{1}{2}\right) e^{0t} = C_1 + C_2 \left(t + \frac{1}{2}\right)
\]
\[
y = 2 \cdot C_1 \cdot e^{0t} + C_2 \cdot (2t + 2) e^{0t} = 2C_1 + C_2 (2t + 2)
\]
Ну и по методу вариации постоянных:
\[
x = C_1(t) + tC_2(t) + \frac{1}{2}C_2(t)
\]
\[
y = 2C_1(t) + 2tC_2(t) + 2C_2(t)
\]
\[
x' = C_1'(t) + C_2(t) + tC_2(t)' + \frac{1}{2}C_2(t)' 
\]
\[
y' = 2C_1(t)' + 2tC_2(t)' + 2C_2(t)' + 2 C_2(t)
\]
Подставляем:
\[
\begin{cases}
x' = -2x + y + t \ln t \\
y' = -4x + 2y + 2 t \ln t
\end{cases}
\]
\[
\begin{cases}
 C_1'(t) + C_2(t) + tC_2(t)' + \frac{1}{2}C_2(t)' = -2(C_1(t) + tC_2(t) + \frac{1}{2}C_2(t)) + (2C_1(t) + 2tC_2(t) + 2C_2(t)) + t \ln t\\
2C_1(t)' + 2tC_2(t)' + 2C_2(t)' + 2 C_2(t) = -4(C_1(t) + tC_2(t) + \frac{1}{2}C_2(t)) + 2(2C_1(t) + 2tC_2(t) + 2C_2(t)) + 2t \ln t
\end{cases}
\]
Много букв, что аж не влезает, но там все сокращается и остается просто:
\[
\begin{cases}
C_1'(t) + tC_2(t)' + \frac{1}{2}C_2(t)' =t \ln t  \\
2C_1(t)' + 2tC_2(t)' + 2C_2(t)' = 2t \ln t
\end{cases}
\]
Из второго уравнения вычитаем первое, умноженное на два:
\[
C_2(t)' = 2t \ln t -2t \ln t
\]
\[
C_2(t)' = 0
\]
\[
C_2(t) = D_2
\]
Отсюда:
\[
C_1(t)' = t \ln t
\]
\[
C_1(t) = \int t \ln t = \frac{1}{4}t^2 (2 \ln t - 1) + D_1
\]
Подставляем:
\[
x = \frac{1}{4}t^2 (2 \ln t - 1) + D_1 + t \cdot D_2 + \frac{1}{2} D_2
\]
\[
y = 2 \cdot \left(
 \frac{1}{4}t^2 (2 \ln t - 1) + D_1
\right) + 2 t D_2 + 2 D_2
\]
\clearpage
\section*{Номер 4}
\subsection*{a)}
\[
\begin{cases}
x' = x + y + 3t + 6 \\
y' = -10x -y + 6t + 3
\end{cases}, \; x(0) = y(0) = 0
\]
Применяем преобразование Лапласа:
\[
x(t) \overset{L}{\rightarrow} X(p) 
\]
\[
x'(t) \overset{L}{\rightarrow} pX(p) - x(0) = pX(p)
\]
\[
y(t) \overset{L}{\rightarrow} Y(p)
\]
\[
y(t)' \overset{L}{\rightarrow}pY(p) - y(0) = pY(p)
\]
Ну и для функций справа:
\[
3t + 6 \overset{L}{\rightarrow}\frac{6p + 3}{p^2}
\]
\[
6t + 3 \overset{L}{\rightarrow} \frac{3p + 6}{p^2}
\]
Тогда получаем:
\[
\begin{cases}
pX(p) = X(p) + Y(p) + \frac{6p + 3}{p^2} \\
pY(p) = -10X(p) - Y(p) + \frac{3p + 6}{p^2}
\end{cases}
\]
Выражаем $Y(p)$:
\[
\begin{cases}
Y(p) = pX(p) - X(p) - \frac{6p + 3}{p^2}\\
p\left(pX(p) - X(p) - \frac{6p + 3}{p^2}\right) = -10X(p) - \left(pX(p) - X(p) - \frac{6p + 3}{p^2}\right) + \frac{3p + 6}{p^2}
\end{cases}
\]
Поработаем со вторым уравнением, пусть $X = X(p), Y = Y(p)$, чтобы не городить символов, раскрываем скобки, все с $X$ переносим в левую часть:
\[
pX(p - 1) + 10X + pX - X= \frac{9p + 9}{p^2}  + \frac{6p+3}{p} 
\]
\[
X \left(
p(p-1) + 10 + p - 1) 
\right) =\frac{9p + 9}{p^2}  + \frac{6p+3}{p} 
\]
\[
X = \frac{\frac{9p + 9}{p^2}  + \frac{6p+3}{p} }{\left(
p(p-1) + 10 + p - 1) 
\right)}
\]
\[
X = \frac{\frac{9p + 9 + 6p^2 + 3p}{p^2}}{p^2 - p + 9 + p}
\]
\[
X = \frac{\frac{6p^2 + 12p + 9}{p^2}}{p^2 + 9}
\]
\[
X = \frac{6p^2 + 12p + 9}{p^4 + 9p^2}
\]
Вспоминаем про нашу систему и ищем $Y(p) = Y$:
\[
Y= pX - X - \frac{6p + 3}{p^2}
\]
\[
Y = \frac{p(6p^2 + 12p + 9)}{p^4 + 9p^2} - \frac{6p^2 + 12p + 9}{p^4 + 9p^2} - \frac{6p + 3}{p^2}
\]
\[
Y = \frac{p(6p^2 + 12p + 9) - (6p^2 + 12p + 9)}{p^2(p^2+ 9)} - \frac{(6p+3)(p^2 + 9)}{p^2(p^2 + 9)}
\]
\[
Y = \frac{6p^3 + 12p^2 + 9p - 6p^2- 12p - 9 -6p^3 - 3p^2 - 54p - 27}{p^2(p^2 + 9)}
\]
\[
Y =  \frac{3(p^2 - 19p -12)}{p^2(p^2 + 9)}
\]
Итого получили:
\[
\begin{cases}
X(p) = \frac{6p^2 + 12p + 9}{p^2(p^2 + 9)} \\
Y(p) =  \frac{3(p^2 - 19p -12)}{p^2(p^2 + 9)}
\end{cases}
\]
Теперь надо сделать обратное преобразование, упростим дроби до чего-то более понятного, начнем с $X(p)$:
\[
\frac{6p^2 + 12p + 9}{p^2(p^2 + 9)} = \frac{A}{p} + \frac{B}{p^2} + \frac{Cp + D}{p^2 + 9}
\]
Ищем коэффы:
\[
6p^2 + 12p + 9 = Ap(p^2 + 9) + B(p^2 + 9) + (Cp+D)p^2
\]
\[
6p^2 + 12p + 9  = Ap^3 + Ap9 + Bp^2 + B9 + Cp^3 + Dp^2
\]
\[
6p^2 + 12p + 9  = p^3(A + C) + p^2(B + D) + p9A + 9B
\]
\[
\begin{cases}
A + C = 0 \\
B + D = 6 \\
9A = 12 \\
9B = 9
\end{cases} \sim 
\begin{cases}
C = - \frac{4}{3}\\
D = 5 \\
A = \frac{4}{3} \\
B = 1
\end{cases}
\]
Итого:
\[
\frac{6p^2 + 12p + 9}{p^2(p^2 + 9)} = \frac{4}{3p} + \frac{1}{p^2} + \frac{5 - \frac{4}{3}p}{p^2 + 9}  =  \frac{4}{3p} + \frac{1}{p^2} + \frac{5}{p^2 + 9}  - \frac{\frac{4}{3}p}{p^2 + 9}  \overset{L^{-1}}{\longrightarrow} \frac{4}{3} + t + \frac{5}{3} \sin 3t + \frac{4}{3} \cos 3t
\]
Теперь тоже самое проделываем с $Y(p)$:
\[
 \frac{3(p^2 - 19p -12)}{p^2(p^2 + 9)} = \frac{A}{p} + \frac{B}{p^2} + \frac{Cp + D}{p^2 + 9}
\]
\[
 3(p^2 - 19p -12) = (A + C)p^3 + (B + D)p^2 + 9Ap + 9B
\]
\[
\begin{cases}
0 = A + C \\
3 = B + D \\
-57 = 9A \\
-36 = 9B
\end{cases} \sim 
\begin{cases}
C = \frac{19}{3} \\
D = 7\\
A = - \frac{19}{13}\\
B = -4\\
\end{cases}
\]
Итого:
\[
\frac{3(p^2 - 19p -12)}{p^2(p^2 + 9)} = -\frac{19}{13p} - \frac{4}{p^2} +  \frac{\frac{19}{3}p}{p^2+ 9} + \frac{7}{p^2 + 9} \overset{L^{-1}}{\longrightarrow} - \frac{19}{3} - 4t + \frac{19}{13} \cos 3t + \frac{7}{3} \sin 3t
\]
Получаем:
\begin{center}
\textbf{Ответ: } 
\end{center}
\[
\begin{cases}
x(t) = \frac{4}{3} + t + \frac{5}{3} \sin 3t + \frac{4}{3} \cos 3t \\
y(t) = - \frac{19}{3} - 4t + \frac{19}{13} \cos 3t + \frac{7}{3} \sin 3t
\end{cases}
\]
\clearpage
\subsection*{b)}
\[
\begin{cases}
x' = -x - y+ e^{2t} \\
y' = 2x + 2y + 2e^{2t}
\end{cases}, \; x(0) = y(0) = 1
\]
Делаем тоже самое:
\[
x(t) \overset{L}{\rightarrow} X(p) 
\]
\[
x'(t) \overset{L}{\rightarrow} pX(p) - x(0) = pX(p) - 1
\]
\[
y(t) \overset{L}{\rightarrow} Y(p)
\]
\[
y(t)' \overset{L}{\rightarrow} pY(p) - y(0) = pY(p) - 1
\]
\[
e^{2t}\overset{L}{\rightarrow}  \frac{1}{p - 2}
\]
\[
2e^{2t}\overset{L}{\rightarrow}  \frac{2}{p - 2}
\]
Получаем систему:
\[
\begin{cases}
pX(p) - 1 = -X(p) - Y(p) + \frac{1}{p - 2}\\
pY(p) - 1 = 2X(p) + 2Y(p) + \frac{2}{p - 2}
\end{cases}
\]
Выражаем $Y(p)$, пусть аналогично $X = X(p), Y = Y(p)$:
\[
\begin{cases}
Y = 1 - X - pX + \frac{1}{p - 2} \\
p \left(1 - X - pX + \frac{1}{p - 2} \right) - 1 = 2X + 2\left(1 - X - pX + \frac{1}{p - 2} \right)+ \frac{2}{p - 2}
\end{cases}
\]
Второе выражение:
\[
X(-p^2 + p) = 3 - p + \frac{-p + 4}{p - 2}
\]
\[
X = \frac{p^2 - 4p + 2}{p(p-1)(p-2)}
\]
Тогда:
\[
Y = 1 -  \frac{p^2 - 4p + 2}{p(p-1)(p-2)} - p  \frac{p^2 - 4p + 2}{p(p-1)(p-2)} + \frac{1}{p-2}
\]
\[
Y  = \frac{p^2 + 3p -2}{p(p-1)(p-2)}
\]
Суммируем:
\[
\begin{cases}
X(p) = \frac{p^2 - 4p + 2}{p(p-1)(p-2)} \\
Y(p) = \frac{p^2 + 3p -2}{p(p-1)(p-2)}
\end{cases}
\]
Начнем с $X(p)$:
\[
 \frac{p^2 - 4p + 2}{p(p-1)(p-2)}  = \frac{A}{p} + \frac{B}{p - 1} + \frac{C}{p-2} 
\]
Ищем коэффы:
\[
p^2 -4p + 2 = A(p-1)(p-2) + Bp(p-2) + Cp(p-1)
\]
\[
p^2 - 4p +2 = A p^2 - 3 A p + 2 A + B p^2 - 2 B p + C p^2 - C p
\]
\[
p^2 - 4p +2 = p^2(A + B + C) + p(3A - 2B - C) + 2A
\]
\[
\begin{cases}
A + B + C = 1 \\
3A -2B - C = -4 \\
2A =2
\end{cases} \sim 
\begin{cases}
C = -1 \\
B = 1\\
A = 1
\end{cases} 
\]
Значит:
\[
 \frac{p^2 - 4p + 2}{p(p-1)(p-2)}  = \frac{1}{p} - \frac{1}{p - 1} + \frac{1}{p-2} \overset{L^{-1}}{\longrightarrow} 1 + e^t - e^{2t}
\]
Теперь разбираемся с $Y(p)$:
\[
\frac{p^2 + 3p -2}{p(p-1)(p-2)} = \frac{A}{p} + \frac{B}{p - 1} + \frac{C}{p-2} 
\]
\[
p^2 + 3p - 2 = p^2(A + B + C) + p(3A - 2B - C) + 2A
\]
\[
\begin{cases}
A + B + C = 1 \\
3A -2B - C = -4 \\
2A =2
\end{cases} \sim 
\begin{cases}
C = 4\\
B = -2\\
A = -1
\end{cases} 
\]
Значит:
\[
\frac{p^2 + 3p -2}{p(p-1)(p-2)} =- \frac{1}{p} - \frac{2}{p - 1} + \frac{4}{p-2} \overset{L^{-1}}{\longrightarrow} - 1 - 2e^t + 4e^{2t}
\]
Получаем:
\begin{center}
\textbf{Ответ: } 
\[
\begin{cases}
x(t) = 1 + e^t - e^{2t} \\
y(t) = - 1 - 2e^t + 4e^{2t}
\end{cases}
\]
\end{center}
\end{document}
