\documentclass[a4paper,12pt]{article}

%%% Работа с русским языком
\usepackage{cmap}					% поиск в PDF
\usepackage{mathtext} 				% русские буквы в формулах
\usepackage[T2A]{fontenc}			% кодировка
\usepackage[utf8]{inputenc}			% кодировка исходного текста
\usepackage[english,russian]{babel}	% локализация и переносы
\usepackage{xcolor}
\usepackage{hyperref}
 % Цвета для гиперссылок
\definecolor{linkcolor}{HTML}{799B03} % цвет ссылок
\definecolor{urlcolor}{HTML}{799B03} % цвет гиперссылок

\hypersetup{pdfstartview=FitH,  linkcolor=linkcolor,urlcolor=urlcolor, colorlinks=true}

%%% Дополнительная работа с математикой
\usepackage{amsfonts,amssymb,amsthm,mathtools} % AMS
\usepackage{amsmath}
\usepackage{icomma} % "Умная" запятая: $0,2$ --- число, $0, 2$ --- перечисление

%% Номера формул
%\mathtoolsset{showonlyrefs=true} % Показывать номера только у тех формул, на которые есть \eqref{} в тексте.

%% Шрифты
\usepackage{euscript}	 % Шрифт Евклид
\usepackage{mathrsfs} % Красивый матшрифт

%% Свои команды
\DeclareMathOperator{\sgn}{\mathop{sgn}}

%% Перенос знаков в формулах (по Львовскому)
\newcommand*{\hm}[1]{#1\nobreak\discretionary{}
{\hbox{$\mathsurround=0pt #1$}}{}}
% графика
\usepackage{graphicx}
\graphicspath{{pictures/}}
\DeclareGraphicsExtensions{.pdf,.png,.jpg}
\author{Бурмашев Григорий, БПМИ-208}
\title{Диффуры, дз -- 7}
\date{\today}
\begin{document}
\maketitle
\section*{Номер 2}
\[ 
y_{t + 2} + 5(t + 1)y_{t + 1} + 6(t + 1)ty_t = 20(t + 1)!2^t
\]
Подстановка:
\[
y_t = (t-1)!z_t 
\]
Ну собственно и подставляем:
\[
(t+1)!z_{t + 2}+ 5(t + 1)!z_{t+1} + 6(t + 1)t \cdot (t-1)!z_t  = 20(t + 1)!2^t
\]
Сокращаем:
\[
z_{t+2} + 5z_{t + 1} + 6 z_t = 20 \cdot 2^t
\]
Сначала найдем общее решение:
\[
\lambda^2 + 5\lambda + 6 = 0
\]
\[
(\lambda+ 2)(\lambda  + 3) = 0
\]
Отсюда корни:
\[
\lambda_1 = -2, \lambda_2 = -3
\]
Ну и получаем решение:
\[
C_1 (-2)^t + C_2 (-3)^5 
\]
Теперь будем искать частное решение, ищем как:
\[
(at + b)2^t 
\]
Подставляем и находим:
\[
(a(t+2) +b)2^{t+2} + 5(a(t+1) + b)2^{t+1} + 6(at + b)2^t = 20 \cdot 2^t
\]
\[
a(10t + 9) + 10b = 10
\]
\[
10t \cdot a + 9a + 10b = 10
\]
Отсюда получаем систему:
\[
\begin{cases}
a = 0 \\
9a + 10b = 10
\end{cases}
\]
 \[
\begin{cases}
a = 0 \\
b = 1
\end{cases}
\]
Тобишь получаем частное решение:
\[
(0 \cdot t + 1)2^t = 2^t
\]
Итого суммируем:
\[
z_t = C_1 (-2)^t + C_2 (-3)^5  + 2^t 
\]
Вспоминаем про подстановку и получаем:
\[
y_t = (t-1)! \cdot z_t = (t-1)! \cdot (C_1 (-2)^t + C_2 (-3)^5  + 2^t ) 
\]
\begin{center}
\textbf{Ответ: } 
\[
y_t = (t-1)! \cdot (C_1 (-2)^t + C_2 (-3)^5  + 2^t ) 
\]
\end{center}
\clearpage
\section*{Номер 3}
\subsection*{a)}
\[
\begin{pmatrix}
x_{t + 1} \\
y_{t + 1} \\
z_{t + 1} 
\end{pmatrix}
=
\begin{pmatrix}
1 & 2 & 2 \\
2 & 1 & 2 \\
2 & 2  & 1 
\end{pmatrix}
\begin{pmatrix}
x_{t } \\
y_{t} \\
z_{t } 
\end{pmatrix}
\]
Ищем собственные значения для матрицы:
\[
\begin{vmatrix}
1 - \lambda & 2 & 2 \\
2 & 1 - \lambda & 2 \\
2 & 2 & 1 - \lambda 
\end{vmatrix} = -\lambda^3 + 3\lambda^2 + 9\lambda + 5
\]
По скобкам:
\[
-\lambda^3 + 3\lambda^2 + 9\lambda + 5 = -(\lambda - 5)(\lambda + 1)^2
\]
Получили значения $\lambda_{1} = 5$, $\lambda_2 = -1$, теперь ищем векторы, сначала для $\lambda = 5$:
\[
\begin{pmatrix}
-4& 2 & 2 \\
2 & -4 & 2 \\
2 & 2 & -4
\end{pmatrix} 
\cdot
\begin{pmatrix}
x \\
y \\
z 
\end{pmatrix} 
=
\begin{pmatrix}
0 \\
0 \\
0
\end{pmatrix}
\]
Отсюда:
\[
h_1 = \begin{pmatrix}
1 \\ 1 \\ 1
\end{pmatrix}
\]
Теперь для второго значения:
\[
\begin{pmatrix}
2& 2 & 2 \\
2 & 2 & 2 \\
2 & 2 & 2
\end{pmatrix} 
\cdot
\begin{pmatrix}
x \\
y \\
z 
\end{pmatrix} 
=
\begin{pmatrix}
0 \\
0 \\
0
\end{pmatrix}
\]
Отсюда получаем два вектора:
\[
h_2= \begin{pmatrix}
-1 \\ 1 \\ 0
\end{pmatrix}
\]
\[
h_3 = \begin{pmatrix}
-1 \\ 0 \\ 1
\end{pmatrix}
\]
Получаем решение:
\[
\begin{pmatrix}
x_{t } \\
y_{t} \\
z_{t } 
\end{pmatrix} = C_1 \cdot 5^t \cdot h_1 + C_2 \cdot (-1)^t \cdot h_2 + C_3 \cdot (-1)^t \cdot h_3 = 
\]
\[
=
C_1 \cdot 5^t \cdot \begin{pmatrix}
1 \\ 1 \\ 1
\end{pmatrix} + C_2 \cdot (-1)^t \cdot  \begin{pmatrix}
-1 \\ 1 \\ 0
\end{pmatrix} + C_3 \cdot (-1)^t \cdot  \begin{pmatrix}
-1 \\ 0 \\ 1
\end{pmatrix}
\]
\begin{center}
\textbf{Ответ: } 
\[
\begin{pmatrix}
x_{t } \\
y_{t} \\
z_{t } 
\end{pmatrix} 
=
C_1 \cdot 5^t \cdot \begin{pmatrix}
1 \\ 1 \\ 1
\end{pmatrix} + C_2 \cdot (-1)^t \cdot  \begin{pmatrix}
-1 \\ 1 \\ 0
\end{pmatrix} + C_3 \cdot (-1)^t \cdot  \begin{pmatrix}
-1 \\ 0 \\ 1
\end{pmatrix}
\]
\end{center}
\clearpage
\subsection*{b)}
\[
\begin{pmatrix}
x_{t + 1} \\
y_{t + 1} \\
z_{t + 1} 
\end{pmatrix}
=
\begin{pmatrix}
1 & -1 & -1 \\
-2 & 2& 1 \\
4 & 2  & 3 
\end{pmatrix}
\begin{pmatrix}
x_{t } \\
y_{t} \\
z_{t } 
\end{pmatrix}
\]
Делаем все тоже самое:
 \[
\begin{vmatrix}
1 - \lambda & -1 & -1 \\
-2 & 2 - \lambda& 1 \\
4 & 2  & 3 - \lambda 
\end{vmatrix} = -\lambda^3 + 6\lambda^2 - 11 \lambda + 6 = -(\lambda -1)(\lambda -2)(\lambda -3)
\]
Получили $\lambda_1 = 1, \lambda_2 = 2, \lambda_3 =3 $, считаем:
\[
\begin{pmatrix}
0& -1 & -1 \\
-2 &1& 1 \\
4 & 2  & 2
\end{pmatrix}
\begin{pmatrix}
x \\
y \\
z 
\end{pmatrix} = 
\begin{pmatrix}
0 \\ 0 \\ 0
\end{pmatrix}
\] 
Отсюда:
\[
h_1 = \begin{pmatrix}
0 \\ 1 \\ -1
\end{pmatrix}
\]
Для второго:
\[
\begin{pmatrix}
-1& -1 & -1 \\
-2 &0& 1 \\
4 & 2  & 1
\end{pmatrix}
\begin{pmatrix}
x \\
y \\
z 
\end{pmatrix}  = 
\begin{pmatrix}
0 \\ 0 \\ 0
\end{pmatrix}
\]
\[
h_2 = \begin{pmatrix}
1 \\ -3 \\ 2
\end{pmatrix}
\]
Для третьего:
\[
\begin{pmatrix}
-2& -1 & -1 \\
-2 &-1& 1 \\
4 & 2  & 0
\end{pmatrix}
\begin{pmatrix}
x \\
y \\
z 
\end{pmatrix} = 
\begin{pmatrix}
0 \\ 0 \\ 0
\end{pmatrix}
\]
\[
h_3 = \begin{pmatrix}
1 \\ -2 \\ 0
\end{pmatrix}
\]
\begin{center}
\textbf{Ответ: } 
\[
\begin{pmatrix}
x_{t} \\
y_{t} \\
z_{t} 
\end{pmatrix}
=
C_1 \cdot 
\begin{pmatrix}
0 \\ 1 \\ -1
\end{pmatrix}
+ C_2 \cdot 2^t \cdot
\begin{pmatrix}
1 \\ -3 \\ 2
\end{pmatrix}
+ C_3 \cdot 3^t \cdot
\begin{pmatrix}
1 \\ -2 \\ 0
\end{pmatrix}
\]
\end{center}
\end{document}
