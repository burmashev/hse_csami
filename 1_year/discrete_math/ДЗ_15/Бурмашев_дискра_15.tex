\documentclass[a4paper,12pt]{article}

%%% Работа с русским языком
\usepackage{cmap}					% поиск в PDF
\usepackage{mathtext} 				% русские буквы в формулах
\usepackage[T2A]{fontenc}			% кодировка
\usepackage[utf8]{inputenc}			% кодировка исходного текста
\usepackage[english,russian]{babel}	% локализация и переносы
\usepackage{xcolor}
\usepackage{hyperref}
 % Цвета для гиперссылок
\definecolor{linkcolor}{HTML}{00FFFF} % цвет ссылок
\definecolor{urlcolor}{HTML}{4682B4} % цвет гиперссылок

\hypersetup{pdfstartview=FitH,  linkcolor=linkcolor,urlcolor=urlcolor, colorlinks=true}

%%% Дополнительная работа с математикой
\usepackage{amsfonts,amssymb,amsthm,mathtools} % AMS
\usepackage{amsmath}
\usepackage{icomma} % "Умная" запятая: $0,2$ --- число, $0, 2$ --- перечисление

%% Номера формул
%\mathtoolsset{showonlyrefs=true} % Показывать номера только у тех формул, на которые есть \eqref{} в тексте.

%% Шрифты
\usepackage{euscript}	 % Шрифт Евклид
\usepackage{mathrsfs} % Красивый матшрифт

%% Свои команды
\DeclareMathOperator{\sgn}{\mathop{sgn}}

\usepackage{enumerate}
%% Перенос знаков в формулах (по Львовскому)
\newcommand*{\hm}[1]{#1\nobreak\discretionary{}
{\hbox{$\mathsurround=0pt #1$}}{}}
% графика
\usepackage{graphicx}
\graphicspath{{pictures/}}
\DeclareGraphicsExtensions{.pdf,.png,.jpg}
\author{Бурмаzшев Григорий, 208. \href{https://teleg.run/burmashev}{@burmashev}}
\title{Дискретная математика. Коллок -- 1. Определения и задачи по ним.}

\begin{document}
{\Large \begin{center}
Бурмашев Григорий.  208. Дискра -- 15
\end{center}}
Пусть A -- благоприятные исходы
\section*{Номер 1}
\begin{equation*}
\begin{gathered}
\Omega  = \{
\overline{x_1 x_2}; 0 \leq x_1 \leq 9, 0 \leq x_2 \leq 9
\}\\
|\Omega| = 10 \cdot 10 = 100 \\
A = \{0, 11, 22, 33, \ldots, 99 \} \\
|A| = 10 \\
P(A) = \frac{10}{100} = 0.1
\end{gathered}
\end{equation*}
\begin{center}
\textbf{Ответ: } $0.1$
\end{center}

\section*{Номер 2}
\begin{equation*}
\begin{gathered}
\Omega = \{
(x_1, x_2, x_3), x_i \in [1, 6], i = 1, 2, 3
\} \\
|\Omega| = 6^3 = 216
\\
A = \{ (x1, x2, x3), \text{ где элементы нечетные}\} \\
|A| = 3^3 = 27
\\
P(A) = \frac{27}{216} = \frac{1}{8} = 0.125
\end{gathered}
\end{equation*}

\begin{center}
\textbf{Ответ: } $0.125$
\end{center}
\section*{Номер 3}
\begin{equation*}
\begin{gathered}
\Omega = \{ 
(x_1, \ldots x_{36}), x_i \in (0, 8) 
\}
\\
|\Omega| = 7^{36}
\}
\\
A = \{ (x_1, \ldots x_{36}), x_i \neq x_j \forall i = j - 1
\} \\
|A| = 7 \cdot \overbrace{6 \cdot  \ldots \cdot 6}^{35} = 7 \cdot 6^{35}
\\
P(A) = \frac{7 \cdot 6^{35}}{7^{30}} = \frac{6^{35}}{7^{35}} \approx 0,004537
\end{gathered}
\end{equation*}
\begin{center}
\textbf{Ответ: } $\approx 0,004537$, 2 знака после запятой
\end{center}
\section*{Номер 4}
\begin{equation*}
\begin{gathered}
\Omega = \{
1000, 1001, \ldots, 9999
\} \\
|\Omega| = 9000 \\
A = \{\text{сумма цифр 8}
\}
\end{gathered}
\end{equation*}
Для подсчета A разобьем наше четырехзначное число на две пары чисел (a+b и c+d), заметим, что $a \neq 0$
\\\\
Посчитаем количество способов получения сумм от 0 до 8 для (a + b):
\begin{equation*}
\begin{gathered}
0 - \text{невозможно, т.к a $\neq 0$}
\\
1 - 1 \text{ способ } (10)\\
2 - 2 \text{ способа } (20, 11)\\
\ldots \\
8 - 8 \text{ способов}
\end{gathered}
\end{equation*}
Посчитаем количество способов для c + d:
\begin{equation*}
\begin{gathered}
0 - 1 \text{ способ } (00)\\
1 - 2 \text{ способа } (01, 10) \\
\ldots \\
7 - 8 \text{ способов } \\
8 - \text{нам не нужно, т.к в первой паре будет как минимум 1}
\end{gathered}
\end{equation*}
Чтобы получить восемь, нужны пары:
\[
8 + 0 \text{ (но такого быть не может)}, 7 + 1, 6 + 2, 5 + 3, 4 + 4
\]
А значит по итогу будет 8 пар:
\begin{equation*}
\begin{gathered}
|A| = 8 \cdot 1 + 7 \cdot 2 + 6 \cdot 3 + 5 \cdot 4 + 4 \cdot 5 + 3 \cdot 6 + 2 \cdot 7 + 1 \cdot 8 = 120\\
P(A) = \frac{120}{9000} = \frac{4}{300} = \frac{1}{75} < \frac{1}{100}
\end{gathered}
\end{equation*}
\begin{center}
\textbf{Ответ: } $\frac{1}{75}$, меньше, чем $\frac{1}{100}$
\end{center}
\newpage
\section*{Номер 5}
У нас есть 20 первых чисел, 8 чисел в середине и 20 чисел в конце, всего у нас 48 чисел и $\Omega = \{ \text{все перестановки из 48 чисел} \}, |\Omega| = 48!$. Посмотрим на 40 чисел, которые оказались в начале и в конце. Среди них есть ровно 1 максимум, этот максимум может находится либо среди первых 20, либо среди последних 20, а значит подходит нам $|A| = \frac{48!}{2}$. Тогда:
\[
P(A) = \frac{\frac{48!}{2}}{48!} = \frac{1}{2}
\]
\begin{center}
\textbf{Ответ: } $ \frac{1}{2} $
\end{center}
 \section*{Номер 6}
\begin{equation*}
\begin{gathered}
\Omega = \{матрица 4 на 4 из нулей и единиц\} \\
|\Omega| = 2^{16} \\
A = \{
\text{верхняя строка из нулей}
\}
\\
B = \{
\text{правый столбец из нулей}
\}
\\
C = \{
\text{нижняя строка из нулей}
\}
\\
D = \{
\text{левый столбец из нулей
}
\}
\\
X = \{\text{хотя бы один из четырех вариантов A,B,C,D}\} \\
|X| = |A| + |B| + |C| + |D| - |AB| - |AC| - |AD| - |BC| \\
- |BD| - |CD| + |ABC| + |ABD| + |ACD| + |BCD| - |ABCD| 
\end{gathered}
\end{equation*}
Один из четырех:
\begin{equation*}
\begin{gathered}
|A| = |B| = |C| = |D| = 2^{12}
\end{gathered}
\end{equation*}
Напротив друг друга:
\begin{equation*}
\begin{gathered}
|AC| = |BD| = 2^8
\end{gathered}
\end{equation*}
Cоседние (строка + столбец):
\begin{equation*}
\begin{gathered}
|AB| = |AD| = |BC| = |CD| = 2^9
\end{gathered}
\end{equation*}
Три из четырех:
\begin{equation*}
\begin{gathered}
|ABC| = |ABD| = |ACD| = |BCD| = 2^6
\end{gathered}
\end{equation*}
Все четыре:
\begin{equation*}
\begin{gathered}
|ABCD| = 2^4
\end{gathered}
\end{equation*}
Итого:
\begin{equation*}
\begin{gathered}
P(X) = \frac{2^{12} \cdot 4 - 2^8 \cdot 2 - 2^9 \cdot 4 + 2^6 \cdot 4 - 2^4}{2^{16}} = \frac{2^{14} - 2^9 - 2^{11} + 2^8 - 2^4}{2^{16}} =
\end{gathered}
\end{equation*}
\begin{equation*}
\begin{gathered}
= 
\frac{14064}{65536} = \frac{879}{4096}
\end{gathered}
\end{equation*}
\begin{center}
\textbf{Ответ: } $\frac{879}{4096}$
\end{center}
\end{document}