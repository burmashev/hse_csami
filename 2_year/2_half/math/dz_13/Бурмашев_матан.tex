\documentclass[a4paper,12pt]{article}

%%% Работа с русским языком
\usepackage{cmap}					% поиск в PDF
\usepackage{mathtext} 				% русские буквы в формулах
\usepackage[T2A]{fontenc}			% кодировка
\usepackage[utf8]{inputenc}			% кодировка исходного текста
\usepackage[english,russian]{babel}	% локализация и переносы
\usepackage{xcolor}
\usepackage{hyperref}
 % Цвета для гиперссылок
\definecolor{linkcolor}{HTML}{799B03} % цвет ссылок
\definecolor{urlcolor}{HTML}{799B03} % цвет гиперссылок

\hypersetup{pdfstartview=FitH,  linkcolor=linkcolor,urlcolor=urlcolor, colorlinks=true}

%%% Дополнительная работа с математикой
\usepackage{amsfonts,amssymb,amsthm,mathtools} % AMS
\usepackage{amsmath}
\usepackage{icomma} % "Умная" запятая: $0,2$ --- число, $0, 2$ --- перечисление

%% Номера формул
%\mathtoolsset{showonlyrefs=true} % Показывать номера только у тех формул, на которые есть \eqref{} в тексте.

%% Шрифты
\usepackage{euscript}	 % Шрифт Евклид
\usepackage{mathrsfs} % Красивый матшрифт

%% Свои команды
\DeclareMathOperator{\sgn}{\mathop{sgn}}

%% Перенос знаков в формулах (по Львовскому)
\newcommand*{\hm}[1]{#1\nobreak\discretionary{}
{\hbox{$\mathsurround=0pt #1$}}{}}
% графика
\usepackage{graphicx}
\graphicspath{{pictures/}}
\DeclareGraphicsExtensions{.pdf,.png,.jpg}
\author{Бурмашев Григорий, БПМИ-208}
\title{Матан, дз -- 13}
\date{\today}
\begin{document}
\maketitle
\section*{Номер 1}
Надо понять, что такое $\frac{1}{z}$ (если $z = x + iy)$:
\[
\frac{1}{z} = \frac{x - iy}{x^2 + y^2} = \frac{x}{x^2 + y^2} +  i \cdot \frac{-y}{x^2 + y^2}  
\]
Тогда заменим:
\[
x' = \frac{x}{x^2 + y^2}
\]
\[
y' = -\frac{y}{x^2 + y^2}
\]
\[
z' = x' + iy'
\]
\[
|z|^2 = \frac{1}{x^2 + y^2}
\]
Знаем:
\begin{center}
\includegraphics[scale=0.4]{1.png}
\end{center}
Тогда подставляем:
\[
\zeta = \frac{ \frac{1}{x^2 + y^2}}{ \frac{1}{x^2 + y^2} + 1} = \frac{1}{x^2 + y^2 + 1}
\]
\[
\xi = \frac{ \frac{x}{x^2 + y^2}}{\frac{1}{x^2 + y^2} + 1} = \frac{x}{x^2 + y^2 + 1}
\]
\[
\eta =  \frac{ -\frac{y}{x^2 + y^2}}{\frac{1}{x^2 + y^2} + 1} = -\frac{y}{x^2 + y^2 + 1}
\]
\begin{center}
\textbf{Ответ: } 
координаты для $M\left(\frac{1}{z}\right)$:
\[
\left(
\frac{1}{x^2 + y^2 + 1}; \; \frac{x}{x^2 + y^2 + 1}; \; -\frac{y}{x^2 + y^2 + 1}
\right)
\]
\end{center}
\clearpage
\section*{Номер 2}
skip
\clearpage
\section*{Номер 3}
Подставляем в известное нам с семинара уравнение:
\[
\frac{z - 0}{z - i} \cdot \frac{\infty - i}{\infty - 0} = \frac{w - (-1)}{w - i} \cdot \frac{\infty - i}{\infty - (-1)}
\]
Замечаем:
\[
\frac{\infty - 1}{\infty - 0}  = \frac{1 - \frac{1}{\infty}}{1} = 1
\]
\[
\frac{\infty - i}{\infty - (-1)} = \frac{1 - \frac{i}{\infty}}{1 + \frac{1}{\infty}} = 1
\]
Тогда получаем:
\[
\frac{z - 0}{z - i} = \frac{w + 1}{w - i}
\]
\[
z(w - i) = (z -i)(w + 1)
\]
\[
zw- zi = zw + z - iw - i
\]
\[
- zi =  z - iw - i
\]
\[
iw = z + zi -i 
\]
\[
-w = zi - z + 1
\]
\[
w = -zi + z - 1
\]
\[
w = z(-i + 1) - 1
\]
\[
w = (1 -i) z - 1
\]
\begin{center}
\textbf{Ответ: } 
\[
w = (1 -i) z - 1
\]
\end{center}
\clearpage
\section*{Номер 4}
Аналогично предыдущему номеру:
\[
\frac{z - 1}{z - 0} \cdot \frac{i - 0}{i - 1} = \frac{w - 1}{w + 1} \cdot \frac{i + 1}{i - 1}
\]
\[
\frac{z-1}{z} \cdot i = \frac{w-1}{w + 1} \cdot (i + 1)
\]
\[
\frac{zi - i}{z} = \frac{wi + w -i -  1}{w + 1}
\]
\[
(zi -i)(w + 1) = (wi + w -i -1) z
\]
\[
ziw + zi - iw - i = wiz + wz - iz - z
\]
\[
 zi - iw - i =  wz - iz - z
\]
\[
zi - i + iz + z = wz + iw 
\]
\[
2iz - i + z = w(z + i)
\]
\[
w = \frac{2iz -i + z}{z + i}
\]
\[
w = \frac{z(1 + 2i) - i}{z + i}
\]
\begin{center}
\textbf{Ответ: } 
\[
w = \frac{z(1 + 2i) - i}{z + i}
\]
\end{center}
\end{document}
